\documentclass[]{book}
\usepackage{lmodern}
\usepackage{amssymb,amsmath}
\usepackage{ifxetex,ifluatex}
\usepackage{fixltx2e} % provides \textsubscript
\ifnum 0\ifxetex 1\fi\ifluatex 1\fi=0 % if pdftex
  \usepackage[T1]{fontenc}
  \usepackage[utf8]{inputenc}
\else % if luatex or xelatex
  \ifxetex
    \usepackage{mathspec}
  \else
    \usepackage{fontspec}
  \fi
  \defaultfontfeatures{Ligatures=TeX,Scale=MatchLowercase}
\fi
% use upquote if available, for straight quotes in verbatim environments
\IfFileExists{upquote.sty}{\usepackage{upquote}}{}
% use microtype if available
\IfFileExists{microtype.sty}{%
\usepackage{microtype}
\UseMicrotypeSet[protrusion]{basicmath} % disable protrusion for tt fonts
}{}
\usepackage[margin=1in]{geometry}
\usepackage{hyperref}
\hypersetup{unicode=true,
            pdftitle={《世说新语》的解释和疑问},
            pdfauthor={linyi518},
            pdfborder={0 0 0},
            breaklinks=true}
\urlstyle{same}  % don't use monospace font for urls
\usepackage{natbib}
\bibliographystyle{apalike}
\usepackage{longtable,booktabs}
\usepackage{graphicx,grffile}
\makeatletter
\def\maxwidth{\ifdim\Gin@nat@width>\linewidth\linewidth\else\Gin@nat@width\fi}
\def\maxheight{\ifdim\Gin@nat@height>\textheight\textheight\else\Gin@nat@height\fi}
\makeatother
% Scale images if necessary, so that they will not overflow the page
% margins by default, and it is still possible to overwrite the defaults
% using explicit options in \includegraphics[width, height, ...]{}
\setkeys{Gin}{width=\maxwidth,height=\maxheight,keepaspectratio}
\IfFileExists{parskip.sty}{%
\usepackage{parskip}
}{% else
\setlength{\parindent}{0pt}
\setlength{\parskip}{6pt plus 2pt minus 1pt}
}
\setlength{\emergencystretch}{3em}  % prevent overfull lines
\providecommand{\tightlist}{%
  \setlength{\itemsep}{0pt}\setlength{\parskip}{0pt}}
\setcounter{secnumdepth}{5}
% Redefines (sub)paragraphs to behave more like sections
\ifx\paragraph\undefined\else
\let\oldparagraph\paragraph
\renewcommand{\paragraph}[1]{\oldparagraph{#1}\mbox{}}
\fi
\ifx\subparagraph\undefined\else
\let\oldsubparagraph\subparagraph
\renewcommand{\subparagraph}[1]{\oldsubparagraph{#1}\mbox{}}
\fi

%%% Use protect on footnotes to avoid problems with footnotes in titles
\let\rmarkdownfootnote\footnote%
\def\footnote{\protect\rmarkdownfootnote}

%%% Change title format to be more compact
\usepackage{titling}

% Create subtitle command for use in maketitle
\newcommand{\subtitle}[1]{
  \posttitle{
    \begin{center}\large#1\end{center}
    }
}

\setlength{\droptitle}{-2em}
  \title{《世说新语》的解释和疑问}
  \pretitle{\vspace{\droptitle}\centering\huge}
  \posttitle{\par}
  \author{\href{http://www.tianya.cn/1907780}{linyi518}}
  \preauthor{\centering\large\emph}
  \postauthor{\par}
  \predate{\centering\large\emph}
  \postdate{\par}
  \date{2018-03-21}

\usepackage{booktabs}
\usepackage{amsthm}
\makeatletter
\def\thm@space@setup{%
  \thm@preskip=8pt plus 2pt minus 4pt
  \thm@postskip=\thm@preskip
}
\makeatother

\begin{document}
\maketitle

{
\setcounter{tocdepth}{1}
\tableofcontents
}
\chapter*{Motivation}\label{motivation}
\addcontentsline{toc}{chapter}{Motivation}

多年前第一次在天涯读到 linyi518
的\href{http://bbs.tianya.cn/post-books-108487-1.shtml}{《世说新语》}
愤慨于天涯纯文字,毫无排版的风格,又正好开始学markdown,于是就开始剪刀浆糊的试着排版,做了这篇。

这么多年过去,帖子还没有结束,似乎烂尾,而我也至今还没读完新语。放在GitHub聊做回忆吧

\chapter*{序}
\addcontentsline{toc}{chapter}{序}

白天,我常常是在写讲话稿和公文。晚上,一般就是打游戏和看书了。现在在玩《精忠报国岳飞传》,令人绝望的游戏,陆陆续续打了有一年了吧,即使用了修改器还是反反复复save-load,所以不但不能弥补工作中缺乏的成就感,反而更加郁闷。
君子疾没世而名不称,为了使自己活得更有意义些,总得找点事干,我突然打算在网上对《世说新语》进行解释。当然,有不少学者对《世说新语》进行了校注、笺疏,似乎没有必要重复劳动,不过这些学者的书籍多为繁体竖排本,多用文言,对普通人阅读有一定的障碍;其次,有一些译注或考虑版权问题,白话本的翻译往往过于简单、丧失文雅,不能很好地使读者了解人物、故事的背景,使《世说》成为平淡的书,不能让人满意;还有,《世说》人物繁多,又常常变换使用名、字、封地、号什么的,让普通读者摸不着头脑。而我恰巧少年时期在书籍中度过,又陆续收集了几十本关于魏晋的书,现在又有互联网搜索的便利,吾乡有句谚语:``山高高不过树,路长长不过步'',即使我写的再不如人意,但比从前的总有些优势。这些因素使我下了这个决心。当然,还有一个小小的愿望,我的孩子平时是由母亲和妻子抚养,除了休息日,我甚至很少和他见面,他的童年比我更加贫乏和封闭,我注释《世说》,希望他在长大一点后阅读,从而理解他的父亲,对现实世界保持一定的热情和一定的距离。每个人都走在各自的路上,即使我这点愿望或许是泡影,但总得做出一点努力。
所以,这本网络书首先献给我的孩子和那些怀着我一样想法的父亲;其次献给一位厦门的网友,她看着我打字已有三年时间,我颇为感激;最后要献给我的父亲,他让我的童年在图书馆长大,并从不吝啬地满足我购买书籍的要求(居然包括古代艳情小说。现在回忆起来,是多么古怪的父亲!),尽管现在他已经退休回老家种菜,似乎再也不看书了。
在写作过程中,网友们给了我鼓励,希望我不要做下面没有的人。是啊,大多数人总是希望做事善始善终,我也多么希望在一两年后,通过打字得到内心的满足,最后喊出``Game
over'',再开始另一场追逐。

\chapter{德行}

\section{1.1}\label{section}

\begin{quote}
陈仲举言为士则,行为世范,登车揽辔,有澄清天下之志。为豫章太守,至,便问徐孺子所在,欲先看之。主簿白:``群情欲府君先入廨。''陈曰:``武王式商容之闾,席不暇暖。吾之礼贤,有何不可!''
\end{quote}

\begin{itemize}
\tightlist
\item
  \emph{陈仲举}:就是陈蕃,是东汉时有名的人物。说他有名,因为我们中学时要写命题作文,常常会引用他的一个故事:据说他年少的时候,他父亲的一个朋友叫薛勤的来看他,看到他房子里比较脏,就对陈蕃说:``小朋友啊,为啥不洒扫以待宾客?''陈蕃当即回答:``大丈夫处世,当扫除天下,安事一室乎!''这回答很豪迈,薛勤感慨之余,劝道:``一屋不扫,何以扫天下?''这个故事没有下文,陈蕃到底打扫卫生没有。不过我猜大概没有。因为后来陈蕃做了很大一个官,叫太傅,太子老师的意思,正一品,虚衔,有时候可以掌管全国的军政大权。东汉末太监当权,陈蕃和大将军窦武打算消灭所有宦官,打扫面太大,居然还上奏,行事不密,不幸走漏风声,宦官们抢先一步,诡称陈、窦二人要废帝,骗得汉灵帝下诏诛杀他们。陈蕃志向很高,但在小细节上出了纰漏,可能就是不扫一屋的毛病。不过扫一屋和安定天下有没有必然联系,按照逻辑学的说法``异类不相比'',好像也没什么关系。不过中国人的思维是跳跃性的,往往都是这样。看了这个故事,不由让人想到汉初的陈平,有一次家族祭祀,推举陈平主持祭社神,为大家分肉。陈平分得很公平。大家赞扬说;``分得真好,太称职了!''陈平感慨道:``假使我能有机会治理天下,也能像分肉一样。''所以,我对屠户和清洁工保持着少有的敬意,自己在干体力活的时候,也常常这样自勉。
\item
  \emph{徐孺子}:即徐稚,字孺子,世人称``南州高士'',孺子本意是小孩子的意思,稚大约是幼稚的意思,所以他名和字是解释的关系。父母把自己的孩子叫孺子,大约是希望好养些的意思。陈蕃、徐稚的故事在《世说》中还有不少。
\item
  \emph{言为士则、行为世范}:这种修辞手法叫互文,如不以物喜、不以己悲,其实就是不以物己喜悲。这句话让人想起我们常说的学高为师,身正为范的``师范''说法。
  我曾经纳闷,为什么叫师范学校,不叫教师学校,而警察叫警官学校、技术叫技术学校。可见教师身担表率重任,是世间的规范,天地君亲师,民间要求很高的,不过当代中国斯文扫地,社会不知师道尊严为何物,教师自视甚底,所获甚少,自我要求甚宽。``学高为师,身正为范''在汉魏时候还有不少说法,如德高为师、言高为师、文为士范、文为德表等,出典大约都是这里。
\item
  \emph{登车揽辔、有澄清天下之志}:有两个解释,一是登高望远,树立雄心大志;二是走马上任,决心有一番作为。这个情景常常被画成一幅图像,一些伟大人物骑着高头大马,指点江山,意气风发。柳永作了一首《望海潮》,据《鹤林玉露》载,此词流播,金主完颜亮闻歌,欣然有慕于``三秋桂子,十里荷花'',起了渡江南猎的念头,派人潜入杭州画了一张西湖图,带回金国后制成一扇屏风,仍觉不过瘾,又在画中添上自己策马立于吴山的画像,并题词其上:``万里车书尽混同,江南岂有别疆封?
  提兵百万西湖上,立马吴山第一峰。''
\item
  \emph{为豫章太守}:唐王勃在《滕王阁序》中说有两句话:``豫章故郡,洪都新府。\ldots{}\ldots{}人杰地灵,徐孺下陈蕃之榻。''说的就是陈蕃和徐稚。陈蕃每次请徐稚过来,两人相谈甚欢,忘了时间。陈蕃特意给徐稚准备了一张卧榻,让他留下过夜。等徐稚一走,陈蕃就把卧榻悬挂起来,直到徐稚再来,他才放下来。这个举动表明其他人不配。至于王勃不称徐稚而用徐孺,可能是照顾平仄。
\item
  \emph{主簿}:官名,主官属下掌管文书的佐吏,相当于现在的办公室主任或秘书长。《文献通考》卷六十三:``盖古者官府皆有主簿一官,上自三公及御史府,下至九寺五监以至郡县皆有之。''隋、唐以前,因为长官的亲吏,权势颇重。魏、晋以下统兵开府之大臣幕府中,主簿常参机要,总领府事。
\item
  \emph{群情欲府君先入廨}:府君是对太守的尊称。``百官所居曰府'',``郡守所居曰府,府者尊高之称'',``汉人称太守为明府''。
  廨,官衙。``大家的意思是希望你先进官署(与大家见面或视察)''。现在新官上任,一般组织部领导陪同,致辞,然后合乎身份的同僚一起下馆子吃上一顿。陈蕃一上任就去拜访当地的学者名人,这大概是当时的风气,``宣室求贤访逐臣'',
  据当时的徐干形容说,``自公卿大夫,州牧郡守,王事不恤,宾客为务,冠盖填门,儒服塞道。饥不暇餐,倦不获己,殷殷沄沄(水流动貌),俾(使)夜作昼''(官僚们不顾政事,一心招揽宾客,府上都是士人。和他们交谈通宵达旦,殷情备至,连疲劳也不顾,连饭都不吃)。现在天下英雄尽入彀中,四海已无遗贤,领导自己就是本国、本省、本市最大的贤人,不然他怎么会当上领导呢?
\item
  \emph{武王式商容之闾}:式,通``轼'',以手抚轼,表示尊敬的礼节,这里大概引申为拜访、敬礼。闾,小巷。《尚书
  -
  武成》中说周武王:释箕子囚,封比干墓,式商容闾。商容,据说直谏纣王被免职,《封神演义》里有这个人的,忠臣。《淮南子》说他是老子的老师,那么果然是神人了。这种风气现在还有影子,如胡锦涛上任后去西柏坡,习近平在上海是去一大纪念馆,不过已经等而下之,不向贤人故里致敬,而是强调道统和执政合法性。
\item
  \emph{席不暇暖}:屁股还没坐热。中国人讲话往往比较夸张、曲折,由此可见。
\end{itemize}

徐稚是个隐士,他的老师、朋友当上官了,按汉代的官制是实行举荐的,所以常常辟他当各式各样的官,据说其中有个太守的官送给他,他都不要。而且这些人当官后徐稚就不太和他们往来了。但徐稚喜欢吊唁,老师、朋友死了,他会不远万里去哭一番。而且在吊唁的时候,他自己带一块棉布,浸上酒,晒干后包住一只鸡,到了墓前也不和死者家人打招呼,把棉布浸上水,绞干洒在坟头,再大哭。这样的事他做过好几次。这个奇怪的举动也许是表达他用自己的东西来祭奠,心意诚恳。
陈蕃有很高的声望,被杀后,他的宗亲宾客姻属,也几乎都被杀光了。《后汉书》的作者范晔说陈蕃以遁世为非义,所以屡退而不去,以仁心为己任,虽道远而弥厉。虽然大功未告成,但其信义足以携持民心。百余年间,汉室乱而不亡,陈蕃功劳最大。
本篇是《世说新语》的第一章:德行。在现代汉语中,如果有人说:``德行!''隐含着没有德行的意思,就是骂人了。这个语言现象很有趣。

\section{1.2}\label{section-1}

\begin{quote}
周子居常云:``吾时月不见黄叔度,则鄙吝之心已复生矣!''
\end{quote}

\begin{itemize}
\tightlist
\item
  \emph{周子居}:周乘,当过太守,很有些官声,后面陈蕃还会继续夸他,这里是他夸别人,而且是反反复复夸,一心一意要把黄宪夸向全国。东汉人喜欢品评人物,叫做``月旦'',大约是每月初一大家聚会,开始评论人物,此风以汝南为最,所以也叫``汝南月旦''。这种做法当然是为了适应汉代的举荐制度,互相吹捧的确是成名的要诀,我几乎没有看到过人物品评说别人坏话的,就像现在的文学批评。汉代做官有四个要求,第一就是德行高妙、志行洁白,其他如学问大,懂法令,会决断都是次要的,那些也需要实践检验才行。刘义庆把德行放在第一,固然是《论语》孔门弟子的体例,但大概也是受到时代的影响。至今,我们选拔干部,表面上还是品德第一。不过好像黑格尔说过:``道德,不过是小人物对大人物的妒忌罢了'',当然也有另外的说法,专制社会中由于机制问题,就像狼群,越往上越凶残。
  时月:隔一段时间。
\item
  \emph{黄叔度}:黄宪,汝南人。宪,大约是法律的意思,度也是法规。伯、仲、叔、季,黄宪排行老三,所以叫叔度。当时不少人称他是``颜子复生'',一方面说他家穷,父亲是牛医,一方面说他德行高。据说当时有个叫戴良的,平时自视甚高,见了黄宪自叹不如,怅然若失。戴良母亲见他不太开心,就问他:``你又不高兴了,是不是又见到那牛医的儿子了?''戴良说:``瞻之在前,忽焉在后,黄宪就是我的老师啊。''戴良就直接把黄宪当作孔子,自己是颜回。可是``颜回''见到``孔子''
  还不高兴,是既生瑜,何生亮的意思。孔子说``未见好德如好色''的,看到漂亮姑娘怦然心动,看到德行比自己高的,心里就不舒服了:``妈妈的,德行比我高,怎么行啊,可追不上啊!''黄宪也不当官,当时有人称他为``征君'',就是被官府征召过但不接受官职的人。黄宪在历史上没有留下什么著述,但他是当时隐逸派的领袖。
\item
  \emph{鄙吝}:小家子气,庸俗气。
\end{itemize}

这个故事《后汉书》中是陈蕃和周举说的,古代常常有这样的情况,主人公换来换去,要么是年代久远记错了,要么是故事人物就像现在写讲话稿,抄来抄去的。余嘉锡说应该是周乘和黄宪,因为他们同时举孝廉,《世说》后面也提到周乘和黄宪交好。
周乘表扬人很有修辞手法,叫反衬,用自己来烘托黄宪,更加突出黄宪的高洁,同时也说明自己有时候也挺高洁的。大约宋代黄庭坚也说过类似的话,说自己三天不看竹子,就鄙俗起来。鲁迅嘲笑那些写``我的朋友胡适之''的人,这种借名人自重的做法,现在还有。再``鄙吝''一点,不少单位里的领导,办公室里挂着自己和中央领导、省委书记、省长等握手之类的合影,我看了很有趣,没有人挂自己和雷锋、张海迪一起的合影,恐怕张海迪家里倒有与各位领导的合影,时代就这样变化,没有办法的。

\section{1.3}\label{section-2}

\begin{quote}
郭林宗至汝南,造袁奉高,车不停轨,鸾不辍轭;诣黄叔度,乃弥日信宿。人间其故,林宗曰:``叔度汪汪如万顷之陂,澄之不清,扰之不浊,其器深广,难测量也。''
\end{quote}

\begin{itemize}
\tightlist
\item
  \emph{郭林宗}:郭泰,为什么字林宗,我不知道,也许是泰山就是大树林的意思。东汉有个大名人叫李膺的(后面马上提到),曾经说过:``我见的士人多了去了,但没有人比得上郭泰'',于是郭泰出名了。蔡邕经常给人写墓志铭,人称``谀墓''赚钱,他曾经说:``我写墓碑,基本上自己也很羞愧,但只有写郭泰的墓碑,写那些表扬的话我问心无愧。''有本书叫《寰宇记》,说南北朝时候的周武帝要铲除天下的墓碑,特意下诏说要保留郭泰的墓碑。(不过我实在弄不懂雄才大略的宇文邕怎么会跟墓碑过意不去。后来看到资料,说曹操也下过禁碑令,理由是``妄媚死者、增长虚伪,浪费资财、为害甚烈''。)郭泰没当过什么官,是教育家,有些学生还是走卒农夫、刑释解教人员等,但在他的引导下,都成为名士。《后汉书》中说,有一天郭泰走在路上,遇到雨,就折头巾的一角挡雨,后来这种头巾的款式就叫郭巾,我猜那就像电视里不少店小二的头巾式样吧。这个故事可以想见郭泰的潇洒风采。
  造、诣:都是拜访的意思。
\item
  \emph{袁奉高}:袁阆,阆就是高大的意思,在汝南慎阳也算是为名士。后面还有他的故事。在当时大概已经做了汝南郡功曹(管组织人事等)。
\item
  \emph{车不停轨,鸾不辍轭}:都是没有停车的意思,这里也许应该理解为没有下车或者停留时间很短。鸾是凤凰的一种。这里使用的修辞手法是隐喻和借代。鸾是鸟,鸟是很好的使者,使者要坐车子,车子上有个小铃铛就叫鸾铃,鸾铃就代表车子了。轨、轭都是车的一部分。车不停轨、鸾不辍轭就是车不停车,这种修辞手法就说明汉语的繁复啰嗦,仅仅为了追求对仗之美,后人有所诟病,但始终改不过来,因为好看、好听,还体现语言的智慧。在《世说》的注中,郭泰解释了没有停车的理
  由,理由是袁阆的才德(譬诸氿滥,虽清易挹)像小山泉,虽清,却容易舀光,大概是说袁阆器局没有深度、广度。
\item
  \emph{弥日信宿}:弥,长、满;弥日就是整天的意思。信宿,见《诗经 -
  有客》``有客宿宿,有客信信'',据说是过了两夜的意思。这里是说郭泰整日和黄宪待在一起,过了好几天。
\item
  \emph{陂}:蓄水池叫陂,万顷之陂就是100平方公里的湖泊。
\end{itemize}

这一条是说黄宪气度宽广,深不可测,大概是像刘备,喜怒哀乐不露于形色,但似乎与``德行''没有什么关系,而应该把这条放到``赏誉''中。但前三条的主人公都是拒绝做官的,也许在刘义庆眼里,拒绝当官是最高的德行。孔子说``智者乐水,仁者乐山'',把人比喻成水,言下之意就是智者,有大智慧的人可能对水特别有感悟。《道德经》说:``上善若水,水善利万物而不争,此乃谦下之德也;故江海所以能为百谷王者,以其善下之,则能为百谷王。天下莫柔弱于水,而攻坚强者莫
之能胜,此乃柔德;故柔之胜刚,弱之胜强坚。因其无有,故能入于无之间,由此可知不言之教、无为之益也'',等等。这样理解,黄宪的德行像大水,行的是不言之教,所以放在德行中也解释得通。本篇使用对比手法,用袁阆的单纯来衬托黄宪的深厚,用停车的长短来反映人格魅力的持久吸引力,谁叫袁阆和黄宪住一个地方的,袁阆就当了牺牲品。

\section{1.4}\label{section-3}

\begin{quote}
李元礼风格秀整,高自标持,欲以天下名教是非为己任。后进之士,有升其堂者,皆以为``登龙门''。
\end{quote}

\begin{itemize}
\tightlist
\item
  \emph{李元礼}:李膺,字元礼,河南颍川人。好像汝南也在河南,当时中原人才辈出,不像现在河南人名声不太理想,还有本书反映这个问题,我想现在河南人太多,气候变迁,童山濯濯,比较穷,生存空间小,官瘾太大,风气就不太好。古代叫元礼的人很多,唐代武则天时有个叫索元礼的,专审谋反案,手段很残忍,杀人无算,给人印象更深,真是好事不出门,坏事传千里。李膺也管刑法,当过司隶校尉(曹操很喜欢当这个官,当了丞相后还兼着,由此可见曹操的法家气质),类似于监察部、公安部、司法部领导,他曾经把张让(有名的太监头子)的弟弟张朔当场格杀,汉桓帝责问李膺为什么不走程序,李膺说:``过去孔夫子做司寇,上任七日就诛少正卯(《论衡》说,孔子和少正卯一起办学,孔子的教室`三盈而三虚',三千学生全跑到少正卯那个教室里去了,因为孔子有点口吃,又喜欢讲大道理,学生不爱听。后来孔子当了司法部门领导,立即打击报复,找了一堆莫名其妙的理由把少正卯杀了。后人认为这个故事是编的,主要在独尊儒术的背景下流行起来,完全是儒生行法家之事时的借口。因为古时候判案,也讲案件引证,用六经大义和孔孟的行为作为审理案件的立足点)。今天我到任十天了,才杀张朔,我还以为除害不速而有过呢!如今只请求陛下让我再活五日,除掉张让,然后再烹杀我,我也心甘情愿。''据说后来平时趾高气扬的宦官假日里也不敢出宫门玩耍,桓帝感到奇怪,宦官说:``畏李校尉。''还有个说法,李膺当青州刺史的时候,刚下任职命令,整个州70多县官全自动离职,因为屁股不干净怕被李膺干掉。好像英国还是美国有个法官,也很严厉,待审判的嫌疑人知道是他主审,就先自杀了,法官说:``因为我遍使正义,罪犯畏罪自杀。''这种故事当然可以从正反两方面看。李膺军事指挥能力很好,不怕死,和少数民族打过几个胜仗,可称文武全才。李膺是东汉``党锢之祸''的主要人物之一,表现很突出,后面还有他的故事。李膺名和字的关系我看不懂。
\item
  \emph{风格秀整}:风采出众,品性端正。
\item
  \emph{高自标持}:自视甚高,把自己当标尺。
\item
  \emph{名教}:以儒家主张的正名定分为准则的礼教,
  要求各阶层按照合乎自己等级身份的方式生活,大意就是三纲五常。东汉末期道德崩溃,不能各安其位,李膺``欲以天下名教是非为己任''就是说自己担负着建立正确的社会荣辱观,使社会秩序按照儒家的``理想国''思想运行的责任。按北宋张载的说法就是:``为天地立心,为生民立命,为往圣继绝学,为万世开太平。''魏晋时期虽然十分混乱,但直接打杀在名教大棒子之下的人比任何时代都多一些,这现象很矛盾,但想想现在,也好理解。现在的中国,是历史上处分贪官污吏最广泛、
  最严厉的时期之一,同时又是最不知道德为何物的时期之一。
\item
  \emph{升其堂}:登上他的厅堂,这里要理解成有机会接受李膺的教诲。《论语
  -
  先进》子曰:``由(子路)也升堂矣,未入于室也。''升堂本来是学业有所成的意思,入室就是学到家了,学问很深。
\item
  \emph{龙门}:在现在的山西河津县。黄河在此有较大的收口,水势很汹涌,《三秦记》中说:龟鱼有能游上去的,就会变成龙。这个故事很有意思,乌龟也能变成龙?估计是偏义复词,只能落实到鱼上。当然,龙性好淫,一向胡乱交配,而且往往一次受精成功,就有种很像乌龟的动物叫赑屃,据说是龙的九子之一。汉代经常把人比喻成龙,并不犯忌,如果是满人统治之下,这些李膺的学生都得砍头了,想想都怕。我没去过龙门,估计龙门不过尔尔,因为我多次看钱塘江涨潮,就觉得古诗词实在夸饰了,要么地理发生了变化,要么古人对大自然特别敬畏,一点鸡毛蒜皮般的自然现象还以为天塌下来了。
\end{itemize}

李膺被太学生誉为``天下楷模''。``党锢之祸''的时候,有人劝其走之,他像谭嗣同一样拒绝了,然后自投牢狱,被杀。李膺的了不起,只要比比李大钊就可以想见,李大钊道德据说也算高洁,不过还要躲在苏联使馆避风头,没料到张作霖土匪出身,不守游戏规则,俄国人又一向滑头,结果大学者命丧大强盗之手。当时有个学子叫荀爽(荀彧的叔父,据说是荀氏家族最杰出的人,少年时就是英才),有一天他得意洋洋地说:``哈哈,今个儿老百姓,真呀真高兴,我居然能给李膺赶马车,太荣幸了。''袁宏在《后汉纪》中说:``李膺话出口,没有人能反对的。如果有人非议李膺,则舆论会一边倒地谴责。李膺如果与人坐同一驾马车,那人立刻名闻天下。''爱因斯坦在对居里夫人的悼词中说:``第一流的人物对于时代和历史进程的意义,在其道德品质方面,也许比单纯的才智成就方面还要大。即使是后者,它们取决于品格和程度,也远超过通常所认为的那样。''在东汉末期,李膺当得起爱因斯坦的评价。只是让人感到苦涩的是,李膺他们死后,历史进程大大加速了------东汉终于随之覆灭。

\section{1.5}\label{section-4}

\begin{quote}
李元礼尝叹荀淑、钟皓曰:``荀君清识难尚,钟君至德可师。''
\end{quote}

\begin{itemize}
\tightlist
\item
  \emph{李元礼}:李膺。
\item
  \emph{荀淑}:字季和,颖川人,曾任朗陵侯相,所以又叫他荀朗陵,古人称呼人,虽然一般称字,但有时候也用官职、谥号、当官的地点、出生地、家族兴盛地等,有时甚至是他的作品和讲过的某句话来称呼,往往让人摸不着头脑。宋代有本《事林广记》,说刘某很喜欢读书,他很纳闷:``班固《汉书》中文章很好啊,为什么他的作品没有进《文选》呢?''旁边有人告诉他``《两都赋》、《燕然山铭》就是班固的作品啊,怎么说没有?''刘某不懂:``那上面明明写的是班孟坚,怎么是班固
  啊!''古人尚且如此,更不要说现在了。在中国,怎么称呼人常包含有微妙的信息,据说温家宝在国务院内部,所有工作人员一般称呼他家宝同志,算是很少见的。
  我们平时一般称呼领导用职务,基本平级的领导之间不叫姓,只叫名加同志,如果称职务,说明生疏或彼此之间希望保持距离,等等,这些还依稀有古人的影子。不过我孩子叫我,一般直呼姓名,有时甚至是小名,弄得小学的老师直看我。我看回忆录,好窥人隐私,可书上往往一说到坏事,就是王某、李某,我就要开动脑子去对号入座。前些年看到罗瑞卿女儿回忆录的片段,说开国陈将军某长期霸占他的侄女,搞得我八卦心大起,在网上一片狂搜。闲话不提,在东汉,颖川和汝南是士人荟萃之地,估计是他们拉帮结派,形成了风气。当时就有人评论颖川和汝南哪个地方士人更优秀一些,一般认为汝南人骨头更硬一点。其实颖川和汝南都在河南一带,只是汝南被称为``天中'',是古豫州的中部。荀淑据说是荀子的后代,他当了朗陵侯相后,处事非常明晰,被誉为``神君'',为荀氏家族博得了很好的名声,当然,他生了八个好儿子,人称``八龙'',
  连他们所居住的地方这八个杰出的儿子而被颖阴令改名为高阳里,因为传说中的帝王高阳氏也有八个好儿子,不过我疑心是荀爽当了司空(很大的官,在东汉相当于组织、监察部部长,三公之一)后,县令才干这种锦上添花的事。荀淑有个孙子叫荀彧,《三国演义》中曹家帮的最佳男配角之一,祖、父辈为荀彧打了很好的舆论基础,荀彧也替祖、父辈露了脸,算是很难得的。因为李膺比荀淑、钟皓要小二十多岁,称他们的字也不合适,所以用了``君''这个称呼,当然,``君''也有人说只能用于平辈间的尊称,但这里不像。荀淑去世后,李膺向朝廷上表,说自己的老师去世了。在《世说》刘孝标的注中,说``(荀淑)所拔韦褐刍牧之中,执案刀笔之吏,皆为英彦'',这句话很费解,表述存在问题,拔是提拔、选拔,韦褐刍牧借代布衣农夫等贫贱之辈,执案刀笔指文书小吏,似乎应该这样表述:所拔多在韦褐刍牧之中,执案刀笔之吏,后皆为英彦。据《后汉书》中说,有一天,年长的荀淑遇到14岁的黄宪,交谈很久,荀淑高兴地说:``黄宪真是我的老师啊。''不过按照我卑鄙地揣摩,这可能是在演孔子和项橐那场戏。为了让这场街头话剧传播出去,故事还有续集,荀淑告别黄宪,又去找袁阆,见面就问:``你这里出了个颜回,你知道吗?''
  袁阆知情识趣,就反问:``您是不是见到黄宪了?''后来当然由袁阆负责广泛传播这段佳话。这是一群多么优秀的演员啊!只是天可怜见,袁阆在我才开始的文章中,已经跑了两次龙套,他的心情是多么郁闷。
\item
  \emph{钟皓}:字季明,也是颖川人,书法家钟繇的祖父,小人钟会的曾祖父。钟皓与李膺有一点亲属关系,他当过县令,据说当得很不错,后来看到形势太险恶,就辞职了,他对侄子说:``春秋时齐国大夫国武子喜欢当面说别人的缺点,以致使人怨恨他。做坏事到一定程度
  就会败露,当没有败露前你说人家,他们就会恨你。所以你要保全姓名和家族,还是不要出头的好。''不要出头,但要有名,不然会被狼吃掉的。钟皓退休后就开馆授课,学生不少。《世说》的注中说,钟皓官位小,但德行却绰绰有余(人位不足,天爵有余,语出《孟子》)。
\item
  \emph{清识难尚}:见识透彻很难比得上。这可能是说荀淑善于识人,也可能是说形势判断准确,拿得起、放得下。据说荀淑生意做得很好,挣了不少产业。
\item
  \emph{至德可师}:汉、魏、晋时代的至德应该是指出奇的孝,汉武帝后当官是``举孝廉''(孝廉也是偏义复词,还没当官不存在廉的问题)。据说张角起义,侍中向栩担心兴兵扰民,便建议朝廷:``但遣将于河上北向读《孝经》,贼当自灭。''二十四孝中,十之八九的变态故事发生在那个时期。钟皓家族一向以孝著名,传统深厚。古时候的孝做起来有点难度,譬如长辈去世,守孝三年,不能和妻妾同房,而汉代房中术很盛,俨然矛盾,结果是``直如弦,死道边;曲如钩,反封侯。''
\end{itemize}

李膺有三个老师,上面是其中两位,还有一位是陈稚叔,也是至德之人,自然孝貌也很昂然,不过古代的孝顺是现代的笑柄。

\section{1.6}\label{section-5}

\begin{quote}
陈太丘诣荀朗陵,贫俭无仆役,乃使元方将车,季方持杖后从。长文尚小,载著车中。既至,荀使叔慈应门,慈明行酒,余六龙下食。文若亦小,坐著膝前。于时太史奏:``真人东行。''
\end{quote}

\begin{itemize}
\tightlist
\item
  \emph{陈太丘}:名寔(shí),字仲弓,也是颍川人,当过太丘的县令。陈寔明辨是非,他当乡绅的时候,乡里有争讼都请他来评判,大家常说:``甘为刑罚所加,莫为陈君所短。''``梁上君子''典故的主人公也是他,这些都说明他善启他人的羞耻之心。他的孙子叫陈群,就是《三国》中刘备早期在徐州经常向他征求意见的人,《世说》后面还会提到。陈寔有个老师叫樊英,一直拒绝做官,面对汉顺帝也毫无畏惧,说:``我虽然是平民,住在陋室,却能怡然自得,无异于帝王的尊贵。陛下怎么能使我显贵,又怎么能使我低贱!''他对妻子也很讲礼貌,有平等思想。这些品性都在陈寔这里有所反映。
\item
  \emph{荀朗陵}:就是前面的荀淑,大家都认识他了。
\item
  \emph{元方将车,季方持杖后从}:大儿子陈纪和三儿子陈谌。他们的名字告诉我们,古人名和字的关系有时候没联系,不过是陈寔在希望孩子有条理、守信用的同时,又寄望他们为人方正。将车,拉车,既然家里穷,估计只能拉了,老子坐上面。持杖,不知是挑担还是柱着手杖跟在后面,但如果是挑担,古文当中大概应该写成``杖荷'',
  柱着手杖走说明陈谌年纪比较小,路还比较远。陈纪在《世说》中会多次露脸,的确很方正,陈谌出场机会很少,因为早夭。陈寔、陈纪、陈谌在当时人称之为``三个君子'',他们往往同时接受朝廷表彰和任职,大臣们赠送的小羊和大雁成堆(``每宰府辟召,羔雁成群''。《礼记-曲礼》:``凡挚,天子鬯,诸侯圭,卿羔,大夫雁。)
\item
  \emph{长文}:陈群,陈纪的儿子。这是一个很识时势的人,曹魏的三朝老臣,永远站在胜利者的一方,建立``九品中正制''是他的主要成绩,但舆论当然认为他德行不及祖、父辈。譬如曹丕立国后,大肆赐爵,问陈群:``我应天受禅,大伙儿个个喜悦,露于形色,惟独华歆和你没笑,这是为了甚么呢?''陈群便离席长跪道:``臣与华歆曾为汉朝之臣,内心虽替陛下感到非常喜悦,但在义理上,臣等的神色实应畏惧、甚至憎恨陛下才对。''曹丕听后大悦。陈群是组织部门领导,自然喜欢臧否人物,曾与崔林共论冀州人士,崔林说冀州当以崔琰(因顶撞曹操被处死)为首。陈群却以``智不存身''加以讽刺,崔林立即反驳:``大丈夫为人但有邂逅而已(命运是有偶然性的),即使如卿等几人,何足为贵!''
\item
  \emph{叔慈应门,慈明行酒,余六龙下食}:荀靖在门口等待接送客人,荀爽倒酒,其余六个儿子给客人上菜。东汉时期龙比较多,李膺的学生都是龙,荀淑的儿子也都是龙,后来还有细分什么某某龙头、某某龙躯、某某龙尾的``佳话''。余嘉锡说:``八龙之中,慈明名最著,叔慈次之,余六龙碌碌无所短长,足见纯盗虚声,
  原非实录。''
  王符在《潜夫论》中说:``今观俗士之论也,以族举德,以位命贤'',慈明名最著,那是因为荀爽官最大,叔慈次之,因为荀靖拒绝做官,一心隐居,其他几个做不大不小的官,不说也罢。上面说过,陈寔家穷,荀淑家富,名声在外,荀淑一定要安排大餐胡吃海喝。
\item
  \emph{文若}:荀彧。彧就是文的意思。荀彧被曹操喻为张良,言下之意自己像刘邦。曹操出兵征伐,常留下荀彧处理朝廷事物。可是这个张良最后不识趣,居然反对曹操当高祖。``妈的,老子十几年前一见面就夸你是张良,表明自己的雄心大志,这些年你也算支持,想不到到关键时刻摆我一刀,自己只能当周文王了!''荀彧受到曹操的猜忌和作弄,据说最后曹操送食物给荀彧,荀彧打开食器,空无一物,因此服毒自尽(不由让人想起传说中徐达吃蒸鹅的故事,君子闻弦歌而知恶意)。荀彧的妻子是太监唐衡的女儿,因为荀彧的确有才华,才免于时讥。程炎震说:``荀淑年六十七,建和三年卒。荀彧以建安十七年卒,年五十,则当生于延熹六年。距荀淑之卒已十四年矣。''所以这个故事是假的,起码这些主人公不能出现在同一个场景,年纪对不上。
\item
  \emph{太史}:《后汉书》中介绍,汉代设太史令一人,六百石。本注曰:掌天时、星历。凡岁将终,奏新年历。凡国祭祀、丧、娶之事,掌奏良日及时节禁忌。凡国有瑞应、灾异,掌记之。
\item
  \emph{真人}:正人,贤人,品行端正的人。真人还有不少意思,如仙人、帝王。《世说》注中说:``陈仲弓从诸子侄造荀父子,于时德星聚,太史奏:`五百里贤人聚。'\,''德星,黄色的星体,又叫景星,据说凡景星、庆云为大瑞,后来汉灵帝就在颍川敕建``德星亭''。这种星座学中西方都有,名人就是明星。按文章意思理解,这里的真人、德星应该是指陈寔,与荀淑无关,是啊,富人上天堂比骆驼过针眼还难。文章最后一句是修辞学的点睛之笔,现代人看来荒诞不经,可在那个时候,一次寻常的聚会,就顿时有化腐朽为神奇的美妙。
\end{itemize}

\section{1.7}\label{section-6}

\begin{quote}
客有问陈季方:``足下家君太丘有何功德而荷天下重名?''季方曰:``吾家君譬如桂树生泰山之阿,上有万仞之高,下有不测之深;上为甘露所沾,下为渊泉所润。当斯之时,桂树焉知泰山之高,渊泉之深,不知有功德与无也。''
\end{quote}

\begin{itemize}
\tightlist
\item
  \emph{客}:一种是来访的客人,一种是家里养的门客。古代养门客之风是现在不能想象的兴盛(现在私人养的那叫二奶,再没有人养智勇之士供他们驱使了),我们现在的政府参事室、人大、政协组织是党养的门客,有时候主人当到退休,还恋栈那些公车、吃喝、交际,也甘愿再当一阵子门客,说几句新主人爱听的话,营造主人受拥戴的氛围。前两天政协的领导来办公室清谈,谈及一句顺口溜:``党委说了算,政府算了说,人大说算了,政协算说了。''大伙儿大笑。不过我不知道那时候陈寔已当过县令,家里发了没有,本文中的客到底是客人还是门客,但无论什么身份,他俨然是很好的捧哏。在东汉,无论当什么客都要有点责任意识。
\item
  \emph{陈季方}:陈谌。见上文。
\item
  \emph{家君}:对自己或他人父亲的尊称。家,居然能表示你家、我家两个意思,这种用法在现代汉语中不知道有没有。据说有些北方人自来熟,和朋友聊天,谈及对方父母,用``咱爸咱妈'',了不起,也许是同一种语法现象。
\item
  \emph{足下}:对同辈、朋友的尊称,``您''。我也不敢看你的脸,只敢看你的脚下泥土。因为你人格太高,面子太大,我就低头和你说话。
\item
  \emph{太丘}:陈寔。刘备人称刘豫州,李白写信自荐于韩朝宗,称呼他韩荆州。
\item
  \emph{功德}:可能是个佛教词语,功指善行,德指善心。也可能出于《左传》:``太上有立德,其次有立功,其次有立言,虽久不衰,此之谓不朽。''这个词语看似简单,但联系下文很难把握其意思。
\item
  \emph{荷天下重名}:能担当(获得)天下高尚的名声。去年还是前年,中央党校有个叫龚育之的死了,我在《学习时报》上被那些阿谀奉承的悼文恶心了一阵,好像中国真死了个文曲星似的,不就是一个翰林院的小丑吗?照道理说,我不了解龚育之,不能刻薄。不过我在报纸上看过他的一篇文章,大舔去世的邓小平先生的痔疮倒还罢了,据文章说,南巡讲话后《解放日报》有一篇对改革开放产生``重大影响''的问答式文章,作者是中央党校三个副校长,龚育之其中之一。龚育之得意洋洋地回忆,其实三个作者就是我一个,我模仿其他两个同事的口吻写的,为了增加文章的权威性。我的文章一出,天下舆论易变,从此改革的春风吹遍大江南北。我呸!文章千古之事,玩这种不足称道的小伎俩。
\item
  \emph{阿}:山的拐角。\emph{渊}:深。
\end{itemize}

陈谌对客人的提问早有准备,我可以想见他拿出稿子流利地说了一段文采很好的言辞,哦,不!陈谌记性好,会背诗的。为什么这样说呢?因为枚乘《七发》中有一段:``龙门之桐,高百尺而无枝。中郁结之轮囷,根扶疏以分离。上有千仞之峰,下临百丈之溪。湍流溯波,又澹淡之。其根半死半生,冬则烈风漂霰飞雪之所激也,夏则雷霆霹雳之所感也。''陈谌这段话文辞优美,是赋体,只是模仿枚乘,有点投机取巧。但桐树孤独高洁,桂树就显得芬芳高雅,因为陈寔那是和光同尘、德化天下,用桂树的比喻比较恰当。美好的言辞总是以模糊为特征。陈谌结尾用了排比和反问这种修辞,排比、反问在文本学中,是攻击性的。(这一段有人认为``吾家君''应该为``吾于家君'',多一个``于''字,也有种说法是``吾家君''
应该删去,如果这样就好理解。陈谌把自己比喻成桂树,把父亲比喻成泰山,桂树不知道泰山的高深。不过把自己比喻成桂树是非常自信的,是否符合当时道德要求,存疑。)这段话大意是说``我父亲得天独厚,天人合一,生而知之,根本不知道、不关注自己有没有功德。''陈谌的话当然让人想起左思的那首名诗《咏史》:``郁郁涧底松,离离山上苗。以彼径寸茎,荫此百尺条。世胄蹑高位,英俊沉下僚。地势使之然,由来非一朝。金张籍旧业,七叶珥汉貂。冯公岂不伟,白首不见招。''汉代除了用龙比人,现在我们开始接触用树比人了,后面还有各式各样的比喻,包括豺狼虎豹这些美好的禽兽,正如亚里士多德说的:``比喻是天才的象征。''

\section{1.8}\label{section-7}

\begin{quote}
元方子长文,有英才,与季方子孝先各论其父功德,争之不能决,咨于太丘。太丘曰:``元方难为兄,季方难为弟。''
\end{quote}

\begin{itemize}
\tightlist
\item
  \emph{陈元方子长文}:陈纪的儿子陈群。
\item
  \emph{季方子孝先}:陈谌的儿子陈忠,显然不像陈群出众,因为陈寔曾经对族人说:``陈群必定兴盛我陈氏家族。''
\item
  \emph{英才}:《礼记 辨名记》``德过千人曰英。''《淮南子 -
  泰族》``智过万人者谓之英。''陈群有出众的才智,这句话
  放在这里比较突兀,按道理说与文章无关,应该省略。我们再进一步说,甚至儒家的伦理认为,子女不应该讨论父亲的品德。所以,作者把``有英才''加在这里别有用意。
\end{itemize}

两个小孩子比较父亲的优劣,居然比的是``功德''而不是力气、学问、金钱、官位大小,咄咄怪事,也许古人寿命短、结婚早,这种对生命的紧迫感,自然要早熟一点。陈寔是陈纪、陈谌的父亲,在对话中称自己儿子的字,这在古代比较少见,不很合礼仪,说明该故事存在一点疑问,或者说明《世说》本来就写的比较草率,是街头巷语的小说而不是严谨的历史传记。难兄难弟很好理解,就是不相上下,一样优秀的意思,但仔细品味,``难为兄,难为弟''具体怎么逐字翻译便于理解,我把握不好,有人说兄弟就是高低上下的意思,有个有趣的网友看不惯汉末互相吹捧的风气,直接批评道:就是当哥哥的不像哥哥,当弟弟的不像弟弟。难兄难弟的词义在现代发生转移,往往是指两个品行同样不端的人或者处于同样困境中的人。词义转移现象在现代汉语中比较常见,譬如涕泪纵横,其实``涕''就是泪,后来成了鼻涕,譬如毙于车,就是倒在车上,``毙''后来成了死亡。现代人望文生义,说话容易露马脚。当年纪宝成大校长``七月流火''事件,就犯了想当然的错误,而后就有范曾大师打出词义转移的旗子为他辩解。不过难兄难弟转移的早,``七月流火''还来不及转移,范大师解释未免牵强。2008年7月4日,杭州气温最高37度,热情的何止是天气啊,范大师拍纪校长马屁也很热情。学者奴役化、学校衙门化,此乃时代潮流,浩浩荡荡,顺之者昌,逆之者亡。范先生1989愤然投法,不料举步维艰,旋尔复献媚于旧党,返乡后大卖字画,大造别墅,当年文革期间也曾怒斥反动文人和恩人沈从文,善识时务,真当今海内宗师也。

\section{1.9}\label{section-8}

\begin{quote}
荀巨伯远看友人疾,值胡贼攻郡,友人语巨伯曰:``吾今死矣,子可去!''巨伯曰:``远来相视,子令吾去;败义以求生,岂荀巨伯所行邪!''贼既至,谓巨伯曰:``大军至,一郡尽空,汝何男子,而敢独止?''巨伯曰:``友人有疾,不忍委之,宁以我身代友人命。''贼相谓曰:``我辈无义之人,而入有义之国!''遂班军而还,一郡并获全。
\end{quote}

\begin{itemize}
\tightlist
\item
  \emph{荀巨伯}:不知道是谁,《世说》注中说:``汉桓帝时人也。亦出颍川,未详其始末。''看下文巨伯是名而不是字。该人不可查就对了,因为这么大的场面,最后胡族大军未尽兴而返,其退敌之离奇,写进史书,不就显得朝廷之无能了吗?其
  故事之感人,不编入小说,不就显不出中华民族文化之精神!
\item
  \emph{胡}:古代对西方、北方各少数民族的统称。《尔雅 -
  释鸟》:鶟鶦,似雉,青身白头。看解释像美国的国鸟------白头雕。可能当时有个少数民族的图腾是这种猛禽,有段时间因为战术、技术先进,在先秦赵国等多次抢掠,杀得能止小儿啼,从此以后七国把那些骑马短服的少数民族统称为胡。好像民国前把马匪叫做``胡子'',也是这个原因。由于长期的砍杀,``胡''在汉语中带有天生的一种杀气,如北方清冷的空气。所以金庸小说中有个胡一刀,而不是苗一刀,胡人凤。姜夔《扬州慢》``自胡马窥江去后,废池乔木,犹厌言兵。渐黄昏,清角吹寒,都在空城。''其中有多少的感慨,本文后面说的``一郡并获全''恐怕也不过是一座矗立在秋天黄昏下的空城。
\item
  \emph{子}:对对方的尊称,其他意思还很多,要看具体的语境。
\item
  \emph{男子}:古代对没有官职的成年男人不敬称呼。
\item
  \emph{委}:抛弃,舍弃。《孟子-公孙丑下》:委而去之。它还有个意思是托付、交付:委身相随。
\item
  \emph{国}:地方。《诗经-硕鼠》:逝将去女,适彼乐国。文章中少数民族首领谈论``义''的概念,奇怪。义的含义很广,儒家曾多次解释,荀巨
  伯打算与朋友同生共死,仅仅是义中极微小的``江湖义气'',而``我辈无义之人''却是广泛的公平和正义概念。要匪徒承认自己不讲义气,很难想象。《庄子-外篇》中说,跖之徒问与跖曰:``盗亦有道乎?''跖曰:``何适而无有道耶?夫妄意室中之藏,圣也;入先,勇也;出后,义也;知可否,智也;分均,仁也;五者不备而能成大盗者,天下未之有也。''这么好的强盗守则摆在那里不会引用,而居然做深刻的自我批评,只能说,如果这个故事是真的,那么这些少数民族还没有真正接触到中华民族文化的精髓,一知半解地掉入了词语的牢笼。
\item
  \emph{班}:还,回。
\item
  \emph{全}:保全。
\end{itemize}

本篇没有难解之处,而余嘉锡却很顶真,他在笺疏中说道:``桓帝时,羌胡并叛,其胡贼之难如此。然他胡辄为汉所击败,惟鲜卑常自来自去。此条末云`贼班师而还',则巨伯所值者,其鲜卑乎?其事既无可考,不知究在何年、何郡也。''我想,余先生可能不治民间文学和文本学,这种故事在民间传说中有很多类似版本,不过胡贼可以改为黄巢、朱元璋等造反派领袖,荀巨伯改为某男、某女,这种类似的文本故事只是说明汉族试图用道义和文化的力量去对抗突然来临的战争,寄托着儒
家教化学说的美好愿望,寄托着人民对道义的顽固信任。当然,在古时候,也有一些真实的故事,大抵是说军阀战争,知道某地有个贤人,某地有座孔庙,领导者以
示尊敬就避开了,这也许就是``恻隐向善之心''。

\section{1.10}\label{section-9}

\begin{quote}
华歆遇子弟甚整,虽闲室之内,严若朝典。陈元方兄弟恣柔爱之道。而二门之里,两不失雍熙之轨焉。
\end{quote}

\begin{itemize}
\tightlist
\item
  \emph{华歆}:字子鱼,山东高唐人。高唐,有名的地方,大家可能读过《高唐赋》吧?没读过?不要紧,总听说过云雨之欢的成语,挺色情的,据说就在高唐楚王做的春梦,梦中情人是巫山的女神。楚王估计妻妾不少,但没有禁忌的快感,就打神仙的主意。不过此高唐未必就是彼高唐。歆本义是古代祭祀时所用的祭品,其中估计有鱼;当然,也可以作其他理解,歆还有个意思是羡慕,《庄子》中曾经说``鱼出游从容,是鱼之乐也'',在《淮南子》中也说``临河而羡鱼,不如归家结网'',
  像孔子的儿子孔鲤字伯鱼,宋襄公有个弟弟也叫子鱼,都说明那些古人羡慕的对象之一是鱼,或清蒸或红烧,或相忘于江湖,都是快乐的事情。华歆同邴原、管宁曾经是同学,都有些名声,当时人们称他们三人为一龙,华歆是龙头,邴原是龙腹,管宁是龙尾。裴松之说这三人品行有高低,头尾应该换换;洪亮吉说,这是按三人年岁排序的,并没有高低之分。华歆官当得很大,曹丕当皇帝后,他就坐上了丞相的位置,前面我介绍陈群的故事中他出现过的。华歆对钱看的不是很重,``家无担石之储'',但为人很上进,《三国演义》中曹操要杀献帝的老婆伏皇后,叫华歆去执行这个大逆不道的光荣任务,伏皇后躲在壁橱里,华歆眼尖,自己动手把皇后拉了出来,处死。
  遇:对待。
\item
  \emph{整}:整齐、严肃。
\item
  \emph{闲室}:家中。
\item
  \emph{朝典}:朝廷的礼仪。现在正值酷暑,我在家中也光
  着膀子,儿子常爬在我身上嬉闹,叫我像在办公室里一样严正,``上帝啊,杀了我吧!''记得南北朝有个刘琎,和哥哥刘瓛住在一起。哥哥在隔壁想跟弟弟闲谈,就叫了刘琎一声,没听见应答,以为刘琎睡着了。谁知过了半晌,刘琎却应了一声。刘瓛感到奇怪,问他为什么半天才应?刘琎答是因为听到哥哥呼唤,便马上起来穿衣束冠,只因带子没系好,人没站正,才不敢非礼回答,所以应得晚了。这正是《礼记》说的``见人不可以不饰''。
  华歆死后谥号敬侯(夙夜警戒曰敬,合善典法曰敬) 。
\item
  \emph{陈元方}:陈纪,上文见多次了。
\item
  \emph{恣柔爱之道}:恣,放纵、尽情。柔爱,和柔友爱。陈寔有六个儿子,平时一起嬉笑玩乐,比较温馨,前面故事讲陈寔孙子比各自父亲高低,要爷爷当裁判,的确说明陈寔家规宽松。
\item
  \emph{二门}:两家。
\item
  \emph{雍熙}:和谐愉快温暖等。《尚书》``克明俊德,以亲九族;九族既睦,平章百姓;百姓昭明,协和万邦。黎民于变时雍'';《道德经》``众人熙熙,如登春台'',《柳毅传》``红妆千万,笑语熙熙''。
\item
  \emph{轨}:准则。
\end{itemize}

这一段让人很容易联想起《史记》中对李广和程不识的介绍。李广不大讲究部队正规化建设,程不识纪律严明。程不识说:``李广治军简易,但将士们乐于效命。我带兵比较烦扰,然敌人也无机可乘。''司马迁评论说,当时李广、程不识都是名将,不过``士卒亦多乐从李广而苦程不识''。刘义庆虽然在此对华歆、陈纪没有区分优劣,不过按照南北朝的舆论猜想,自然倾向于陈纪一些。
为什么把家庭关系放在``德行''里面,这需要从儒家学说中来考量。``三纲''中两纲是家庭关系,``父为子纲,夫为妻纲'';``五伦''中``父子有亲,君臣有义,夫妇有别,长幼有序,朋友有信'',家庭关系占三个,可见儒学对家庭关系的重视。《礼
记》中大约2/3的篇幅来调整家庭关系,如子女对待父母的礼节,婚礼习俗的安排、极其复杂的丧礼等。儒家显然认为所有这些细节对于维持一个和睦安宁的家庭环境都是必要的,并花费了大量笔墨来为它们辩护。儒学之所以把家庭放在中心,是认为家庭和睦非常重要,它是培养个人道德与维系社会稳定的枢纽。在儒家看来,和睦的家庭不仅是社会安宁的前提,更是造就``君子''的温床。《礼记
-
大学》中说:``从古之欲明明德于天下者,先治其国。欲治其国者,先齐其家,欲齐其家者,先修其身'',它对治学到修身都预设了一个良好的培育环境为前提,而只有和睦的家庭才能提供这个环境。现在一些``新儒家''鼓噪复
古兴国,但如今婆媳不和,父子分居,常为三口之家,其学说依存的基础已经丧失,再谈儒家如何如何,完全没有意义了。

\section{1.11}\label{section-10}

\begin{quote}
管宁、华歆共园中锄菜,见地有片金,管挥锄与瓦石不异,华捉而掷去之。又尝同席读书,有乘轩冕过门者,宁读如故,歆废书出看。宁割席分坐曰:``子非吾友也。''
\end{quote}

\begin{itemize}
\tightlist
\item
  \emph{管宁},字幼安,山东朱虚人,和华歆、邴原都是老乡,看文章的意思,他们几个可能曾经在一起又读书来又生产,可谓知行合一。我小时候读书,学校里也有块菜地,班级承包,还有几棵果树,小组里也养过兔子什么的,现在的孩子可能体验不到其中的快乐了。管宁避祸辽东37年,一辈子拒绝作官,只是被曹丕强封了个太中大夫,作一个咨询顾问。管宁虽然和华歆绝交,但华歆还是要贴上去,多次推荐,提出把自己太尉的差事让给管宁,管宁当然不会接受(这种故事古时候倒常听说,但在我的印象中好像没有一样能达成协议的,不过一定是双赢的局面------大家都获得了好名声),笑着说:``华歆当年本来就渴望做个大官僚,所以把自己的位置当个宝贝(子鱼本欲作老吏,故荣之耳)。''正是李商隐《安定城楼》所说:``不知腐鼠成滋味,猜意鹓雏竟未休。''
\item
  \emph{捉而掷去之}:拿起来又扔掉。
\end{itemize}

这个故事用现代人的眼光看,管宁的举动有点出格,华歆倒是更符合人性一点,无论这个金是指黄金也好、白银也好、黄铜也好,它起码是货币,而且是在自己家菜园子挖出来的的,其所有权没有疑问,郭巨埋儿,天可怜见,挖出一坛子钱,毫不犹豫占为己有,不照样是二十四孝之一?再说了,挖到货币又不利用,GDP不就消失了,对社会的经济建设也造成负面影响。记得有故事说,楚共王打猎时把其弓掉了,左右要去找。共王说:``止!楚人遗弓,楚人得之,又何求焉!''孔子听
说后叹息说,何必加个楚字啊。可见孔子对捡东西也不反对。当然,对捡东西儒家也有不同的说法,《后汉书》中有个故事,说乐羊子也捡到一块金,拿回家给妻子(快乐总是要和别人分享的)。妻子说:``我听说志士不饮盗泉之水,廉者不受嗟来之食,何况是捡拾失物、谋求私利来玷污自己的品德呢!''羊子听后十分惭愧,就把金扔弃到野外,出外拜师求学去了。

\begin{itemize}
\tightlist
\item
  \emph{席}:供坐卧铺垫的用具,古人坐的时候是跪着的,膝盖上总的垫个垫子。用椅子可能是唐代以后的事情。
\item
  \emph{轩冕}:偏义复词,轩,指曲辕而有帷幕的马车,是古代大夫以上的坐车;冕,高帽,大夫以上带的帽子。这里指车子,借代有达官贵人过门。我有几个朋友平时看到什么保时捷、林肯、悍马啊等车子停着,往往要绕车半圈,口中啧啧,报出一些数字,对尊贵表示敬意,为了不失去朋友,我也要应和几声。
\item
  \emph{宁、歆}:上文称管,这里称宁;上文称华,这里称歆。古文惯例,人名已见于上文时,就可以单称姓或名。
\item
  \emph{废}:放弃;放下。
\item
  \emph{友}:《礼》、《易》的注中都说``同门曰朋,同志曰友。''
\end{itemize}

也许很难说管宁有道德洁癖,心与心的距离未必比星与星的距离更近。管宁对那个乱世越来越失望,所以对朋友要求越来越严格,``常笑邴原、华子鱼有仕宦意''(《世说》注),正像法国大革命时的罗兰夫人说的``我了解人越多,我越喜欢狗'',我很难想象管宁如果掌控至高权力,这个世界究竟是血流成河还是光明一片。
古希腊有个学派叫斯多葛主义,其主力有一群道德圣徒和厌世者组成,他们对历史的真正影响,或微乎其微,或破坏深重。我想,对于那些与社会无害的生活方式,不要拿完美的标准去衡量,这样做同样是被外物所累。不过像范美忠这样最近频频上网推销他人生哲学的网友,我还是感到不舒服,只能理解成他用不恰当的方式来
努力消除内心的愧疚。

\section{1.12}\label{section-11}

\begin{quote}
王朗每以识度推华歆。歆蜡日尝集子侄燕饮,王亦学之。有人向张华说此事,张曰:``王之学华,皆是形骸之外,去之所以更远。''
\end{quote}

\begin{itemize}
\tightlist
\item
  \emph{王朗}:字景兴,山东郯县人,原来叫王严。照道理说,名字授之于父母,古人一般不会改的,只有譬如身体不好、官位不大、冲犯忌讳、躲避仇杀等,才先祭祀占卜一番,再慎重改名。三国时李严、王严都改名了,一叫李平,一叫王朗,一定有什么特殊的原因。王朗是个学者,对不少儒家典籍做过解释,但没有流传下来;王朗也曾是个失败者,他当会稽太守的时候,被孙策杀败并狠狠地羞辱,不过在魏国他当上了司徒,``三公''之一,大司法部部长;王朗在民间被耻笑,因为在《三国演义》中,大学者居然在一场辩论赛中自告奋勇地当一辩,被诸葛亮活活骂死,这当然是假的,不过王朗怎么看都像个伪君子真小人,他多次向吴国、蜀国大臣写信喊话劝降,《魏略》中记载,王朗给许靖的信中说,自己能够亲身经历圣主受终(曹丕篡位),十分开心,``如处唐虞之世'',
  我们往他身上泼泼脏水,是多么快乐。
  蜡:音``吒'',古代年终大祭万物,《礼记》``天子大蜡八,水庸居其七'',天子在冬至三戌之日期间,必定会举行祭祀诸农神与土神的祭典,这祭典仪式称之为``蜡''。蜡后来也做腊,据说本义是打猎的意思。汉应劭《风俗通义》中说:
  ``夏曰嘉平,殷曰清祀,周用大蜡,汉改为腊。腊者,猎也,言田猎取禽兽,以祭祀其先祖也。''或曰:``腊者,接也,新故交接,故大祭以报功也。''
\item
  \emph{燕饮}:宴饮。
\item
  \emph{张华}:河北人,字茂先,西晋名臣,被人誉为``子产''(凡是拍总理的马屁,我们一律可以用子产的)。年轻时写了一篇《鹪鹩赋》,以小鸟自喻,``色浅体陋,不为人用,形微处卑,物莫之害,繁滋族类,乘居匹游,翩翩然有以自乐也\ldots{}\ldots{}静守约而不矜,动因循以简易。任自然以为资,无诱慕于世伪'',一举成名,慢慢地麻雀变凤凰,最终当上了晋国的``三公''之一------司空。张华博闻广识,著有《博物志》,我们现在还能读到。据说龙泉、太阿宝剑(就是干将、莫邪)就是他指点别人找到并得到雄剑(《滕王阁序》中``(剑)气冲斗牛之间''典故所在),他死后雄剑不知所终,最后两剑相遇,化龙而去,给后人留下无穷的遐想。张华曾经掌管过国家宝库,有一次失火,据说``累代之宝及汉高斩蛇剑、王莽头、孔子屐等尽焚焉'',十分可惜。张华最后在西晋``八王之乱''中遇害。
\item
  \emph{形骸之外}:形骸,身体。指外在形式。
\end{itemize}

王朗是华歆的Fans,他当然知道东施效颦的故事,但齐白石生得太晚了,所以王朗没有听到``学我者生,似我者死''的劝告。故事的真相可能是这样,华歆有个绰号叫``华独坐'',是说他酒量太好,别人已经喝得东倒西歪了,可华歆没事,不会失态。因为华歆喜欢喝酒,所以在腊月召集子弟喝,图个热闹。王朗不知其所以,也叫了一家子人喝,结果喝趴下了,被张华嘲笑。学习模仿别人其实是件危险的事,大家可能还记得查普曼,他曾经刺杀了披头士主唱列侬,其中一个说法是查
普曼具有``我就是列侬''的身份妄想。查普曼原是列侬的歌迷,在列侬退隐年间,他开始学习列侬,连日常生活的细节都模仿列侬,``我就是列侬''的妄想为他沉闷无趣的生活增添了不少乐趣和光彩。但等到列侬再度复出后,查普曼即产生严重的认同危机,他再也无法退回原来的身份和生活中,于是他只好去枪杀那个真正的列侬。还有一个说法是,查普曼刺杀列侬后没有跑,依旧冷静地看《麦田里的守望者》,他是如此认同于塞林格创造的这个人物,他不久前还在试着将名字合法地改为
霍尔顿 -
考菲尔德。他说:``这本书是写我的。''他觉得自己像霍尔顿,他们同样憎恨虚假,之所以枪杀列侬,是因为列侬是一个彻头彻尾的伪君子。

\section{1.13}\label{section-12}

\begin{quote}
华歆、王朗俱乘船避难,有一人欲依附,歆辄难之。朗曰:``幸尚宽,何为不可?''后贼追至,王欲舍所携人。歆曰:``本所以疑,正为此耳。既已纳其自托,宁可以急相弃邪!''遂携拯如初。世以此定华、王之优劣。
\end{quote}

\begin{itemize}
\tightlist
\item
  \emph{避难}:躲避战乱,据华峤《谱叙》中说,是指献帝在长安时事,王允刺杀董卓后,李傕、郭汜兵变,天下大乱。东汉末年黄巾起义,我们且不谈农民起义的正义性,也不谈太平道的邪教性质,有历史资料指出,像张角、张梁这些个领袖,和宦官的联系很紧密的,曾与太监密约一起起事,只是消息走漏,所以起义非常仓促。
  那个时候中原不少地区可算是人间地狱,曹操说的``白骨露于野,千里无鸡鸣,生民百遗一''并无夸张之处,东汉黄巾起义前本来人口在6000万左右,到了西晋建立,人口在1600万左右。王粲《七哀诗》写道``出门无所见,白骨蔽平原。路有饥妇人,抱子弃草间'',
  曹丕《令诗》:``丧乱悠悠过纪,白骨纵横万里'',曹植《送应氏》:``中野何萧条,千里无人烟'',这些我们在世界大战和外星人入侵等影片中都见识过了。活着的还要逃命,要见识经历许多丧乱别离。``哲学是死亡的练习'',``如果没有死亡的问题,恐怕哲学也就不成其为哲学了'',要把握魏晋时候某些士人的心态,需要我们设身处地的设想无数死亡对于他们的教育意义。现在网上玄幻小说之重生主题之一就是回三国当大英雄,可事实往往是,如果重生在东汉末的中原地区,被秒杀的概率大大超过活着;即使活着,按现代人的思维活动,也大都会厌世而崩溃。
\item
  \emph{纳其自托}:接受了他托身的请求,指同意他搭船。
\item
  \emph{拯}:挽救。
\end{itemize}

在《世说》注中,华歆孙子华峤回忆了另外一个版本,主人公是华歆和郑太,那个倒霉鬼是掉在井里了,华歆主张把他救起来。程炎震说,故事中不会有王朗,因为当时他在徐州陶谦那里。华歆小节可能还可以,但是进取心很强,人品也就那么回事。章太炎曾评论,汉魏之交替,人们认为这些士大夫人品越来越差,(``公惭卿,卿惭长''。儿子、孙子官越来越大,可是德行越来越差)独于华歆,魏、晋间皆颂美不容口,歆之得誉,亦缘峤之《谱叙》。就是说,华歆有个好孙子华峤,历史学家,全靠他写了本《谱叙》,大大赞美爷爷,牢牢掌握话语权,所以舆论一边倒了。华歆的那些丑事,全靠吴国人写曹操传而保留下来。余嘉锡作出批评:``后汉末至六朝,士人小廉曲谨,以邀声誉;逮至闻望既高,四方宗仰,虽卖国求荣,犹翕然以名德推之。华歆、王朗、陈群之徒,其作俑者也。''眼下有个毛三,学习华峤,大写爷爷的伟大(我想因为智力问题恐怕是请人捉刀的),不过现在的历史,一部分掌握在网络手中,毛三想重塑祖先的荣光,已是时过境迁。

\section{1.14}\label{section-13}

\begin{quote}
王祥事后母朱夫人甚谨。家有一李树,结子殊好,母恒使守之。时风雨忽至,祥抱树而泣。祥尝在别床眠,母自往暗斫之;值祥私起,空所得被。既还,知母憾之不已,因跪前请死。母于是感悟,爱之如己子。
\end{quote}

\begin{itemize}
\tightlist
\item
  \emph{王祥}:字休征(兵者不祥,要祥自然要不兴征伐),山东琅琊人,长寿,经历了汉、魏、晋三个朝代,是中国历史上最著名的孝子,其经历完全可以和舜一比高下,如果是原始部落的禅让制,王祥就是理所当然的氏族头领。不过舜到儿子商均这一代就湮没无闻,而王氏家族可以说是中国历史上的第一世家。王家的发迹,就是从王祥哥俩(王祥、王览)开始的,王祥哥俩出名,全是因为他们的母亲朱氏集天下恶毒后母之大成,白雪公主后妈、灰姑娘后妈都要``让她一头'',《世说》仅仅选取了其中两个小故事,在其他史书里朱妈妈还有打骂,下毒等故事,而且这个后母也没有改悔,一如既往地考验着王氏兄弟一直到死。王祥后来做了曹髦的老师,接着很配合地转向司马昭、司马炎,从此家族兴盛三百年,可谓辉煌无比,形成了``王与马,共天下''的局面。
\item
  \emph{时风雨忽至,祥抱树而泣}:一个二三十岁,也许四五十岁的大男人经常在雨中抱着李树哭泣,孝气冲天,其画面是何等地感人。据说后来这几株果树居然每次在风雨中都毫无损失,其他家的果树却全遭殃了。在今天山东临沂孝感河边还有一块碑,纪念王祥的风雨守李。
\item
  \emph{母自往暗斫之}:因为这个故事太离奇,所以《晋书》上没有采用,而采用了和``守李''异曲同工的故事:``母又思黄雀灸(小鸟烧烤),复有黄雀数十飞入其幕,复以供母。''还有故事说,朱氏虐待王祥和媳妇,自己的亲生儿子王览便一起承担,让朱氏狠不起来。王祥父亲死了以后,王祥孝的名声大了。朱氏更恨,用毒酒要毒杀王祥。王览见机不对,拿起来自己喝,王祥也要抢,最后朱氏倒掉酒。从此以后,王览为了防止母亲把王祥毒死,每次饭前他都要先尝一遍。不过儒家说孝``小杖则受,大杖则走''《孔子家语》,王祥自动请死,可能是非常悲愤的,要陷朱氏于不义,朱氏终于觉醒了。多么好的后妈啊,坚持数十年如一日地当恶人,这是一种怎样的执着精神!
\end{itemize}

王祥最著名的表演在《二十四孝》中:朱氏在一个天寒地冻的夜晚突然要求吃鱼,王祥二话不说,立刻来到河边,脱掉衣服用身体
融化坚冰,结果冰面终于裂了个缝,从中蹦出两条鲜活的鲤鱼来到王祥脚下,这就是``卧冰求鲤''。隔壁的邻居晚上闻到王祥家飘来鱼香,口水禁不住往下流,急忙打听,十分感动这份孝心,义务广泛传播,后来那条河被乡里乡亲改名作``孝感河''。余嘉锡说,故事不是这样的:《初学记》三引师觉《孝子传》中说``祥解褐扣冰求之,忽冰少开,有双鲤出游,祥垂纶获之而归。''王祥砸冰激动得身体发热,脱衣服继续砸,钓到了鱼。余先生扫兴!

\section{1.15}\label{section-14}

\begin{quote}
晋文王称阮嗣宗至慎,每与之言,言皆玄远,未尝臧否人物。
\end{quote}

\begin{itemize}
\tightlist
\item
  \emph{晋文王}:司马昭,司马懿之子,司马师的弟弟,死的早了点,儿子司马炎是正儿八经的武帝。因为按伦理司马师的儿子更具有当皇帝的合法性,所以为后来的``八王之乱''埋下了伏笔,西晋迅速衰败。
\item
  \emph{阮嗣宗}:阮籍,字嗣宗,河南陈留人,父亲阮瑀是建安七子之一,他是竹林七贤之一。阮籍当过步兵校尉(俸禄二千石,算是中高级干部了),所以后人也叫他
  ``阮步兵''。
  阮瑀是曹操的秘书,文章很合曹操的心思,也有很好的音乐才能,但去世早,那时候阮籍才四岁。阮瑀的才能完全被阮籍继承下来,而且更加出色。阮籍是正始文学的代表作家之一,它区别于建安文学,是因为反映民生疾苦和抒发豪情壮志的作品减少,抒写个人忧愤的作品增多,作品逐渐与玄理结合,风格由建安时的慷慨悲壮为主变为寄托深远为主。如阮籍的《咏怀》诗,就消解了建安文人建功立业的豪情,而是仔细体味生命的悲哀,去看透名利的虚幻,它的``深情''体现出独特的艺术风貌。阮籍的道德和我们前面介绍的道德完全不同,东汉中后期大多数士人自我意识高涨,以天下为己任,对自己的道德要求是严谨;而正始期间的一些士人的自我意识高涨,以自由为追求,对自己的道德要求是本性,这在阮籍身上有比较充分的反映,我们会在后面陆陆续续地了解他的所作所为,他也许并不喜欢这样,但不得不这样,不然他就找不到自己了。不过,人是都是传统的产物,矛盾的结合体,阮籍也有他的另一面,譬如这个故事。
\item
  \emph{臧否}:善恶;得失。 《诗 大雅
  抑》:``於呼小子,未知臧否。''这里做动词,称赞批评。宋代司马光说:``党人生昏乱之世,不在其
  位,四海横流,而欲以口舌救之,臧否人物,激浊扬清,撩虺蛇之头,跷虎狼之尾,以至身被淫刑,祸及朋友,士类歼灭,而国以随亡,不亦悲夫!''臧否人物就像``用手去撩拨毒蛇的头,用脚践踏老虎和豺狼的尾巴'',确实是找死。
\end{itemize}

阮籍``终身履薄冰'',虽然在司马昭面前``至慎'',不谈论时势人物,但其实在别人面前却``至狂'',耐不住性子,用青白眼看人,也会讽刺别人,``礼法之士疾之若仇''。阮籍志向很大,自视甚高,但他又感到世事已不可为,没有自己的舞台(徐稚说``大树将颠,非一绳所维''),``但恨处非位,怆恨使心伤'',所以多次称病辞官,辞官一次官升一次,后来居然爵位到了关内侯,司马昭也要和阮籍做亲戚,官运比李广好多了。越是这样,阮籍的屈辱感就越强,背上了沉重的良知、人性与情感的重担,所以写了《咏怀》宣泄内心的痛苦,写了《大人先生传》追求属于自己的道路。他在《大人先生传》说,现在的君子``心若怀冰,战战栗栗。束身修行,日慎一日。择地而行,唯恐遗失。
颂周、孔之遗训,叹唐、虞之道德,唯法是修,为礼是克'',最后``奉事君上,牧养百姓。退营私家,育长妻子。卜吉宅,虑乃亿祉。远祸近福,永坚固己'',似乎生活的不错。但在大人眼里,不过是``虱之处於裤中,逃乎深缝,匿乎坏絮,自以为吉宅也。行不敢离缝际,动不敢出裤裆,自以为得绳墨也。饥则啮人,自以为无穷食也。然炎丘火流,焦邑灭都,群虱死於裤中而不能出。汝君子之处区内,亦何异夫虱之处裤中乎?悲夫!''。你们战战兢兢,克己复礼,得到了所谓的地位和美好生活。在我眼里,所谓的大房子,不过是裤裆里的破棉絮;所谓的规矩,不过是裤裆里的规矩;所谓的俸禄,不过是吸了点肮脏的血食。阮籍说的很痛快,可我们这些后辈,实在迈不出陶渊明那一步,就只能继续浑浑噩噩地生活。

\section{1.16}\label{section-15}

\begin{quote}
王戎云:``与嵇康居二十年,未尝见其喜愠之色。''
\end{quote}

\begin{itemize}
\tightlist
\item
  \emph{王戎}:字浚冲,山东琅琊人,属王祥家族中的一员,竹林七贤之一,他的爵位是安丰侯,所以也叫他王安丰。据说他们七个曾经在河南山阳竹林边一起居住许多年,交游甚密;但也有观点认为七个人有的年纪相差较大,排历史年谱的话会存在问题(王戎加入竹林集体可能还是个娃娃),志趣各异,不太可能成为好朋友。王戎在七贤中名声最不好,但官至司徒。王戎在《世说》中故事很多,是焦点人物之一。王戎是毫无疑问的神童,他后来的随波逐流(``慕蘧伯玉之为人,与时舒卷,无蹇谔之节'',典出《论语》),无非是保全生命的一种办法。他曾经掌管全国的人事组织,一切按门第来,不用考评,但他平时的鉴赏评论人物却非常出名;他官至司徒,吝啬贪财的名声流传全国,不过平时不管事,穿着便衣骑着小马到处游玩,普通人根本不知道他是政府的最高首脑。事实上,经济生活的富裕本来就是士大夫人生目标之一,它与士人追求思想的自由相反相成,往往只有``居有良田广宅''的物质基础,才能始及``老庄之玄虚''的精神生活。王戎晚年遭遇战乱,亲接锋刃,打了败仗,在危难之间,谈笑自若,也看不出有什么害怕的。王戎可谓享受生活的代表。
\item
  \emph{嵇康}:字叔夜,竹林七贤之一。本来姓奚,祖籍会稽,先人避仇迁安徽,改姓嵇,以纪念祖籍会稽,也有说安徽有嵇山,就以山为姓。他曾经娶了曹魏的宗室长乐亭主,任闲官中散大夫,世称嵇中散。嵇康作为曹家宗室的一员,自然对于司马氏采取不合作态度,而且他的声望很大,是全国著名的文学家、思想家、音乐家、书法家,再加上别人的挑拨,最后被司马昭杀害,年仅40岁。阮籍诗歌写得好,嵇康散文写得好,比阮籍的诗歌还好,他的《与山巨源绝交书》能够鼓舞很多人,与之共鸣。
\item
  \emph{与嵇康居二十年}:嵇康比王戎大十岁,就是说王戎起码在10岁就和嵇康生活在一起,这显然有些夸张,我想更可能是认识二十来年的意思。看文章的语气,好像是在回忆往事,表明王戎和嵇康的亲密关系。嵇康早已被害,回忆就成了王戎的专利,嵇康再也不可能从坟墓里跳出来澄清事实。
\item
  \emph{未尝见其喜愠之色}:这与《与山巨源绝交书》中的嵇康似乎是两种人,嵇康自称:``直性狭中,多所不堪'',``刚肠疾恶,轻肆直言,遇事便发'',就是说自己性格直爽,心胸狭窄,对很多事情不能忍受;性格倔强,憎恨坏人坏事,说话轻率放肆,直言不讳,碰到看不惯的事情脾气就要发作。王戎如此过滤式地回忆往事,也许有什么环境背景和内心的隐秘之事,也许是为嵇康鸣不平,也许是为了怀念友谊。《晋书》中描写阮籍,开头也是``喜怒不露于形色'',大概那个时候人们把这一点非常看重,事实上,无论刘备、阮籍、嵇康都是很情绪化的人,我理解``喜怒不露于形色''是他们不在意琐事,不惧怕挫折,不怨天尤人,不向命运屈服。
\end{itemize}

《世说》中没有讲嵇康之死的故事,那么就在这里解释一下:起初,嵇康与吕巽、吕安兄弟都是朋友。吕安的妻子徐氏貌美,吕巽乘吕安不在时,灌醉奸污徐氏。吕安找嵇康商量是否要告官,嵇康说出自家门名誉考虑,息事宁人,保全兄弟之情,由他出做工作,让吕巽保证绝不再犯,也不准向吕安下毒手。吕巽答应了,不料为消除这块心病,而后恶人先告状,向司马昭诬吕安有打母之举。司马昭当时以孝治天下,忤逆是条大罪。吕安被囚,援引嵇康为他作证,嵇康义不负心,保明其事,于是一起被收监。结果三千太学生力保嵇康,引起司马昭不快,小人钟会又在那里推波助澜,说嵇康曾打算参与一次军事政变,劝司马昭处决吕安和嵇康,罪名是``言论放荡,害时乱教''(孔子杀少正卯的典故又一次在此案中被引用),嵇康就这样胡里胡涂被杀了。
嵇康、吕安的好朋友向秀若干年后写了篇《思旧赋》,文字很短,感情很深,王戎这样走样地回忆嵇康,也许和向秀的心情是一样的:``余与嵇康、吕安居至接近,其人并有不羁之才;然嵇志远而疏,吕心旷而放,其后各以事见法。嵇博综技艺,于丝竹特妙。临当就命,顾视日影,索琴而弹之。余逝将西迈,经其旧庐。于时日薄虞渊,寒冰凄然。邻人有吹笛者,发音寥亮。追思曩昔游宴之好,感音而叹,\ldots{}\ldots{}''

\section{1.17}\label{section-16}

\begin{quote}
王戎、和峤同时遭大丧,俱以孝称。王鸡骨支床,和哭泣备礼。武帝谓刘仲雄曰:``卿数省王、和不?闻和哀苦过礼,使人忧之。''仲雄曰:``和峤虽备礼,神气不损;王戎虽不备礼,而哀毁骨立。臣以和峤生孝,王戎死孝。陛下不应忧峤,而应忧戎。''
\end{quote}

\begin{itemize}
\tightlist
\item
  \emph{和峤}:字长舆,汝南人。祖父和洽曾任魏国尚书令,父亲和逌(悠)曾任吏部尚书。和峤曾任西晋的中书令、尚书令等职,可谓世门高第。曹操本性比较朴素,当时的风气就特别讲究节俭,和洽评论说,当官不能只以清廉为评判标准。俭朴作为个人的处世原则没什么不对,但以此衡量为官的标准就有失偏颇。现在有官员着新衣、乘好车,都被说成贪官,而那些穿破衣驾破车出入公府的人都说成了清吏,致使大小官员故意故作姿态,大臣们每天上朝都是自己带着茶具、提着饭篮出出进
  进,太可笑。其实立教观俗,贵处中庸,才是治国的道理。以偏概全,勉而为之,必有它终结的时候。况且从古至今,务在通人情而已,大凡过激的做法,则最容易隐伪矣。这在一定程度上扭转了官场虚伪的风气。当然和洽这样说是有资本的,他以廉洁出名。和峤在《世说》里的故事很多,我们可以陆续接触到。
\item
  \emph{大丧}:父或母去世。儒家本来就有厚葬之风,汉末、魏晋更是变本加厉,几千、几万人的吊唁场面经常出现。虽然有个别人要求并践行薄葬,但不是主流。
\item
  \emph{鸡骨支床}:支,支离。鸡骨大概是说瘦骨嶙峋的样子,也可能鸡当作离,雞,離。病得很瘦小,只能躺在床上了,意同下文的哀毁骨立。史书上说王戎在服孝期间虽然不禁酒肉,照样娱乐,但内心非常悲痛,走路都成了问题。这当然是母亲去世,如果是父亲不至于到这种地步。
\item
  \emph{刘仲雄}:刘毅,汉家子弟,为人刚直,在魏国时当检察院干部,与上司不和辞职。本来不想在司马昭底下当官,但被恐吓,只能再出山。他常常直言不讳,使``王公贵人皆惮之'',也曾多次批评九品中正制,指出``上品无寒门,下品无势族''的不合理性。,刘毅甘于清贫,以至司马炎也不好意思,经常赐钱接济他,每天派使者送米肉救济。刘毅退休后回青州,由于名望很高,青州的官员选拔由他来决定,部分圆了他不实行九品制的梦想。刘毅去世后,连司马炎也长叹:``我失去一位名臣,可惜他生不能为三公!''最后以三公之礼安葬。
\item
  \emph{省}:探望,``省亲''。不:同否,疑问词。
\item
  \emph{生孝}:指遵守丧礼而能注意不伤身体的孝行。和峤按照礼的要求行孝,做得很周全,但身体没有损伤。
\item
  \emph{死孝}:对父母尽哀悼之情而至于死的孝行。
\end{itemize}

东汉末对守孝本来就存在两种方式,一种是谨守《礼记》的要求,如《仪礼》说:父母死后三日不食。出殡后,可以食粥,朝暮各一溢米。百日后,可以``疏食水饮''。1年(8个月)以后,可以``食菜果''。2年后,可以用酱醋调味。3年(25个月)丧服期满,恢复正常饮食,等等。一种是学习庄子鼓盆而歌的旷达,如我在前面介绍看到黄宪就闷闷不乐的戴良:及母卒。良一个人吃肉饮酒,悲伤的时候就哭两声。有人问良:``子之居丧,礼乎?''良曰:``是的。礼是用来制约感情放荡的,如果我的感情不散佚,要礼来拘束干嘛?那些吃很差的东西,无非是为了使自己显得憔悴一些;我现在食不知味,吃酒肉有什么关系!''像戴良、王戎、阮籍这样守孝的,虽然有时被普通人非议,但由于他们的感情很真挚,别人也不能拿大棒子打人了。可是余嘉锡先生这时候又跳出来说,和峤、王戎都是西晋著名的吝啬鬼,对家人都十分刻薄,很难想象他们会对父母孝顺。

\section{1.18}\label{section-17}

\begin{quote}
梁王、赵王,国之近属,贵重当时。裴令公岁请二国租钱数百万,以恤中表之贫者。或讥之曰:``何以乞物行惠?''裴曰:``损有余,补不足,天之道也。''
\end{quote}

\begin{itemize}
\tightlist
\item
  \emph{梁王}:司马肜(肜,和乐),字子徽,司马懿的第三子。
\item
  \emph{赵王}:司马伦,字子彝,司马懿的第九子。司马懿有好几个儿子以地名作字。现在``八王之乱''的中心人物之一出场了。据说,司马亮(司马懿的第四子)本来接受司马炎的遗诏,做晋惠帝司马衷的辅政大臣,可是司马炎的丈人杨骏修改了遗诏,单独辅政。而司马衷的妻子贾南风又是有深厚家族背景的奇特女性,要求分享权力,于是矛盾爆发。但深层次原因是司马昭重新实行古代儒家赞不绝口的分封制,几个叔叔和兄弟都是一方有实权的诸侯;再往前推,司马懿大儿子司马师意外身亡,老二司马昭接替其职位,这给其他王室作出了榜样,而惠帝司马衷恰巧是以愚笨出名的人,大大刺激了他们的权力欲,马上撕去了``礼''这块遮羞布,兄弟叔侄相残。从历史资料看,司马师、司马昭、司马炎都可以说有杰出的才能,可司马家族其中有几个却比较冲动、迷糊,以国家为私器,结果内部反复厮杀16年,天下大乱,西晋迅速灭亡。这个故事的积极意义是,有了西汉、西晋的鲜活教材,从此以后各王朝对家族内部的王侯全力实行打压,分封制从此没落。现代的中国农民家族企业,也存在分封制的影子,事实上,历史早证明,外人比亲人可靠一些,不太会质疑和挑战合法性。司马伦是``八王之乱''的中心人物,他本来统领禁军,是贾南风一党,曾劝贾南风赐死天资卓越的太子司马遹,而后以此为借口杀死贾南风,逼司马衷禅位给自己,以惠帝为太上皇(滑稽,孙子辈的司马衷这是倒成了他的父亲辈),最后``席不暇暖'',司马家族的另外几个王侯一起起兵反对,司马伦兵败被赐死。
\item
  \emph{国之近属}:国,帝王,指皇帝司马衷。
\item
  \emph{裴令公}:裴楷,字叔则,山西人。官至中书令。中书掌管文件机要,最为清贵华重,尊称为令公。裴家在当时也是一大世家,在后面的《世说》称``八裴''。裴楷、王戎都是钟会赏识和提拔的,虽然钟会谋逆兵败,但司马昭、司马炎用人不疑,依旧重用裴楷、王戎。裴楷学问、道德、风采都很出众,当时被誉为``子产''(张华和他同掌朝廷,同时有两个子产了,结果晋朝也这么乱,不是能力问题,而是制度问题)和``玉人''。
  由于位置和亲戚关系,裴楷卷入了``八王之乱'',虽然多次要求外放和隐退,未能如愿,最后病死,儿子却在动乱中被杀了。
\item
  \emph{中表}:指中表亲,跟父亲
  姐妹的子女和母亲的兄弟姐妹的子女之间的亲戚关系。这里有两种理解,一是要求司马肜、司马伦散财给其他穷皇外戚,二是要求两位王爷散财给裴楷的穷外戚。我理解应该是给裴楷自己的亲戚,不然别人也很难非议。据《晋书》上说,裴楷和富翁贵族们交游,见到好的就径直拿来了,随后随意散发给穷人。有一次他造了个大宅子,堂兄王衍参观后赞不绝口,裴楷头脑一热,连新房子也送给王衍了。
\item
  \emph{天之道}:语出《道德经》``天之道其犹张弓(弹琴)乎?高者抑之,下者举之;有余者损之,不足者与之。天之道,损有余,而补不足。''裴楷小时候就以善谈《老》、《庄》闻名,自然信手拈来。不过《道德经》还有下文:``人之道则不然,损不足以奉有余,孰能有余以奉天下(哪个能做到将富余的财富奉送给天底下的贫穷者呢)?''老子的``人道''就是``马太效应'':``凡有的,还要加给他让他多余;没有的,连他所有的也要夺过来。''揭露了经济学中贫者愈贫、富者愈富、赢家通吃的分配不公现象。``当代子产''裴楷实行天道,截留了《道德经》的下文,回避了真子产所说的``天道远,人道迩,非所及也''。其实裴楷这一招很了得,每年向司马肜、司马伦要这么多钱,那么可见他和这两位王爷的关系多铁(不铁也铁了),又送了他们很好的名声。我看过一则故事,说二战时一个犹太人要避难,儿子建议向他帮助过的人求助,他说:``不,应该向曾经给予我恩惠的人求
  助。''事实证明他是对的。``天道''是个宇宙问题,因为我不懂,不展开了。
\end{itemize}

\section{1.19}\label{section-18}

\begin{quote}
王戎云:``太保居在正始中,不在能言之流。及与之言,理中清远,将无以德掩其言!''
\end{quote}

\begin{itemize}
\tightlist
\item
  \emph{太保}:大孝子王祥,王戎的祖父辈。王祥曾任太保,正一品的崇高虚衔,太子的老师之一(太师、太傅、太保)。大家可能记得岳飞、于谦,人称``少保'',那是从一品的虚衔。王祥(185年---269年),王戎(233年-305年),看样子祖孙聊过天的。
\item
  \emph{居在}:``居''可能是多余的字。
\item
  \emph{正始}:三国时魏帝曹芳年号(240年---249年)。魏明帝曹睿没有儿子,所以从亲戚这里过继了8岁的曹芳,并任命曹爽、司马懿为辅政大臣。在三国斗争这
  么激烈的年代,把位置交给黄口小儿,大将军曹爽又是喜欢花天酒地的纨绔子弟,哪里是老狐狸司马懿的对手,曹睿有愧``明''的谥号,结果魏国内部派系斗争加
  剧。249年,司马懿诛大将军曹爽,从此曹家帮全面没落。254年,曹芳被司马师所废,另外找了个小孩曹髦当皇帝(高贵乡公)。曹芳最后谥号是``厉'',杀戮无辜曰厉,曹芳倒谈不上杀什么人,只是在命运之手的推动下,承担了别人的错误。
\item
  \emph{能言}:指能清谈,阐述玄理。清谈起源于汉代的清议,清议本来是通过评价人物好坏来选拔官吏,讲的是从品德看人的才能。而品德是往往是表面现象,架不住伪君子,而且汉代举荐错误是要连坐的,大家就不满足与``孝''这个东西,而要看透人物的本质,要看人物的``才性'',由此产生出现``四本才性''问题(四本论),即才性同,才性异,才性合,才性离,使讨论进入了抽象的哲学问题。到了正始年间,清议的主要内容逐渐转变为清谈思想时期,以《老子》、《庄子》、《周易》所谓``三玄''为中心,又叫玄学,当时的主要代表人物有何晏、王弼、嵇康、阮籍、向秀、郭象、钟会等。王祥也许对《礼》有研究,但从他的行为看,是个非常执着保守的人,高渺的玄学他插不上嘴的。
\item
  \emph{理中清远}:《晋书
  王祥传》中``理中''作``理致''(义理和情致),讲名理之学清新深远。
\item
  \emph{将无}:表示测度的语气词,相当于莫非、恐怕、大概,我想这和``莫须有''的``莫须''意思差不多,表示猜测而偏于肯定。``恐怕是德行掩盖了他的善谈吧'',这条表面上是讲王祥的清谈能力,其实是衬托王祥的德行实在高,所以纳入了``德行''这一章。
\end{itemize}

平心而论,即使王祥孝顺的故事如何添油加醋,离奇古怪,但事情的影子大抵是在的,那么为什么后世对他有所非议?主要是因为曹髦被司马昭杀害,作为太学生老师、掌管全国道德礼仪的太常王祥只是痛哭了一阵,没有死节,也没有谴责司马昭,未免和平时的形象不符。本来内心忐忑不安的司马昭见此大喜过望,又狠狠升了王祥的官。王祥之所以这样做,我们应该注意到,到了玄学时代,保全性命已经高于忠君思想,到了世家门阀时代,家族利益已经高于国家利益。魏晋南北朝直至隋朝,这种个人利益和家族思想始终左右着政坛,所以朝代频繁更替。王祥可能在玄学上并不出色,但他无疑把握了玄学的精髓和时代的要求,另外一些玄学大家,
却往往死于非命。

\section{1.20}\label{section-19}

\begin{quote}
王安丰遭艰,至性过人。裴令往吊之,曰:``若使一恸果能伤人,浚冲必不免灭性之讥。''
\end{quote}

\begin{itemize}
\tightlist
\item
  \emph{王安丰}:王戎,字浚冲。当时爵位安丰侯,职务吏部尚书。
\item
  \emph{艰}:父母丧,王戎母亲去世。《诗经
  蓼莪》:``父兮生我,母兮鞠我,拊我畜我,长我育我,顾我复我,出入腹我。欲报之德,昊天罔
  亟''。父母生下我,抚育我、养活我、拉拔我、教育我、照顾我、舍不得我,时时都怀抱着我。父母的恩情如天一样广大,怎么报答得尽。``瓶之罄矣,维罍之耻。
  鲜民之生,不如死之久矣。无父何怙,无母何恃。出则衔恤,入则靡至。''小瓶的酒倒光了,是大酒坛的耻辱。一个人孤苦伶仃地活著,还不如早些死去好。没了父母,我依靠谁,我仰赖谁?出门在外,心怀忧伤;踏进家门,魂不守舍。爱午睡的宰予问孔子,三年守丧是不是长了点?孔子说,孩子在父母怀里呆了三年,用守孝三年来报答父母的养育之恩,你说长吗?假如你觉得一年够了,那你就一年吧!按儒家丁忧的规矩,王戎要离职守孝25个月。古时候这方面非常重视,除了很特殊时期,皇帝要求``夺情'',三次下诏,才可以继续留在职位上。守孝的来源可能是祖先崇拜,但经过儒家改造,对守孝和祭祀的规矩要求就变得很高,就连守孝期间哭几声、跳几次等等都有具体明确的规定,很有些宗教活动的意味。儒家是试图通过这些仪式,引起人对生命的尊重,实现由人自然的朴素情感向道德信仰体系转化,从而起到教民化俗的目的,正如孔子在《论语
  -
  学而》中说:``其为人也孝悌,而好犯上者,鲜矣;不好犯上,而好作乱者,未之有也。''孝道上升到政治哲学高度,成为国家维稳定保平安求和谐的工具。
\item
  \emph{至性}:纯真的天性。
\item
  \emph{裴令}:中书令裴楷。裴楷和王戎是发小,娃娃的时候就穿一条开裆裤了。他们有一次结伴去青年学者钟会家玩,顷刻就受到钟会的赞赏。钟会成为司马昭心腹得宠后,立刻举荐两人同时进入政坛,当司马昭的副官。裴楷和王戎两人是姻亲,裴楷的侄子裴頠是王戎的女婿,两人也常常一起升官,甚至同列三公,执掌朝廷。
\item
  \emph{恸}:本义是大哭,引申为悲痛。
\item
  \emph{灭性}:指因为哀伤过度伤害身体,以至丧失性命。《曲礼》:``居丧之礼,毁瘠不形,视听不衰,不胜丧,乃比于不慈、不孝。''《孝经》``身体发肤,受之父母,不敢毁伤'',``毁不灭性,此圣人之政也'',所以哀伤过度而伤害生命,古人认为是不合圣人之教的,也是一种不孝。裴楷看着好像是批评指责,其实是反过来劝说。不过我们可以再恶意揣度一下,王戎是名士,嵇康的粉丝,家里又有钱,自然要和嵇康一样服用五石散(寒食散),据鲁迅先生讲,五石散这个东西需要喝大量的酒、吃大量的冷食来缓解它的毒性;据苏轼讲,``服寒食散以济其欲'',这还是一种助欲之药。王戎在居丧期间想必还在服用五石散,但总要控制一下食欲和性欲,结果人搞得憔悴不堪,我们只要看看毒品教育片和宣传画就可以想见王戎骨瘦如柴的模样了------珍爱生命,远离毒品,社区常常打出这样的标语。
\end{itemize}

\section{1.21}\label{section-20}

\begin{quote}
王戎父浑,有令名,官至凉州刺史。浑薨,所历九郡义故,怀其德惠,相率致赙数百万,戎悉不受。
\end{quote}

\begin{itemize}
\tightlist
\item
  \emph{王浑}:字长源,浑,大水涌动声,《荀子 -
  富国篇》``财货浑浑如泉源''。父亲王雄是幽州刺史,与王祥是同族弟兄。魏晋时候还有一个王浑,是太原王家的,徐州刺史,在灭吴国的时候是晋国的重要将领。当时吴国投降,镇南大将军杜预等在吴宫宴请吴国旧臣,酒过半巡,志得意满的王浑挑衅地说:``诸君,吴国灭亡了,你们没有一点忧愁吗?''吴国旧臣低头无语。这时候周处应对道:``汉末分崩,三国鼎立,魏国灭亡在前,吴国灭亡在后,说到亡国的忧愁,怎么只是我们才有,而阁下却没有呢?''王浑非常没趣,比吴国旧臣更加羞愧。《世说》后面还有这位徐州刺史王浑和妻子、儿子的故事,王戎的父亲王浑在《世
  说》中只有这次遗体告别机会。
\item
  \emph{令}:美好。《论语》``巧言令色'',《孔雀东南飞》``年始十八九,便言多令才''。
\item
  \emph{凉州刺史}:凉州是现
  在的甘肃、河西走廊一带,气候寒凉,所以称凉州。凉州在后来的唐代诗歌中是一个醒目的符号,无论是王之涣的``黄河远上''还是王晗的``葡萄美酒'',都是七绝中的杰出作品,形成了固定的曲调。刺史本来是西汉时中央派到州的督察专员,多用年轻人,品级并不高,但权力很大。到东汉品级也提高了,超越了太守,成为州的第一号人物,往往统领所有事务。汉代把天下分为十三州,
\item
  \emph{薨}:古代王侯死叫做薨。王浑曾封为贞陵亭侯,所以他的死称薨。汉朝的侯爵分为三个等级:县,乡,亭。亭是秦汉时的基层行政单位,``大率十里一亭,十亭一乡'',一般封了一个县就叫县侯,一个乡就叫乡侯。王戎安丰侯是县侯,比父亲高二级。
\item
  \emph{九郡}:据《晋书 -
  地理志》,凉州管辖八个郡,所以有以为这里的九郡应是八郡。但是也有说《御览》是引作``州郡''的,认为``九''是``州''的误字。
\item
  \emph{义故}:义从和故吏。指自愿受私人招募的人员和老部下。东汉豪强有招募私兵的做法,这些人没有朝廷的俸禄,是主人的门客和打手,等豪强得势以后,这些人就是亲信,往往最受重用。三国时候最有名的义从是公孙瓒的``白马义从'',杀得少数民族心惊胆战,望风而逃。
\item
  \emph{赙}:送给别人办丧事的财物。《礼记》:``吊丧弗能赙'',《仪礼》:``知死者赠,知生者赙'',
  《说苑》:``玩好曰赠,货财曰赙''。
\end{itemize}

虞预的《晋书》中说,因为这件事,``戎由是显名''。王戎后来是著名的小气鬼,这时候的表演反倒叫人跌破眼球,是演戏还是复杂的人性?晋代是非常奢侈的时代,照道理说天下初平,百废待兴,君臣应该励精图治,重振山河,可事实上朝廷上下享受成风,武帝司马炎光妻妾就要一万多人。官僚都过着时不待我,朝不图夕,痛饮生命中黑暗甘泉的生活,如此甜蜜而又如此绝望。正如《古诗十九首》中唱的:人生天地间,忽如远行客。斗酒相娱乐,聊厚不为薄。\ldots{}\ldots{}昼短苦夜长,何不秉烛游!为乐当及时,何能待来兹?

\section{1.22}\label{section-21}

\begin{quote}
刘道真尝为徒,扶风王骏以五百疋布赎之,既而用为从事中郎。当时以为美事。
\end{quote}

\begin{itemize}
\tightlist
\item
  \emph{刘道真}:刘宝,字道真,山东山阳人,可能是孔子的老乡。刘宝文武全才,对汉书有研究有作品,而且勇敢善骑射,后来做到安北将军、领护乌丸校尉、都督幽并州诸军事等职,关内侯。我看过一本笑话书,大概叫《启颜录》,里面有他的故事,刘宝有点像徐文长,出句嘲笑农村妇女,结果自讨没趣,被狠狠羞辱了一番,这种故事未必可信,因为对联从历史上说是唐代以后的事情,但从另一方面理解,刘宝可能是个滑稽和热情的人。故事也告诉我们,三步之内,必有芳草,读了点书也得谦虚。
\item
  \emph{徒}:刑徒,罪犯。我不知道刘宝犯过什么事,不过估计不是见不得人的罪,也许是勾引女孩子,因为史书上说刘宝音乐才能很好,许多人喜欢,闻者无不流连。
\item
  \emph{扶风王骏}:司马懿的第七子司马骏,字子臧,封扶风王。据《晋书》上说,司马骏是宗室中最文采风流的人,很小的时候就通诗书了,年岁大了声望也首屈一指。
  他当王时倡导农桑,身先士卒,自己和属下每人限耕40亩。司马骏本来是汝阴王(河南),西晋时少数民族侵扰,秦州刺史胡烈、凉州刺史牵弘先后败死,司马骏都督雍、凉等州诸军事,``善抚御,有威恩'',多次给予打击,出现了``遣入质子''和
  ``二十万口来降''的局面,后汝阴王徙封为扶风王(陕西),使他的王国与都督所在地相近,并且一直在那里当了18年都督,直到死去。死的时候``泣者盈路,百姓为之树碑,长老见碑无不下拜,其遗爱如此。''
\item
  \emph{疋}:《说文》匹,四丈也。
\item
  \emph{赎}:用财物来抵销罪过,解除刑罚。在唐以前,布匹是一种货币,因为农民交税,布是其中之一,于是成为中间货币。《卖炭翁》中说``半匹红纱一丈绫,系向牛头充炭直''。当然也可能货币本身叫布,《诗经
  -
  氓》:氓之蚩蚩,抱布贸丝。``布''指流行于周、郑、卫、韩、赵、魏、燕、楚等地的镈(钱、铲)形铜币。用钱财抵罪的做法是原始社会遗留下来的,以民事赔偿代替刑事处理,估计有什么条件,我不懂法律史,资料待查。我有朋友在监狱系统,有的农场卖西瓜,罪犯是有指标的,比如说卖掉2万斤给加几分,加到一定的分数后可以减刑几个月,前些年这种做法一定是有的,现在不知道还有没有。据说湖北还是湖南有个监狱,办宾馆效益不好,就搞人性化管理,叫犯人家属来队住,夫妇可以同房,为监狱创收还可以加分,这也可以理解。可是发展到后来宾馆变成了青楼,一篇内参,据说原司法部部长吴文英因此去职。
\item
  \emph{从事中郎}:官名,大将军府的属官,职参谋议,主管文书、谋划。
\item
  \emph{以为美事}:暗含百里奚的典故。百里奚原为虞国大夫,晋灭虞被俘,后作为秦穆公夫人的陪嫁奴隶送往秦国。百里奚以为耻,逃脱到了楚国。在楚国,他开始帮人养牛,养的牛都膘肥体壮,结果楚王知道了,把他招去,让他为自己在南海养马。秦穆公看到礼物中的百里奚不见了,就问公孙支怎么回事?公孙支把百里奚的逃跑的事报告了一遍,然后说:这个百里奚,是个了不起的人,可惜没被重用。秦穆公说,我要备重礼去楚国,把百里奚换回来。公孙支摇头道:大王这么做,就得不到百里奚了。现在楚王叫百里奚牧马,就说明楚王还不知道百里奚的价值。若现在备大礼去赎百里奚,就是在告诉楚王,百里奚是个人才啊。我们只需对楚王说,有个叫百里奚的老奴逃跑了,为了惩一儆百,我们要抓他回来。拿老奴的价钱,五张黑色公羊皮去交换就好啦。后来秦穆公授百里奚以国政,号``五羖大夫'',成为秦穆公称霸的辅助重臣。
\end{itemize}

\section{1.23}\label{section-22}

\begin{quote}
王平子、胡毋彦国诸人,皆以任放为达,或有裸体者。乐广笑曰:``名教中自有乐地,何为乃尔也!''
\end{quote}

\begin{itemize}
\tightlist
\item
  \emph{王平子}:王澄,字平子,山东琅琊王家,王衍的弟弟,王戎、王敦的族弟。王戎怀抱目的认真做隐士,达到目的后认真做官。而王澄没做过隐士,始终在官场,但完全是认真的隐士派头,在《世说》的后文中,他还有奇特的表演。王澄文武全才,曾任荆州刺史,爵南乡侯,但他政事、军事上应该说是个失败者,在文学和思想上也没有留下什么后人可以佩服的东西,但那些目无旁人、不以世务经心的放纵言行和家族势力的影响,使他得到了崇高的声望,最后引起了族中以凶狠著称的王敦的记恨,把他杀害。我讨厌王澄,因为西晋的统治太严酷,流民成为一大问题,有一次流民杀死了王澄治下的一个县令,王澄派兵镇压,流民请降,王澄假装答应,然后解除流民武装,把他们的妻子作为部下的赏赐,沉八千余人于江中。于是益州、梁州四五万家流民俱反,政局大乱。王澄这样胡作非为,横死人手也算老天有眼。
\item
  \emph{胡毋彦国}:姓胡毋,名辅之,字彦国,山东奉高人,曾任陈留太守、建武将军、湘州刺史等职。胡毋这个姓本来就表明是齐国王族的后代,是田姓
  的分支。胡毋辅之家穷,曾经为当县令戒过酒,也表现出能力,但后来环境改变,又开始醉生梦死地生活,枉费父母的美好愿望。胡毋辅之自己不拘小节,也放纵儿子不拘小节。他儿子从不把父亲放在眼里,吃醉酒时,直呼父亲的名字,没得酒喝就骂老头子无能。胡毋辅之听了,就招呼儿子一起来喝酒。
\item
  \emph{任放}:任性放纵。王隐《晋书》中说:``魏末阮籍,嗜酒荒放,露头散发,裸袒箕踞。其后贵游子弟阮瞻、王澄、谢鲲、胡毋辅之之徒,皆祖述于籍,谓得大道之本。故去巾帻,脱衣服,露丑恶,同禽兽。甚者名之为通,次者名之为达也。''这些世家子弟没有了竹林人士的如履薄冰和悲愤压抑,而继承了他们的醉酒疯痴,完全是享乐主义生活,这样过分的生活当然包含着玄学精神,也许和五石散也有必然的关系。魏晋和明代中晚期有点相似,不过魏晋时期的放纵是一种社会认同的普遍状态,如阮
  籍《咏怀》中感叹``开轩临四野,登高望所思。丘墓蔽山岗,万代同一时'',体现了齐万物的思想。而明代官僚的放纵与下层社会对礼教的严格遵守,与对女性贞洁的严酷要求往往并行不悖,糅合在一起。就像现在,官僚们人格分裂,嘴
  上喊着要过好``权力关''、``美色关'',作报告头头是道,日常生活却往往缺乏道德,虚伪成为常态,完全是无哲学意义的放纵。在前面,我屡次提到《晋书》,需要解释的是,我们现在看到的《晋书》是唐代修的,在魏晋时期,还有很多版本的《晋书》,在《世说》的注中就引用了很多部《晋书》。其中一个原因就是那时候的人希望名垂青史的愿望比现在的人要强得多,我们在阅读的时候应该注意到这一点。
\item
  \emph{乐广}:字彦辅,河南南阳人,寒族,得到裴楷、王戎的尝试,走上从政之路,最后接了王戎尚书令的班。应该说他深得中庸之道的精髓,自己要求严格,但通情达理,不讲别人的短处,名望很高。他不谈怪力乱神,成语杯弓蛇影的主人公就是他。他当河南尹的时候,据说官舍多妖怪,前任都不敢住,乐广一个人住着毫无在乎,左右皆惊。后来乐广看见墙上有孔,使人掘墙,得狸而杀之,其怪亦绝。乐广为政无当时功誉,然每去职,人常思其遗爱,不由让人想起《后汉书》中写刘宠的那一段:刘宠在绍兴当太守的时候,不干什么事,后来要调离了,几个老头送来一百钱,说:``我们那里是山区,还不认识太守和衙门。你到任以前,常常有官吏到我们那里,至夜不绝,或狗吠竟夕,民不得安。自你上任后,狗不夜
  吠,民不见吏。年老遭值圣明,今听说你要离开、抛弃我们,故自扶奉送。''宠曰:``吾政何能及公言邪?勤苦父老!''为人选一大钱受之。有人说无为是道家治国之道,其实儒家也讲``礼治'',讲村民自治。乐广当官不管事,是在践行落实儒家的自治理论。
\item
  \emph{名教中自有乐地}:找情人有找情人的痛苦烦恼,一夫一妻有一夫一妻的好处。乐广笑着批评,也算是所谓的``普世价值''中的宽容吧,理解而不赞同。把这条放在德行,那是在表扬乐广而不是赞赏王澄、胡毋辅之。
\end{itemize}

\section{1.24}\label{section-23}

\begin{quote}
郗公值永嘉丧乱,在乡里甚穷馁。乡人以公名德,传共饴之。公常携兄子迈及外生周翼二小儿往食。乡人曰:``各自饥困,以君之贤,欲共济君耳,恐不能兼有所存。''公于是独往食,辄含饭著两颊边,还吐与二儿。后并得存,同过江。郗公亡,翼为剡县,解职归,席苫于公灵床头,心丧终三年。
\end{quote}

\begin{itemize}
\tightlist
\item
  \emph{郗公}:郗鉴,字道徽,山东高平人,王羲之岳父。他在西晋时受到东海王司马越的赏识,西晋灭亡时,其宗族乡人推为主,成为有实力的军阀,琅琊王司马睿(东晋第一任皇帝晋元帝,曾经是司马越的跟班)任命为兖州刺史,后加辅国将军,都督兖州诸军事(西晋由于实行分封制,部队往往成为私人武装)。司马睿儿子司马绍当上晋明帝时,拜安西将军,都督扬州、江西诸军镇合肥,成为东晋皇家部队的主要将领之一,由于他忠于朝廷得到信任,最终打通了流民帅和朝廷的关节。他曾
  经击败钱凤、王含、王敦、祖约、苏峻、刘征等叛乱,最后做到太尉,成为东晋第一号军事领袖,和王导、庾亮同为东晋政坛的三巨头。纵观郗鉴一生的功绩,应该说正是他为人谦虚退让,又坚守原则,做事力求抑制矛盾,得到了皇帝和高门世阀的共同认可,各士族诸门户间的权力平衡状态得以维持,对时局的稳定起到了积极作用,明末大思想家王夫之说:``东晋之臣,可胜大臣之任者,其为郗公乎!''这话应该是不错的。
\item
  \emph{永嘉丧乱}:软弱而迷糊的晋惠帝司马衷吃了一块饼后暴毙,八王之乱中最后一个把持朝政的东海王司马越选中司马炎第二十五个儿子司马炽继位,世称晋怀帝,建元永嘉。永嘉五年(公元311年),匈奴主刘聪派大将石勒、刘曜等南侵,攻破洛阳,先后俘虏怀帝和名相王衍等大臣,掘晋诸帝陵墓,焚毁全城。使魏晋以来花费民力和资财,经过近百年的努力才建设起来的洛阳
  城,距董卓之后又一次化为灰烬,史称``永嘉丧乱''。5年后刘聪攻下长安,俘虏晋愍帝司马邺,西晋灭亡,建国52年。史书上说,后来刘聪志得意满地对司马炽说:``你当豫章王的时候,我和王济去拜访你,王济曾当你的面大大表扬了我一通。你说早就听说我的大名了,还送给我柘木做的弓和银制的砚,你还记得吗?''司马炽说:``臣怎么敢忘呢!可惜当时没能早识龙颜!''刘聪又问:``你家骨肉兄弟为什么要如此自相残杀?''司马炽说:``大汉将要应天受命,所以为陛下自动扫除,此乃天意,非人力所能挽回。何况臣家里要是都能和睦相处,守住武帝创下的基业,陛下又怎么能得到它呢!''刘聪听得高兴,于是将自己宠幸过的女子赏给司马炽为妻,不久后刘聪还是不放心,又杀死司马炽。愍帝司马邺被俘后,刘聪也像对待怀帝那样,出猎时命令司马邺全身披挂,手执长戟,作为前导;喝酒时命令司马邺穿上青衣,替大家斟酒洗怀,甚至在自己小便时,命令司马邺替他揭开便桶盖。晋代被俘大臣见到没有不痛哭的。司马昭父子得到天下用了卑鄙的手段,其子孙失去天下后又如此凄惨,虽说是皇帝家的私事,可是这种耻辱却是凌驾于汉人头上,东晋居然不思进取,往往只顾门阀内斗,可谓良心丧尽。后来北宋、明代又上演了这一幕,恢复始终无望,我们的唯物主义历史学家还大讲民族融合、中国文化博大精深等等,令人感慨。
\item
  \emph{馁}:饥饿。《说文》馁,饥也。
\item
  \emph{传}:轮流。
\item
  \emph{饴}:通``饲'',给人吃。
\item
  \emph{兄子迈及外生周翼}:郗迈,字思远,郗鉴后来重用郗迈超过自己的儿子郗愔、郗昙,因为郗愔、郗昙有名士之风,不喜欢世务,不以才干见长。外生周翼:外甥周翼,字子卿。
\item
  \emph{同过江}:世称``衣冠南渡'',东晋士族南迁,带去比较先进的文明,中原则是少数民族当家,东部文化至此逐渐赶上和超过中原文化。
\item
  \emph{席苫}:《仪礼 -
  既夕礼》``居倚庐,寝苫枕块。''古时父母死了,就要在草垫子上枕着土块睡。唐代贾公彦解释说:``孝子睡草席,是哀悼父母被埋在草下;枕土块,是哀悼父母被埋在土中。''
\item
  \emph{心丧}:古时父母死,服丧三年,外亲死,服丧五个月。周翼完全按照守父母孝的做法,但他是外亲,后来不能再穿丧服,只能在心中服丧,所以叫心丧。据说孔子去世后,弟子们都服``心丧''三年。
\end{itemize}

《世说》就是本小说,班固在《汉书 -
艺文志》中说:``小说家者流,盖出于稗官。街谈巷语,道听涂说者之所造也。''这个故事本来就是
随口说说的,只能表达民众对郗鉴的推崇,不能当真。因为在《晋书》中说,当时郗鉴``于时所在饥荒,州中之士素有感其恩义者,相与资赡。鉴复分所得,以恤宗族及乡曲孤老,赖而全济者甚多,咸相谓曰:今天子播越,中原无伯,当归依仁德,可以后亡。遂共推鉴为主''。正是因为郗鉴的仗义疏财,是家族中的``及时雨'',所以拉起了一支革命队伍,成为他起家的资本。郗鉴一日三餐鼓了个腮帮子给两个亲戚送饭,没提到他的好几个子女,是因为这两口饭显然喂不饱六七口人,
为了增加故事的真实性,所以其他家庭成员过滤性消失了。

\section{1.25}\label{section-24}

\begin{quote}
顾荣在洛阳,尝应人请,觉行炙人有欲炙之色。因辍己施焉。同坐嗤之,荣曰:``岂有终日执之,而不知其味者乎!''后遭乱渡江,每经危急,常有一人左右己。问其所以,乃受炙人也。
\end{quote}

\begin{itemize}
\tightlist
\item
  \emph{顾荣}:字彦先,江苏吴县人(苏州),他的祖父顾雍在孙权那里当过10几年的丞相,有出色的才能和崇高的声望,但为人谦和,所以在《三国演义》中没有张昭出名。司马炎灭吴后,北人鄙视南人为``亡国之余'',南人则有丧亲亡国之痛,故多有反抗之举。司马炎决定下诏招揽南士,顾荣和江南另一望族的陆云、陆机一起应诏到洛阳,去恢复祖先的光荣和开拓家族的发展,当时被北人称为``三俊''。但是由于传统,北方人还是看不起吴国这些亡国奴。顾荣在八王之乱中给朋友杨彦明的信中说:``吾为齐王主簿,恒虑祸及,见刀与绳,每欲自杀,但人不知耳!''常醉酒不问事以避祸,后来侥幸逃得性命回了江南。永嘉丧乱后,晋元帝司马睿南迁,江南士族国仇家恨,看不起被丧家犬司马睿。顾荣和王导积极配合,支持司马睿立足江南,``由是吴会风靡,百姓归心焉。自此以后,渐相崇奉,君臣之礼始定。''顾荣去世后被授予侍中、骠骑将军、开府仪同三司。
\item
  \emph{觉行炙人有欲炙之色}:发觉传递菜肴的仆役有吃烤肉的欲望。现在我不把吃肉当回事,自己的孩子更是只愿意吃虾蟹,可是我看过一本日本人的书大概叫《中国人的气质》,里面说,在晚清中国的穷人可能一辈子也吃不上几次肉,很是怜悯;而厨师、上菜
  工吃不上肉,更是要痛苦加倍的。钱钟书曾经注释梅尧臣的《陶者》:``陶尽门前土,屋上无片瓦;十指不沾泥,鳞鳞居大厦'',引用《淮南子》:``屠者藿羹,车者步行,陶者用缺盆,匠人处狭庐:为者不得用,用着不肯为。''引用唐代于濆《辛苦行》:``陇上扶犁儿,手种腹常饥;窗下掷梭女,手织身无衣。''等等,我看过一首民歌,其中说:泥瓦匠住草房,纺织娘没衣裳,卖盐的老婆喝淡汤,种粮的吃米糠,磨面的吃瓜秧,炒菜的光闻香,编凉席的睡光床,卖鞋婆子赤脚走,抬棺
  材的死路旁。
\end{itemize}

从前书上说那是揭露万恶的旧社会和残酷的剥削现象,可是在物质条件相对比较宽裕的今天,以标榜要解放全人类的今天,在标榜自己伟大、光荣、正确的今天,依旧有人这样没有尊严地生活,这些历史记录还是能够引起强烈的共鸣,多么令人感慨。
- \emph{因辍己施焉}:于是自己停吃让给他。 -
\emph{遭乱渡江}:指永嘉丧乱。 - \emph{左右}:帮助。《史记 -
萧相国世家》:``高祖为亭长,常左右之。'' - \emph{所以}:缘故。

这个故事有很多相似的文本,如成语结草衔环的两个典故,如《战国策》中的故事:中山国的君主宴请都城里的士大夫,大夫司马子期也在座。由于羊羹不够,司马子期没能吃上。他一怒之下,跑到楚国煽动楚王攻打中山。中山是个小国,中山君只能裸奔了。这时只有两个人提着武器保护他。中山君回过头来问:``这时候你们为啥还保护我?''两人回答:``我们的父亲曾经快要饿死的时候,多亏您给了他饭吃。父亲在临死的时候叮嘱我们说:`中山一旦有急难,你们俩一定要冒死去保护。'所以,我们是来保护您,为您献身。''中山君听罢仰天长叹:``给人东西不在于多少,应该在他灾难困苦的时给于帮助;怨恨不在于深浅,关键的是不要使人伤心。我因为一杯羊羹亡了国,却因为一碗饭得到了两个勇士。''
当然,我们还能回忆起楚庄王开断缨会、专诸替阖闾卖命等典故,这种一饭之恩最后得到丰厚的回报,往往发生在穷人和贵族之间的故事,可见穷人是多么缺乏和重视自己的尊严。如果那个穷人后来发达了,也许会出现周星驰无厘头电影的一幕,所以古人常常说
``仗义每多屠狗辈''。不过这个故事未必是这样的,因为它还有一个《晋书》的版本:``初,荣与同僚宴饮,见执炙者貌状不凡,有欲炙之色,荣割炙啖之。坐者问其故,荣曰:`岂有终日执之而不知其味!'及伦败,荣被执,将诛,而执炙者为督率,遂救之,得免。''顾荣首先是看到那个仆役相貌不凡(可惜我活了三十多年,漂亮姑娘倒见识了几个,可是一眼就能看出气度不凡的男人却一个没有,只能怪自己眼神不好),其次是那个仆役后来发达了,当上了死刑监督官(督率,要么是监狱长),暗箱操作救了顾荣一把。而出现两个版本的原因可能是《世说》采用了《顾氏家传》的说法,因为顾氏子孙既要感恩又要回避顾荣曾经是个死刑犯的不光彩历史。历史上后来还有类似的故事,想必西方也有吧,恐怕很难说大家都在编故事,寄托自己美好愿望。凡是美好的故事,请大家相信那是真的,不然人生太残酷了。

\section{1.26}\label{section-25}

\begin{quote}
祖光禄少孤贫,性至孝,常自力母炊爨作食。王平北闻其佳名,以两婢饷之,因取为中郎。有人戏之者曰:``奴价倍婢。''祖云:``百里奚亦何必轻于五羖之皮邪!''
\end{quote}

\begin{itemize}
\tightlist
\item
  \emph{祖光禄}:祖纳,字士言,河北范阳人,闻鸡起舞主人公之一祖逖同父异母的哥哥,曾经的职位是光禄大夫。祖家是北方士族,但他们的父亲、当过太守的祖武去世早,家道中落。祖纳以践行儒家学说而著称,弟弟祖逖却完全是一派豪杰风范。
\item
  \emph{孤}:失去父亲叫孤,孤儿寡母。《说文》孤,无父也。
\item
  \emph{常自力母}:经常亲历亲为?自己动手为母亲?这种表述方法很奇怪,力的意思不好落实,可能有漏字或是当时的语法现象,``常自炊爨作食侍母?''《晋书》中作:常自炊衅以养母。古代男人可能不用做饭,所以一动手就显出特别。最近半个多月父母回老家,妻子孩子回娘家,吃饭、洗衣成为我的一大困扰。我平时做家务,只有和父亲两个人的时候动动手,祖纳母亲还在,居然要他来烧,这样一想,真是奇怪。祖家有六兄弟,父亲死的早,起码有两个妻子,照道理她们的岁数不会
  大到哪里去,难道实行分居了,而且也许古代妇女身体比现代女性要差得多。
\item
  \emph{王平北}:王乂,字叔元,山东琅琊王家王雄的儿子,前面提到王戎父亲王浑的弟弟,名相王衍的父亲,他当时是平北将军,督幽州诸军事,祖纳是他治下之民。乂是收割、治理的意思,不知道和``元''有什么关系。
\item
  \emph{以两婢饷之,因取为中郎}:王乂送了两个女佣给祖纳,并选拔他当自己的侍从。要不然没有俸禄,祖纳更养不活四个人。
\item
  \emph{奴价倍婢}:祖纳相当于两个婢女。《汉书 -
  百官公卿表上》:郎掌守门户,出充车骑,有议郎、中郎、侍郎、郎中,皆无员,多至千人。议郎、中郎秩比六百石,侍郎比四百石,郎中比三百石。祖纳应该是低级侍从,所以近似于奴仆。
\item
  \emph{百里奚亦何必轻于五羖之皮邪}:百里奚何尝比五张黑公羊皮轻贱呢!其典前面已经解释过了,略。
\end{itemize}

这个故事反映了王乂的仗义疏财和祖纳的自尊、自信,祖纳以名相百里奚自诩。后来岁月摧磨,祖纳的地位越来越高,当上中护军、太子詹事、军谘祭酒,封晋昌公等,只是天下越来越乱,于是祖纳平时终日下棋。朋友王隐劝他说,大禹珍惜寸阴,没有听说痴迷下棋的。祖纳回答:聊以忘忧罢了。王隐又说:你见多识广,少长五都,游臣四方,华裔成败,皆当闻见,为什么不记录下来,何必借下棋来忘记忧愁呢?祖纳回答道:你的话我同意,可我没有那种力量。祖纳只是在有一次辩论
会上露出过锋芒,那天他和同事钟雅聊天,钟雅说:``我们河南人,锋利如锥;你们河北人,愚钝如槌。''祖纳说:``那就用我的钝槌,砸你的利锥。''钟雅说:``自有神锥,你打不着。''
祖纳说:``既然有神锥,也就有神槌。''钟雅就说不出话来了。依现在看,这是一篇十分无趣的地域歧视贴,我实在想不通,为什么这样不着边际的聊天居然作为《晋书
-
祖纳传》的最后一节,也许为了保留反映河南、河北人的区域文化特点和矛盾吧。

\section{1.27}\label{section-26}

\begin{quote}
周镇罢临川郡还都,未及上,住泊青溪渚。王丞相往看之。时夏月,暴雨卒至,舫至狭小,而又大漏,殆无复坐处。王曰:``胡威之清,何以过此!''即启用为吴兴郡。
\end{quote}

周镇:字康时,河南陈留人(现在的开封,战国时候是楚国的首都大梁,大家可能记得《三国演义》中汉献帝曾经是陈留王,曹操起兵反董卓的根据地也是陈留。
我没去过河南和开封,但看过曹锦清写的《黄河边的中国》,真有抚今悼昔之感)。史书上说周镇``清约寡欲,所在有异绩。''古人说``君子寡欲,则不役于物,可以直道而行'',现在我们的政府考评官员制度,哪里是叫人做君子啊,简直是逼他们去当骗子、强盗、篾片、皮条客,这样才能有异绩的,不过日光底下无新事,古
代想必也是如此。
罢临川郡:临川是现在的江西抚州一代,《世说》的编辑者刘义庆就是临川王。《礼记
- 王制》``诸侯之于天 下也,比年一小聘,三年一大聘,五年一朝'',《尚书 -
舜典》``三载考绩,三考黜陟幽明''(上古时期一个职位要九年一迁,可见官位之
少,衙门之小)。古代官员考评往往三年一大考,一般分品德、政绩、才能、年龄等方面,晋代估计也差不多,太守是五品至三品的官(地区不同品级也有区别,我买过一本古代官僚品级制度情况变化的书,书房太乱一时找不到,估计临川太守和吴兴太守的品级是不一样的,所以有这个故事,待核实)。这次周镇坐船回建康述职,三品以下由丞相、尚书等负责考核。
王丞相:王导,字茂弘,王览的孙子。王家三代经营,终于出了个王导,把王祥、王览的风光推向了新的高潮。王导是东晋朝的实际创造者,最后形成``王与马,共天下''的局面,门阀制度此时定型。王导曾任扬州刺史、录尚书事,丞相等职,总揽元、明、成帝三朝国政。在晋成帝司马衍时候,司马衍见了王导,每次都要下拜。下手令给他,要加上``惶恐言'',诏书中也添上``敬问''的字眼。王家势力太大,几个亲戚都有不臣之心和不臣之举,而王导始终坚持维护正统,这在一定程度上维护了东晋的稳定,使老百姓少受一点兵火之祸,这是王导的历史贡献。
卒:通``猝'',突然。
舫至狭小,而又大漏:鲁迅诗``破帽遮颜过闹市,漏船载酒泛中流'',鲁迅是潇洒,周镇是俭朴。
吴兴郡:现浙江湖州。
胡威:字伯武,一名貔,安徽寿春人,在西晋时当过右将军、前将军、豫州刺史,尚书、青州刺史,加奉车都尉、封平春侯等。父亲胡质在魏国时候曾是征东将军、荆州刺史。父子俩都以清廉著称,后来司马炎问父子俩谁更加清廉一点,胡威说自己不如父亲,因为``臣父清恐人知,臣清恐人不知,是臣不及远也。''这话非常得体,直指人心,不过另外有个故事却反映胡质矫情做作。胡威小的时候告别父亲回洛阳,胡质给了他一匹绢作差旅费,胡威就问:``父亲清高,我不理解你哪里
来的钱。''胡质说:``这是我俸禄里省吃俭用攒下来的。''故事到这里到还让人佩服,但是后来就过分了。胡威在路上遇到一个人,经常帮助他,胡威奇怪,多次盘问,才知道那个人是胡质手下的都督,因为心怀感激就请假与胡威一起回京。胡威把绢给了那位都督作为报酬。后来胡威写信告诉胡质这件事,胡质居然``质杖都督一百,除吏名''。胡质哪里是恐人不知啊,就是拿手下作秀来抬高自己虚假的声誉。
古代官员任用、考评的确是个难题,汉代举荐制度大概是每年每郡从十万人里面选出一个孝廉,魏晋逐渐推行九品中正制,隋唐后实行科举。白居易曾经写过一首诗:``赠君一法决狐疑,不用钻龟与祝蓍;试玉要烧三日满,辨材需待七年期(七年在这里大概是个虚数)。周公恐惧流言日,王莽谦恭未篡时;向使当年身便死,一生真伪复谁知?''王导虽然以和稀泥著称,曾经考察官员``吞舟是漏'',但这次突击考察得到自己满意的结果,立即启用周镇,我们可以从中体会到政治家的无奈和微薄的努力。

\section{1.28}\label{section-27}

\begin{quote}
邓攸始避难,于道中弃己子,全弟子。既过江,取一妾,甚宠爱。历年后,讯其所由,妾具说是北人遭乱;忆父母姓名,乃攸之甥也。攸素有德业,言行无玷,闻之哀恨终身,遂不复畜妾。
\end{quote}

\begin{itemize}
\tightlist
\item
  \emph{邓攸}:字伯道,山西襄陵人。西晋时先后任淮南太守、汝阴太守、河东太守等。永嘉之乱时,邓攸被石勒俘虏,侥幸逃脱,于是发生了下面的故事。
\item
  \emph{弃己子}:石勒俘虏邓攸后,赏识他的文辞,征召邓攸为自己的参军。后来邓攸一家人寻机逃脱,其中有自己的儿子及他弟弟的儿子邓绥。他觉得没有能力保全两个孩子,对妻子说:``我的弟弟早亡,只有这一个儿子,天理不可以绝后,只好自弃我儿了。如果有幸能够存活,我今后应当有子。''于是扔了自己的孩子。他的儿子可能有点大了,追了一天赶了上来。第二天,邓攸把儿子绑在树上离开。
\item
  \emph{取一妾}:《曲礼》``取妻不取同姓,故买妾不知其姓,则卜之。''古代同姓不通婚,所买之妾若出于微贱,不知其氏族之所出,就要卜筮,以决其疑。邓攸没有走这个程序,而且过了一年才问小妾的姓氏,不免有伪君子的嫌疑。
\item
  \emph{攸素有德业}:德业,偏义复词,德行。邓攸七岁时死父母和祖母,他小小年纪就按儒家的方式守孝九年,声名远扬。邓攸在东晋时得到吴郡太守的肥缺,他像当今
  美国的几个州长,载着米粮前去就任,表示只喝吴水,不接受俸禄。邓攸受到了百姓的拥戴,去职的时候下属、百姓们送钱财数百万,邓攸不受一钱。数千人留牵邓
  攸的船,邓攸于是小停,夜晚偷偷离去。当时江苏人歌唱道:``紞如打五鼓,鸡鸣天欲曙。邓侯挽不留,谢令推不去。''(紞如:击鼓声。五鼓:五更,3点到5点。谢令:应该是前任太守,估计钱捞了不少,所以引起了朝臣的羡慕,吴郡太守于是炙手可热。)邓攸后来当上侍中、吏部尚书、护军将军、会稽太守、尚书右仆射等。
\end{itemize}

这两个故事都让人不舒服,甚至足够引起常人的愤恨。古人说父母是``在家佛'',对孩子的要求总会竭尽所能地给与满足,邓攸把孩子绑住离
去,其坚决和残忍超过了一般人的想象能力。我也不知道邓攸的外甥女(外孙女?)怎么面对这件乱伦事件,但在当时的社会氛围中,也许她承担本应该由邓攸承担的命运------自杀。刘义庆把这两个故事放在一起,列入``德行''一章,省略了邓攸捆绑孩子的细节,也不交代其外甥女的下落,恐怕也是有所顾虑的。在``赏誉''一
章中,讲到邓攸最终没有儿子,谢安说:``天道无知,使邓伯道无儿。''千年后的今天,我们还可以想见那个在荒野中被捆绑在树上的孩子,眼睁睁地看着自己的父母离去时发出的撕心裂肺的哭喊;我们还可以触摸到那个以为逃脱死亡命运、一心一意对邓攸曲意承欢的女孩子,得知真相后害怕绝望的心。在《晋书》中编撰者也极度不满,批评道:``邓攸赢粮以述职,吴隐酌水以厉精,晋代良能,此焉为最。而攸弃子存侄,以义断恩,若力所不能,自可割情忍痛,何至预加徽纆,绝其奔走
者乎!斯岂慈父仁人之所用心也?卒以绝嗣,宜哉!勿谓天道无知,此乃有知矣。''现在我们见到的《晋书》是李世民要求编撰的,总编是房玄龄等,他们的态度是邓攸无子``乃天道有知'',这是唐代比晋代认识上的进步。我小时候看民间故事,大概是讲黄巢吧,他一路烧杀抢掠,见路边一个妇女背着大孩子、拉着小孩子逃命,心中好奇,一问,原来大孩子是亲戚的孩子,小孩子是自己的孩子,非常感动,饶了他们,这种类似的事在民间故事中还有不少,现在想来那妇女这样做恐怕也不合适,同样是不公平和虚伪的,它违背了人性。何谓真正的道德,需要伟大人物进一步明确。前天索尔仁尼琴去世了,世界上又失去了一位伟大人物,我回忆着他的那些作品,考虑着现在的中国,思绪烦乱------憎恶、无奈和厌世。

\section{1.29}\label{section-28}

\begin{quote}
王长豫为人谨顺,事亲尽色养之孝。丞相见长豫辄喜,见敬豫辄嗔。长豫与丞相语,恒以慎密为端。丞相还台,及行,未尝不送至车后。恒与曹夫人并当箱箧。长豫亡后,丞相还台,登车后,哭至台门;曹夫人作簏,封而不忍开。
\end{quote}

\begin{itemize}
\tightlist
\item
  \emph{王长豫}:王悦,字长豫,是王导的长子,死的较早,但在王氏家族中得到肯定,曾经陪太子读书,是东宫侍讲。
\item
  \emph{色养之孝}:色,《论语》中子夏问什么是真正的孝。孔子说,最难的是始终对父母和颜悦色(色难)。有事的时候,年轻人承担些体力活,有酒饭的时候,让长辈先享用,这些并不难做到。难的就是始终如一的好态度。养,《论语》中子游问什么是真正的孝。孔子说,如今所谓的孝,只是说能够赡养父母罢了。但即便是犬马都能够得到饲养,如果不尊敬父母,那么赡养父母与饲养犬马又有什么区别呢?所以在本文中``色养之孝''是偏义复词,主要是指色孝。
\item
  \emph{敬豫}:王恬,
  字敬豫,王导的次子,王恬年轻的时候好武,不拘礼法,王导不太喜欢他。悦、恬、豫都是愉快的意思。王导不喜欢孩子能带兵,有武功,也许是为了避帝皇之忌讳,也许是东晋名士之风,看不起当兵的人。王恬其实多才多艺,书法、围棋都很出色,他的棋艺被认为是当时东晋第一,``中兴第一''。
\item
  \emph{还台}:台,中央机关的官署,这里指尚书省,当时王导录尚书事。每次王导去上班,那可是早上3、4点钟,王悦都起来送他上车。
\item
  \emph{曹夫人}:王导的妻子。王导很怕老婆,但还是喜欢女性,就在另外地方养小妾。曹夫人知道了,带领婢女杀向王导的别馆。王导看远处烟尘滚滚,一打听是妻子杀来了,美妾也不顾了,``飞辔出门,左手攀车栏,右手提尘尾,以柄打牛'',抱头鼠窜,成为一时的笑谈。后来司徒蔡谟对王导开玩笑说∶``朝廷欲加公九锡(九锡,古代帝王赐予诸侯大臣的最高礼遇。属嘉礼,指衣服、朱户、纳陛、舆马、乐则、牙贲之士、鈇钺、弓矢、秬鬯等九种器物及待遇,多授予对国家有大功的权臣)。''王导当真了,连说不配。蔡谟接着说∶``九锡中我不记得其他东西,惟有短辕、犊车、长柄尘尾。''王导又羞又怒,无可奈何。曹夫人可谓爱之深,责之切,同时也说明晋代风气的开放,女性自觉意识的高涨------不甘心当男人的附庸和玩物。
\item
  \emph{并当箱箧}:整理收拾箱子。
\item
  \emph{长豫亡后}:王悦身体不好,《晋书》上说,有一次王导梦见有人用百万钱向他买王悦,王导梦里答应了,醒来很后悔。过了不久,王导家里挖地,挖出一百万钱,王导命人赶快掩埋上,心情很糟糕。不久王悦病重,王导忧心忡忡,几天吃不下饭。一天王导又梦见一人,形状伟岸,披甲持刀而来。王导问:``您是谁?''说:``我是蒋侯(东汉蒋子文,官至秣陵县尉,因追盗贼负伤致死,葬于南京钟山(现中山陵处)。蒋子文死后,他原来的部下曾在路上看见他``乘白马,执白羽,侍从如生'',疑为他已是神灵,故建``蒋王庙''。
  三国时期吴帝孙权追封蒋子文为蒋侯,因避祖父孙钟之名讳,遂改钟山为``蒋山'',蒋王庙易名蒋山寺。于是东吴民间广建``白马将军庙''供奉蒋侯塑像,除设庙宇,民间家家均有``白驮将军''或``白马大将军''牌位,专供驱邪避凶之用),知道王悦身体不佳,特地而来,希望大人不要忧心了。''王导醒来很高兴,于是开始吃饭,竟然吃了几升。后来又梦见蒋子文说:``大人的忧患,我无法相救了。''不久王悦病逝。
\end{itemize}

这是《世说》中难得的几个感人的故事之一,老年丧子,睹物思人,情不能堪。文章通过日常生活中平凡温馨的两个细节,让读者能真切地体会王导和妻子无限的悲痛,这种悲痛,千百年后一样能得到我们的理解和体味。

\section{1.30}\label{section-29}

\begin{quote}
桓常侍闻人道深公者,辄曰:``此公既有宿名,加先达知称,又与先人至交,不宜说之。''
\end{quote}

\begin{itemize}
\tightlist
\item
  \emph{桓常侍}:桓彝,字茂伦,安徽怀远人。他在晋明帝王敦叛乱时曾任散骑常侍,是皇帝左右的顾问官,从此建立了与皇族深厚的情意,后来结为姻亲。苏峻叛乱,桓彝义不投降,力战而死。他有个儿子叫桓温,晋明帝司马绍的女婿,是曹操一类的人物,有勇有谋,武功卓著,曾经废了皇帝司马奕,另立简文帝。当时东晋第一大势力就不在是王家,王、谢、庾三大世家全都需仰仗桓温鼻息。桓温的儿子桓玄甚至自己当了两年皇帝玩玩,东晋司马家的正统地位从此烟消云散,由此而灭。
\item
  \emph{深公}:竺法深,名潜,或称道潜,字法深,俗姓王,据说是琅琊王氏一族,他18岁出家,师从名僧刘元真。晋代大乘佛教的传译已经大备,玄学家们借般若思想来丰富玄学的理论,而佛教学者则迎合玄学思潮以取得生存和发展的条件,于是社会上很快兴起一股佛玄合流的思潮,僧侣中涌现出一批与玄学名士的风貌接近的僧人,竺法深便是最先出现的这类名僧之一。``永嘉之乱''后,竺法深南渡,和朝廷皇帝大臣保持长期交往,后来隐居绍兴嵊州,以弘法为己任,讲学30余年,牵引
  老、庄思想阐释大乘般若学说,创立本无异宗,大意谓``诸法无本''是佛家第一义谛,但``无''能生万物,即``无''在``有''先,从``无''出``有''。他的这一宗派
  都养成了``内(佛学)外(世俗学问,主要指老、庄玄学)兼洽''的治学特点。
\item
  \emph{宿名}:宿,老,长久。宿名,久为人知的名望。
\item
  \emph{先达}:前辈贤达。
\item
  \emph{先人}:先父,桓彝的父亲桓颢。桓彝,276-328年;竺法深,285-374年,所以程炎震说:根据两人年龄考订,桓彝比竺法深大十岁,这话可能是桓温说的,而不是桓彝说的。
\end{itemize}

竺法深本来不姓竺,这个姓在佛教中有使用,就像现在的僧人姓释一样,属于外来姓氏,古代印度称天竺国,有天竺僧人来中国传教,以国名``竺''为姓。后来,中国僧侣有的随印度僧人学经,也随师父的姓,成为竺姓。佛教有时候也称为竺教。桓温不评价竺法深,因为他是长辈,评论不符合儒家的学说。佛教传入中国有个过程,有些舆论认为他们殊服异制,不拜帝王和父母的作风是不忠不孝、违法乱俗的行为,朝廷曾下令要求沙门跪拜帝王,引起纠纷;而且僧人交游帝王大臣,不符
合佛教出世的宗旨,有其矛盾之处。这一则放在``德行''里面,是说桓温``孝'',同时也是说讲别人好话也是德行。这些天普陀山的胖和尚、``武僧''戒忍,日子也不好过,因为他引起了上层领导的反感,其中内幕我没打听,恐怕要问全国政协的人士,不过当今寺庙之乱象,佛法之堕落,是很好的谈资,真可谓``人间佛教''。

\section{1.31}\label{section-30}

\begin{quote}
庾公乘马有的卢,或语令卖去。庾云:``卖之必有买者,即复害其主,宁可不安己而移于他人哉!昔孙叔敖杀两头蛇以为后人,古之美谈。效之,不亦达乎!''
\end{quote}

\begin{itemize}
\tightlist
\item
  \emph{庾公}:庾亮,字元规,河南颖川人。东晋四大世家王、谢、庾、桓之一的族长。他妹妹是是东晋明帝司马绍的皇后。当时司马宗主掌禁兵,曾在司马绍生病之时,拒绝给庾亮宫中的钥匙,并斥责他说:``这是皇宫,难道是你家的门户吗?''司马绍二十几岁就早逝了,司马绍病危时,一概不见大臣,颇有托孤宗室之意,庾亮强进宫门,直入寝室,劝阻司马绍不要单独托孤宗室,一定要与大臣共同辅政,才能确保政局稳定。司马绍终于被说服。庾亮受遗诏与王导等辅立成帝,任中书令,执
  掌朝政。应该说,统治阶级内部民主制更适合东晋王朝的实际情况,从后世晋孝武帝加强皇权成功而导致东晋王朝走上末路来看,庾亮为东晋立了大功。后来王家的王敦以``清君侧''的名义叛乱,王家受到沉重打击,王导心中有愧,步步退让,庾亮一改前任王导辅政以宽和得众的做法,不讲情面,均按法规办事,于是庾家一跃成为当时的第一世家。庾亮对宗室进行大清洗,诛杀司马宗。直到苏峻之乱后,晋成帝还不知道司马宗已被杀害,他问舅舅庾亮,``往常那位白头公何在?''(司马
  宗一头白发)庾亮奏道:``司马宗谋反伏诛''。晋成帝流泪道:``舅舅说人家做贼,便杀之,如果人言舅舅做贼,又该如何?''庾亮脸色大变,无言以答。一般认为,庾亮为人正直,但是过于理想主义,世家观念较重,不能很好的整合力量,执政成果并不理想。
\item
  \emph{的卢}:马名。马白额人口至齿者叫的卢。按迷信说法,这是凶马,它的主人会得祸。刘孝标引用散佚的《伯乐相马经》中说
  ``奴乘客死,主乘弃市,凶马也(下等人骑坐客死他乡,上等人骑坐被处以死刑)。''说到的卢,大家当然记得刘备的故事,刘备有匹好马的卢,刘表喜欢上它,刘备将此马送于刘表。可是刘表的重臣蒯越认为此马是凶马,表示了疑问。于是刘表将马又还给了刘备。接着伊籍告诉刘备的卢不祥,刘备说``但凡人死生有命,岂马所能妨哉。''后来的卢马跃檀溪,救了刘备的命。刘备被蔡瑁追杀的故事也许仅仅是个故事,未必能当真。的卢还有种说法,就是它眼下有泪线,就像有些相书中说
  人眼下有痣,象征着会不停地流泪,命运孤苦。
\item
  \emph{或}:有人,《晋书》等中说是庾亮的好朋友兼下属殷浩。
\item
  \emph{宁可不安己而移于他人哉}:怎么可以因为对自己不利就转嫁给别人呢?就像收到假钞,再把它用掉就是害人了,这个道理大家都懂,但能做到的人很少。
\item
  \emph{孙叔敖}:春秋时代楚国的令尹(宰相)。据贾谊《新书》载,孙叔敖小时候在路上看见一条两头蛇,回家哭着对母亲说:听说看见两头蛇的人一定会死,我今天竟看见了。母亲问他蛇在哪里,孙叔敖说:我怕后面的人再见到它,就把它打死埋掉了。他母亲说:不要忧虑。我听说积了阴德的人,一定有善报。
\item
  \emph{达}:旷达、通达。
\end{itemize}

如今迷信之风又盛,戴高帽的说法是继承传统文化,其实不过是人无法掌控自己的命运,无论在丛林法则下被强食的还是侥幸获胜的,在生活中都茫然不知所措罢了。葛洪有句名言``我命由我不由天''才是中国传统文化的正解。

\section{1.32}\label{section-31}

\begin{quote}
阮光禄在剡,曾有好车,借者无不皆给。有人葬母,意欲借而不敢言。阮后闻之,叹曰:``吾有车而使人不敢借,何以车为!''遂焚之。
\end{quote}

\begin{itemize}
\tightlist
\item
  \emph{阮光禄}:阮裕、字思旷,阮籍的族弟(同一个爷爷),曾经被召为金紫光禄大夫(三品的高级文官),不肯俯就。不过也因此而用官名称呼他。有人问阮裕,那你为什么当过两郡的太守呢?阮裕说,我从小没有当官的情致,但也没有什么谋生的本领,要生活下去就要有点收入,所以当过太守。他也曾经深受王敦的赏识,当他的主簿,但觉察出王敦的不臣之心,就和堂兄阮籍一样终日饮酒避祸,最终得以全身。
\item
  \emph{剡}:绍兴嵊县,现在的嵊州。晋唐之间,嵊州可是出名的风光美好的地方,阮裕就在此隐居。据说``东南山水越为最,越地风光剡领先'',我在嵊州、新昌住过个把月(新昌以前也属于嵊县),山水之美令人怀念,杜甫曾经说:``剡溪蕴秀异,欲罢不能忘。''不过现在那里厂很多,有些污染。
\end{itemize}

这个故事没有难解之处,这个细节很能反映阮裕大气和助人为乐的品性。让人想起季札子挂剑的典故。季札子是春秋时吴国贵族,他奉命出使徐、鲁、齐等国。他经过徐国时,徐君十分欣赏季札子身上佩戴的宝剑,却说不出口。季札子明白徐君的意
思,不过,他还要出使其他的国家,宝剑暂时不能送给徐君,他打算等他出使各国后,再送给徐君。不料,待季札子出使完各国,再途经徐国回国时,徐君已经死了。季札子感到非常遗憾,想到自己曾打算把宝剑送给徐君,便解下身上的宝剑,挂在徐君墓前的树上。还有一个典故,据说一天孔子要出门,下起了雨;有个弟子说:``子夏有把漂亮的伞,您向他借了伞再出门吧!''孔子笑笑说:``子夏为入比较吝啬,我听说和人打交道要扬其长避其短,这样才能交往长久。我不是不知道子夏有伞,若向他借伞,借,他心里不痛快;不借,不是把他的缺点暴露给众人了吗?''所以后人说:``爱见人长,共服宣尼休假盖;恐彰己吝,谁知阮裕竟焚车。''

\section{1.33}\label{section-32}

\begin{quote}
谢奕作剡令,有一老翁犯法,谢以醇酒罚之,乃至过醉,而犹未已。太傅时年七八岁,著青布绔,在兄膝边坐,谏曰:``阿兄,老翁可念,何可作此!''奕于是改容曰:``阿奴欲放去邪?''遂遣之。
\end{quote}

\begin{itemize}
\tightlist
\item
  \emph{谢奕}:终于谢家出现在《世说》里面了,而且是谢安打头阵,哥哥谢奕配角。谢奕,字无奕,河南阳夏人,最后当到都督豫司冀并四州军事、安西将军、豫州刺史。谢奕为人任性,和桓温是布衣之好,即使桓温权倾朝野之时,他还是一如既往地拉着桓温喝酒。桓温怕他撒酒疯,每次最后都要跑到老婆那里避风头,谢奕就拉过桓温帐下的老兵一起喝,别人责怪谢奕这样做有失身份。谢奕说:``跑掉一个老兵,又拉来一个老兵,有什么好责怪的!''(在魏晋时期直呼别人老兵是很侮辱人的。)谢奕死得较早,有个聪明的女儿谢道韫,是有名的咏絮之才,还有个儿子谢玄是淝水之战中的名将,这些名人的性子应该是受谢安的熏陶,和谢奕区别很大。
\item
  \emph{醇酒}:度数高、味道厚的酒。成语``醇酒妇人''正是男子及时行乐的标志。老人家犯了罪罚喝酒,可见其罪尚轻和谢奕行事滑稽。冯梦龙《古今谭概》中有个笑话:李载仁,唐之后也,避乱江陵高季兴,署观察推官。性迂缓,不食猪肉。一日,将赴召,方上马,部曲相殴。载仁怒,命急于厨中取饼及猪肉,令相殴者对餐之。复戒曰:``如敢再犯,必于猪肉加之以酥!''不过还不真要说,无论是喝酒还是吃肉,多了真是一种可怕的惩罚。
\item
  \emph{太傅}:谢安,字安石,谢奕的弟弟,后任中书监、录尚书事,进位太保,死后追封太傅兼庐陵郡公,东晋时期最伟大的人物之一。小时候看过他的一本传记,其中淝水之战后他的无奈给我留下了深刻的感受,谢安的种种故事我们在《世说》中会陆续接触到,不用在这里展开。值得一提的是,谢安病重的时候回南京,是从西州路进城的。谢安死后,他的外甥兼朋友羊昙很悲痛,平时就不走西州路了,有一次酒喝醉了走到那里,下人告诉他,到西州门了。羊昙悲感不已,以马鞭打着门,歌咏曹植的诗歌``生存华屋处,零落归山丘'',恸哭而去。后来苏轼有首词《水龙吟》中唱到:``记取西湖西畔,正暮山好处,空翠烟霏。算诗人相得,如我与君稀。约他年、东还海道,愿谢公、雅志莫相违。西州路,不应回首,为我沾衣。''歌咏的就是这件事。南京想必还有西州路吧,曾经出差南京,只在一个莫名其妙的宾馆里洗了个所谓的温泉浴,外加开车
  兜了一圈南京城。哎,不去也罢,玄都观里桃千树,尽是刘郎去后栽,逛什么名胜古迹啊,不如看书!
\item
  \emph{著青布绔}:此句突兀,不过正是写小说的技巧之一。契河夫告诫一位作家朋友避免概括化和平常化:``我认为,对于自然的真正描写应该相当简略并与主题存在相关性。应该避免落人俗套的描写,\ldots{}\ldots{}在描写自然时,要抓住细节,而且要达到这样一种程度,即使闭上双眼,也仍能看到你所描写的场景。''
\item
  \emph{可念}:可怜。关汉卿《窦娥冤》``念窦娥身首不完全''。
\item
  \emph{阿奴}:晋代对亲近的人的第二人称称呼,这里是哥哥称呼弟弟。
\end{itemize}

上等人往往对下等人通过恶作剧的行动来达到自己的快乐,而谢安年纪很小的时候就体现出道德高尚者的怜悯之情------不要作弄弱者。

\section{1.34}\label{section-33}

\begin{quote}
谢太傅绝重褚公,常称``褚季野虽不言,而四时之气亦备。''
\end{quote}

\begin{itemize}
\tightlist
\item
  \emph{谢太傅}:谢安。
\item
  \emph{褚公}:褚裒,字季野,他的女儿是司马岳晋康帝的妃子,可惜康帝实在短命,只活了22年。褚裒死后赠侍中、太傅。他大谢安17岁,是谢安的堂姐夫。从《世说》后面的故事看,褚裒不愧是一个温和洒脱的人,不过实在不是军事家。公元349年,后赵石虎去世,征讨大都督褚裒带着三万人趁机去光复中原,其中的一支军队在代陂一战即败,全军败退京口。此时二十多万饱受少数民族奴役的汉人趁乱渡过黄河,恳求东晋接他们渡过长江,回到自己的国家。褚裒却只能退兵江南。之后没过多
  久,二十多万汉人就被追上的鲜卑人等赶尽杀绝。褚裒深以自己的无能忧愤发病,在京口看到很多人在哭,更加自责,47岁就病逝了。
\item
  \emph{常}:经常,也有人说通``尝'',曾经。
\item
  \emph{四时之气}:春夏秋冬,风雨阴晴。《晋书 - 外戚传 -
  褚裒传》说:桓彝认为``季野有皮里阳秋'',就是说他虽然口不臧否人物,可心里是有褒贬的(阳秋,指《春秋》,孔子表面上不评价人物,但在用词上隐藏着自己的爱憎褒贬)。这两者是一个意思,都是说褚裒深沉。我看过一本管理学的书,其中说,批评与事无益,没有人会因此改变,人永远只需要激励。
\end{itemize}

\section{1.35}\label{section-34}

\begin{quote}
刘尹在郡,临终绵惙,闻阁下祠神鼓舞,正色曰:``莫得淫祀!''外请杀车中牛祭神,真长答曰:``丘之祷久矣,勿复为烦!''
\end{quote}

\begin{itemize}
\tightlist
\item
  \emph{刘尹}:刘惔,字真长,汉家子弟,曾任侍中、丹阳尹等。惔应该通``淡'',淡则长久。刘惔是当时永和名士之代表,``清言冠世'',我们可以在后面陆续接触到他的清谈风采。丹阳尹相当于现在的北京市市长,三品官,因为刘惔36岁去世,所以官位没有进一步的发展。刘惔和桓温是连襟,都是晋明帝司马绍的女婿。刘惔做官以``居官无官官之事,处事无事事之心''著称,在东晋由于门阀政治的关系,如果行法家严峻事,就不能保持大局的稳定,一味进取内部必定先生祸乱,有见识的政治人
  物一般是``全欲德被江汉,耻以威刑率物''。
\item
  \emph{绵惙}:绵、惙都是软弱的意思,引申为病重弥留之际。
\item
  \emph{阁}:这里应该不是是指官衙,而是供神的阁楼。东晋的五斗米道(天师道)很是盛行,名士豪门也多信这一套。
\item
  \emph{祠神鼓舞}:击鼓舞蹈来祭祀神灵,替刘惔除病祷告。五斗米道除以符箓咒术为人治病,以祈禳斋醮为人谢罪除灾。即使现在的很多宗教、邪教都认为人生病是上天的惩罚和磨难,不用上医院看病的,而是要通过信仰神灵和祷告得到拯救。
\item
  \emph{淫祀}:滥行祭祀。《礼记 -
  曲礼》``非其所祭而祭之,名曰淫祀。淫祀无福'',
  祭祀不合礼制或祭祀不在国家祀典当中的神明叫淫祀。儒家对祭祀有比较严格的规定。看来刘惔虽然是清谈的领袖人物,但骨子里还是正统的,不认可民间迷信文
  化,但家里人和下属还搞那一套。
\item
  \emph{杀车中牛祭神}:晋代官僚一般坐牛车不坐马车,所以庾亮有匹好马有人劝他不要,还有的贵族见到马以为是老虎,非
  常恐惧。到了隋唐以后贵族们又开始用马车了。杀牛祭祀一般是贵族的祭祀典礼,《礼记
  - 玉藻》:君无故不杀牛,大夫无故不杀羊,士无故
  不杀犬豕。君子远庖厨。《孟子 -
  齐桓晋文之事》中说:臣闻之胡龁曰:王坐于堂上,有牵牛而过堂下者。王见之,曰:牛何之?对曰:将以衅钟。''王曰:舍之!吾不忍其觳觫,若无罪而就死地。对曰:然则废衅钟与?曰:何可废也?以羊易之。唐以后杀牛渐渐少了,唐玄宗下《禁屠杀马牛驴诏》说:自古见其生,不食其肉,资其力,必报其功。马牛驴皆能任重致远,济人使用,先有处分,不令宰杀。
\item
  \emph{丘之祷久矣}:《论语 -
  述而》孔子得了重病,子路请求允许向神祷告,孔子说:``丘之祷久矣''(我早就祷告过了),委婉拒绝了子路的请求。刘惔也很委婉和自信:我一生行为无愧于鬼神,没有需要忏悔的,我的所作所为本身就是对上天的祷告。刘惔深通老庄,不会把生死放在心上,不需要``终极关怀''的,这也算中国文化的一个特色,但能做到这样的人终究是少数,今后我们还得面对这个终极关怀的问题。
\end{itemize}

\section{1.36}\label{section-35}

\begin{quote}
谢公夫人教儿,问太傅:``那得初不见君教儿?''答曰:``我常自教儿。''
\end{quote}

\begin{itemize}
\tightlist
\item
  \emph{谢公夫人}:谢安的妻子刘氏,是上文中刘惔的妹妹,管丈夫是一把好手,经常教育自己的先生,正应了``成功的男人背后一定有位伟大的女性''的俗语,这次家庭辩论会谢夫人小负,算是很少有的。是啊,看看自己,和妻子讲过几次道理,哪次不是进退失据,郁闷填膺,落荒而走。受过现代教育的女性,一定会很好地反诘谢安。
\item
  \emph{教儿}:谢安的儿子谢瑶、谢琰。谢瑶早逝,谢琰带兵过于自信,死于孙恩之乱。谢安的几个侄子倒很有出息。此处的``教儿''可删去。
\item
  \emph{那得}:为什么。问渠那得清如许。
\item
  \emph{初}:从来,一点也(不)。
\item
  \emph{我常自教儿}:我常常以我的言行来告诉他们应当怎样做。这话说的很听,但恐怕不是这么回事。刘孝标在注中说,当年太尉刘寔刘子真,道德高洁、一尘不染,行为处处符合礼的要求,而他的两个儿子却因为贪污罪太出名了被处理,刘寔也因此多次被免官。有人对刘寔说:``君行高一世,而诸子不能遵。何不旦夕切磋,使知
  过而自改邪!''刘寔回答:``我的一言一行,他们是能够清清楚楚看到的,却不能够向我学习,难道教育他们几句就能够改变吗?''男人难免过于自信,言传身教,对孩子来说其实不可偏废,希望孩子自学成才,就像三岁的孩子,直接扔给他一本书,叫他自己学习,怎么可以做得到呢!再说,孩子可以受你的影响,也一样会受别人的影响,而且格调低下的东西才是相对吸引人的。等他长大后,虽然家长自己做得很好,但孩子已经形成自己的价值观,再回头很难了------教育是这一代对下一代的债务。
\end{itemize}

\section{1.37}\label{section-36}

\begin{quote}
晋简文为抚军时,所坐床上,尘不听拂,见鼠行迹,视以为佳。有参军见鼠白日行,以手板批杀之,抚军意色不说。门下起弹,教曰:``鼠被害,尚不能忘怀;今复以鼠损人,无乃不可乎?''
\end{quote}

\begin{itemize}
\tightlist
\item
  \emph{晋简文}:简文帝司马昱,字道万。他本来是琅琊王、会稽王,职务是抚军将军(三军分别为中军、镇军、抚军)、丞相等,当时权臣桓温以晋废帝司马奕不能经人道,而且让宠臣与自己的妃子乱搞借种,废了司马奕,另立司马昱。但司马昱在位不过一年时间就病逝了。司马昱有名士之风,继位后听命于桓温,死前下遗诏,叫桓温仿效诸葛亮和王导。这还算好,起初的遗诏是同意桓温``少子可辅者辅之,如不可,君自取之'',这话对忠臣说当然是收买人心,对桓温这样的枭雄说正投其下怀,这份遗诏被王坦之撕了。(这个故事流传出来估计是王家为了标榜自己的功劳。)司马昱风采不错,但当皇帝有些窝囊。正如后人评价李煜和宋徽宗那样:``做个名士真绝代,可怜薄命做君王。''
\item
  \emph{坐床}:胡床,从西域引进,一开始指折叠椅(马扎),后来发展成现在的椅子,所以用``坐''字。椅子常常是用来体现身份等级的工具,宋以前往往一家之长或军队首领才能坐椅子。像后来的皇帝登位和和尚当家,也常常叫坐床。
\item
  \emph{尘不听拂}:听,允许。《魏书 -
  世祖纪上》:``民相杀害,守牧以法平决,不听私辄报复。''《魏书 -
  明帝纪》注引《汉晋春秋》:``燕王拥兵南面,不听臣等入。''
\item
  \emph{参军};将军幕府的军事参谋,唐以前是参谋长,唐以后一般是低品级的参谋。
\item
  \emph{手板}:笏,古代下属和上司讲话,不带讲话稿,而是带块板子,上面可以记讲话要点,要是像现在这样,发言动辄上万言,得用毛笔写几天,带多少纸啊。笏是八仙过海中曹国舅的法宝,在这里变成了杀鼠利器。
\item
  \emph{说}:悦,高兴。司马昱不高兴,就开始责备那个不识名士情趣的下属,所以下文才说``以鼠损人''。
\item
  \emph{门下起弹}:门客,虽说贵族家养的帮闲人物一般为捧哏,但除了捧哏以外总还有当直臣谏人的义务,所以站起来批评司马昱了。本文主题是赞扬这个直臣,但居然没有留下名字,实在古怪,难道主题是赞扬司马昱的英明?就像隋唐之间的名臣裴矩,在隋而佞,在唐而忠,他曾经千方百计奉承杨广,后来却常常``犯颜直谏''李世民,后来司马光评论说:``君明臣直。裴矩在隋炀帝面前是个佞臣,在唐太宗面前忠于职守,不是裴矩性格有改变。君主不愿臣下提缺点,则忠臣化为佞臣;君主
  喜欢臣下提批评意见,则佞臣变化为直言忠臣。可见,君主是表率,好像华表,臣下是影随。根子还在君主身上。''以鼠损人,无乃不可乎:因为一只死老鼠而处分人,恐怕不可以吧?其中隐藏着一个典故:当年齐景公喜欢打鸟,派烛邹管养鸟,结果鸟却跑了。景公大怒,命令杀掉烛邹。晏子说:``烛邹有三条罪状,请让我一条一条地指出他的罪状,然后再杀掉他。''景公师出有名,欣然同意,于是把烛邹叫来。晏子说:``烛邹!你替我国国君主管养鸟却让鸟跑了,这是第
  一条罪状;使我国国君因为鸟的缘故杀人,这是第二条罪状;让诸侯听到了这件事,认为我国国君重视鸟却轻视人,这是第三条罪状。''晏子讲完,请求景公立即杀掉烛邹,景公却只能作罢。
\end{itemize}

\section{1.38}\label{section-37}

\begin{quote}
范宣年八岁,后园挑菜,误伤指,大啼。人问:``痛邪?''答曰:``非为痛,身体发肤,不敢毁伤,是以啼耳。''宣洁行廉约,韩豫章遗绢百匹,不受;减五十匹,复不受;如是减半,遂至一匹,既终不受。韩后与范同载,就车中裂二丈与范,云:``人宁可使妇无裈邪?''范笑而受之。
\end{quote}

\begin{itemize}
\tightlist
\item
  \emph{范宣}:字宣子,河南陈留人,后来住在江西,虽然家境贫寒,但很小的时候就通儒家经典,恪守礼教,后来名声在外,多次被朝廷应召,但是都不接受,成为东晋初的隐逸派领袖人物。
\item
  \emph{挑菜}:挖菜。
\item
  \emph{身体发肤,不敢毁伤}:《孝经》``身体发肤,受之父母,不敢毁伤,孝至始也。立身行道,扬名於后世,以显父母,孝之终也''。当时越国断发纹身,就被认为是
  蛮夷的做法,中原人就不需要理发师的,幸亏满族入侵,从此三百六十行多了一门行业。以前海外华侨一般从事三种行业:剃刀、剪刀、菜刀。
\item
  \emph{韩豫章}:韩伯,字康伯,玄学家,历任豫章太守、丹阳尹、吏部尚书等。儒家和道家同习《易经》,虽然范宣从不看老庄的文章,也从不入官门,但在《易》上和韩伯有共同爱好,所以交上了朋友。
\item
  \emph{遗绢}:送绢匹,多了是送钱,送少是做衣服。一匹三丈,给两丈就要撕了。送东西这样反复讨价还价、不厌其烦地送,也算是有情致的。
\item
  \emph{裈}:裤子。据说先秦以前的汉族着装像现在的苏格兰人,穿裙子,再在腿部绑上套布,叫做``胫衣'',后来渐渐使用满档裤,但内裤很晚才有。有关这方面的知识要请教方家。韩康伯这话也有点奇怪,因为古代朋友之间的家属一般来说不大见面的,除非关系非常好。在《世说》中我们可以屡次看到女性在内室偷窥丈夫或儿子的朋友,然后品评人物。我想可能范宣家穷,没有佣人,所以妻子被迫出来招呼客人,而她裸露的腿部给韩康伯留下了印象;范宣妻子的腿被别人看去了,所以尴尬地笑,尴尬地接受了必要的礼物。不过依我看,年轻的女性恐怕平时还是穿短裙子的好,因为它透露着美丽和健康的消息。
\end{itemize}

\section{1.39}\label{section-38}

\begin{quote}
王子敬病笃,道家上章,应首过,问子敬由来有何异同得失。子敬云:``不觉有余事,唯忆与郗家离婚。''
\end{quote}

\begin{itemize}
\tightlist
\item
  \emph{王子敬}:王献之,字子敬,王羲之的小儿子,大书法家,官至中书令,掌管皇家机要,近似于宰相了,我们上面见到的和峤、裴楷、庾亮、谢安都承担过这个显要
  的职务。后人也称呼王献之为``王令''。
  王献之做官的事迹我们大多不了解,可是他《鸭头丸帖》、《中秋帖》的摹本,练过两天字的人大都看过,的确是让人又羡又妒的作品。王献之续弦是一位公主,女儿是皇后。他40几岁就去世了,如果天假以年,我们必定能欣赏到他更多书法作品。东晋直到唐初,一般舆论认为他的书法造诣超过了父亲王羲之,后来唐太宗竭力褒扬王羲之而贬抑王献之,一些书法评论家才开始认为王献之的书法比不上王羲之。如果没有《兰亭序》,我也认为王献之更加伟大。
\item
  \emph{病笃}:首先是五石散,重金属中毒;其次是脚痛(痛风?)。因为他的书法作品中多次说:``昼夜十三四起,所去多,又风不差,脚更肿'',``奉承问,近雪寒,患面疼肿,脚中更急痛'',``脚重痛,不得转动。左脚又肿,疾候极是不佳''等,这些话我们现在还能看到,因为实在是书法出众。
\item
  \emph{道家上章,应首过}:五斗米道治病方式是``加施静室,使病者处其中思过。又使人为鬼吏,主为病者请祷。请祷之法,书病者姓名。说服罪之意,作三通:其一上之天,着山上;其一埋之地;其一沈之水,谓之三官手书。使病者家出五斗米以为常。''这种坦白自己的罪过,写下来向上天祷告除难消灾(上章首过)的精神疗法完全依赖病人求生的愿望和信念,自然是很不牢靠的。王献之又刚好没做过几件缺德事,更加没有效果了。
\item
  \emph{由来};向来;一向。
\item
  \emph{异同得失}:偏义复词,异和失,特殊的过失。
\item
  \emph{与郗家离婚}:王献之的婚姻故事比较曲折的。他小时候表现很突出,得到很多人的赏识,认为是王羲之几个儿子中最优秀的一个。他与表姐郗道茂从小青梅竹马,到了婚嫁年龄,王羲之就替王献之向大舅子郗昙求亲:``您家女孩有没有中意的对象?如果没有,就和我家小郎结婚,那是再好不过了。希望你考虑一下(中郎女颇有所向不?今时婚对,自不可复得。仆往意,君颇论不?大都此亦当在君耶。)婚后两人恩爱,可是这时候简文帝的女儿司马道福和桓温的儿子桓济离婚了,她一向
  喜欢王献之,就央求皇帝哥哥司马曜将自己改嫁给王献之。司马曜一向欣赏王献之,就一道诏书令王献之休妻再娶。王献之后来官越来越大,但心里还是牵挂着前妻,曾经写信说:``虽奉对积年,可以为尽日之欢。常苦不尽触额之畅。方欲与姊极当年之匹,以之偕老,岂谓乖别至此!诸怀怅塞实深,当复何由日夕见姊耶?俯仰悲咽,实无已已,惟当绝气耳!''(我和表姐生活多久都不会厌的。即使是年复一年地相对,也可以当作是一日之欢。那种额头触着额头的欢畅,只是遗憾不能再
  尽兴一点、更尽兴一点。正想着要和表姐成双成对,白头偕老,哪知道命运如此不顺,分离到这个地步!实在是伤心惆怅啊,什么时候才能白天晚上都见到表姐呢?我只能仰首低头悲叹呜咽,实在没有办法啊,要跟表姐见面,只能等到我断气罢了。)
\end{itemize}

王献之的爱情虽然真挚,但其他故事中说他还有一个女朋友叫桃叶,桃叶常常过河来和王献之幽会,王献之对桃叶往返于河流两岸很不放心,每次都亲自在渡口迎送,还作歌数首吟唱:``桃叶复桃叶,桃树连桃根。相怜两乐事,独使我殷勤。桃叶复桃叶,渡江不用楫。但渡无所苦,我自来迎接(我自迎接汝)。桃叶映红花,无风自婀娜。春花映何限,感郎独采我''。后来那个南京的渡口就叫桃叶渡。爱情这个东西实在奇怪,从生理上讲,男人很少能接连与多名女性接触,而女性却能够接纳不同的男性,但从心理上讲,女性比男性忠贞得多。我曾经说过,爱情本身无可指责,它超越了生活的无聊和卑微,它使我们保持了对世界。对存在的一定的向往。爱情,也许就是与时间和遗忘作斗争的一种仪式。

\section{1.40}\label{section-39}

\begin{quote}
殷仲堪既为荆州;值水俭,食常五碗盘,外无余肴;饭粒脱落盘席间,辄拾以啖之。虽欲率物,亦缘其性真素。每语子弟云:``勿以我受任方州,云我豁平昔时意,今吾处之不易。贫者,士之常,焉得登枝而捐其本!尔曹其存之!''
\end{quote}

\begin{itemize}
\tightlist
\item
  \emph{殷仲堪}:河南人,祖父殷融曾经当过吏部尚书。殷仲堪淝水之战时曾经在谢玄帐下当长史(幕僚长),后来很得晋孝武帝司马曜的赏识,授以东晋最为重要的荆州刺史岗位。殷仲堪和上面介绍过的韩伯同是当时玄学的领袖,对《道德经》、《周易》研究很有些成就。他父亲病了,就自学中医,成为有名的医生,据说他给父亲煎药,一边煎一边哭,最终哭瞎了一只眼睛。孝武帝死后,在朝廷的分化牵制的政策引导下,殷仲堪和桓温的儿子桓玄斗,兵败被杀,一般认为他文化水平太高,与军事政治多谋少断,故有此败。
\item
  \emph{水俭}:因水灾而年成不好。俭:歉收,旧时称青黄不接之时为``俭月'',荒年为``俭岁''。公元392年,殷仲堪任荆州刺史,394年以后,荆州地区多次水灾,人民疲敝不堪,399年,桓玄乘荆州水灾缺粮进攻,大胜。
\item
  \emph{五碗盘}:据说是一种成套食器,由一个托盘和放在其中的五只碗组成,应该就像现在的快餐盘,四菜一汤吧。
\item
  \emph{肴}:一般指熟肉食。
\item
  \emph{率物}:为他人做出表率
\item
  \emph{真素}:真诚无饰;质朴。
\item
  \emph{方州}:指州郡长官。《资治通鉴 -
  宋顺帝升平元年》:``诉以其私用人为方州。''
  胡三省注:``古者八州八伯,谓之方伯,后世遂以州刺史为方州。''王安石诗:``他年佐方州,说将尚不纳。''方州还有的意思如大地、帝都和州郡等,看具体语境。
\item
  \emph{豁}:消除、放弃,如豁出性命。豁也可能是害的意思,违背。
\item
  \emph{平昔时意}:平日往昔的生活习惯。
\item
  \emph{贫者,士之常}:刘向《说苑 -
  杂言》古代高士荣启期``衣鹿皮裘,鼓瑟而歌'',孔子见之,问其何乐,他回答说:``天生万物,唯人为贵,
  吾既已得为人,是一乐也。人以男为贵,吾既已得为男,是为二乐也。人生不免襁褓,吾年已九十五,是三乐也。夫贫者士之常也,死者民之终也。处常待终,当何忧乎?''贫穷疾病,乃是人生常态;死亡,乃是人生终结。能够处于常态而等待死亡,有何不乐?
\item
  \emph{登枝而捐其本}:登上高枝就抛弃树干,身居高位忘掉做人的根本。
\item
  \emph{曹}:等;辈:尔曹(你等,你辈);吾曹(我等,我辈)。
\item
  \emph{其存}:其,表命令、劝告的语气副词,``还是、要'';存,存想,记得。
\end{itemize}

殷仲堪是个非常有特点的人,《晋书》中说他不是节约,而是吝啬;爱行小惠收买下属,而没有执政思路;精通老庄,而笃信天师道。``及在州,纲目不举,而好行小惠\ldots{}\ldots{}仲堪少奉天师道,又精心事神,不吝财贿,而怠行仁义,啬于周急。及玄来攻,犹勤请祷。然善取人情,病者自为诊脉分药,而用计倚伏烦密,少于鉴略,以至于败'',后人评论说,打仗的时候还向天师祈祷保佑,那么上天不就是一个受贿之宵小的吗?``异端杀身,故学者当先明器识。''他爱行小惠,不就像现在的一些政府官员,照顾亲信,过年过节也表演一些爱心慈善家的活剧,但在行使职务的时候,你能见到他的哪些为民的举措?总之,殷仲堪是个读书人,一个处于有用无用之间的知识分子,很多人可以在他身上找到自己的影子。

\section{1.41}\label{section-40}

\begin{quote}
初,桓南郡、杨广共说殷荆州,宜夺殷觊南蛮以自树。觊亦即晓其旨。尝因行散,率尔去下舍,便不复还,内外无预知者。意色萧然,远同斗生之无愠。时论以此多之。
\end{quote}

\begin{itemize}
\tightlist
\item
  \emph{桓南郡}:桓玄,字敬道,他虽然是桓温的小儿子,但那时候讲究相术,估计由于长得合符大贵之相,母亲生他的时候又有过感星而孕的神话故事,所以5岁时继承了桓温死后的爵位,封南郡公。桓玄七岁守丧期满,原桓温属下的文武佐吏聚会,桓玄叔父、荆州刺史桓冲抚摸着桓玄的头告诉他:``这些人都是你家的故吏!''桓玄应声痛哭,哀动左右,众人无不惊异。想来桓玄从小就被桓温的事迹所洗脑,以英雄自诩,所以依靠桓温的遗泽,一生跌宕起伏,风云变幻,在东晋末期门阀内斗
  中是第一男主角,最后居然当了两年皇帝玩玩,兵败身死,成全了寒族的刘裕。桓玄性格的缺点就是在小事上过于计较来体现自己的小聪明,生活上过于享受来体现自己的豪门身份,做事随心所欲,骄横跋扈,是典型的纨绔子弟,和父亲桓温不可以道里计。
\item
  \emph{杨广}:东汉太尉杨震(天知地知、你知我知的主人公)之后,陕西华阴人,他的父亲杨亮南渡较晚,又有在少数民族王朝当过官的历史,所以受其他豪门排挤。而杨广、杨佺期兄弟又以为自己的家世比当时的王谢这些要显赫得多,自己武功又好,心中多有不平。殷仲堪为荆州刺史时,因为他是名士,不通军事,所以把军旅之事交给杨广、杨佺期兄弟掌握。
  殷荆州:殷仲堪。
\item
  \emph{殷觊}:字伯通,殷仲堪的堂兄,当时任南蛮校尉,掌管南蛮地区。殷仲堪想邀他起兵谋取更大的势力,殷觊不同意;杨广劝仲堪杀了殷觊,殷仲堪不同意。殷觊后忧郁终。
\item
  \emph{树}:树立;建立。殷觊自动离任后,杨广接替其职务。
\item
  \emph{因行散}:趁着散步。服五石散后要走路以便散发。
\item
  \emph{率尔}:急促或随意。《论语 - 先进》子路率尔而对;《晋书 -
  袁宏传》:`` 谢尚时镇牛渚,秋夜乘月,率尔与左右微服泛江 。''
\item
  \emph{去下舍}:离开自己住的房屋。
\item
  \emph{斗生}:指春秋时楚国令尹子文,子文姓斗,名谷於菟,字子文。於菟指老虎,鲁迅诗``知否兴风狂啸者,回眸时看小於菟''。
  《论语 -
  公冶长》:``令尹子文三次做令尹,不显露喜色;三次被罢官,不显露愠色。''(每次交职)一定要将自己这任的政令全部告知新来
  的令尹。生:先生的简称。《史记索隐》:``自汉以来,儒者皆号生,亦先生者省字呼之耳。''
\item
  \emph{多}:称赞。《韩非子 - 五蠹》``以其犯禁也罪之,而多其有勇也''。
\end{itemize}

当不当官都无所谓,是因为殷觊家里薄有积蓄吧,``人如果能一世沉醉,爱爱,喝喝酒,打打架,谁不愿一早起来?谁不愿一晚就睡下?''工作以来,总有几个志得意满的领导在耳边聒噪``有为才有位,有位更有为''之类不着边际的文字游戏,算了吧,自古``黄钟弃毁,瓦釜雷鸣'',生于尘土,心在星空。

\section{1.42}\label{section-41}

\begin{quote}
王仆射在江州,为殷、桓所逐,奔窜豫章,存亡未测。王绥在都,既忧戚在貌,居处饮食,每事有降。时人谓为试守孝子。
\end{quote}

\begin{itemize}
\tightlist
\item
  \emph{王仆射}:王愉,字茂和,在桓玄朝当上尚书左仆射(尚书、中书、门下三省虽然同为三个部门,但尚书省总领百官,更像宰相。仆、射都是主管、掌管的意思,仆射就是副宰相。唐以后因为李世民当过尚书令,所以尚书省只设仆射。至于左右谁更尊贵一些,情况比较复杂,一般来说,官位以左为贵,吃饭、凶事、兵事等以右为尊,不过我以为这些是糟粕,斤斤计较所谓的官场礼仪皆碌碌之辈)。王愉是太原王氏,不是琅琊王氏,性急的王述之孙,与谢安同时的王坦之之子。当时出任江
  州(大约相当于江西省)刺史、都督江州及豫州之四郡军事。桓玄灭亡后,因为王愉、王绥世家骄横气到处散发,曾经看不起``老兵''刘裕,被这位草莽英雄宋武帝
  以谋反罪诛杀。
\item
  \emph{殷、桓}:殷仲堪、桓玄。皇族、世家之间打打杀杀,朝为兄弟、暮为仇雠家常便饭,解释这次事件的前因后果麻烦,总是为了一块骨头一群恶狗撕咬,现在我们不用分什么正义非正义的,总之王愉在这次军事行动中逃到临川,被妹夫桓玄俘虏。
\item
  \emph{王绥}:字彦猷,王愉的儿子,在桓玄任太尉时,他任太尉右长史,后来位至中书令、荆州刺史,王愉父子又和桓玄关系亲密得不得了。
\item
  \emph{试守孝子}:试守,试用期。譬如现在官员正式任命前,先主持其事以试其能,当时称为试守,这里等于说孝子实习。王绥在父亲存亡不知时就服丧,所以人们模仿职官称谓,称他为试守孝子。
\end{itemize}

应该说实习孝子的绰号总带有点调侃嘲讽性质,刘义庆把这条放在德行里面,难道是赞扬?刘义庆反反复复在德行中谈论孝道,故事迭出,不少是匪夷所思的,仿佛晋代人特别孝一点。但完全不是这么回事,晋代士人往往只知有家,不知有国,王绥对父亲孝得未雨绸缪,东晋国灭却不知其耻,对篡位的舅舅一味奉承,这是当时的风气,求忠臣必出孝子之门成了一句笑话。我们也可从中体会儒家学说的空疏,孝和忠并没有什么必然联系,现在有些市哗众取宠,说什么考察干部要考察他是
否孝顺,也算是我国一大滑稽事,让我们一起努力建设有中国特色的社会主义事业吧。王绥这样荒唐的举动,也许包含着恐怕追究父亲兵败的责任,要朝廷早一点给
王愉一个烈士称号,如果有这样的心机,倒要高看他一眼了。

\section{1.43}\label{section-42}

\begin{quote}
桓南郡既破殷荆州,收殷将佐十许人,咨议罗企生亦在焉。桓素待企生厚,将有所戮,先遣人语云:``若谢我,当释罪。''企生答曰:``为殷荆州吏,今荆州奔亡,存亡未判,我何颜谢桓公!''既出市,桓又遣人问欲何言。答曰:``昔晋文王杀嵇康,而嵇绍为晋忠臣;从公乞一弟以养老母。''桓亦如言宥之。桓先曾以一羔裘与企生母胡;胡时在豫章,企生问至,即日焚裘。
\end{quote}

\begin{itemize}
\tightlist
\item
  \emph{桓南郡既破殷荆州}:前面介绍过,公元399年,桓玄趁荆州水灾大荒进攻殷仲堪,获胜,同时击杀雍州刺史杨佺期。公元400年,东晋朝廷承认事实,任命桓玄为都督荆、司、雍、秦、梁、益、宁、江八州及扬、豫八郡诸军事、后将军、荆、江二州刺史、假节,从此桓玄尽占长江中游一带。
\item
  \emph{收}:收捕,俘虏。
\item
  \emph{罗企生}:字宗伯,江西南昌人,得殷仲堪赏识,任咨议参军,掌管谋划。殷仲堪败走,文武官员没人跟从,只有罗企生随从,但被弟弟用计拖住。
\item
  \emph{谢我}:向我谢罪、认错。
\item
  \emph{市}:闹市,刑场。因为古代多在闹市行刑,以为警示。《晋书 -
  嵇康传》``康将刑东市,太学生三千人请以为师,弗许。''
\item
  \emph{嵇绍为晋忠臣}:嵇康被杀害后,儿子嵇绍才10岁,一直待在家中。后来山涛向司马炎举荐,嵇绍最后官至侍中(相当于三公了),名望很高。公元304
  年晋惠帝亲征成都王司马颖,败于荡阴,晋惠帝的侍卫全被箭射跑了,居然是嵇绍一人为保护惠帝中箭阵亡,鲜血溅在惠帝的衣服上。后来惠帝说,不要洗去衣服上的血,那是嵇侍中的血啊!西晋时候像嵇绍这样的忠臣算是个奇迹,以弱智著称的晋惠帝也算是个有温情的人。
\item
  \emph{一弟}:罗遵生。罗企生曾经对他说:``今日之事,我必死之。汝等奉养不失子道,一门之中有忠与孝,亦复何恨!''后来罗遵生也按照要求侍奉母亲。刘裕杀桓玄后赐罗家书以``一门忠孝''。罗企生引述嵇绍事,是拍桓玄的马屁,比之为晋武帝司马炎,同时希望桓玄搞感情投资,不要斩尽杀绝。《晋书》中还记载罗企生对桓玄的怒斥:``(你桓玄)升坛盟誓,口血未干,而生奸计。(我罗企生)自伤力劣,不能翦灭凶逆,恨死晚也。''形象就更高大了,但和这里委婉的说法矛盾,似不确,所以刘义庆没有采用。《晋书》虽
  然是唐太宗命令修撰的,但取材似乎没有仔细推敲,后人多有诟病。
\item
  \emph{问}:(罗企生被害的)消息。《晋书 -
  陆机传》``既而窃寓京师,久无家问''。
\item
  \emph{即日}:当日,立刻。罗母胡氏立刻烧裘表明与桓玄一刀两断的决心,并不领桓玄不屠一门的情。桓玄终究是纨绔子弟,气量窄小,其实罗企生不足以为殷仲堪报仇或捣蛋,放他一马又当如何,那样又能占道德的制高点,故由此可见桓玄不能成事。
\end{itemize}

《世说》看似散乱,但各个故事之间有脉络可寻,它的写法类似《儒林外史》的连环结构,人物、故事之间往往有承上启下的作用,一般来说,下一个故事和上一个故事有一定的联系,我们已经读了40多个故事,应该能体会到这一点。

\section{1.44}\label{section-43}

\begin{quote}
王恭从会稽还,王大看之。见其坐六尺簟,因语恭:``卿东来,故应有此物,可以一领及我。''恭无言。大去后,即举所坐者送之。既无余席,便坐荐上。后大闻之甚惊,曰:``吾本谓卿多,故求耳。''对曰:``丈人不悉恭,恭作人无长物。''
\end{quote}

\begin{itemize}
\tightlist
\item
  \emph{王恭}:字孝伯,太原王氏,他的姑姑是东晋哀帝司马丕的皇后,妹妹是晋孝武帝司马曜的皇后。他在当著作郎是说:``仕宦不为宰相,才志何足以骋!''所以称病辞职,后来历任中书令、青州、兖州刺史等。王恭受到谢安的赏识,曾评价说:``王恭人地可以为将来伯舅'',伯舅的意思是一方诸侯,王恭后来是两州刺史,也许有谢安讲话的分量在里面。周朝时候,对分封各地的诸侯,周天子称同姓为``伯父'',``叔父'',异姓为``伯舅'',``叔舅''。
  王恭说过很有名的话,在《世说》后面会提到:``名士不必须奇才,但使常得无事,痛饮酒,熟读离骚,便可称名士。''后来他以清君侧的名义去打同宗的王愉,部下、淝水之战的英雄刘牢之反水,王恭兵败被杀,死时也是一派名士风范,``临刑犹诵佛经,自理须鬓,无惧容。''
\item
  \emph{从会稽还}:王恭父亲王蕴当过都督浙江东五郡、镇军将军、会稽内史的职务。会稽相当于现在的绍兴、杭州、宁波的部分地区。王恭和父亲从会稽回来。
\item
  \emph{王大}:王忱,小名佛大,也称阿大,王坦之的儿子,王恭的同族叔父辈,官至荆州刺史。为人放旷嗜酒,连月不醒,常常裸体出席各种场合。王忱和王恭是亲戚加朋友,并称一时,相互欣赏,后来翻脸。王忱的兄弟王国宝、王愉是王恭的对头仇人。王坦之家非常奢侈,王国宝有``后房伎妾以百数,天下珍玩充满其室'',而王蕴家风却比较俭朴,所以闹出误会。
\item
  \emph{簟}:竹席。
\item
  \emph{卿}:六朝时,在对称中,尊辈称晚辈,或同辈熟人间的亲热称呼。
\item
  \emph{故应有此物,可以一领及我}:从老头子的地盘旅游回来,总得带点土特产送送亲戚朋友,我主动向你要,也算是看得起你。官场迎来送往,常常是拿公帑当礼物送来送去,此是世界古往今来之惯例,倒不是中国特色。可以:是两个词,``可''是可以,``以''是拿。
\item
  \emph{荐}:草席。
\item
  \emph{大闻之甚惊}:王忱听说了很吃惊。为什么是听说,按道理是第二次见面,王忱看到草席才对,也许是为了行文简便,也许是有捧哏出场。``甚惊''用得挺好,王恭回来不带土特产送人,有违``常理''。龙应台写过一篇文章,说她小时候住在台湾农村,当邻家孩子送来一篮自家树种出的枣子时,她母亲会将枣子收下,然后一定在那竹篮里放回一点东西,几颗芒果、一把蔬菜。家里什么都没有时,她一定将篮子填满白米,让邻家孩子带回。问她为什么,她说,``不能让送礼的人空手走开。''
\item
  \emph{丈人}:古时晚辈对长辈的尊称。《论语》中有``荷蓧丈人''章节。
\item
  \emph{作人无长物}:平生没有多余的东西。在现实生活中,往往只有摒弃物质的羁绊,才能在精神上获得较为广阔的自由。王恭有很好的志趣。据说他读《左传》至``奉王命讨不庭'',每辍卷而叹,他后来屡次兴兵打仗,也许不是为了权力和地盘,真的是为了国家和理想,这就超越了东晋时候大多数的门阀世勋。
\end{itemize}

\section{1.45}\label{section-44}

\begin{quote}
吴郡陈遗,家至孝。母好食铛底焦饭,遗作郡主簿,恒装一囊,每煮食,辄贮录焦饭,归以遗母。后值孙恩贼出吴郡,袁府君即日便征。遗已聚敛得数斗焦饭,未展归家,遂带以从军。战于沪渎,败,军人溃散,逃走山泽,皆多饥死,遗独以焦饭得活。时人以为纯孝之报也。
\end{quote}

\begin{itemize}
\tightlist
\item
  \emph{吴郡}:江苏、浙北一代,治所苏州。
\item
  \emph{家至孝}:可能少``贫''字,家贫至孝。
\item
  \emph{焦饭}:锅巴比饭香,,而且不容易变质。
\item
  \emph{孙恩}:东晋末五斗米道道师,其叔父孙泰据说有法术,谋反被揭发后处死,孙恩继承其衣钵起兵革命,一时云集响应,连妇女也把婴儿扔在水里,祝告说:``孩子先登仙界,我随后就来。''跟随孙大仙人造反。孙恩不费举手之劳打下会稽,踌躇满志地说:``孤当与诸君朝服而至建康。''后来听说大将刘牢之来了,他又说:``就算我只割据浙东这块地方,总也能做个勾践!''又过了几天,听说刘牢之已经带着军队渡江,他又放低了目标,说:``就算逃走,也没什么丢人的!''他曾经打
  败了谢家当时的军事首领、谢安的儿子谢琰,斩杀王羲之的儿子、会稽内史王凝之等。孙恩兵败投水自杀,信徒们认为他尸解当了水仙,继续跟随孙恩的妹夫卢循追求现实的``天国''。之后,道教发生了重大变化,后来道教徒很少从事革命事业,造反大多是民间佛教徒的兼职了。
\item
  \emph{袁府君}:袁山松,又名袁崧,当时任吴国内史(诸侯国太守),音乐家、史学家,399年于沪渎(吴淞口)战死。上海附近不知道有什么高山大泽,能逃的一条性命。
\item
  \emph{未展}:``展''很难理解,疑为误字,大约总是还来不及的意思。
\item
  \emph{遗独以焦饭得活}:吃独食?还不被人砍了?可见当时民风之厚淳。
\end{itemize}

后来的《南史》和一本《孝子传》还有续集:``母昼夜泣涕,目为失明,耳无所闻。遗还入户,再拜号咽,母豁然即明。''可谓是母慈子孝,更感人了,不过哭瞎眼听说过,不知道还带哭聋耳的,大约也是``家兄塞北亡''吧,``对仗亲切耳''!不知大家注意到没有,读到现在,讲``德行''中的主人公,全是男性,孝顺的对象当然也有女性,但只是孩子获利的工具,让人很不舒服。报应之说,不知起于何时,为了让``报应''逻辑自圆,有编出了现世报、死后报、来世报、多世报、后人报
等林林种种,陈遗实实在在的现世报可能是生活中极为罕见的个例,庆幸啊。余嘉锡先生的笺疏中有一段认真的评论,``因其纯孝足贯神明,不以微贱而遗之也'',
老先生实在有点迂,幸亏1955年就去世了,不然活过``大革命''时期,不免成了伪君子。

\section{1.46}\label{section-45}

\begin{quote}
孔仆射为孝武侍中,豫蒙眷接。烈宗山陵,孔时为太常,形素羸瘦,著重服,竟日涕泗流涟,见者以为真孝子。
\end{quote}

\begin{itemize}
\tightlist
\item
  \emph{孔仆射}:孔安国,327---408年,会稽人,不是西汉那个献尚书的孔安国。仆射、侍中前面都介绍过了,晋代侍中相当于现在的中央办公厅主任,可能权力还要大一点,极为尊贵和清要。
\item
  \emph{晋孝武帝}:司马曜,362~396年,母亲李陵容是个黑奴,后来司马曜当皇帝后仍尊为皇太妃,故称孝;因为打胜了淝水之战,称为武也在情理之中。司马曜好酒,``醒日既少'',据说某次酒后和一个贵妃开玩笑,宣称自己以后不再宠信近三十岁的她,该贵妃居然杀了他。
\item
  \emph{豫蒙眷接}:豫,喜悦;幸福;眷接:恩宠和接待。荣幸地受到恩宠礼遇。
\item
  \emph{烈宗}:晋孝武帝庙号,即死后立室奉祀时起的名号。
\item
  \emph{山陵}:帝王的坟墓,这里指山陵崩,死亡。
\item
  \emph{太常}:主管祭祀社稷、宗庙和朝会、丧葬等礼仪,并主管皇帝的寝庙园陵及其所在的地区,祭祀时充当主祭人皇帝的助手。太常事重职尊,其位列于诸卿之首。这
  里连续描写了孔安国的三个职务,有点啰唆,但充分表示孔安国的的确确、完完整整是皇家的亲信,就因为对皇帝的``孝'',尽管孔安国年龄可以当司马曜的祖、父
  辈。
\item
  \emph{羸瘦}:羸就是瘦,两同义之词叠用,是汉语的一大特点。
\item
  \emph{重服}:孝服中之重者,即父母丧时所穿的孝服,用极粗的生麻布为丧服,不缝衣旁及下边,要穿三年。
\item
  \emph{竟日}:整日。苏轼:春风自恨无情水,吹得东流竟日西。
\item
  \emph{涕泗}:自目曰涕,自鼻曰泗。这里应该是偏义复词,眼泪。
\item
  \emph{真孝子}:《世说》有些地方比较奇怪,70岁的老头子被称为35岁死者的孝子,现代人看来是在开玩笑,带有讽刺的意味,刘义庆却表现得一本正经,可能是时代变迁的关系吧。
\end{itemize}

\section{1.47}\label{section-46}

\begin{quote}
吴道助、附子兄弟居在丹阳郡。后遭母童夫人艰,朝夕哭临,及思至,宾客吊省,号踊哀绝,路人为之落泪。韩康伯时为丹阳尹,母殷在郡,每闻二吴之
哭,辄为凄恻。语康伯曰:``汝若为选官,当好料理此人。''康伯亦甚相知。韩后果为吏部尚书。大吴不免哀制,小吴遂大贵达。
\end{quote}

\begin{itemize}
\tightlist
\item
  \emph{吴道助、附子}:吴坦之,小名道助;吴隐之,小名附子,兄弟俩是河南濮阳人,祖上有了叫吴质的,曾经出过风头,史书上说,有一次曹操出征,曹丕、曹植送行。
  曹植出口成章,口诵华美的辞章赞美、预祝曹操之功德,而曹丕是慢热型,急就诗赋相形见绌,就紧张了。吴质对曹丕说:``你什么都不要说,只管哭泣就行了。''
  曹丕听了吴质的话,哭的非常伤心。曹操于是认为曹植有点华而不实,不如曹丕诚实孝顺。兄弟俩出身没落世家,当时可能都担任功曹(管组织人事的),应该说弄点钱和礼物都不是大的问题,但这两兄弟一向以清廉著称。我们应该读过一个故事,吴隐之当广州刺史的时候,广州石门有口泉叫``贪泉''(山东还有个``盗泉'',真是使英雄豪杰能纵横天下的两口好泉),据说不管谁喝了这泉水,都会变得贪得无厌。吴隐之说:``没有贪婪的欲望,心就不会乱。越过五岭丧失廉洁,我知道其
  中原因的'',他特意来到贪泉,掬水而饮,并赋诗为志:``古人云此水,一歃怀千金。试使夷齐饮,终当不易心''。意思是:人们都说喝了这泉水,就会贪财爱宝,假若让伯夷叔齐那样品行高洁的人喝了,我想终究不会改变那颗廉洁的本心。后来司马曜表扬说:``吴隐之在家孝敬父母严谨不怠,在官清廉之操凛如风霜,这确实是一个人立身处世难于做到的,而这正是君子的美德。''于是赏钱升官。吴隐之官运很好,但据史书上说,他始终保持清廉的作风,家里穷,有时候一天只能吃一餐
  (俸禄常常救济穷亲戚);有一次他吃咸酸菜,感到味道美,便抛掉了,怕以后上瘾(欧阳修好像也有类似一出,是喝稀饭不愿意吃大餐);吴隐之寒冬读书,常身
  披棉被御寒。
\end{itemize}

要在皇家兄弟中出类拔萃,常常要出奇招的,曹丕以眼泪胜出,康熙以出过天痘胜出,杨广就和杨勇、李治和李承乾比谁孝胜出,咸丰和奕訢比谁箭法差胜出,都是一时之佳话。
- \emph{艰}:去世。前面解释过了吧。丁忧、丁艰。 -
\emph{哭临}:以痛哭流涕向死者悲歌致哀的悼念方式,现多见于农村。《史记 -
孝文本纪》:``毋发民男女哭临宫殿。宫殿中当临者,皆以旦夕各
十五举声;礼毕罢,非旦夕临时,禁勿得擅哭。''看来那是葬礼已经结束,所以这时哭已经有时间段了,早晚各一次。因为根据书上说``孝子自父母始死至殡,哭不绝声;殡后思及父母即哭,不择时间。祭日结束后改为朝夕各一次哭奠。''
-
\emph{及思至}:这里解释不一,有人说``思至''当做``葬''字,有人说思至应为``周忌'',就是做``七七''的意思,有人说思至就是思念至深的意思。
-
\emph{号踊}:号哭跳跃,一种哭的方法,想来和非洲原始部落的舞蹈差不多。《礼记
-
丧服小记》:奔母之丧,不括发,袒于堂上,降踊,袭免于东方,绖即位成踊。《仪礼注疏》:凡冯尸,兴必踊。
- \emph{韩康伯}:韩伯,韩豫章等,见1.38。 -
\emph{母殷}:韩伯的母亲是殷浩的妹妹。殷浩下面会多次出场。 -
\emph{选官}:主管铨选的官。 - \emph{料理}:照顾。 - \emph{相知}:交好。
-
\emph{大吴不免哀制}:吴坦之经不起丧亲的悲痛而死。宗躬《孝子传》``吴坦之,隐之兄也。母葬夕,设九饭祭,坦之每临一祭,辄号痛断绝,至七祭,吐血而死。''九饭祭不知其内容,估计失传了。这个葬礼还有故事,吴家贫寒,没人奏哀乐,兄弟俩每当哭丧的时候,总有一对鹤飞来伴舞鸣叫,在周年祭祀之际,又有一群大雁聚在一起,当时人们都认为是孝心所感。不过``子不语怪力乱神'',刘义庆对这些神话故事总还有点免疫力。

\section{\texorpdfstring{``德行''小结}{德行小结}}

《世说》的体例,受到《论语》的影响,他的前四章就是所谓的孔门``四科十哲'',东汉后批评人物,大约就是根据这四科来的,德行就是其中第一科。孟子曾经说:``人之所以异于禽兽者几希'',人与禽兽的区别,就是道德,就是仁义。人要摆脱动物的属性,并不是劳动,而是对待世界的态度。康德说过:``有两种东西,我对它们的思考越是深沉和持久,对它们日久弥新和不断增长之魅力以及崇敬之情就越充实心灵,这就是------我头上的星空和内心的道德定律。''星空灿烂,是对宇宙的敬畏;道德神圣,是对人类的信心。
尽管道德有其时代属性,但其基本价值却总是稳定的。按《圣经》的说法,就是七德:谦卑,温纯,善施,贞洁,适度,热心,慷概。按孟子的说法,就是``恻隐之心''、``羞恶之心''、``辞让之心''和``是非之心''。按现在的说法,知识分子是``社会的良心'',他除了专
业知识外,必须超越私利,深切同情世界、国家、社会的一切公共利害之事。尽管在具体的德行践行上,可能存在不同的做法,譬如有人积极用世,有人认真避世;有人孝而不免哀制,有人放旷于礼法,但其正如沧浪之水,清浊自用,他们心灵相同,惺惺相惜。
《世说》中的德行内容,大约都是以孔子门下以德行
著称的学生为范本。颜回``一箪食、一瓢饮、在陋巷,不改其乐'',
闵损对后母很孝顺,父亲要休妻,他劝说``母在一子寒,母去三子单''(后母又生了两个孩子),并且视官职、金钱``如坛土矣'',
冉耕尊师重道,冉雍``通则一天下,穷则独立贵名,天不能死,地不能埋,桀跖之世不能污'',有南面而为诸侯大宰的能力。这一些关于道德的内容,大体就反映在这一章节中。当然,颜回的木讷和善于吹捧,闵损的受虐倾向,冉耕的认真拘泥,也一样可以从《世说》``德行''章中找到他们的影子,为《世说》平添了更让人推敲的趣味。是啊,我们站在高处,我们的内心更加精细、复杂,也更加迷茫。但是,我在解释《世说》的时候,也是摆脱尘世杂念的美好时刻,常常处以微笑和宁静之中------我伸手,抓住了两个世界。是为记。

\chapter{二、言语}

\section{2.1}\label{section-47}

\begin{quote}
边文礼见袁奉高,失次序。奉高曰:``昔尧聘许由,面无怍色。先生何为颠倒衣裳?''文礼答曰:``明府初临,尧德未彰,是以贱民颠倒衣裳耳!''
\end{quote}

\begin{itemize}
\tightlist
\item
  \emph{边文礼}:边让,字文礼,河南陈留郡人,曾任九江太守。边让从小以才华、风采著称,蔡邕曾给大将军何进写信《与何进书荐边让》,认为边让``天授逸才,聪明贤智,\ldots{}\ldots{}诗、书、易、礼先通,\ldots{}\ldots{}使让生于先代,在唐、虞则、元凯之比,当仲尼则颜、冉之亚,岂徒世俗之凡偶兼浑,是非讲论而已哉!''天下大乱后避祸兖州,因为言语中多有对兖州大寨主曹操的讽刺,被人告发,``操闻而杀之,并其妻子''。曹操灭政敌满门的描述不绝于书,的确是伟大人物,所谓``唯才是举''不过
  是一种发家后故作姿态和狂妄的自信。曹操杀了边让之后,引发了兖州士族的巨大反弹,兖州叛乱,曹操弄得半死,侥幸身免。
\item
  \emph{袁奉高}:袁阆。在1.3等处跑过龙套,有名望,但不算第一流的人物。从文中看,袁阆大概当过年轻时候边让的父母官(陈留郡大守?)。
\item
  \emph{失次序}:举止失措。所以用``颠倒衣裳''来代替,就是上衣和下衣穿错了。前面说过,古人在早先没有裤子的,衣服分为到膝盖的下衣和半身的上衣,上衣在外,下衣在里,一时匆忙容易穿反。《诗经
  -
  东方未明》:``东方未明,颠倒衣裳。颠之倒之,自公召之。''不知道边让具体举止不当的情况,估计没有严格遵守尊卑礼节,但边让``名士自风流'',
  袁阆责怪,未免量浅,只是问的还算文雅。
\item
  \emph{尧聘许由}:儒家、道家的传说。虽然各家说法有点出入,不过大体情况是这样:尧原始部落的大酋长当腻味了,一次路上碰到农民许由,头脑发热,一定要把位置让给他,``日月出矣,而烛火不息,其於光也,不亦难乎?时雨降矣,而犹浸灌,其於泽也,不亦劳乎?''(许由是太阳,尧是蜡烛;许由是天雨,尧是挑水工,一定要让位给太阳和天雨。)在古代传说中,尧这个酋长很不尽职,当时洪水已经泛滥,但尧精力不济、办法不多,怕别人说闲话,多次要求内退。不过这番作秀,野人们都比较配合,没有上演燕王哙与相国子之的大闹剧。
\item
  \emph{面无怍色}:没有露出羞愧、吃惊等脸色,就是说许由应对落落大方。据庄子讲,许由听说要得到这个大烧饼,说,我是小鸟,不过栖息一根枝条,我是一只小老鼠,不过喝河里的一口水,还是算了吧。还有说法是许由当即跑路,洗耳朵去了,因为尧的请求太肮脏。
\item
  \emph{明府}:对太守的尊称,高明的府君,前面解释过了。
\item
  \emph{尧德未彰}:你又不是尧,还没有表现出尧的风范、德行,何必要求我像许由呢!至少在宋以前,官员把自己比作古代的帝王,并不犯忌。曹操等人就多次把自己比作刘邦,把别人比作张良,互相吹捧,这样的例子我们在《世说》中也可以看到一些。如果是文字狱时期,大家互相这么一说,就是灭门之祸。是啊,何必对别人要求太高啊,有些群众说中央领导是好人,就是基层官吏不行,中央也以为自己是好的,下发多少个文件提这提那的要求,一脸正气,权为民所用、情为民所系、利为民所谋的样子。据我所知------``觑多时认得,险气破我胸膛:你身须姓刘,你妻须姓吕,只道刘三谁肯把你揪扯住,白甚么改了姓、更了名、唤做汉高祖。''
\end{itemize}

\section{2.2}\label{section-48}

\begin{quote}
徐孺子年九岁,尝月下戏。人语之曰:``若令月中无物,当极明邪?''徐曰:``不然,譬如人眼中有瞳子,无此必不明。''
\end{quote}

\begin{itemize}
\tightlist
\item
  \emph{徐孺子}:即徐稚,见1.2。世说大抵每章中的人物是按年代排下来的。
\item
  \emph{月中无物}:月亮中阴影太明显了,必须给出解释,于是产生一系列神话。最早的说法大概是认为月亮中有蟾蜍。屈原在《天问》中就说``夜光何德,死则又育?厥利惟何,而顾菟在腹?''古人也许认为,月亮在黑暗中出来,有盈有缺,就有死后复生的能力。``顾菟''据说是癞蛤蟆的意思。古人不会认为癞蛤蟆恶心,因其出色的繁殖能力,被奉为神物。后来又有兔子、嫦娥、桂树、吴刚等传说,我们可以发现,这些传说都和绵绵不绝、死后复生、长生不死有关。张衡在《灵宪》里说:
  ``月者,阴精,积而成兽,像蛤兔焉。''
  古代印度有月亮里有兔子修行的传说,还有认为它是阎浮树的影子,或认为它是大海里鱼鳖等影子在月轮里的显现等。我想大约各民族都有类似的解释性神话。
\item
  \emph{当极明邪}:杜甫《一百五十夜对月》``斫却月中桂,清光应更多''。
  向子湮《洞仙歌》``谁道斫却桂,应更光辉?无遗照,泻出山河倒影。''极在这里应该是更加的意思,但这样的用法不知其他文献中可有佐证。
\end{itemize}

徐稚的回答偷换概念,别人在讨论光明产生的问题,如何更加明亮的问题,他回答的是看到光明的问题。不过作为孩子,回答比较机巧,很难得了。中国古人的思维常常大而无当,跳跃性很大,也由此可见。我有时候注意到孩子清澈的眼睛,就想,是不是不知满足的欲望和纷杂的心思,使成人的眼睛如此混浊、闪烁和浮浅?

\section{2.3}\label{section-49}

\begin{quote}
孔文举年十岁,随父到洛。时李元礼有盛名,为司隶校尉;诣门者、皆俊才清称及中表亲戚乃通。文举至门,谓吏曰:``我是李府君亲。''既通,前坐。元礼问曰:``君与仆有何亲?''对曰:``昔先君仲尼与君先人伯阳有师资之尊,是仆与君奕世为通好也。''元礼及宾客莫不奇之。太中大夫陈韪后至,人以其语语之,韪曰:``小时了了,大未必佳。''文举曰:``想君小时,必当了了。''韪大踧踖。
\end{quote}

\begin{itemize}
\tightlist
\item
  \emph{孔文举}:孔融,字文举。融有羽毛的意思,举也有羽毛的意思。孔融是我们熟知的历史上最出色的神童之一,他的不少幼年故事我们耳熟能详,但现在情景消失,多有不同意见。孔融是孔子的二十世代孙,东汉末突出的名流、文学家,有一次刘备应邀出兵救援,受宠若惊:``想不到孔融还知道我刘备啊!''由此,可想见孔融的名望。孔融历任侍御史、虎贲中郎将、北海相、青州刺史等。后来在曹操那里担任将作大匠、少府、太中大夫等职。孔融是``建安七子''之首,虽然保留下来的文章不多,但笔法纵横,非常自信,令人钦佩。他的《与曹公论盛孝章书》,要求曹操帮助盛孝章,开头就是``岁月不居,时节如流。五十之年,忽焉已至。公为始满,融又过二(我比你大两岁)''。他讽刺曹丕纳甄氏``武王伐纣,以妲己赐周公''与曹丕得禅位后``舜禹之事,朕知之矣''有异曲同工之妙。曹操有禁酒令,其中一条理由是喝酒可导致亡国,孔融写了封信给曹操:
  ``也有因妇人失天下的,何以不禁婚姻?''这些言辞的确锋芒毕露,气势逼人。孔融为人耿直,不愿意向曹操低头,举荐人才众多,和祢衡、杨修交好,这些都大大触怒了曹操,所以因言辞行为违反礼教为借口灭了满门(上朝不带帽子,说孩子不过是父母追求性快乐后的附带品,父母谈不上什么生恩,而是如东西寄在瓶中,不孝)。
\item
  \emph{父}:孔宙。
\item
  \emph{李元礼}:李膺,见1.4等。
\item
  \emph{诣门者皆俊才清称及中表亲戚乃通}:疑有漏字,皆``有''俊才清称。登让拜访而给与通报的,都必是些才子、名流和内外亲属。因为李膺是``龙门'',政党意识觉醒,有意识结党,打造一个团队。
\item
  \emph{李府君}:太守俸禄二千石,司隶校尉是比二千石(二千石的月俸是一百二十斛,比二千石是一百斛),同样有官衙,所以用同一个称谓。譬如现在调研员、巡视员一般称为张处、王厅、李局等,一种尊称。
\item
  \emph{仆}:谦辞,我。
\item
  \emph{先君仲尼与君先人伯阳}:我的祖上孔子和您的祖上老子。老子,姓李,名耳,字伯阳。老子曾经是东周的图书馆馆长,无论是儒家还是道家,都认为孔子向老子学习过``礼''等。儒家和道家的学说一向不是泾渭分明的,在历史上往往处于相互交融认可的状况,当时人就是这样认为。
\item
  \emph{奕世为通好}:奕,大、累、重。几十世的交情。
\item
  \emph{太中大夫}:从四品上,《汉书 -
  百官公卿表》``郎中令所属有太中大夫等,秩比千石,掌议论。''
\item
  \emph{陈韪}:不详。
\item
  \emph{了了}:了有清楚的意思,了了就是明白通晓,聪明。
\item
  \emph{踧踖}:窘迫,不知所措。
\end{itemize}

由于这个典故,小时了了就成为贬义词,因为它后面隐含着大未必佳的尾巴。在《三国志》中,陈寿认为诸葛亮``应变将略,非其所长'',这本来带有贬义的色彩,但正是由于是形容``完人''诸葛亮的,所以它成了褒义词,这种语言现象并不多见。在10多年的工作中,往往有些领导对当面评价鄙人``某某不错,是个老实人'',其实不过隐含``老实是无用的别名''的意思,幸好小弟无所追求,不然岂不是要洗面革心,重新做人了。
陈韪的说法是中平之论,孔融的词锋则从小就十分锐利,体现出很强的攻击性。这个故事在《后汉书》中还有下半段,李膺多次向孔融问难,最后叹息说:``你长大必定是伟器,可惜我将老死了,看不见你以后的富贵。''融回答说:``您一定不会死!''李膺奇怪地问为什么,孔融回到:``鸟之将死,其鸣也哀;人之将死,其言也善。您一直对我说话,没有什么善的内容,所以知道您不会死。''几年以后,党锢之祸发生,李膺身死宦官之手,另一名士张俭逃亡到孔家,当时孔宙已去世,兄长孔褒不在家,孔融收留了他。后来事泄,孔融、孔褒入狱,孔融说是我收留,请处死我。孔褒说张俭本来是来找我的,与兄弟无关。两人的母亲说,先夫身亡,我是一家之主,家里一切行为都是我的意见。郡县疑不能决,呈请上级判罪,最后孔褒被杀。由此可见,孔融嘴上虽然厉害,但内心纯良。史书上说,有谁当面指出他的缺点,他在背后便要称道这个人的优点;如果了解别人的才能而未加荐举,就认为是自己的过失;其余哪怕有一毫之善的,也无不受到他的礼遇。像这样的名士被杀,也是一种淘汰规律吧。

\section{2.4}\label{section-50}

\begin{quote}
孔文举有二子:大者六岁,小者五岁。昼日父眠,小者床头盗酒饮之,大儿谓曰:``何以不拜?''答曰:``偷,那得行礼?''
\end{quote}

此事与2.12钟毓、钟会的故事一样,刘义庆再粗心,也会注意到。当然小孩子的游戏完全可能一样。刘义庆不加辨析地把两个故事放在一个章节,也许想表达神童都是相似的。神童的故事自然是越小越神。
因为钟毓、钟会的故事更详细一点,所以不在这里解释了。一般来说,神童都留下了名字,可惜的是,孔融的孩子没有留下名字来,因为他们很快就被曹操杀了。嗯,除善务尽,省得有像赵氏孤儿那样的续集。

\section{2.5}\label{section-51}

\begin{quote}
孔融被收,中外惶怖。时融儿大者九岁,小者八岁,二儿故琢钉戏,了无遽容。融谓使者曰:``冀罪止于身,二儿可得全不?''儿徐进曰:``大人岂见覆巢之下复有完卵乎?''寻亦收至。
\end{quote}

\begin{itemize}
\tightlist
\item
  \emph{孔融被收}:208年,曹操东征刘表期间,御史大夫郗虑(他后人有个叫郗虑的,为人比祖宗强)、丞相军谋祭酒路粹表奏孔融罪状,认为孔融有违反孝道的言论,收押后被诛杀。郗虑曾经和孔融有私怨,因为有一次献帝召见郗虑和孔融,问孔融:``郗虑何所优长?''孔融说:``可与适道,未可与权。''(语出《论语》,
  意思大概是可以和郗虑一起学习追求``道'',但郗虑这人不知轻重,处理事物能力不强,不能给与权变之责任,担当一面。)但这仅仅是一方面,因为主要还是孔融心存汉室,在他身旁聚集了一批政治反对派,孔融又多次和曹操当面唱反调,特别是祢衡事件,大大地羞辱了曹操,两人之间早已产生了巨大的裂痕。所以后人对曹
  操此举是很不以为然的,称这是``微法、构成''。
\item
  \emph{中外惶怖}:孔融是海内名流,而且比李膺要好客,每日都宾客盈门,他感慨地说:``坐上客恒满,尊中酒不空,吾无忧矣。''
  孔融打仗失败,治理政府失败,但其气质出众,言论高明,``懔懔焉,皓皓焉,其与琨玉秋霜比质可也'',所以``海内英俊皆信服之''也。孔融被害时,虽然``融有高才清名,世多哀之'',但许昌没有人敢收尸的,只有好友脂习前往抚尸而哭:``孔融你离我而死,我活着还有什么意义!''曹操闻后大怒,收押脂习。
\item
  \emph{琢钉戏}:据清代周亮工《因树屋书影》讲,这是一种小孩子的游戏,``画地为界,琢钉其中,先以小钉琢地,名曰签;以签之所在为主,出界者负;彼此不中者负,中而触所主签亦负。''大约是一种投掷游戏,居然沿袭了上千年。
\item
  \emph{遽容}:惊惧、慌张的神色。《楚辞 -
  九章》``众骇遽以离心兮,又何以为此伴也?''。
\item
  \emph{冀}:希冀,希望。
\item
  \emph{徐进}:慢慢上前,从容上前。从容不迫要很高的修养才能做到。
\item
  \emph{覆巢之下复有完卵}:或作``安有巢覆而卵不破者哉'',语出《说苑》``覆巢毁卵,则凤凰不翔''。鸟巢打翻了,里面还有完整的蛋吗?
\item
  \emph{寻}:不久,很快。《桃花源记》``未果,寻病终。''
\end{itemize}

这个故事有其他版本,其中一种说法是曹操开始还不打算灭门的,后来有人把孩子的言论及时报告,曹操立起杀心。在《后汉书》中主人公之一是孔融的女儿:孔融女儿七岁,男孩九岁,当时在别人家暂住。主人家给肉汁喝,哥哥喝了。妹妹说:``今日之祸,岂得久活,何赖知肉味乎?''兄号泣而止。后来临刑时,妹妹对哥哥说,``如果死者有知,得见父母,岂非至愿!''乃延颈就刑,颜色不变。不过赵一清在《三国志注补
-
十二》中说:``晋书羊祜传`祜前母孔融女,生兄发''',可见孔融起码出嫁的女儿没有被杀绝。无论后人怎么样高度评价曹操,孔融孩子的死亡不应该被忘记,天才儿童的横死,比一般人的死亡更加令人惋惜。当我们看曹操故事的时候,就能瞥见他手里滴滴答答淌着孩子的鲜血。

\section{2.6}\label{section-52}

\begin{quote}
颖川太守髡陈仲弓。客有问元方:``府君何如?''元方曰:``高明之君也。''``足下家君何如?''曰:``忠臣孝子也。''客曰:``《易》称:`二人同
心,其利断金;同心之言,其臭如兰。'何有高明之君,而刑忠臣孝子者乎?''元方曰:``足下言何其谬也!故不相答。''客曰:``足下但因伛为恭,而不能答。''
元方曰:``昔高宗放孝子孝己,尹吉甫放孝子伯奇,董仲舒放孝子符起。唯此三君,高明之君;唯此三子,忠臣孝子。''客惭而退。
\end{quote}

\begin{itemize}
\tightlist
\item
  \emph{髡}:古代的一种肉刑,将人头发全部或部分剃掉,以人格侮辱的方式对犯人实施惩罚,使他处于一种明显的非常人形状,这是一种耻辱刑,主要流行于中国古代夏商周到东汉。因为古人认为身体发肤受之父母,不能毁的。我们只要想想文革时给失败者
  剃阴阳头、戴高帽,就可以知道髡刑对人精神的巨大打击了。古人头发这么长,所以要在15岁的时候束发带冠,而且为了保持清洁,官府大概每十天放一次假,让
  官员洗头发。
\item
  \emph{陈仲弓}:陈寔,见1.6等,他后来社会声誉极高,但据史书上讲,他被捕两次,一次被怀疑为杀人犯,一次受党锢之祸牵连。虽然陈寔在乡间有很高的声誉,是决断是非之能人,但这次髡刑一定是有重大打击的,所以家里人要收集古人受冤枉的例子来排遣个体与社会关系的紧张。
\item
  \emph{客}:不会是上次的那个吧?老是来挑衅,客观上又成就了陈寔和儿子陈纪、陈湛的名声。
\item
  \emph{府君}:指颖川太守,陈所在地的长官,如果陈纪说不满的话,可能就会形成构陷,进一步采取动作。所以我猜想陈寔这次髡刑,不可能是因为杀人案,而是政治案,长官意志就很容易置人于死地。
\item
  \emph{``《易》称''句}:《周易 -
  系辞上》``同心协办的人,他们的力量足以把坚硬的金属弄断;同心同德的人发表一致的意见,说服力强,人们
  就像嗅到兰花的芬芳之味,欣然接受。''既然太守和你父亲有一致性,怎么会出这种事?臭的原意气味,后来变成了现在的专指臭味,这是一种词义缩小现象,如``丈人''在古代是对老前辈的尊称,前面有句``丈人不悉恭'',现代专指岳父;``金''先秦泛指金属,``金就砺则利'',到了银、铁、锡都出现以后,``金''就逐渐
  地专指黄金了。
\item
  \emph{伛为恭}:伛,弯曲。余嘉锡在《笺疏》中说,左氏昭七年传``正考父佐戴、武、宣,三命兹益共。故其鼎铭云:`一命而偻,再命而
  伛,三命而俯。'\,''御览四百三十二引作``滋益恭'',并引贾逵曰:``俯恭于伛,伛恭于偻。(其本意是指正考父三朝元老,官职一步步上升,态度也愈加谦虚)''
  这话的意思就是因为问及陈纪的长官和父亲,所以陈纪不得不虚词褒扬,不是诚恳的回答。就像那些驼背的人,见人都弯着腰,看上去很恭敬,其实是不得不这样,
  驼背人的内心也许对人并没有恭敬的意思。
\item
  \emph{昔高宗句}:连用三个典故反驳,就增加了气势。殷高宗武丁的儿子孝己,侍奉父母很孝顺(孝己事亲,一夜
  而五起,视衣厚薄,枕之高下也),后来高宗受后妻的迷惑,把孝己放逐致死。东周著名政治家尹吉甫的儿子伯奇,遭受后母诬陷,被父亲放逐,最后尹吉甫(《诗经》中多次出现的人物)醒悟射杀妻子。董仲舒儿子符起,被误为不孝而赶出家门。
\end{itemize}

大家应该注意到,在前面解释了几个《言语》的故事,几乎每句话都是化用别人的话或者是使用典故,这样讲话比较典雅,需要双方或者三方都有一定的文学修养,光从字面理解虽然能够揣度出大意,但把握起来就不一定准确。所以孔子说,不学诗,无以言。

\section{2.7}\label{section-53}

\begin{quote}
荀慈明与汝南袁阆相见,问颖川人士,慈明先及诸兄。阆笑曰:``士但可因亲旧而已乎?''慈明曰:``足下相难,依据者何经?''阆曰:``方问国士,而及诸兄,是以尤之耳!''慈明曰:``昔者祁奚内举不失其子,外举不失其仇,以为至公。公旦《文王》之诗,不论尧、舜之德而颂文、武者,亲亲之义也。《春秋》之义,内其国而外诸夏。且不爱其亲而爱他人者,不为悖德乎?''
\end{quote}

\begin{itemize}
\tightlist
\item
  \emph{荀慈明}:荀爽,荀淑的第六个儿子,荀彧的叔父。前面说过,荀爽非常荣耀地给李膺赶过车,当时舆论认为,``荀氏八龙,慈明无双''。他自幼是神童,精通《易》、《礼》、《书》等,与郑玄、虞翻并称为汉末易学大家。荀爽长期隐居著作,后来董卓执政,强辟,委任平原相、光禄勋、司空,95天就升任三公之一。但他联合王允谋除董卓,举事前病卒。
\item
  \emph{汝南袁阆}:《世说》中突出的配角。
\item
  \emph{士但可因亲旧而已乎}:我问你颖川的杰出人士,你只能依靠亲戚朋友来表现罢了。言下之意一是偏袒,二是颖川无人。汝南、颖川是当时河南人才辈出的地方,两地士人经常别苗头;袁阆作为分管干部工作的领导,自然忌讳宗族势力抱团,把持郡事,故有此问。事实上,朝廷对这种豪强现象长期以来都有所警惕,有一次汉光武帝就怒斥:``汝南子欲复党乎!''仲长统在《潜夫论》中也指出:``今观俗士之论也,以族举德,以位命贤,兹可谓得论之一体矣,而未获至论之淑真也。''
\item
  \emph{何经}:``经''可能是衍字。
\item
  \emph{尤}:责难。怨天尤人。
\item
  \emph{祁奚}:春秋时晋国人,任中军尉(掌管军政的长官)。祁奚告老退休,晋悼公征求他继任者人选意见,他推荐了对头解狐。刚要任命,解狐却去世了。祁奚接着推荐儿子祁午。此事多为人传颂。
\item
  \emph{公旦《文王》之诗}:公旦就是著名的周公,文王之子、武王之弟,孔子的崇拜对象;《文王》指《诗经》中的《文王之什》,包括《文王》、《大明》等十篇,分别歌颂周文王、周武王之功德,当时人认为是周公所创。我们熟知的``周虽旧邦,其命维新''就出于此。
\item
  \emph{亲亲之义}:亲爱自己亲人的大义。《礼记》``君子贤其贤而亲其亲'',这是儒家学说的基础,完全区别与道家、墨家。儒家认为,对亲人的爱为最基本、最原始的爱,是其他一切爱的基础、源泉和延伸。《孟子
  -
  尽心上》中说:``孩提之童,无不知爱其亲者,及其长也,无不知敬其兄也。亲亲,仁也;
  敬长,义也;无他,达之天下也。''孟子又举例说,舜的弟弟象,``至不仁'',又要抢嫂子又要杀大哥,但舜还是封他为诸侯,孟子的弟子万章不解地问:``所谓的
  仁人就能这样做?如果他人这种行为就要被杀头,而因为是自己的弟弟却能当诸侯?''孟子解释说:``仁人之于弟也,不藏怒焉,不宿怨焉,亲爱之而已矣。亲亲,欲其贵也;爱之,欲其富也。''(《孟子
  -
  万章上》)虽然大义灭亲备受推崇,但古代自有亲亲相隐的法律;虽然我们对官吏任用亲人多有不满,但``一人得道,鸡犬升天''的现状并不会因此而改变。即使从现代的眼光看,自己春风得意而亲人流离颠沛,亲人犯罪而自己积极检举揭发,无论从什么理由出发,都是不妥当的。
\item
  \emph{内其国而外诸夏}:《春秋公羊传》中说,《春秋》的立场就是``内其国而外诸夏,内诸夏而外夷狄'',就是指内外有别,感情是不断延伸的过程,爱父母然后爱故土,爱故土然后爱国家。
\end{itemize}

这段话很有意思,是书本上所说的中国是宗法社会的注脚。我们再联想一下,我们党不总是这样自我歌颂吗,过年过节还安排子女们捧一下场,偶尔对世界舆论普遍谴责的民贼独夫表扬一番。有些人听了、看了不以为然,``风光长宜放眼量'',我们的文化传统就是这样:中国文化讲感情,西方文化讲功利。

\section{2.8}\label{section-54}

\begin{quote}
祢衡被魏武谪为鼓吏。正月半试鼓,衡扬枹为《渔阳掺挝》,渊渊有金石声,四座为之改容。孔融曰:``祢衡罪同胥靡,不能发明王之梦。''魏武惭而赦之。
\end{quote}

\begin{itemize}
\tightlist
\item
  \emph{祢衡}:这是个像彗星一样的人,他的一生不过26年,他所闪耀的光辉就是``击鼓骂曹''那一刻,尽管事实并不是京剧中表现的那样。也许很多人会认为祢衡是个充满缺点、不自量力、难以接近的小人物。但只要我们设身处地地想想,在现实生活中,有谁能够这样高傲地生活,毫无畏惧地表达自己的意见,在至高无上权力面前,用嘹亮的声音进行反抗。祢衡就能这样紧紧地盯住别人的缺点,向那个浑浊不堪的社会和同样污浊的人群投出自己鄙夷的目光------我绝不和你们站在一起!
\item
  \emph{祢衡},字正平,山东人。有过目不忘、倚马可待的本领,读的书多,记性又好,把理想国当成曾经的世界,难免看不起一般人。孔融说他``忠果正直,志怀霜雪。
  见善若惊,疾恶若仇'',在那个乱世中祢衡就以鹦鹉做喻:``心怀归而弗果,徒怨毒于一隅。苟竭心于所事,敢背惠而忘初!托轻鄙之微命,委陋贱于薄躯。期守死以抱德,甘尽辞以效愚。''从《鹦鹉赋》看,祢衡并不是没有自知之明,而决然选择了属于自己的道路,他在怀念张衡的文章中就讲到:``石坚而朽,星华而灭,惟道兴隆,悠永靡绝。''他以继承道统为己任,对生命的羁绊不屑一顾!那些嘲笑祢衡的人应该扪心自问,除了世故,谁比祢衡更加高超?
\item
  \emph{魏武}:曹操初封魏王,死后溢为武。曹丕夺帝位魏国后追尊为武帝。因为孔融多次推荐祢衡,曹操做出姿态征辟。但祢衡毫不领情,时有诽谤,所以祢衡被拉去当追鼓掾了。在这里评论曹操很不合适,略。
\item
  \emph{正月半试鼓}:这可能是古代官场的一种风俗,鼓是古代一种非常重要的乐器,我到过几个城市,都有叫``鼓楼''的地名。在古代,每天黄昏来临,各个城市都会响起鼓声,代表着公务人员一天工作的结束。在寺院里,也有暮鼓晨钟的传统。但鼓的用途很多,白天也是要打的,杜甫有句诗说:``岁暮阴阳催短景,天涯霜雪霁寒宵。五更鼓角声悲壮,三峡星河影动摇。''也许我们记得很多悲壮的诗歌,都与鼓有关。在刘孝标的注中,时间为八月朝会。
\item
  \emph{枹}:鼓槌。
\item
  \emph{渔阳掺挝}:鼓曲名,也作渔阳参挝。我们当然记得白居易的诗歌``渔阳鼙鼓动地来,惊破霓裳羽衣曲。''相传此曲是祢衡所创.取名渔阳,是借用东汉光武帝时彭宠据渔阳反汉的故事。彭宠据幽州渔阳反,攻陷蓟城,自立力燕王,后被手下的人杀死。祢衡击此鼓曲,也许有讽刺曹操有叛逆之心的意思。后来渔阳鼓成为著名的典故,既有名士之风流,又有叛乱风起等意思。
\item
  \emph{渊渊有金石声}:鼓声深沉寂寥,有金属撞击的质地之音。后来这个曲调应该是有流传的,唐代李颀有句诗说:``边城晏开渔阳掺,黄尘萧萧白日暗。''估计这鼓声不会是现今反映中国特色社会主义核心价值理念的集体主义精神的``威风锣鼓''吧?
\item
  \emph{罪同胥靡,不能发明王之梦}:这句话我不懂,胥靡据说本意是相随,因为古代犯人是用铁链联系跟随的,所以胥靡代指刑徒。发明王之梦据说是傅说的典故,当年商朝君主武丁梦见上天赐给他一个贤人,就画出相貌全国去找,结果正是犯人傅说。但是后人考证,关于傅说从政经历的神话传说,出现于东晋时期的《伪古文尚书》中有《说命》上中下三篇,孔融是否使用了这个典故存疑问。但即使用了这个典故,孔融想解释分辩什么呢?是否是说``祢衡和傅说一样才德兼备,应该委以重
  任,只是曹操(尊称曹操为明王?明王是不是佛教词语?事实上当时估计是建安12年,曹操不过是大将军,当魏王是213年的事情,祢衡被黄祖杀害在建安13
  年,198年,这种称呼混乱《世说》中每每可见。)梦中未得见而已。''
\end{itemize}

刘义庆把这个故事放在``言语''中而不是在``简傲''中,是因为孔融解释得好,不过我看不清楚,遗憾。历史上击鼓骂曹还有很多细节的出入,是否裸衣、是否当场开口大骂,祢衡的其他事迹等,因为是短文,不展开讲述。不过在《世说》
中,刘义庆对曹操总体是不以为然的,这点我们应该可以把握。

\section{2.9}\label{section-55}

\begin{quote}
南郡庞士元闻司马德操在颖川,故二千里候之。至,遇德操采桑,士元从车中谓曰:``吾闻丈夫处世,当带金佩紫,焉有屈洪流之量,而执丝妇之事!''德操曰:``子且下车。子适知邪径之速,不虑失道之迷。昔伯成耦耕,不慕诸侯之荣;原宪桑枢,不易有官之宅。何有坐则华屋,行则肥马,侍女数十,然后为奇?此乃许、父所以慷慨,夷、齐所以长叹。虽有窃秦之爵,千驷之富,不足贵也。''士元曰:``仆生出边垂,寡见大义。若不一叩洪钟、伐雷鼓,则不识其音响也。''
\end{quote}

\begin{itemize}
\tightlist
\item
  \emph{庞士元}:庞统,字士元,湖北襄阳人,18岁去见司马徽,得到他的赏识,从此以后庞统成为名流。庞统曾经当过周瑜治下的功曹,后来投奔刘备当了个县令,由于诸葛亮和庞统是亲戚、鲁肃和庞统也有交情,多次推荐庞统,终于引起刘备注意和重用。214年,庞统在随刘备征伐四川途中被流矢射杀。
\item
  \emph{司马德操}:司马徽,字德操,河南颖川人,和庞统的叔叔庞德公关系极好,据说有一次司马徽去见庞德公,不在,司马徽就直接吩咐庞德公的妻子佣人:``待会还有客人要来,你们准备饭菜''。庞德公有识人之鉴,最看得起司马徽、诸葛亮和庞统,分别赠以人鉴、卧龙、凤雏的绰号。这次估计是庞德公要求庞统去见司马徽,大家互相抬轿子,长长声望。据说司马徽小庞德公10岁,大庞统5岁,所以也算是平辈,讲话就没有什么顾忌。
\item
  \emph{带金佩紫}:带金印、紫绶,指高官。在汉代,官员在袍服外要佩挂组绶,用彩带打个大回环,下面系上官印,装于腰间的袋子中,故称印绶。印和绶都要通过质地、颜色、宽窄、长短等反映级别。汉代的《汉旧仪》中说,皇帝用玉为玺,列侯、丞相、大将军、御史大夫、匈奴单于黄金印,二千石银印,千石、六百石、四百石铜印。《后汉书
  -
  舆服志》:``乘舆(皇帝)黄赤绶;诸侯王赤绶;公、侯、将军紫绶''等。在故事中,当年司马相如、朱买臣等一朝时来运转,飞黄腾达,就穿着便衣回老家当官,在某个场合突然拿出随身的官印炫耀,羞辱故人。不过后来官印越做越大,估计随身带着就没有了。
\item
  \emph{洪流之量}:大约是比喻气度胸襟像长河大江
\item
  \emph{子且下车}:庞统在车上说话是不礼貌的行为,所以前后都做了强调,突出细节。
\item
  \emph{邪径}:斜径,小路。《汉书 -
  五行志》:``邪径败良田,谗口乱善人。''皇甫谧《高士传 -
  焦先》:``冬夏袒不著衣,卧不设席,又无蓐,以身亲土,其体垢污,皆如泥滓。不行人间,或数日一食,行不由邪径,目不与女子。''引申为比喻不正当的行径。明何景明《何子》:``正道坏则邪径成,公室衰则私门盛。''美国诗人弗罗斯特1915年写下《未选择的路》:黄色的树林里两路分岔,真遗憾我不能分身全走。久久地我站在岔道口前,沿一条羊肠道极目远眺。直到它隐没在密林深处,再将眼转移向另一条道。也许它招引人更想去走,因为它少人踏长满野草。虽如此仍有人在此走过,地面上给踏得彼此难分。两条路公平地摆我面前,清晨里无人走落叶片片。噢,我将第一条留待他日,却知道路连路永无止境。此生里怕只怕再难回走,多年后在某个遥远所在。我只能长叹息将往事回首,密林里两条路彼此岔开。我选的一条路更少人走,此选择使结局全然变改。
\item
  \emph{伯成耦耕}:伯成子高,相传尧治天下,他立为诸侯。大禹上台后,他认为``德自此衰,刑自此立,后世之乱自此始'',就隐居耕种。耦耕:古代的一种耕作方法,即两人各扶一张犁,并肩而耕,泛指务农。
\item
  \emph{原宪桑枢}:原宪是孔子的72个及格生之一。他穷,住茅草屋,用桑树干做门上的转轴(桑树材质不好),漏风漏雨,毫不在意,不废弦歌。子贡见他一副落魄的样子,问他是不是病了,原宪说:``吾闻之,无财者谓之贫,学道而不能行者谓之病。若宪,贫也,非病也。''
\item
  \emph{肥马}:《论语 - 雍也》``赤之适齐也,乘肥马,衣轻裘'',《孟子 -
  梁惠王上》``庖有肥肉,厩有肥马,民有饥色,野有饿莩,此率兽而食人也'',杜甫《奉赠韦左丞丈二十二韵》``朝扣富儿门,暮随肥马尘'',形容富裕阔绰,中性词转变为贬义词。
\item
  \emph{然后为奇}:这样才能算与众不同?
\item
  \emph{许父夷齐}:许由、巢父、伯夷、叔齐,视当公仆为蛇蝎四大名人,这里不解释了。
\item
  \emph{窃秦之爵}:指吕不韦。《战国策》中说,有一次吕不韦请教他父亲:``农耕的回报几何?''其父回答:``十倍吧。''他又问:``珠宝买卖赢利几倍?''答道:``一百倍吧。''他又问:``如果拥立一位君主呢?''他父亲说:``这可无法计量了。''吕不韦说:``如今即便我艰苦工作,仍然不能衣食无忧,而拥君立国则可泽被后世。我决定去做这笔买卖。''因为吕不韦是商人,当官动机不纯,在当时的文化中是看不起他的。
\item
  \emph{千驷之富}:指齐景公,《史记》中说他``好治宫室,聚狗马,奢侈,厚赋重刑'',《论语
  - 季氏》:``
  齐景公有马千驷,死之日,民无德而称焉。''现存齐景公墓已被盗,但殉葬坑中还有600匹马的尸体,数量之多,规模之大,前所未见,可见《论语》诚不我欺。
\item
  \emph{边垂}:边陲,边疆,湖北相对于中原河南,是上古的蛮夷之地。《国语
  -
  周语上》中说``夫先王之制,邦内甸服,邦外侯服,侯、卫宾服,蛮夷要服,戎狄荒服'',当年齐桓公提出``尊王攘夷''口号,其中就有``北御夷狄,南制楚蛮''的说法。
\item
  \emph{叩洪钟、伐雷鼓}:击打巨钟大鼓。《礼记 -
  学记》中说:``君子之教如钟,不扣不鸣;小扣小鸣,大扣大鸣'',``善待问者,如撞钟,扣之以小则小鸣,扣之以大者则大鸣,待其从容,然后尽其声;不善答问者反此。此皆进学之道也。''唐太宗褒奖孔颖达说:``洪钟待扣,扣无不应;幽谷发响,声无不答。''雷鼓是古代古代祭祀天神时所用的八面鼓。《周礼
  - 地官 -
  鼓人》:``以雷鼓鼓神祀。''曹丕在曹操出征赤壁时曾写《浮淮赋》:``建安十四年,王师自谯东征,大兴水军,泛舟万艘。\ldots{}\ldots{}乃撞金钟,爰伐雷鼓。白旄冲天,黄钺扈扈。武将奋发,骁骑赫怒。''这句话是说,如果我庞统不提问,就认识不到司马徽您的胸襟境界胸怀,自己也不会有所受益而使自己得到教益。这种回答也是非常得体的。
\end{itemize}

这也许是庞统和司马徽的第一次会面,庞统故意表现得比较傲慢,来考验叔叔口中赞不绝口的名士,后来就像刘孝标的注中所说,两人坐在桑树下说话,从白天到黑夜,司马徽最后赞叹道:``生当为南州士人之冠冕''。于是当时河南最璀璨的明星诞生了。``遥想公瑾当年,羽扇纶巾,谈笑间,强虏灰飞湮灭'',可惜善于兵行奇险的庞统中道陨落,在雏凤坠落的前一年,司马徽也因病去世。``凤兮凤兮,何德之衰?''

\section{2.10}\label{section-56}

\begin{quote}
刘公干以失敬罹罪。文帝问曰:``卿何以不谨于文宪?''桢答曰:``臣诚庸短,亦由陛下网目不疏。''
\end{quote}

\begin{itemize}
\tightlist
\item
  \emph{刘公干}:刘桢,字公干,桢和干是一个意思,房屋的支柱。刘桢是汉家宗室,建安七子之一,和曹丕、曹植交好,以五言诗著称,其中有首``亭亭山上松,瑟瑟谷中风。风声一何盛,松枝一何劲。冰霜正惨怆,终岁常端正。岂不罹凝寒,松柏有本性''是托物言志的名诗,后人的诗歌中常常化用这个典故。217年,刘桢与陈琳、应玚等大手笔同染瘟疫而亡,时年30。
\item
  \emph{失敬}:有一次刘桢和曹丕等一起喝酒,当时曹丕抢的妻子甄氏(204年从袁绍儿子袁熙那里霸占来的),美貌非常,也许正是``仿佛兮若轻云之蔽月,飘飘兮若流风之回雪。远而望之,皎若太阳升朝霞;迫而察之,灼若芙蕖出渌波'',名声在外,心中得意,请出来炫耀,座上客人多拜伏在地不好意思看,独独刘桢目不转睛地平视(这是不礼貌的行为),曹丕名士风流,也不在意,因为后来还有一次,曹丕召集吴质、曹休等人欢聚,他又把一个妻子郭氏请出来见客,吩咐臣下说:``卿仰谛视之。''这也许是缅怀刘桢,也许曹丕有开无遮大会的嗜好。可是曹操听说了,把刘桢逮捕下狱,判罚做苦工(从这个故事看,曹操没有楚庄王绝缨会的雅量,十有四五和甄氏关系暧昧。后来甄氏这样的绝色,也经不住时光的锉刀,失宠被曹丕赐死,``被发覆面,以糠塞口'')。
\item
  \emph{罹}:遭遇不应该遭遇的苦难。《诗经 -
  兔爰》``我生之后,逢此百罹,尚寐无吪'',(在我成年这岁月,各种苦难竟齐集。只有长睡而把嘴闭起。)李朝威《柳毅传》``远罹构害''。
\item
  \emph{文帝}:曹丕。此事不确,曹丕220年才夺位,刘桢早死。所以这段对话如果是真的,那是刘桢对答曹操。这个故事还有个版本,刘孝标注中等说,当时刘桢劳动改造,工作是磨石头。曹操来视察,刘桢这时也不抬头了,一心低头干活。曹操走上前问刘桢:``磨石头感觉怎么样?''刘桢跪直身体回答说:``石出荆山悬岩之颠,外有五色之文,内含卞氏之珍,磨之不加莹,雕之不增文,禀气坚贞,受之自然,顾其理枉屈纡绕,而不得申。''曹操感觉刘桢态度有所端正,回答也好,又把他从劳改农场放做管理员去了。
\item
  \emph{文宪}:法令条文、礼法。张华《答何劭诗》:``缨緌为徽纆,文宪焉可逾''士人被加上刑具,哪有人可以凌驾于法律呢?
\item
  \emph{网目不疏}:法网严密。《史记 -
  酷吏列传序》:``网漏于吞舟之鱼,而吏治烝烝,不至于奸,黎民艾安。''袁宏道《经下邳》:``诸儒坑尽一身余,
  始觉秦家网目疏。枉把六经灰火底, 桥边犹有未烧书。''
\end{itemize}

这个故事激起了后人的愤慨,看美女是人之常情,凭什么甄氏就只能你曹家父子共享?和尚摸得,我就看不得?蒲松龄的《聊斋志异》就有一篇叫《甄后》,说刘桢后来投胎为洛阳书生刘仲堪,甄后下凡与之发生一夜情,``息烛解襦,曲尽欢好''。而曹操在故事里变成了一只因嫉妒而狂吠的狗。爽快!这种意淫的笔法,现多见于网络色情小说,多少求之不得的女子,在网友笔下呼之即来,挥之即去。

\section{2.11}\label{section-57}

\begin{quote}
钟毓、钟会少有令誉。年十三,魏文帝闻之,语其父钟繇曰:``可令二子来!''于是敕见。毓面有汗,帝曰:``卿面何以汗?''毓对曰:``战战惶惶,汗出如浆。''复问会:``卿何以不汗?''对曰:``战战栗栗,汗不敢出。''
\end{quote}

\begin{itemize}
\tightlist
\item
  \emph{钟毓},字稚叔,可见家里排行老三,老大叫钟劭,无名之辈。有书说钟毓大钟会7岁,两人都是钟繇70岁左右生的。他们是前面介绍过的汉末名士钟皓的曾孙,钟繇老而弥坚,居然还生了两个神童,确实令后辈仰慕(浙江现在有个书法家沈定庵,据说80来岁还兴致勃勃),可见书法和养气之道密不可分,服用五石散也未必不长寿。
\item
  \emph{钟会},字士季,钟家是河南颍川人,为望族。钟会比钟毓更出色一些,我们从这个故事就可以看出。有一次大臣蒋济到钟繇家做客,看到5岁的钟会就非常吃惊,对钟繇说:``这孩子必定不是一般的人,你看看他的眼睛就知道了。''他22岁任中书侍郎,29岁封关内侯,31岁为司隶校尉,是著名的玄学家、军事家、书法家,被当时人认为是司马昭的张良,``谋无遗策,举无废功。凡所降诛,动以万计,全胜独克,有征无战(他所谋划的事,没有不周全的,他所从事的事,也没有不成功的。被他征服的部队,总以万人为计算单位,他每次都能获得全胜,甚至于只要出兵,不用战斗,就能取得胜利了。)''但钟会害死过两个伟大人物------嵇康和邓艾,所以历史上认为他德不驭才。也许钟会一生太顺利,就觉得一切都是那么囊中取物,据说他在姜维的怂恿下,莫名其妙地想割据蜀地自立为王,被下属所杀。钟会之事其中恐怕有点问题,他写过一篇著名的檄文,其中说``明者见危于无形,智者窥祸于未萌'',既然他欲独立,那么总不会是临时起意,首鼠两端,结果军中哗变。也许造反正是件很辛苦的事。
\item
  \emph{令誉}:令,美好,《论语》``巧言令色'',《孔雀东南飞》``年始十八九,便言多令才''。
\item
  \emph{年十三}:缺``毓''字。
\item
  \emph{魏文帝}:曹丕(187~226年),所以此事又不确,钟会是225
  -264年,《世说》故非信史,如果这个故事是史实,魏文帝当作魏明帝曹叡(205---239年)。
\item
  \emph{锺繇},字元常,著名的书法家,历任御史中丞,侍中尚书仆射,封东武亭,司隶校尉、相国、太尉、太傅,封成侯等。他成为杰出的书法家是有原因的,一往深情。他发现同僚韦诞有蔡邕的真迹,向他借阅而苦求不得,难过得捶胸顿足,哭闹三天,
  ``其胸尽青,因呕血'',亏得曹操派人送五灵丹救他。韦诞死后,钟繇竟派人盗他的墓,终获蔡邕手迹。他的小妾张昌蒲也是神童,四岁读《孝经》,七岁读《论语》,十岁读《尚书》,等等,因为基因好,家教好,所以又生了两个神童,为了这个女子钟繇以自杀要挟,抗曹丕之命与原配离婚。可能当时人也很佩服钟繇的能力,在《三国志》的注中,居然捎带了一个钟繇和美女妖怪的故事。
\item
  \emph{战战惶惶}:战战惶惶、战战栗栗、战战兢兢都是一个意思,害怕、戒惧、发抖的样子。《诗经
  - 小雅 - 小旻》:``战战兢兢,如临深渊,如履薄冰。''《韩非子 -
  初见秦》:``战战栗栗,日慎一日。''小孩子见皇帝紧张是人之常情,勇士秦舞阳入秦宫尚且``色变振恐'',孙叔通帮刘邦恢复朝廷的礼仪,众臣``莫不振恐肃敬'',刘邦得意地说:``吾乃今日知为皇帝之贵也。''
\item
  \emph{汗不敢出}:钟毓的解释已算是得当,钟会则更是胆大心细,反应敏捷。他这样说既是帮助哥哥解围,又抬举了曹叡,言辞风趣,也很押韵,是了不起的举动。
\end{itemize}

这次会面后,一则父亲是太傅,二则君臣之间关系融洽,给曹叡留下了好印象,第二年钟毓就当上了散骑侍郎,作为皇家五品顾问,并发挥了自己的才干。钟毓后来做到青州刺史,都督徐州、荆州诸军事等。

\section{2.12}\label{section-58}

\begin{quote}
钟毓兄弟小时,值父昼寝,因共偷服药酒。其父时觉,且讬寐以观之。毓拜而后饮,会饮而不拜。既而问毓何以拜,毓曰:``酒以成礼,不敢不拜。''又问会何以不拜,会曰:``偷本非礼,所以不拜。''
\end{quote}

\begin{itemize}
\tightlist
\item
  \emph{昼寝}:《论语》中说,宰予睡午觉,孔子很不高兴,认为朽木不可雕,粪土之墙不可圬也。老先生特别珍惜时间,所以看不惯。不过在故事中,刘备去拜访诸葛亮,诸葛亮在午休。郗鉴挑女婿,王羲之也照样毫不在意地白天睡觉。可以说午睡本来就是我们大多数人平常的习惯,大家觉得对孔子如此严厉批评白天睡觉很不好理解,自己就睡不踏实。有人就提出,宰予``昼寝''当作``画寝'',因为繁体的``晝''和``畫''很相似,画寝就是装修卧室很富丽堂皇,大约是不符合当时的礼制,孔子特别愤怒。照样解释虽然能自圆其说,但所谓不和礼制缺乏证据,和孔子后面讲到的宰予言行不一缺乏联系,聊备一说罢了。还有过分解释的,我一直对现在的红人南怀瑾不以为然,其中一条就是他喜欢故作姿态,哗众取宠,譬如他对宰予昼寝有另一番高论,宰予身体不好要午休,孔子爱护他,认为身体是革命的本钱,可不能影响宰予休息,不然身体垮了,成为朽木和粪土之墙。南怀瑾这样作惊人之语,牵强附会,在他长篇累牍的文集中屡屡可见,不是学者的态度。
\item
  \emph{因}:于是;就。柳宗元《黔之驴》:驴不胜怒,蹄之。虎因喜,计之曰:``技止此耳!''
\item
  \emph{服药酒}:用``服''字,就应该是五石散了。五石散本来是药剂,应该是煮着喝的吧,后来大概经过改良,变成十全大补酒,所以钟繇坚持长期服用,活了80多岁。
\item
  \emph{讬寐}:讬通托,但似乎当做``诈'',诈装睡着了。
\item
  \emph{拜而后饮}:起码在周以后,饮酒就要讲礼节。西周周公旦借鉴纣王喝酒亡国的教训,颁布《酒诰》,明确指出天帝造酒的目的并非供人享用,而是为了祭祀天地神灵和列祖列宗,所以饮酒前要很隆重祭拜。在《礼记》中关于酒礼的记载也很多,其中《乐施》篇说到:``一献之礼,其宾主百拜''。如果遵照《礼记》的要求,喝酒的规矩比现在要复杂得多,一杯酒没有来来回回10分钟喝不下去。``酒以成礼''是严格的礼仪训练,所以前面有个故事说王朗学华歆在家庭里办酒会。
\end{itemize}

从这个故事看,钟毓要规矩一些,有盗亦有道的味道,而钟会小小年纪就有当名士的潜质。不知诸君小时候在家长午休时做过哪些``坏事'',反正我小时候最快乐的时光之一就是老爹中午睡觉的时候,可以偷偷跑出去为所欲为,摘野草莓啊,抓知了了,捉迷藏等等。现在自己的孩子就只能偷偷玩电脑游戏了。

\section{2.13}\label{section-59}

\begin{quote}
魏明帝为外祖母筑馆于甄氏。既成,自行视,谓左右曰:``馆当以何为名?''侍中缪袭曰:``陛下圣思齐于哲王,罔极过于曾、闵。此馆之兴,情钟舅氏,宜以渭阳为名。''
\end{quote}

\begin{itemize}
\tightlist
\item
  \emph{魏明帝}:曹叡,甄妃的儿子。甄妃被曹丕赐死后,曹睿对母亲十分思念。有一次打猎,曹丕射杀了一只母鹿,旁边的小鹿围著母鹿哀鸣。曹丕催促曹叡射鹿,可是曹叡哭着说:``您已经杀了它的母亲,我怎忍心再杀害这失去母亲的小鹿呢?''
  曹丕听罢,把弓扔到地上,对曹叡另眼相看,``而树立之意定''。
\item
  \emph{于甄氏}:疑衍文,可删去。
\item
  \emph{缪袭}:字熙伯,山东兰陵人,后官至尚书、光禄勋等。
\item
  \emph{罔极}:无极;无穷无尽。前面引用过的《诗经 -
  蓼莪》有``欲报之德,吴天罔极''句,所以罔极可指父母的恩德像天那样无穷无尽。
\item
  \emph{曾、闵}:前面介绍过的大孝子曾参、闵子骞。曾参曾经被父亲曾皙痛打,孔子劝说曾参小杖则受,大杖则走。历史上还有很多故事描写曾参的孝。譬如据说曾参每外出,其母若有事,便会自咬手指,曾参即会感到疼痛马上回来,可谓是心电感应的佐证。想来曾母是无疾而终的,不然曾参岂不是也要病死。还有故事说,曾皙喜欢吃羊枣,曾皙死后,曾参再也不吃这种枣了,因为一吃羊枣,就会睹物思人,悲痛万分。曾参曾说过:``当我官轻禄薄的时候,我很高兴,而是因为父母都健在。当父母去世后,我去南方做了高官,尽管得到厚禄,却依然不时地向着北方哭泣,哭的是父母都不在了啊!所以家里若贫穷而双亲年老,也不要选择做什么高官。''
  曾参提出,儿女不要等父母去世后才想到要孝顺他们,那样已经追悔莫及。他还将孝道分为能养(使双亲温饱无忧)、弗辱(使双亲不受屈辱)、尊亲(让双亲地位尊崇)三个等级,并谦虚地说自己仅仅能养,不敢说孝。闵子骞也是``二十四孝''中的``芦花孝子'',对凶恶的后母很好。在(《论语
  -
  先进》)中,孔子赞颂闵子骞时说:``闵子骞是一位真正的孝子,别人对于他的父母、兄长赞赏他的话都不持异议和怀疑。''
\item
  \emph{情钟舅氏}:感情集中在母家。
\item
  \emph{渭阳}:渭水北边。语出《诗经 -
  渭阳》:``我送舅氏,曰至渭阳。何以赠之?路车乘黄。我送舅氏,悠悠我思。何以赠之?琼瑰玉佩。''这首诗据说是描写春秋时秦康公为送别舅舅晋文公重耳,思念起亡母时作的,后人以此说明见舅氏而怀念亡母之情。曹叡为舅家建大宅子,也是为纪念亡母,因此缪袭以为应该根据这两句诗的意思来起名。这种引用典故的手法非常机巧、文雅,令人回味,是高明的修辞手法。
\end{itemize}

\section{2.14}\label{section-60}

\begin{quote}
何平叔云:``服五石散,非唯治病,亦觉神明开朗。''
\end{quote}

\begin{itemize}
\tightlist
\item
  \emph{何平叔}:何晏,字平叔,河南南阳人,汉末大将军何进的孙子,曹操的养子兼女婿(其母貌美,夫死后被曹操收纳为妻妾)。他勤于专研老庄,引导一代学术风气,成为最早的玄学家和名士的领袖之一。据说何晏风采极为出众,是当时最英俊的小白脸,人见人爱。他后来成为曹魏王朝的骨干之一,历任侍中、吏部尚书等,最终被司马师所杀。
\item
  \emph{五石散}:又叫寒食散,据说是汉代神医张仲景留下的主治伤寒病的一个方子,何晏经过改良后发扬广大。古代的伤寒病与现代的病名有区别,它包括一切时冷时热的传染病,不仅仅是现在的肠道感染疾病。五石散其成分说法不一,如《抱朴子》中说为丹砂、雄黄、白矾、曾青、磁石;而《诸病源候论》则认为是石钟乳、硫黄、白石英、紫石英、赤石脂。中国古人的思维方式是吃啥补啥,石性当坚,认为经过处理服用这些石头就能够得到``轻身益气、不老延年''。而且也许这些东西的确能够压制冷热病症,但现在看来,重金属中毒大约是免不了的。由于身体差异、及时发泄和其他药物的中和作用等原因,五石散也未必会致命,甚至可能也有些好处,不然很难解释有些炼丹师和服用者长寿的现象。
\item
  \emph{治病}:五石散能增加体力,提高性能力,就像晋代皇甫谧说:``何晏耽声好色,始服此药,心加开朗,体力转强。京师翕然,传以相授。历岁之困,皆不终朝而愈。''(何晏服用五石散,马上体力转强。这一下可轰动了京城,大家争相服用。多年的烦恼一下子就解决了。)唐代孙思邈在《备急千金要方》中说:``有贪饵五石,以求房中之乐''。五石散也有很大的坏处,皇甫谧也谈到药性太热:``又服寒食药,违错节度,辛苦荼毒,于今七年。隆冬裸袒食冰,当暑烦闷\ldots{}\ldots{}或暴发不常,夭害年命,是以族弟长互,舌缩入喉;东海王良夫,痈疮陷背;陇西辛长绪,脊肉溃烂;蜀郡赵公烈,中表六散,悉寒石散之所为也。''不过这些危害是潜伏的,人往往先图一时欢愉,``众人喜于近利,未睹后患。晏死之后,服者弥繁,于时不辍。''
\item
  \emph{神明开朗}:其实就是服用后感觉异常敏锐,神游四海,灵感爆发。据说历史和现实中不少文艺工作者都是食用毒品中爆发创作激情的。这些矿物质价格不菲,所以主要流行在上流社会,到了唐代以后,大概有新的特效药发明,五石散逐渐退出市场。在魏晋时代,五石散也是抵抗礼教的一种武器,成仙的理由超越了恪守圣人之道的要求,贵族以此为借口争取到了堂皇的行为自由。
\end{itemize}

何晏的这句话放在``言语''中很难理解,难道是因为解释五石散功效很到位,有效掩盖了其私密性?

\section{2.15}\label{section-61}

\begin{quote}
嵇中散语赵景真:``卿瞳子白黑分明,有白起之风,恨量小狭。''赵云:``尺表能审玑衡之度,寸管能测往复之气。何必在大,但问识如何耳。''
\end{quote}

\begin{itemize}
\tightlist
\item
  \emph{嵇中散}:中散大夫嵇康。
\item
  \emph{赵景真}:赵至字景真,山西人。小时候流亡到河南,受到母亲的鼓励有志于学,14岁时认识了嵇康,仰慕交好,跟随学习。据说他识断高明,判案精当。后来母亲去世,悲痛而死。
\item
  \emph{白起之风}:晋孔衍《春秋后语》中记载,公元前262年,韩割上党地与秦,韩守将冯亭不愿降秦,献上党十七邑与赵国,欲引赵国抗秦。孝成王大喜,但平阳君赵豹认为不可接受。孝成王召平原君赵胜等商议此事。平原君赵胜对赵孝成王说:``在秦赵的渑池之会中,我观察了武安君白起的相貌。他小头而锐,瞳子白黑分明。小头而锐,断敢行也;瞳子白黑分明者,视事明也;视瞻不转者,执志强也(看东西往往直视,目不转睛,说明他意志坚定)。可与持久,难与争锋。廉颇为人,勇鸷而爱士,知难而忍耻,与之野战则不如,持守足以当之。''于是赵国接受了上党,最终爆发长平之战,赵括上台演出失败。
\item
  \emph{恨量小狭}:长相像白起,不过就是眼睛小了点。
\item
  \emph{玑衡}:璇玑玉衡的简称,古代天文学术语。大概就是像张衡制造浑天仪观察天象,或者用日晷月晷来探求日月星辰和地球的时光季节变化规律。
\item
  \emph{寸管}:古代考察季节的办法。``其候气之法,十有二月,每月为管,置于地中。气之来至,有浅有深,而管之入地者,有短有长。十二月之气至,各验其当月之管,
  气至则灰飞也。其为管之长短,与其气至之浅深,或不相当则不验。上古之圣人制为十二管,以候十二辰之气,而十二辰之音亦由之而出焉。''大意是说在地上埋管子,由于时间变化,十二根管子的反映是不一样的。
\end{itemize}

赵至回答的大意是:一把尺子、一张图表、一根管子可以观察度量复杂的天象。你何必在乎我眼睛的大小,只考评我的学问识见如何就是了。赵至曾经写过一篇著名的文章给嵇康的侄子嵇蕃《与嵇茂齐书》,其中说自己``顾影中原,愤气云踊。哀物悼世,激情风烈。龙睇大野,虎啸六合。猛气纷纭,雄心四据。思蹑云梯,横奋八极。披艰扫秽,荡海夷岳。蹴昆仑使西倒,蹋太山令东覆,平涤九区,恢维宇宙,斯亦吾之鄙原也。时不我与,垂翼远逝,锋钜靡加,翅翮摧屈,自非知命,谁能不愤悒者哉?''看来是有远大志向的人物,文笔也的确令人荡气回肠,可惜悼母而死。

\section{2.16}\label{section-62}

\begin{quote}
司马景王东征,取上党李喜以为从事中郎。因问喜曰:``昔先公辟君,不就,今孤召君,何以来?''喜对曰:``先公以礼见待,故得以礼进退;明公以法见绳,喜畏法而至耳。''
\end{quote}

\begin{itemize}
\tightlist
\item
  \emph{司马景王}:司马师,司马懿的大儿子,继承了司马懿的事业,司马懿死后封长平乡侯,任大将军。公元255年,他废除曹芳,立曹髦为皇帝。扬州刺史毌丘俭在起兵反对他,被他打败。司马师途中病死。拘于条件,司马家的篡位事业有弟弟司马昭继承。晋朝建立后,追尊为景王、景皇帝。
\item
  \emph{东征}:东征大概就是指攻打扬州刺史毌丘俭的事情,扬州在洛阳的东面。
\item
  \emph{李喜}:或作李憙,字季和,山西沁县人,司马懿时候屡辟不就,司马师一征就来,因为他已经看出世道变了,再不给面子下场不妙。李喜据说当官清正,不惮强御,家无积蓄,后来再晋朝做到太子太傅、尚书仆射、金紫光禄大夫等。
\end{itemize}

这段对话有嘲讽的意味,天下哪有不愿当官要法律制裁的道理,由此可见司马师当时的强势和统治的名头,所以也我们不难理解阮籍的诸多咏怀诗表达的痛苦心情,``终身履薄冰,谁知我心焦''。《三国演义》中刘备三顾茅庐,张飞就说``今番不须哥哥去;他如不来,我只用一条麻绳缚将来!'',这倒不是什么笑话,后来朱元璋。康熙等大英雄就是这样做的,强拉人家当官,一定要大学者捧场,不然就咔嚓了。现在绝大多数官员,为争一个领导职位,嗯,甚至为争一块骨头也能互相倾轧,丑态百出,也算是另一种焦灼。

\section{2.17}\label{section-63}

\begin{quote}
邓艾口吃,语称``艾艾''。晋文王戏之曰:``卿云`艾艾',定是几艾?''对曰:``\,`凤兮凤兮',故是一凤。''
\end{quote}

\begin{itemize}
\tightlist
\item
  \emph{邓艾}:字士载,河南新野人。早年丧父,家贫,但素有大志。因为出身和口吃一直沉于下僚,40多岁时候有机会就本地的农业问题向太尉司马懿汇报,得到赏识,从此舒展抱负,成为三国时候著名的军事家、政治家,对魏国的经济发展和攻克蜀国作出了决定性的贡献。邓艾20年间飞黄腾达,从一个没有家族背景的屯民到历任关内侯,镇西将军,兖州刺史、邓侯、太尉等,与司马家族的赏识是分不开的。但邓艾和大多数同事的关系并不好,后来因此受到钟会、卫瓘的陷害被杀。
\item
  \emph{艾艾}:有个成语叫``期期艾艾'',就是描写历史上著名的两个口吃者。另外一个叫周昌,是西汉刘邦时候的御史大夫,刘邦要另立太子,周昌坚决抵制,说``臣期不奉诏''。连续用了两个``期''字,本来一件非常剑拔弩张,甚至会血溅庙堂的大事,在周昌口吃的情况下缓和了气氛。
\item
  \emph{晋文王}:司马昭,字子上,儿子司马炎建晋朝,追尊司马昭为文帝。
\item
  \emph{定}: 确定,引申为到底。
\item
  \emph{凤兮凤兮}:《论语》中说,孔子周游列国时候,楚国陆通(字接舆),对孔子唱歌道:凤兮凤兮,何德之衰?往者不可谏,来者犹可追!已而!已而!今之从政者殆而!在《庄子》中,这首歌还要长一些和激烈一些。
\end{itemize}

《世说》中有很多时候是用典故来表达自己的意见,这种修辞手法大量见于文言文和诗歌中,有时候这些典故过于冷僻,以至于很多人看不懂作者到底是什么意思。有的作者因担心自己写的别人不懂,就自注,这个习惯好,譬如我,曾经读过几首陈寅恪的诗,实在无法理解。据说好的诗歌是化用典故而读者不觉阅读障碍,陈先生显然做不到这一点。

历史上很多名人都口吃或曾经口吃,委婉的说法叫口讷。钱钟书曾介绍过韩非、司马相如、杨雄、范晔等大作家,其实现代的学者像巴金、柳亚子、朱自清、沙叶新、余华等也是这样。西方也有很多名人表达不是十分顺畅,有名的像伊索、达尔文、华盛顿、韦尔奇等。鲁迅先生曾经在《故事新编》中用口吃来嘲弄顾颉刚,文风是油滑不厚道,可见鲁实在是恨极了顾。也有回忆文章说鲁迅也有点口吃的。  

\section{2.18}\label{section-64}

\begin{quote}
嵇中散既被诛,向子期举郡计入洛,文王引进,问曰:``闻君有箕山之志,何以在此?''对曰:``巢、许狷介之士,不足多慕。''王大咨嗟。
\end{quote}

\begin{itemize}
\tightlist
\item
  \emph{向子期}:向秀,字子期,河内怀县人,经山涛引见结识嵇康,极为知己。魏晋时期名士虽多,一天到晚谈天,但大多没有留下理论文章,向秀作《庄子注》,好朋友吕安就夸他``庄周不死'',算是扎扎实实推动玄学的。嵇康、吕安被杀后,他被迫做官捧场,后来历任散骑侍郎、黄门散骑常侍、散骑常侍等。``在朝不任职,容迹而已''。
\item
  \emph{举郡计}:当时每个郡要设立一个郡计椽和郡计吏,大概相当于现在的统计局领导。每年年末,按要求要到洛阳送统计报表。这句话有点费解,按当时规定每个郡每年要向朝廷推荐一个``贤良'',举大概指向秀是这年的人选,难道他平时职业就是统计员?
\item
  \emph{箕山}:传说中尧要把酋长的位置出让许由,许由就跑到箕山洗耳朵,当时箕山还有个隐士巢父,两人合演一处极为做作的话剧。后来箕山代指归隐山名,在今河南省登封县。尧时巢父、许由在箕山隐居。这里说箕山之志,就是指归隐之志。
\item
  \emph{狷介}:《论语 - 子路》子曰:``狂者进取,狷者有所不为也。''《庄子
  - 庚桑楚》;``夫函车之兽,介而离山,则不免于网罟之患。''《汉书 -
  注》:``申徒狄,殷之末世介士也。''
  狷介同义复词,孤独、耿直、洁身自好。
\item
  \emph{咨嗟}:赞叹。欧阳修《相州昼锦堂记》:``夹道之人,相与骈肩累迹,瞻望咨嗟。''《世说新语
  - 言语》:``玄咨嗟称善。''咨嗟常见的还有叹息的意思。
\end{itemize}

向秀这话隐含着赞美司马昭是尧舜,夸当权者是鸟生鱼汤本来就是最好的恭维,自然``龙颜大悦''。向秀出山后,标志着竹林七贤的``广陵散绝矣''。现代社会,只要新闻管制和职称评定一出,不要说``绝响'',连出``七贤''也没有可能。

\section{2.19}\label{section-65}

\begin{quote}
晋武帝始登阼,探策得一。王者世数,系此多少。帝既不说,群臣失色,莫能有言者。侍中裴楷进曰:``臣闻天得一以清,地得一以宁,侯王得一以为天下贞。''帝说,群臣叹服。
\end{quote}

\begin{itemize}
\tightlist
\item
  \emph{登阼}:又叫践阼,登位称皇帝。阼原指大堂前东边的台阶,皇帝登上太庙台阶主持祭祀。
\item
  \emph{探策}:占卜。策指占卜用的蓍草。用龟甲占卜称为``卜'',用蓍草占卜称为``筮''。古人很重视占卜,《史记》中就有《龟策列传》,其中说``闻蓍生满百茎者,其下必有神龟守之,其上常有青云覆之。传曰:`天下和平,王道得,而蓍茎长丈,其丛生满百茎。'方今世取蓍者,不能中古法度,不能得满百茎长丈者,取八十茎已上,蓍长八尺,即难得也。人民好用卦者,取满六十茎已上,长满六尺者,既可用矣。''据说占卜时用五十根长得好的蓍草,去掉一根,随意分半后反复抽数,推算数十次,最后算出一爻一卦(我们似乎没有必要知道操作细节,现代迷信的人也已直接简化抽签或随手扔制钱了)。司马炎这次占卜得到的是第一爻乾卦初九,乾卦虽然总体是吉卦,但其初九爻爻辞是``潜龙勿用'',一般理解为你司马炎还是潜龙,还不是当皇帝的时候,击中了司马炎篡位的要害;他问的又是帝位能传多少代,答案只能传一代,非常不吉利。大家吓得不敢触霉头,说不出话来。
\item
  \emph{世数}:指帝位传承多少世代的数目。
\item
  \emph{裴楷}:见1.18。裴楷以精通《老》、《易》著名。如果是汉儒,应该按照象数学说来解释卦辞,而裴楷用道家学说注释,就上升到了哲学高度,不再拘泥于事实。其大意是说天得道而清明,地得道而宁静,侯王得道而成为天下的首领。
\item
  \emph{``天得一''句}:引自《道德经》三十九章。这里的``一''就是``道'',以为天地侯王都是来源于道,有了道,才能存在。裴楷引用略有删节,当时玄学家王弼写过《老子注》,其中说:``一者,数之始,物之极也。各是一物,所以为主也。各以其一,致此清宁贞'',裴家与王弼为师友之间,想来知道这本书,所以裴楷在引用时省略了``神得一以灵,谷得一以盈''句。按道理朝廷中有的是大学者,《老子》自然很熟悉,但只有裴楷跳出来应急安慰兼拍马,估计当时用道家学说贯通汉代《易》学还是很少见的。
\end{itemize}

占卜古人是很当真的,这样行为有时就不免受到羁绊。现在流传下来的故事往往都是占卜如何神妙,不准的就自动过滤掉。我们现在还可以看到,有些人有迷信思想,常讲一些应验的事情,大肆宣传。即使向钱钟书这样的学者也不能免俗,在杨绛的《干校六记》里就写过钱钟书占卜的故事。当然,如果大政方针定了,古人也未必会一味迷信,而采取另一种解释。譬如武王伐纣,占卜得``大凶'',吕尚就说,吊民伐罪,天下大道!当为则为,当不为则不为,何祈于一方朽物。后来无论下大雨还是风断旗杆,都采用了不利于商的解释,``天洗兵''、``天落兵''------天助我也。后世北魏孝文帝想迁都,有人提议要卜筮,孝文帝立即反驳说:``卜筮是古代的做法。过去周公、召公用卜筮的方法,选择伊、洛地区建都城。可现在没有周公、召公那样的人了,用不着占卜了。况且占卜的目的是解除疑难,而现存迁都洛阳是既定的方针,占卜还有什么用呢?''名将吴起也说过,出兵作战有八种情况不需要占卜,如天气恶劣、物质紧缺、人心浮动、兵处险地等,因为一占卜更加可能士气大落。这对我们也很有启示,占卜这个东西,得意时得个凶卦,警醒一下;失意时得个吉卦,保留点希望;倘若以此作为行动指南,必受其害。

说句题外话,台湾实行民主政治是好的,但台湾文化氛围中迷信思想很甚,陈水扁、谢长廷等上演过一出出滑稽剧,民间也是处处讲神佛,这样的氛围很不恰当,虽然可以美其名曰``有敬畏感'',但终究缺乏光明的前途。

\section{2.20}\label{section-66}

\begin{quote}
满奋畏风。在晋武帝坐,北窗作琉璃屏,实密似疏,奋有难色。帝笑之。奋答曰:``臣犹吴牛,见月而喘。''
\end{quote}

\begin{itemize}
\tightlist
\item
  \emph{满奋}:字武秋,他的祖父叫满宠,曾与曹仁一起坚守樊城,抵挡住了关羽的进攻。满奋先后当过吏部郎、司隶校尉、冀州刺史、尚书令等。满奋胖得夸张,据说夏天热得肌肤开裂,流出膏油,小妾把油脂收集起来,用来当蜡烛。满奋长得胖,又怕风又怕热,不知是什么疾病。
\item
  \emph{琉璃屏}:应该就是水晶玻璃窗,由于工艺问题,它在清以前一般是珍贵物品,常人没接触过。晋代葛洪的《西京杂记》中曾说:``﹝昭阳殿﹞窗扉多是绿琉璃,亦皆达照,毛发不得藏焉。''但司马昭宫殿的玻璃也许是无色的,因为同样在《西京杂记》中说:``白光琉璃为鞍,鞍在暗室中,常照十余丈如昼日。''有些东西在古时候是宝贝,而现在却很平常,据说铝锅在拿破仑时代比金银锅要珍贵得多,在列宁笔下,黄金也就当当马桶了:``我们将来在世界范围内取得胜利后,我想,我们会在世界几个最大城市的街道用黄金修建一些公共洗手间,这样使用黄金,对于当今几代人来说最是`公正'而富有教益的,因为他们没有忘记,怎样由于黄金的缘故,在1914-1918年``伟大的解放的''战争中,即在为了解决是布列斯特和约坏些还是凡尔赛和约坏些这个重大问题的战争中,曾使1000万人死于非命,3000万人变成残废;怎样又是由于黄金的缘故,不知是在1925年前后还是在1928年前后,是在日美之间还是在英美之间的战争中,或者在诸如此类的战争中,一定还会使2000万人死于非命,6000万人变成残废。''想当年天庭编制臃肿,大伙儿长生不老,就连一个拉帘子的小厮也熬到了大将级别,结果在兼职酒保时失手打坏玻璃杯,被贬下凡间吃人去了,还每周一次遭百余飞剑贯胸,比普罗米修斯痛苦、无聊和卑微多了。
\item
  \emph{吴牛}:水牛的别称,大约只有长江中下游有,所以叫吴牛。水牛怕热,夏天不用耕田,得成天待在水里凉快。据《太平御览》的《风俗通》中说,水牛到了晚上看到月亮又圆又亮,形成条件反射,还一个劲喘气。
\end{itemize}

满奋自比为喘气的牛,敢于自嘲。自嘲的人往往幽默、智慧和有坚定的信心和相对超脱的胸怀,也容易获得朋友。孔子就认为别人说他是丧家犬,``对极了!对极了!''    

\section{2.21}\label{section-67}

\begin{quote}
诸葛靓在吴,于朝堂大会,孙皓问:``卿字仲思,为何所思?''对曰:``在家思孝,事君思忠,朋友思信,如斯而已。''
\end{quote}

\begin{itemize}
\tightlist
\item
  \emph{诸葛靓}:字仲思。父亲诸葛诞,是诸葛亮的堂兄弟,在魏国当扬州刺史,大约忠于魏国,受到司马昭的猜疑,要召入朝廷当司空,于是起兵造反,并把儿子诸葛靓送到吴国作为合作的人质,旋兵败被杀。诸葛靓在吴国做到了右将军、大司马。
\item
  \emph{孙皓}:字元宗。孙权的长孙,吴国第四代皇帝兼末代皇帝。这是一个奇特的人物,主要特点是随心所欲。他当皇帝有些侥幸,上台后不知收敛,极尽个人的欢乐,没有统一天下的雄心壮志,因为个人意气大杀臣僚,穷奢极欲,做了不少让人费解的事情,不免身俘国灭。但也有不少故事说明孙皓有真性情,率真直接,对得起自己的人生。
\item
  \emph{``三思''}:在这里诸葛靓对《论语》中的``三思而后行''作出了自己的解释,这个解释也沿用了同样出自《论语》中曾参的``吾日三省吾身,为人谋而不忠乎?与朋友交而不信乎?传不习乎'',不过是把孝突出出来。曾参不讲孝,是不用讲孝,因为孝已经内化为曾参的自觉力量,是他的金字招牌。而旁人一般做不到,得不断提醒自己。
\item
  \emph{如斯而已}:字面上没有难以理解的,但同样出于《论语》:子路问君子。子曰:``修己以敬。''曰:``如斯而已乎?''曰:``修己以安人。''曰:``如斯而已乎?''曰:``修己以安百姓。修己以安百姓,尧舜其犹病诸?''
  诸葛靓的回答现代人一般不会了解其中《论语》的背景,但当时人却很能会心赞赏。
\end{itemize}

吴亡后,诸葛靓回到洛阳,因为父亲是被司马炎的父亲杀死,所以发誓不仕,据说终生背向皇宫而坐,因为有``孝''这块大牌子档着,朝廷也拿他没办法。诸葛靓和皇帝司马炎是儿时的玩伴,按辈分还是司马炎的舅舅,所以司马炎想见他叙旧,但诸葛靓不合作。司马炎就叫自己的婶婶、诸葛靓的姐姐琅玡王妃宴请诸葛靓,自己也出席。于是故事有了两个版本,一个是诸葛靓一见司马炎,就借口上厕所,拍拍屁股走人------我就是不见。另一个是在宴会中司马炎问诸葛靓:``你还记得我们儿时一块玩的欢乐吗?''诸葛靓毫不客气地说:``臣不能吞炭涂身,今日复睹圣颜!''他的意思就是,我很惭愧不能学习春秋时期的豫让,吞炭涂身来暗杀你!这个故事可以作为诸葛靓思孝的注脚。

\section{2.22}\label{section-68}

\begin{quote}
蔡洪赴洛。洛中人问曰:``幕府初开,群公辟命,求英奇于仄陋,采贤俊于岩穴。君吴楚之士,亡国之馀,有何异才而应斯举?''蔡答曰:``夜光之珠,不必出于孟津之河;盈握之壁,不必采于昆仑之山。大禹生于东夷,文王主于西羌,圣贤所出,何必常处!昔武王伐纣,迁顽民于洛邑,得无诸君是其苗裔乎?''
\end{quote}

\begin{itemize}
\tightlist
\item
  \emph{蔡洪}:字叔开,江苏人,吴亡后由本州举荐为秀才,到京都洛阳。这个故事另见于《晋书
  -
  华谭传》,主人公华谭,挑衅的人落实为王济。华谭家世好,名声大,王济出身高贵,生活奢侈,一向看不起人,好言语伤人,似更有可能。
\item
  \emph{幕府}:为便于行军,中国古时军队主将的府署设在帐幕内,因称幕府,后也称军政大员的官署。
\item
  \emph{仄陋}:或作侧陋。《论衡 - 吉验》:``舜未逢尧,鳏在侧陋。''《汉书
  -
  元帝纪》:``延登贤俊,招显侧陋,因览风俗之化。''指处在僻陋之处或微贱地位的贤人。汉代哀帝时有诏书说:``盖闻圣王之治,以得贤为首。其与大司马、列侯、将军、中二千石、州牧、守、相举孝弟厚能直言通政事,延于侧陋可亲民者,各一人。''
\item
  \emph{岩穴}:山中洞穴,指隐居山中的隐士。汉代章帝时有诏书说:``公卿以下,其举直言极谏,能指朕过失者各一人;遣诣公车,将亲览问焉。其以岩穴为先,勿取浮华。''这两句同义反复,为互文。现代公文中文采虽然不如古代,但同义反复的修辞手法更加发扬光大。海南有位不学无术的中级官僚,居然写了本《成语典故》,语言学家、全国人大副委员长许嘉璐先生作序,大肆夸赞,相映成丑,令人作呕,令后学不知斯文为何物哉!其实中国古时候一些皇帝的罪己诏倒可以与美国元首的演讲词相媲美,现代公文则不可以以道计。
\item
  \emph{夜光之珠}:隋侯珠。相传春秋时代隋国国君访问齐国途中救了一条巨蛇,蛇从江中衔来一颗宝珠。孟津:渡口名,在今河南省,这句话好像在讽刺自高自大的中原人。
\item
  \emph{盈握之壁:}一个手掌大小的玉璧,大概是指和氏璧,产于楚国。李斯《谏逐客书》中说:``今陛下致昆山之玉,有隋、和之宝,垂明月之珠\ldots{}\ldots{}''。昆仑山在传说中产美玉,周穆王驾着马车带回,给中原人见识过,现代一般落实为和田玉。
\item
  \emph{大禹}:大禹的出生地历来传说纷纭,一般认为是四川。这里的东夷是指东部沿海一带,可能采用了另一种说法,我们不必在意,反正按顾颉刚的观点,禹不过是图腾崇拜,是条虫。
\item
  \emph{西羌}:代指西部少数民族地区。
\item
  \emph{常}:永久、固定不变。《诗经 -
  大雅》``天命靡常。''《墨子》``故官无常贵,而民无终贱。''《师说》``圣人无常师。''
\item
  \emph{迁顽民于洛邑}:《洛阳城内伽蓝记》中说:``洛阳城东北有上高景,殷之顽民所居处也。高祖名闻义里,迁京之始朝士住其中,迭相几刺竟皆去之,唯有造瓦者止其内,京师瓦器出焉。世人歌曰:洛阳城东北上高里,殷之顽民昔所止。今日百姓造瓮子,人皆弃去住者耻。''《尚书
  -
  传》中说,周武王借口吊民伐罪,推翻商朝,怕不服从周朝统治的商人反抗,``惟殷顽民,恐其叛乱,故徙于洛邑,密近王室,用化其教。''郭沫若《中国古代社会研究》中说:``殷人被征服了以后事实上是作了奴隶,他们算是受尽了轻视和虐待的,周室的人称他们为`蠢殷',称他们为`顽民',一直到春秋战国的时候都还把他们的后人当蠢人看待。''
\end{itemize}

蔡洪的回答前半段来自《谏逐客书》的模样,后半段结合洛阳的来历,骂得痛快,所以刘孝标的注中记载,蔡洪一辈子也就当了个县令。如果主人公是华谭,结局好很多,去世的时候赠光禄大夫,金章紫绶,加散骑常侍,谥号``胡'',书上说:``弥年寿考曰胡
保民耆艾曰胡。''

\section{2.23}\label{section-69}

\begin{quote}
诸名士共至洛水戏,还,乐令问王夷甫曰:``今日戏,乐乎?''王曰:``裴仆射善谈名理,混混有雅致;张茂先论《史》《汉》,靡靡可听;我与王安丰说延陵、子房,亦超超玄著。''
\end{quote}

这是一次像后来兰亭集会那样大腕云集的盛会,里面5个人全是高官兼学者,个个个性鲜明,才华显于当世,为士林的领袖,彼此间还有点血缘或亲属关系。乐令指乐广,字彦辅,后来是朝中的尚书令,王夷甫指王衍,后来是朝中的太尉、尚书令;裴仆射指裴頠,字逸民,朝中的尚书左仆射;张茂先指张华,中书令,广武候;王安丰指王戎,吏部尚书,安丰侯。这里称呼非常混乱,也许有什么春秋笔法在里面,我不了解。也有版本说里面不是王衍,而是王济。

从这个故事中我们也可大约知道四个清谈者擅长的学问是什么。裴頠谈名理,就是考察名与实的关系,魏晋时期讲才性,强调辩名析理来考察人物。张华学问广博,大部头的《史记》、《汉书》了如指掌。王衍和王戎谈论季扎、张良,这两个人物都是名声显赫、不愿意据高位的人,可能反映他俩的志趣。这些名角的故事我们曾经接触过,以后还会接触,但总的来说性情相对都比较超脱,渴望在乱世中又能有所作为、又能保全自己和背后的家族、又能保持优美的情趣,但最终基本没有如愿。他们有学问,却把学问当游戏;他们权势极大,却没有力量。他们是混乱时代的参与者、受害者,过着轻飘飘的人生。正始之音在耳,建安风骨已绝!不过回头想想,省部级以上高官业余时间聚会,大谈学问,今天是绝看不到的,打牌、喝酒、字画、女性!
-
\emph{洛水戏}:戏不能理解成游戏,而是专指清谈,《世说》中多处可证。在戴逵《竹林七贤论》中落实为三月三的修禊事也。练过几天字帖的都应该了解修禊之大概。
-
\emph{混混}:形声兼会意字。《说文》``混,丰流也。''水大貌,滔滔不绝。靡靡:靡有美好的意思,指讲话娓娓动听,是啊,生活糜烂是多么令人向往的名词啊。超超:高远拔俗。
-
\emph{延陵}:指贤人季扎。春秋时吴王寿梦有四个儿子:诸樊、余祭、夷昧、季扎。寿梦临死之时欲立季扎,季扎坚辞。于是寿梦将王位传给诸樊,并约定兄终弟及,必使季扎取得王位。诸樊立刻要把王位让给季扎,季扎拒绝。诸樊想:``我要是活到老才死,按次序传位,传来传去,四弟还能继承王位吗?我得另想办法。''于是他亲自率领吴军攻打楚国,战死。接着是余祭,也是同样的做法。三子夷昧死后把王位传给自己儿子僚,于是爆发鱼肠剑的故事。阖闾篡位后还是假装以王位相让季扎。季扎说:``费心谋求,怎能让人。我若嗣位,早成大典。''季扎守臣位,终以政变为国耻,从此退避江苏延陵,老死。据说孔子推崇季扎,曾手书碑文:``呜呼!有吴延陵君子之墓。''当然,说他是君子,还有一个季扎挂剑的故事,不必在此细表。
- \emph{子房}:张良字子房。

\section{2.24}\label{section-70}

\begin{quote}
王武子、孙子荆各言其土地人物之美。王云:``其地坦而平,其水淡而清,其人廉且贞。''孙云:``其山崔嵬以嵯峨,其水浃渫而扬波,
其人磊砢而英多。''
\end{quote}

\begin{itemize}
\tightlist
\item
  \emph{王武子}:王济,字武子,太原王氏,父亲王浑(此王浑不是王戎父亲那个王浑)在征伐东吴时有很大的功劳,王济是司马炎的女婿。王济在《世说》中故事很多,奢侈、骄横、潇洒、口才出众,有识人之明等,被世人认为是``豪俊公子'',曾任中书郎、太仆、骁骑将军、侍中等职。
\item
  \emph{孙子荆}:孙楚,字子荆,同是太原人,受到王济的赏识,40多岁后终于出仕,逐渐做到冯诩太守(今陕西)。孙楚傲气直率,对以品第门阀任人看不惯。有一次传言说有人在井中发现了龙。朝臣有的认为是吉祥之兆,上表称贺。孙楚上表借题发挥,认为这是不能举贤任能的暗示,``夫龙或俯鳞潜于重泉,或仰攀云汉游乎苍昊,而今蟠于坎井,同于蛙虾者,岂能管库之士或有隐伏,厮役之贤没于行伍。故龙见光景,有所感悟。''这样的人当领导则可,当随员需要上级有些雅量的。我们从这段对话中也可略窥其貌。
\item
  \emph{崔嵬以嵯峨}:一个意思,同义反复,高大险峻。《诗经 -
  卷耳》``维山崔嵬'',《史记 - 司马相如传》``崔巍嵯峨''。
\item
  \emph{浃渫而扬波}:一个意思,河水流动貌。水波连续的样子。太原的主要河流应该是汾河,但估计河流或者地段不同,河流估计也有不同的表现。
\end{itemize}

以上所有形容词都是双双同义反复,这也是古汉语特别是诗歌的特点,现在看完全可以省略成``其地平,其水清,其人廉'',如此反复无非就是突出强调和抒发强烈的感情,形成复沓之美,但这个界限很难把握,譬如我们写``一个姑娘,美丽又漂亮'',老师一定是一个大大的问好号,``你以为一棵是枣树,另一棵也是枣树啊!你谁啊!'',不过用的好也有,譬如``弃我去者,昨日之日不可留;乱我心者,今日之日多烦忧'',颇能反映徘徊悱恻的心情。

这是个地域贴,我曾经收集过不少地域与文化、人物之间的关系的评论,认认真真做过笔记,后来马齿渐长,不敢再谈文化的话题了------失之空疏。

\section{2.25}\label{section-71}

\begin{quote}
乐令女适大将军成都王颖。王兄长沙王执权于洛,遂构兵相图。长沙王亲近小人,远外君子;凡在朝者,人怀危惧。乐令既允朝望,加有昏亲,群小谗于长沙。长沙尝问乐令,乐令神色自若,徐答曰:``岂以五男易一女?''由是释然,无复疑虑。
\end{quote}

  乐令:乐广。我们已经多次介绍过他了,在当时他和王衍一样以玄学、风采闻名,但王衍趋向于道家,恐怕有信奉五斗米教的家风,而乐广趋向于儒家,为人恐怕也方正一些,不过这些印象都建立在大量吹捧性的文字之中,也许当不得真。
  适:嫁。《后汉书 -
列女传》:``夫有再娶之义,妇无二适之文'';《孔雀东南飞》:``贫贱有此女,始适还家门。''
  成都王颖:司马颖,晋武帝司马炎的儿子,封成都王。看来在这里还得介绍一下狗屁倒灶的``八王之乱'':西晋之乱早在司马炎实行王族分封制时就埋下了恶因,司马炎的继位者惠帝又恰巧是智力显然有重大缺陷的人,其妻子贾南风(杀死高贵乡公曹髦的黑手贾充之女)恰巧是飞扬跋扈、野心勃勃的女性,与太后家族特别是托孤之臣、太傅杨骏形成矛盾。贾南风得到汝南王司马亮和楚王司马玮的帮助,诛杀了杨骏一系。而后司马亮和卫瓘为消弱诸王的权势,主张``遣诸王还藩'',于是司马玮杀了司马亮和卫瓘,贾南风又杀了司马玮。接着司马伦杀贾南风,废惠帝自立。而后就是其他司马击败司马伦,惠帝复立,在这个阶段,司马乂专权,为太尉,都督中外诸军事。由于分赃不均,司马顒、司马颖与司马乂又大打出手,这个故事的背景就是在这一时期,公元303年前后。我们只要想想一群野狗恶狼在旷野中抢夺血食的场面,再把时间延长成16年,就会不由自主地想起``天地为炉兮,造化为工;阴阳为炭兮,万物为铜'',命运的无情磨盘把司马炎的几个儿子和大部分大臣碾碎,中国好不容易的统一局面很快葬送掉。不久以后,衣冠南渡,中原物华的局面逐渐转移到了南方。
  长沙王:司马乂。除了晋惠帝司马衷,司马炎的其他几个儿子都好像有两下子的,司马乂以少胜多,打了不少漂亮仗,其中一仗就是打败了大名士陆机。但司马乂政治上不``成熟'',后来被东海王司马越出卖,乘其不备发动兵变,将其俘获,送交司马顒、司马颖。司马乂被放在火堆上烤死,时年28岁。
  构兵相图:为各自的图谋交战。司马乂给弟弟司马颖写信:``吾之与卿,友于十人(第6子和16子),同产皇室\ldots{}\ldots{}卿宜还镇,以宁四海,令宗族无羞,子孙之福也。''司马颖回信说,整个天下都将是我的,怎么会和你分享?一定要决一雌雄。``前遣陆机董督节钺,虽黄桥之退,而温南收胜,一彼一此,未足增庆也。今武士百万,良将锐猛,要当与兄整顿海内。''
  允朝望:确实在朝廷中有声望。乐广和王衍俱名重于一时,``天下言风流者,以王乐为称首焉''。
  
``岂以''句:难道会用五个儿子去换一个女儿?意指如果依附司马颖,五个儿子就会被杀。但史书上只记载了乐广的3个儿子。乐广这样的言辞很有说服力,也说明晋代士人以家族为重的特点。不过据《晋书》上讲,``乂犹以为疑,广竟以忧卒'',可见放达的玄学家只是生活方式上的自由散逸,思想并没有到自由无惧的地步,他们对生活的猥琐态度我们还将陆续接触一些。

\section{2.26}\label{section-72}

\begin{quote}
陆机诣王武子,武子前置数斛羊酪,指以示陆曰:``卿江东何以敌此?''陆云:``有千里莼羹,但未下盐豉耳!''
\end{quote}

  陆机:字士衡,江苏人,名将陆逊的孙子,陆抗的儿子,以文采著称于西晋,大概以为把祖宗的大牌一举,下级迎头就拜,敌人望风而逃,居然投笔从戎,一下子就领军20万,当起成都王司马颖的后将军,河北大都督,参与``八王之乱'',欲恢复祖上的军事光辉。但以南人统率北军,指挥乏力,下级军官多不服调遣,不战而溃。以致司马颖举起屠刀,将江南名士领袖陆机、陆云、陆耽三兄弟处死,``不意三陆相携暗朝,一旦湮灭,道业沦丧,痛酷之深,荼毒难言。国丧俊望,悲岂一人'',从此之后,南人与北人更是格格不入,西晋之世南士入北求仕的活动宣告结束。后来的人认为,三世为将,必遭天谴。想当年张华结识两陆时,说道:``征服江东没啥的,好的是获得了两位才俊!''(``伐吴之役,利获两俊。'')可陆机临死时候感叹:``华亭鹤唳,岂可复闻乎!''他是在后悔到洛阳谋求功名了。
  王武子:王济老兄有一次挑起了地域之争,以军事征服者的身份傲视被征服者,以北人的高第藐视南方的世族,这种风气,贯穿整个两晋南北朝。
  斛:一斛是十斗,一斗6.25千克,看样子当时是吃自助餐的,不然放这么多干嘛。不过王济素来是家财巨万的大爷,``供馔甚丰,悉贮琉璃器中'',真是相当的有钱!
  羊酪:羊酪、莼菜羹、豆豉我倒都吃过,虽然趣味无争辩,但要说它们非常好吃,那是昧良心了。孙隆基的《中国文化的深层结构》中曾经提出中国文化特点之一就是``口腔文化'',还有学者说:``中国口腔文化对味觉的片面强调,最终导致了中国人的饥民心理、饥民人格和身体化生存,深刻地塑造了中国人的人格和文化。中国人因此而陷入了周期性的被吃垮的宿命。\ldots{}\ldots{}一个人要想获得灵魂的自由,首先必须摆脱和超越味觉的囚禁。''你看,夸北方好,就是夸东西好吃,是何等轻慢和无聊。
  
``千里''句:此句历来有争议,甚至有人说千里当做``干里'',未下当做``末下'',分别是地名。文言有时真是像猜谜一样,字个个认识,但就是不知道什么意思。我在这里列举其中几种解释:``江苏千里湖的莼菜羹就很好吃,而且不用像用羊酪一样放盐豉'';``我们南方种有千里的莼菜和羊酪一样美味,如果放了盐豉就比羊酪更好吃'';``南方有干里的莼菜羹,有南京的盐豉(比羊酪还多一样)''。我们如果仔细去考究,那就是腐儒了。不管怎样,吃吧,吃吧,一场人肉的盛宴。

\section{2.27}\label{section-73}

\begin{quote}
中朝有小儿,父病,行乞药。主人问病,曰:``患疟也。''主人曰:``尊侯明德君子,何以病疟?''答曰:``来病君子,所以为疟耳!''
\end{quote}

  中朝:这是东晋对西晋的称呼,大概因为相对于东晋,西晋朝廷在中原地区。这个称呼在南北朝继续通用,也表明其他王朝对西晋正统地位的继承。
  行乞药:这个词奇怪,按后面的称呼,虽然我们可以认为尊侯是尊称,但总不能不靠谱吧,它起码比家君要实在一些,因为家君是泛泛之称,只表明当时家族意识的逐渐强盛,而具体能落实为``侯'',就应该有地位的家庭,这样的家庭为什么要去讨药呢。也许有两个原因,一是``主人''可能指官府,当时有对贵族一定的医疗保障制度;二是``主人''可能指养有医生的极为高贵之家,所以有资格调侃。这要我们去了解当时的医疗制度。但我想,主人不会是指医生。
  何以病疟:古代所谓的疟病不等于现在的疟疾,大概是一切寒热病的总称,譬如《释名》中说:``疟,酷虐也。凡疾,或寒或热耳。而此疾,先寒后热,两疾,似酷虐者。''对于这种相对特殊的周期性的时冷时热疾病,古人不免从鬼怪方面做解释,如《论衡
-
订鬼篇》中说:颛顼氏有三子,生而亡去,为疫鬼。一居江水(地名),为虐鬼;一居若水(地名),是为魍魉鬼;一居人宫室区隅沤库,善惊人,小儿(鬼)。据说在汉代,各地每年都举行仪式来驱逐疫鬼,为解除之法。疟鬼行疟,归入疫鬼范围,也成为岁终驱逐之列。古代把疟鬼当做小儿鬼,就利用猛将凶徒为画像,以惊吓疟鬼,望疾病愈。关于疟鬼的故事说法有很多,很有民俗学的研究意义,有心者可以另加考察,在此不予赘述。要解释的是,古人还认为,壮士不病疟,巨人不病疟,疟疾小鬼只会祸害孩子和穷人(这倒也有一定道理,壮汉抵抗力强,富人讲究卫生,得疟疾的几率的确小)。所以本文中的``主人''故有此问:``令尊不是道德高尚的君子吗,怎么也会得这种(小孩子或者小人才会得的)疟病?''像这种富有进攻性的诘问,贯穿于整本《世说》,我们也就领略到故事主人公的辩论技巧。也许随着佛教的传入、清谈之风的盛行,汉民族全民的思辨能力与口才在此跃上了崭新的境界,就连普通的小朋友都有上佳的表现。我们这个故事中的小朋友也很不错。
  所以为疟耳:此处的``疟''当理解成``虐'',虐、疟本来相通。虐就是残暴的意思,《马王堆汉墓帛书》``静作相养,德疟相成。''于是这句话通过同音同意字做出机巧的回答:``因为疟病来祸害君子,所以我们才说它为虐(残暴)啊!''正是``东边日出西边雨,道是无晴却有晴''。

\section{2.28}\label{section-74}

\begin{quote}
崔正熊诣都郡,都郡将姓陈,问正熊:``君去崔杼几世?''答曰:``民去崔杼,如明府之去陈恒。''
\end{quote}

  崔正熊:崔豹,字正熊,河北人,在晋惠帝时当过管理仪仗马匹的太仆丞,他的名和字要解释一下,古人熊豹是不太分的,姜尚号飞熊,据说是周文王做梦遇见``飞熊豹'',在诗文中,熊豹也往往并称。
  都郡:都郡守的省略语,都有大的意思,都郡指大郡。
  都郡将:余嘉锡说:``以他郡太守兼都督本郡军事也'',
就是军政兼职的长官,不知何出典,所以后面崔豹称他为明府。文革时候也有这样的情况,军区司令员兼地方官,民国时候就不用说了。
  崔杼:春秋时代齐国的大夫,又称崔武子。他在朋友的葬礼上,看到其寡妻棠姜有倾国美貌,不顾舆论和占卜不利等因素,娶进门来。崔杼对齐庄公有拥立之功,私交很好,庄公大约常去崔杼府第,一来二去就用权势与棠姜私通。崔杼得知后大怒,就装病。齐庄公食髓知味,主动来``看望''崔杼,结果被杀。崔杼又把平常庄公亲信大臣皆杀死,立庄公的弟弟为景公,又命太史伯把庄公写成死于疾病。太史伯直书:``崔杼弑其君光。''崔杼杀掉太史伯。当时``太史''这个职位相当于``巫'',由家族继承,太史伯的兄弟们依旧不改,崔杼杀了好几个,最终放过了太史家的老四:自己一怒为红颜,为维护大丈夫之气节,想必能得到后人的理解,你们爱怎么写就怎么写吧。故事还没完,后来崔杼立了棠姜前夫的孩子为家族继承者,引起崔家大乱,几个儿子倚仗各自的势力采取了军事行动,接着外祸又至,最后崔家湮灭,棠姜自缢。前几天网上刊登了一高官情妇的照片,据民调说大多数男人看了照片后觉得值得为这样的女子贪污受贿。此所谓``未见好德如好色者也'',棠姜的美貌令人神驰向往。
  崔杼被认为是乱臣贼子,姓陈的郡守就拿崔豹的姓氏开涮,没考虑到崔杼的骨气,没考虑到彼此的姓氏也是难兄难弟,而且有过之而无不及,于是出丑。
  陈恒:他的事业比崔杼搞得大,造反成功,从此齐国的王权由姜姓转移到了陈姓(或者叫田姓)。陈家是齐国的外来户,从陈国来齐国避祸,受到齐国的保护,陈姓也改名为田姓(古音陈、田不分的)。陈家在齐国善于收买人心,据说他们往外借粮的时候,借时用大斗,还时用小斗,原本的动机大约是想主动融入齐国。但后来陈家越来越受到齐国百姓的拥护,地位越来越高,晏子就警告过齐景公,说国君你拼命敛财,可陈家使劲散财,大家的心可全都向着陈家了。到了陈恒这一代他就当上了左相,有一次右相阚止在路上遇到陈恒族人田逆杀人,把他拘捕,但马上被陈家营救出来。阚止害怕陈家势力,在简公支持下,准备驱逐田氏。陈恒先下手为强,发动政变,劫持简公,杀死阚止。后简公夫妇出逃,被陈恒追兵杀死。从此以后,齐国的国君就是陈(田)家的人了,而且没有受到齐国人的抵制。这两件事用现代人的眼光看当然问题不大,但在古代都是极大的罪恶。纪伯伦在《沙与沫》里写道:``每一个人都是已往的每一个君王和每一个奴隶的后裔。''崔豹以子之矛,攻子之盾,言辞不卑不亢,是有力的反击。
  这个故事和《世说》中不少故事一样,说明晋人言辞往往比较轻薄放肆,也说明欲辱人者终受其辱。

\section{2.29}\label{section-75}

\begin{quote}
元帝始过江,谓顾骠骑曰:``寄人国土,心常怀惭。''荣跪对曰:``臣闻王者以天下为家,是以耿、亳无定处,九鼎迁洛邑。愿陛下勿以迁都为念。''
\end{quote}

  元帝:司马睿,字景文,司马家的旁支,原来是不足道的空头世袭琅琊王,在八王之乱中投靠司马越,逐渐成为安东将军,都督扬州诸军事。西晋亡后司马家的嫡长系消灭殆尽,就由司马睿维持局面。司马睿在琅琊世家王导等的帮助下,建立东晋,5年后去世。
  顾骠骑:顾荣,字彦先,见1.25。顾、陆为江东地位最高的士族,顾荣是司马睿最倚重的江东人,死后赠骠骑将军。骠骑将军汉武帝时始置,给霍去病的。汉晋时期皇帝会给一些重臣以骠骑将军称号,位在三公之上,以示尊重,不一定掌握军队。这个故事有点问题,顾荣死在司马睿即位之前,不当称陛下,事情大约是追述的,或者编造。
  王者以天下为家:《公羊传》中``王者以天下为家'',西汉戴圣的《礼记 -
礼运》中说:``今大道既隐,天下为家,各亲其子,货力为已,大人世及以为礼。''其大意就是从大同社会转为小康社会,丧失了原始社会的公有制,``大道不行'',领袖开始实行``家天下''的家族世袭制。
  耿、亳、洛邑:分别指商朝和周朝屡迁国都的情状。商朝从成汤到盘庚,在耿、亳等地共迁都五次,所以说``无定处''。周武王定都镐京,却把传说中作为国家政权象征的九鼎迁到中国中心位置的洛阳。
  司马睿资历很浅,难以得到江南高门望族的认同,大名士王导就让江东贵族顾荣、纪瞻等配合,烘抬司马睿的地位。司马睿这样问顾荣,一方面固然有寄人篱下的忐忑,另一方面大约也想再试探试探江东人的想法。顾荣非常机灵,所以立刻``跪对'',输出自己的忠诚。不过美中不足的是,顾荣只是一味宽慰,没有说什么我们共同努力、光复神州之类的勉励话语来振奋一下,含义单薄。这个故事中顾荣对还没当皇帝的司马睿肯跪对,固然了得,但这番言辞放在``言语''里面不免逊色三分。

\section{2.30}\label{section-76}

\begin{quote}
庾公造周伯仁,伯仁曰:``君何所欣说而忽肥?''庾曰:``君复何所忧惨而忽瘦?''伯仁曰:``吾无所忧,直是清虚日来,滓秽日去耳!''
\end{quote}

  庾公:庾亮,字元规,骑的卢马的庾亮,见1.31。庾亮在东晋时历元帝、明帝、成帝三朝,先后任职中书和六州诸侯,为东晋前期的皇亲国戚、股肱之臣。
  伯仁:周顗,字伯仁,河南汝南人,爵武城侯,少年出仕,曾任吏部尚书,尚书左仆射等,也称周侯。几年前记得天涯社区有个帖子说自己出身显赫,是汝南周家之后,``最不能容忍的是一个暴发户在我们面前装上流社会'',大约就是指这个周家,鲁迅先生据说也是汝南之后。小时候看梁羽生的书,其中有句``我不杀伯仁,伯仁却因我而死'',像什么``凤点头''、``细云巧翻身''一样频繁出现,当时不知道它的准确意思,到后来看了《世说》才了解,原来说的就是周顗啊。周顗口才很好,名望很高,是《世说》中的最主要人物之一,我看他的故事,似乎不像那种热情洋溢的北方人,倒像言辞中暗藏锋芒的会稽人,这个故事就是这样。当时一般舆论认为周顗是一个正义使者,直言无忌,``正体嶷然'',人人都敬畏他,但有时候周顗的直率到了奇特的地步。有一次尚书纪瞻请客,很热情,让自己的爱妾表演歌舞,周顗大约五石散症发,看到美女也太喜欢了,就突然脱得精光扑了上去,要当众发泄性情。开无遮大会也得讲个自愿不是,像周顗这样莫名其妙的,可能是当时风气的极端代表。记得美国前总统卡特有句名言,``我看到漂亮年青的女性时,便有强奸她的冲动'',每次我看到关于卡特的新闻,就心怀敬意。
  直是:正是、只是。杜牧《云梦泽》``直是超然五湖客,未如终始郭汾阳'',卢炳《诉衷情》``无端风雨送清秋。天气冷飕飕。行人先自离索,直是不禁愁''。
  周顗问庾亮为什么高兴得胖起来了,其实暗藏玄机,是在``钓鱼'',如果庾亮回答我最近升官了,所以胖;我成为皇亲国戚了,所以胖,自然大大不妥。周顗早已猜中庾亮的反诘,可以借机羞辱一番。庾亮果然上钩:``您有什么忧伤,竟然瘦了起来?''周顗立刻说自己``清静淡泊之趣日来,污浊之思日去'',言下之意就是:``庾亮啊,你胖就是因为你生活太庸俗、思想太污秽了''。这个故事我猜想大约发生在周顗50来岁的时候,风流倜傥已成为往事,剩下的是老而弥坚。

\section{2.31}\label{section-77}

\begin{quote}
过江诸人,每至美日,辄相邀新亭,藉卉饮宴。周侯中坐而叹曰:``风景不殊,正自有山河之异!''皆相视流泪。唯王丞相愀然变色,曰:``当共戮力王室,克复神州,何至作楚囚相对!''
\end{quote}

  过江诸人:指南渡的诸位大臣。
  美日:可以理解为风光美好的日子,但这种解释有点牵强,美日这种用法似乎不见于其他典籍,或当做``春'',或当做``春日''。敦煌《世说》残本做``暇日'',也比美日容易理解。
  新亭:地名,大约在南京附近的长江边,据说三面环山,是当时的战略要地和著名景点,估计现在的南京不会忘记重修吧。这个故事在成语中也叫``新亭对泣''。
  藉卉:一般认为是坐在草地上的意思,一大帮高级干部席地而坐,体现名士的亲近自然风采。
  周侯:周顗。
  ``风景''句:想必当年诸君也一样经常在洛水附近宴饮,洛阳、南京都以风光著称,登高望远,美景相似,河山异名,故国难回,不由悲从中来。正自,只是。辛弃疾《水调歌头》:``酒亦关人何事,正自不能不尔,谁遣白衣来。''姚云文《紫萸香慢》:``正自羁怀多感,怕荒台高处,更不胜情。''
  王丞相:东晋名相王导。   戮力:并力,合力。《国语 -
晋语四》:``吾先君武公与晋文侯戮力一心,股肱周室,夹辅平王。''《史记 -
项羽本纪》:``臣与将军戮力而攻秦。''
  楚囚:春秋时楚国音乐家钟仪作战被俘到晋国,他多年依旧穿戴楚国的服饰。有一次晋侯视察时发现了这个楚国人,问他是谁,以前是干什么,并叫他再弹曲子。钟仪也弹奏楚国的曲调,同样表示不忘故国。后以``楚囚''
多表示拘囚异乡或怀土思归者意思。这里形容遭遇国难,相对无策,徒然悲伤。
  王导虽然话说的不错,但终他一生,并没有做出任何北伐的举动,他始终致力于调和东晋各门阀之间矛盾,维护内部和平的工作,当然,他首先维护的是自家琅琊王氏的地位,以至于著名的占卜家郭璞做出这样的预言:``淮流竭,王氏灭''(除非淮河水流尽了,王氏家族才会灭亡!)在东晋期间,南北两朝主要精力都放在内部厮杀之中,无论是北伐还是南征,都无力改变隔江而治的局面。 

\section{2.32}\label{section-78}

\begin{quote}
卫洗马初欲渡江,形神惨悴,语左右云:``见此茫茫,不觉百端交集。苟未免有情,亦复谁能遣此!''
\end{quote}

  
卫洗马:卫玠,字叔宝,山西人,祖父是杀害邓艾的卫瓘,丈人是尚书令乐广。卫玠是历史上著名的美男子,打小就是那种人人都要忍不住捏捏脸蛋亲热一番的小白脸,《世说》中屡有提及,``总角乘羊车入市,见者皆以为玉人,观之者倾都''。长大后也许因为长相出众,当了太子洗马(洗马就是先马的意思,仪仗队员)。在晋代,很多男人是靠脸蛋和肤色扬名立万的,比现在的美丽女性还容易受到关注,如果还有点才能,那就仿佛当今的贝克汉姆,更了不得了。卫玠恰恰有杰出的思辨能力和口才,``亲友时请一言,无不咨嗟,以为入微'',为``中兴名士第一'',
王导、周顗、庾亮等还排不上号,如果参加什么``超级男生''、``谁是好男儿''的比赛,其他同学也就只有争夺争夺亚军的资格了,以至于后来传说,卫玠是被一帮男女Fans看死的,``看杀卫玠'',年仅27岁。当然,这只是一种夸张的说法,卫玠虽然平时喜愠不露于形色,但从小体格羸弱,``不堪罗绮'',思虑过度,又好玄理,清谈不休,终于病死。
  
渡江:永嘉四年(310年),洛阳陷落前,卫玠经江夏赴南昌,举家南迁,见到长江,此时天下大乱,正是``问君能有几多愁''之时。
  
惨悴:忧伤憔悴。李华《吊古战场文》:``黯兮惨悴,风悲日曛。蓬断草枯,凛若霜晨。鸟飞不下,兽铤亡群。''
  
百端交集:百感交集的出典。从魏晋以来,两汉三百年之后遇到道德沦丧,社会动乱,儒家信仰逐渐瓦解,玄学、佛学地位逐渐上升,国人的哲学思辨能力有了很大的提升,看到江水不仅仅局限于``逝者如斯夫''了,而是对宇宙本体``有无''关系反复辩论,对生命的终极意义反复拷问,于是百感交集,是喜是悲,是真是幻,是有是无,是幸是不幸,只要心中还有点感情,谁又能排遣得了这种情绪!我小时候听大人说,不要久站在江边思考,不然会控制不住自己的。

\section{2.33}\label{section-79}

\begin{quote}
顾司空未知名,诣王丞相。丞相小极,对之疲睡。顾思所以叩会之,因谓同坐曰:``昔每闻元公道公协赞中宗,保全江表。体小不安,令人喘息。''丞相因觉,谓顾曰:``此子珪璋特达,机警有锋。''
\end{quote}

  顾司空:顾和,字君孝,江苏人,江东顾家之后,祖上顾雍在吴国当了19年宰相,叔叔顾荣我们不久前刚介绍过,顾家子孙有名的有顾恺之、顾炎武、顾颉刚等人,在各自领域都有较高的声望。离我们较近的顾颉刚,其《古史辨
-
自序》的确写得好,但其为人啧啧可惜,毫无祖先的风度。顾和2岁丧父,年少出众,被族叔顾荣誉为``吾家麒麟''。历任吏部尚书、尚书仆射、尚书令等,死后追赠侍中、司空。
  王丞相:王导任扬州刺史时,顾和被王导赏识,召他为从事。
  极:劳累、疲乏。《三国志 -
华陀传》``人体欲得劳动,但不得当使极耳''。在拜访上司时没聊几句上司睡了,的确尴尬,给下人的印象就是你不受欢迎,然后大家一传十、十传百,顾和出头无望哉!必须改变这个局面,所以先拿叔叔的名头镇场子,让下人们知道自己的身份,再夸王导了不起,再关心王导的身体,丝丝入扣。而且说话的声音要响亮,可以把王导叫醒。小小年纪如此机心,可佩!当然,也保不定王导在出题考察干部,自己更要表现一番。看来,自幼丧父也有丧父的好处,敏感,纤巧,能低三下四。
  疲睡:打瞌睡。
  所以叩会之:其义不确,大概是用什么方式来叫醒他的意思。所以,疑做``何以'',所以也有用来的意思;叩,击打;会,使协调一致,符合,``有会于心'',吴均《续齐谐记》``音韵清畅,又深会女心。''
  元公:指顾荣,顾荣溢``元'',``能思辩众曰元,行义说民曰元,始建国都曰元,主义行德曰元''所以称为元公。
  中宗:晋元帝司马睿的庙号,顾和初出仕是在元帝时,还不可能有元帝的庙号,此当做后人补述或编造故事。
  协赞:协助,辅佐。《三国志 -
来敏传》:``(来忠)与尚书向充等并能协赞大将军姜维。''
  江表:长江以南地区,从中原看,地在长江之外,故称江表。
  体小不安:小有稍微的意思。
  喘息:呼吸急促,比喻紧张,这说明即使在古代,领导的健康也被认为是大家的事情,领导不舒服了,人民群众就如丧考妣。``告诉大家一个特大喜讯,主席身体非常健康!''记得有篇回忆录,其中说当年听说主席身体非常健康,医生说起码可以活150岁,是全国人民的最大幸福,把他吓得:``头脑轰地一响,虽然有所怀疑,但当时的直觉是,这一下,我这一辈子都注定要生活在毛泽东时代了!''
  谓顾曰:当做``顾谓曰''。回头对别人说,或对四周众人说。
  珪璋特达:珪和璋是玉器,是诸侯朝见天子时所用的重礼。用珪璋时可以单独送达,不须加上别的礼品为辅。《礼记
-
聘义》:``珪璋特达,德也。''孔颖达疏:``行聘之时,唯执珪璋特得通达,不加余币。言人之有德亦无事不通,不须假他物而成。言圭璋之特达同人之有德,故云德也。''丘光庭《兼明书》:``按朝聘之礼,有珪璋璧琮。璧琮则加束帛,然后能达。而珪璋德重,可以独行,故曰特达。''后以``珪璋特达''喻人资质优异,才德出众。
  机警有锋:机锋的出典。才思敏捷,随即生发,暗藏锋芒。
  在晋书中,下一条中的``不徒东南之美,实为海内之秀''也是王导对顾和的评价,就接在``机警有锋''后面,这样理解更妥当一些,因为下面一条不伦不类,疑有漏句窜句。我们家乡有句谚语:``天下十八省,马屁大通行'',莫里哀在《吝啬鬼》里说:``要人宠信,根据我的体会,最好的方法就是在他们面前,投合他们的爱好,称道他们的处世格言,恭维他们的缺点,赞美他们的行事。你用不着害怕殷勤过分,尽管一望而知,你是在戏弄他们,可是他们一听奉承话,就连最精明的人也甘心上当。''顾和就是此道高高手。

\section{2.34}\label{section-80}

\begin{quote}
会稽贺生,体识清远,言行以礼;不徒东南之美,实为海内之秀。
\end{quote}

  这节文章很奇怪,既然是记录言语,但本节没有对话;主人公是贺生,据刘孝标注中说是贺循,他是为大官僚,为什么既不表字,也不称官名,而用长辈对晚辈的称呼``生'';《晋书》``不徒东南之美,实为海内之秀''是王导夸顾和的,怎么就落到贺循身上?此节必定有误,所以只解释字句,贺循容后再表。
  体识清远:体清识远,为了对仗,所以倒置。禀性清纯,见识高远。
  东南之美:语出《尔雅》``东南之美者,有会稽之竹箭焉''。《三国志 -
虞翻传》``虞翻与少府孔融书,并示以所著《易注》。融答书曰:`闻延陵之理乐,睹吾子之治《易》,乃知东南之美者,非徒会稽之竹箭也\ldots{}\ldots{}。'\,''本指东南地方出产的好东西,后指东南地区的杰出人物。王勃《滕王阁序》:``台隍枕夷夏之交,宾主尽东南之美。''
  海内:泛指中国,在传说中,中国大陆四面环海,由几只大乌龟抗着的。哎呀,这么有名的传说,居然记不住是什么书上写的,见笑。

\section{2.35}\label{section-81}

\begin{quote}
刘琨虽隔阂寇戎,志存本朝。谓温峤曰:``班彪识刘氏之复兴,马援知汉光之可辅。今晋阼虽衰,天命未改。吾欲立功于河北,使卿延誉于江南,子其行乎?''温曰:``峤虽不敏,才非昔人,明公以桓、文之姿,建匡立之功,岂敢辞命!''
\end{quote}

\begin{itemize}
\tightlist
\item
  \emph{刘琨}:10来岁的时候,自己大约常常5点多就起床,姐姐还在睡觉,往往是我烧好早餐叫她的。父亲很欣慰,问我怎么肯起来,我就说:``闻鸡起舞嘛!''如今节假日不躺到8、9点钟不起来的,有时候孩子和我睡,也是他先起床,到时候再叫醒我的。虽然依旧仰慕前辈中流砥柱的抱负,不过``何意百炼钢,化为绕指柔'',霜刃难示,壮志渺渺,随他去吧。刘琨,就是少年时的闻鸡起舞的榜样,其字越石,河北人,当时``与范阳祖逖俱以雄豪著名'',他在洛阳与贾南风的侄子贾谧、大富翁石崇、名人陆云、陆机、潘岳、左思交好,被称为``二十四友'',后来舆论认为``二十四友''是西晋豪强的负面教材,主要由于贾谧骄盈,石崇奢侈,最后身死人手。而他们的生活方式,是符合当时特定时代的,其他几人拥护贾谧无非是想通过捷径得到政治上的腾达罢了,杜甫不也是``朝叩富儿门,暮随肥马尘''?刘琨卷入``八王之乱''很深,常常领兵作战,由此也当上了广武侯、并州刺吏、司空、都督并、冀、幽诸军事。西晋覆灭,刘琨就留在太原,成为被匈奴、鲜卑等包围的一块东晋``飞地'',所以后文说``隔阂寇戎''。历史上认为刘琨志大才疏,``短于控御'',后来被匈奴的刘曜、石勒击败,最终被鲜卑段匹磾所杀。《晋书》中有个非常滑稽的故事,东晋的大枭雄桓温雅重刘琨,``自以雄姿风气是宣帝、刘琨之俦,有以其比王敦者,意甚不平。及是征还,于北方得一巧作老婢,访之,乃琨伎女也,一见温,便潸然而泣。温问其故,答曰:`公甚似刘司空。'温大悦,出外整理衣冠,又呼婢问。婢云:`面甚似,恨薄;眼甚似,恨小;须甚似,恨赤;形甚似,恨短;声甚似,恨雌。'温于是褫冠解带,昏然而睡,不怡者数日。''这哪里是歌姬啊,这个一马三立!
\item
  \emph{存}:思念。身在江湖,心存魏阙。
\item
  \emph{温峤}:字太真,山西太原祁县人,刘琨是他的姨丈,其在刘琨手下任右司马。这次刘琨派他为使臣,劝进司马睿。温峤识大体,心机深,做事果断,在东晋声望很高,曾是平定王敦、苏峻之乱的主力,一生功业多有可述之处,后官拜极品,为骠骑将军,开府仪同三司,封始安公,观其为人与成就,算是两晋第一流的人物。
\item
  \emph{班彪}:班固、班超的父亲,西汉末先后追随隗嚣、窦融等,他认为刘秀可以依靠和追随,常常劝隗嚣、窦融投靠刘秀。
\item
  \emph{马援}:``穷当益坚''、``老当益壮''、``马革裹尸''的主人公,他和王莽时代的造反派隗嚣是好朋友,但认为刘秀才是``人主'',多次劝隗嚣归顺。其在东汉南征北伐,战功赫赫,甚至主动带兵击败了隗嚣,有出色的军事才能和高尚的人品,不愧祖上赵奢的威名。
\item
  \emph{阼}:见2.19。
\item
  \emph{延誉}:传扬美名。延,陈也。《国语 -
  晋语》``使张老延君誉于四方''。
\item
  \emph{桓、文}:齐桓公,晋文公,都是逢凶化吉,白手起家,挽巨澜、扶大厦的角色。这里用王霸功业来推崇刘琨,而不是以管、乐比较,稍稍有点奇怪,因为晋阳已成飞地,与东晋相隔,其局面已经是实际独立的一方诸侯,所以后面用``匡立''的字眼。
\item
  \emph{辞命}:辞命的通常意思是辞令,就是说我愿意当说客,去劝司马睿即位。但这里也可解释成拒绝、推辞命令。
\end{itemize}

刘琨叫温峤去南方传播他的美名,温峤不负所托,即使刘琨兵败身死,温峤也尽心为刘琨辩说呼吁,刘琨最后赠侍中、太尉,谥``愍''。

刘琨也是位诗人,元好问说他的诗歌可以与曹植、刘祯比较,``曹刘坐啸虎生风,四海无人角两雄。可怜并州刘越石,不教横槊建安中。''
王图霸业转瞬即逝,忧愤之作后人难忘,这可能也是刘琨没想到的。

\section{2.36}\label{section-82}

\begin{quote}
温峤初为刘琨使来过江。于时,江左营建始尔,纲纪未举。温新至,深有诸虑。既诣王丞相,陈主上幽越、社稷焚灭、山陵夷毁之酷,有《黍离》之痛。温忠慨深烈,言与泗俱,丞相亦与之对泣。叙情既毕,便深自陈结,丞相亦厚相酬纳。既出,欢然言曰:``江左自有管夷吾,此复何忧!''
\end{quote}

  始尔:就是始,开始。古汉语中,特别是我们在阅读《诗经》时会发现很多看似有意义其实没意义的词缀,譬如今我来思、思娈季女逝兮的``思'',曰归曰归的``曰'',式燕且喜、式微式微的``式'',有些人往往会一不小心去落实意义。
  纲纪:法律制度。
  ``主上幽越''句:晋愍帝司马邺被刘曜监禁、流亡,其事详见1.24注释。洛阳诸帝陵墓被掘,长安、洛阳被焚,司马邺成了小厮,后被杀。现在想来,刘曜等人还比较粗糙,既然绑票了哪有撕票的道理,既可以要钱,也可以搞乱东晋朝廷。
  黍离:《诗经 -
黍离》篇名,``彼黍离离,彼稷之苗。行迈靡靡,中心摇摇。知我者,谓我心忧,不知我者,谓我何求。悠悠苍天,此何人哉!\ldots{}\ldots{}''传说是东周大臣路过西周旧都镐京,见宫阙桑田,不胜感慨悲伤。后来南宋姜夔有首《扬州慢》中说``过春风十里,尽荠麦青青'',在亡国之后,表达这种情感的诗词不胜枚举。
  泗:本指鼻涕,一般可理解为眼泪。李朝威《柳毅传》``悲泗淋漓,诚怛人心。''据《语林》说,温峤不但铺陈亡国之痛,而且建议早立司马睿为帝:``及言天下不可以无主,闻者莫不踊跃,植发穿冠。''这次对话有回忆与建议,有悲痛和激情,有声有色,唱念做打俱佳,内容比较丰满,史书上说温峤``美于谈论,见者皆爱悦之''。
  陈结:述说心意想与之交结为好。
  管夷吾:管仲,名夷吾,字仲,辅佐齐桓公创立霸业。孔子说:``微管仲,吾其被发左衽矣。''
  此复何忧:据说温峤看到王导大事可托,心情愉快,``便游乐不住''。
  从此以后,王导就得到``江左夷吾''的称号。不过观其功业,与管仲的文韬武略相距颇大,盛名之下,其实难副,就像一个小城自称为``东方威尼斯'',一个饭店叫``地中海大酒店'',无论如何总要差上一些的。

\section{2.37}\label{section-83}

\begin{quote}
王敦兄含为光禄勋。敦既逆谋,屯据南州,含委职奔姑孰。王丞相诣阙谢。司徒、丞相、扬州官僚问讯,仓卒不知何辞。顾司空时为扬州别驾,援翰曰:``王光禄远避流言,明公蒙尘路次,群下不宁,不审尊体起居何如?''
\end{quote}

\begin{itemize}
\tightlist
\item
  \emph{王敦}:字处仲,琅邪王氏之后,司马炎的女婿,与堂兄王导一起辅佐晋元帝司马睿。王敦与王导不同,其野心勃勃,又是王家的军事领袖,不愿意``王与马,共天下''的局面,两次以``清君侧''的名义造反。第一次攻入南京,杀司马睿宠臣刁协、戴渊等,安然无事,王导劝阻他立废兴之事。司马睿因此忧愤而死。晋明帝司马绍即位后王敦第二次起兵,这次王导明确表态:``宁忠臣而死,不无赖而生'',王敦中途病死。本节讲的故事应该是王敦第一次起兵之时。
\item
  \emph{王含}:字处弘,王敦的亲哥哥。光禄勋是九卿之一,通常是皇帝的近卫首领。
\item
  \emph{逆谋}:常作``谋逆''。
\item
  \emph{南州}:就是姑孰,今安徽省当涂县,因为在南京的南边,所以叫南州。这里为什么用两个地名,费解。王含在王敦第二次造反时,任元帅。
\item
  \emph{委职}:离职。见1.9。有些文章会把委职当就职,就像当年余秋雨错把致仕当出仕。说到余秋雨,其学历、文笔应在一、二流之间,可惜一失之于作品高产,二失之于为人功利,三失之于心窍聪明。有人爱读他的文章,小心翼翼地纠正其错误,可余先生恼羞成怒,说道:``青年人千万不要拘泥于文史的旧框中,捕捉那些典故中的细枝末稍,这样会限制一个人的创造力和想像力。''并在《苏东坡突围》一文中,以赤壁赋中的赤壁不是赤壁大战中的赤壁为例,放出名言:``大艺术家即便是错,也会错出魅力来。''据说中国人好面子,最难说出口是``不知道''、``我错了'',余秋雨可惜!
\item
  ``王丞相''句:疑少``罪''字。王导当时不是丞相,具体官职大概是扬州刺史、司空、中书监、录尚书事等,他在晋成帝时任丞相。
\item
  \emph{司徒、丞相、扬州官僚}:这里称呼十分混乱,我们不必管它。
\item
  \emph{顾司空}:机警有锋的顾和。
\item
  \emph{别驾}:州刺史的属官,刺史出巡辖境时,别驾另乘驿车随行,其职能大致相当于唐宋的长史、通判。
\item
  \emph{援翰}:执笔。鸟羽之长而劲者为翰,当时可能中国也用鹅毛笔的,后来借指毛笔、文章、书信等。所以给皇帝写文章的机构叫``翰林院''。
\item
  \emph{蒙尘路次}:蒙受路中的风尘。路次,路途中间。《晋书》中记载,王敦第一次起兵之时,``王敦之反也,刘隗劝帝悉诛王氏,论者为之危心。导率群从昆弟子侄二十余人,每旦诣台待罪''。对于王敦的第一次造反,王导其实持支持态度的,因为司马睿利用刁协、刘隗钳制王氏家族,是司马家主动挑引起内乱。王敦第二次造反则是王家主动挑起,于是王导趁王敦病了,立即以家族的名义为王敦发丧,鼓舞朝廷的斗志。
\item
  \emph{审}:知道。《三国演义》:``陛下连降三诏,召臣回朝,未审圣意为何?''
\end{itemize}

对于上司,文章以委婉为贵,顾和讲话都比较婉转文雅,我们还可以多次领略。可惜史书上好像没有记载王导的回复

\section{2.38}\label{section-84}

\begin{quote}
郗太尉拜司空,语同坐曰:``平生意不在多,值世故纷坛,遂至台鼎。朱博翰音,实愧于怀。''
\end{quote}

\begin{itemize}
\tightlist
\item
  \emph{郗太尉}:郗鉴,见1.24,传说中腮帮子里含饭给孩子吃的那位,他多次拒绝举荐,可能无意于朝廷,西晋洛阳沦陷后,他也没有南渡,而是继续于族人结团居住在山东,后来被石勒驱赶,才逐渐南渡,这时候他掌握了一批流民,实力渐大,在王敦之乱中坚定地站在朝廷一边,``遂与帝(晋明帝司马绍)谋灭敦'',有``首策''之功,于是成为第一流士族。325年司马绍死,郗鉴、王导、庾亮等7人成立托孤之臣。329
  年任司空,338年进位太尉,据史书上说,郗鉴两拜三公,不操其柄,无竞于朝,而谦退旨趣前后如一。
\item
  \emph{多}:高、重。
\item
  \emph{世故}:世上的事情。嵇康
  《与山巨源绝交书》:``机务缠其心,世故烦其虑。''李商隐为贺拔员外上李相公启》:``世故推迁,年华荏苒。''
\item
  \emph{台鼎}:指三公或宰相。东汉时大尉、司徒、司空合称三公,是最高的官位。人们拿三台(星名)和鼎足来比喻三公,说成台鼎。《陈书》:``晋氏丧乱,播迁江左,顾荣、郗鉴之辈,温峤、谢玄之伦,莫非巾褐书生,搢绅素誉,抗敌以卫社稷,立勋而升台鼎。''
\item
  \emph{朱博翰音}:朱博是西汉哀帝时的宰相,据说他在以御史升为丞相的任免时刻,``有大声如钟鸣'',有人解释说:``空名得进,有声无形''(有名无实的人登上朝廷,才会有一种无形的声音发出),后比喻名不副实,不应处此高位。不过翰音也有鸡鸣的意思,西汉朝廷本来就有斗鸡的传统,从皇家百兽园里传来鸡鸣也是情理之中。
\end{itemize}

郗鉴出身不过是中等士族,最后能位列三公,并在乱世中取得正面的成就,的确是命运的推手使然。他在晚年病重时留下遗嘱:``我一向崇敬微子、张良和子鱼,希望死后的墓地与他们为邻。''这历史上的三位名人都是以谦退自守的,都葬在山东微山湖,郗鉴的墓地与他们相邻,后人为了纪念他,取名为``郗山''。郗鉴说这样的话,应该是出于本心而不是作秀。记得爱因斯坦有一次发表获奖感言,大意是自己不配有那样高的荣誉,对不起这个奖项,希望人们以宽容的态度对待这件事,说得也很谦虚自然,我曾经整段都抄了下来,不过找笔记本太麻烦,就不``献宝``了。在工作中,我很少看到有自知之明的人,个个认为自己应该升到更高的职位,没有人为感到惭愧。从管理学上讲,每个人都会晋升到他不能胜任的工作为止。

\section{2.39}\label{section-85}

\begin{quote}
高坐道人不作汉语。或问此意,简文曰:``以简应对之烦。''
\end{quote}

\begin{itemize}
\tightlist
\item
  \emph{高坐}:或作高座,据说是新疆龟兹国人,姓帛,名尸梨蜜多罗,西晋永嘉年间到中国,是早期佛教的密宗大师(佛教有显密二宗,密宗习密咒,以此降心伏魔),在东晋受到王导、周顗等人的敬重,称为``吾之徒也''。
  高座和尚人称``卓朗'',就是说他性格洒脱。后来周顗被王敦杀害,其亲友无人敢临丧吊唁,帛尸梨蜜多罗独登门慰问遗孤,诵密咒数千言,神气自若,挥泪告别(在《世说》中我们可以发现,晋代人很喜欢哭,哭大概是真性情的表现)。因为口才好,经常居于上座讲经说法,所以叫他``高座''。在古代,和尚和道人有时通用,高座修习的是头陀苦行,是靠苦行得佛法的,也许不用剃度,说他道人也没错,《水浒传》中也有道人居住和尚庙,《封神演义》中和尚一律称道人。南京雨花台应该还有个高座寺,就是为了纪念他。
\item
  *不作汉语:传说高座不学汉语,与东晋公卿交往要靠传译。但他善于颖悟非常,往往不待翻译,而已神领意得。不过我们能够看到他翻译的经书,可见他大概是不想说汉语或者学习的是哑巴汉语,懂是懂的,当时大家也猜出他其实听得懂。所以简文帝做出了后面的解释。
\item
  \emph{简文}:简文帝。见1.37。他很喜欢清谈的,但据说水平一般。
\item
  \emph{以简应对之烦}:简省应酬的烦扰。一个外国人,毫无利己的动机,不远万里,来到中国,把中国人民的痛苦当做自己的痛苦,虽然不能忍受中国繁琐的应酬,但捏着鼻子交接中国的高级官僚,广泛传授解脱痛苦烦恼的办法,这是一种什么样的精神!现在有白求恩医院,当年有高座寺,白求恩大夫不过是见一个救一个的医生,高座和尚动不动就解脱一大片,好像要高明一些。不知现在还也没有僧人习头陀苦行,能够吃得来那种艰苦,性情一般都很卓绝无畏。
\end{itemize}

\section{2.40}\label{section-86}

\begin{quote}
周仆射雍容好仪形,诣王公,初下车,隐数人,王公含笑看之。既坐,傲然啸咏。王公曰:``卿欲希嵇、阮邪?''答曰:``何敢近舍明公,远希嵇、阮!''
\end{quote}

\begin{itemize}
\tightlist
\item
  \emph{周仆射}:周顗、周侯、伯仁。
\item
  \emph{雍容}:和谐、从容不迫。本来不用解释的,不过书话有位版主叫雍容,前几天看过她写的几篇文章,很感性、也许内心狂热的一位女性。
\item
  \emph{王公}:王导。我前面讲过,《世说》各章的人物出场大抵是按时间为序的,周顗、王导前几章都出现过,那么为什么不把几节放在一起呢?也许是为了文章显得错落有致,便于回忆和消化,新亭对泣和既坐啸咏是反映人物性格的不同侧面,如同朋友小别,见面就会亲切一些,同时又发现人物身上新的东西,更能欣赏。这种写作技巧值得借鉴。当然,古人写文章不像现在这样严密,这里也可能反映行文的跳脱和自由。
\item
  \emph{隐数人}:隐通``倚''或``凭'',就是由别人扶持的意思。据记载,晋人中有不少人出门行走要人做拐杖的,一方面可能是服用五石散体弱,一方面可能也是讲排场,突出自己的风采。
\item
  \emph{含笑看之}:微笑,轻微的讥笑?看后文大概有这个意思。
\item
  \emph{啸咏}:魏晋风采其中一个特点就是当时人喜欢啸咏,一般解释是吹口哨,但我想恐怕不是现代人的那种低靡徘徊的口哨,应该是长声歌咏的意思,也许有词,也许无词,都是为了表达独立高亢的情怀。当时有位叫成公馁的专门写了篇《啸赋》,其中说道:``狭世路之阨僻,仰天衢而高蹈。邈姱俗而遗身,乃慷慨而长啸。\ldots{}\ldots{}乃吟咏而发散,声骆驿而响连。舒蓄思之悱愤,奋久结之缠绵。心涤荡而无累,志离俗而飘然。\ldots{}\ldots{}乃知长啸之奇妙,盖亦音声之至极。''关于啸咏,我们后面还会接触很多。
\item
  \emph{卿}:当时对平辈或晚辈间亲密的称呼。称对方为卿,一般表示自己的身份要高一点。
\item
  \emph{希}:希望引申为仰慕、学习。《隋书 -
  隐逸传》``本无意于希颜(渊),岂有心于慕蔺(相如)''。
\item
  \emph{嵇、阮}:嵇康、阮籍都喜好啸咏,引导了时代潮流,曹植《远游篇》``鼓翼舞时风,长啸激清歌。''
\item
  \emph{何敢近舍明公}:直接排王导的马屁。王导是当时清谈派的核心人物,``内戢强臣,外御狄患。暇则从容谈说,自托风流''。
\end{itemize}

从《世说》上的记载看,周顗和王导、王敦是非常亲密的朋友,大家言辞之间都很放肆,当对方做挚友。可是王敦造反时,王导每天去皇宫请罪,并希望周顗能替他说些好话,周顗不理不睬,还嘀咕:``今年杀诸贼奴,取金印如斗大系肘。''但其实周顗极力在司马睿面前维护王导。结果王敦攻破建康时,司马睿迫于无奈,下诏王敦剪灭刘隗有功,任凭王敦处置朝中大臣。王敦欲杀周顗,征求王导的意见,王导默不作声,王敦便将周顗杀害。王导后来发现事实,痛哭着就对儿子们说:``我虽不杀伯仁,但是伯仁因我而死。幽明之中,负此良友!''信息不对称真要命!在这痛哭当中,有多少往事涌上王导心头!起码《世说》中有好几则王导和周顗相处融洽的故事。周顗是一小撮真正有名士风采的人物。

\section{2.41}\label{section-87}

\begin{quote}
庾公尝入佛图,见卧佛,曰:``此子疲于津梁。''于时以为名言。
\end{quote}

\begin{itemize}
\tightlist
\item
  \emph{庾公}:庾亮,见1.31等。晋代许多名士对佛教很感兴趣,把它当做玄学的分支,把它叫做``浮屠道'',希望从佛教中寻找人生的真谛。
\item
  \emph{佛图}:又作``浮图、浮屠、休屠''等,一般指佛或佛塔,这里借代为佛寺。
\item
  \emph{卧佛}:传说释迦牟尼死的时候用的姿势,又叫做狮子卧,我们在寺庙中可以见到,具体大概是头北面西,右胁而卧,左腿压右腿,右手曲枕头下,左手舒伸放在身体左侧。有些僧人吃不消长期打坐,要睡卧时一般也应该采取这种姿势。因为后来的佛教徒认为,仰着睡是阿修罗卧,趴着睡是饿鬼卧,左侧卧是贪欲人卧,狮子王卧才是最适合修行人的卧姿,会生广大智慧,有不可思议的功德。佛教本来是无神论思想,后来演变成非常繁琐和功利的迷信宗教,这种莫名其妙的演化很奇怪,大约是人的自虐和偶像崇拜倾向。
\item
  \emph{此子}:如果是老子、孔子,估计不会用这个字样,这里也许从一个侧面反映出佛教在当时并不占据像南北朝时那样的地位,名士讲话也往往刻薄。
\item
  \emph{津梁}:又作梁津。津,水;梁,桥。《楚辞 - 离骚》
  ``麾蛟龙使梁津兮,詔西皇使涉予。''
\end{itemize}

大乘佛教讲的是普渡众生,使之超脱痛苦,登上彼岸,佛法就是过河拆桥、登岸舍筏中的``桥''和``筏''。这句话说释迦牟尼为众生说法,当做桥梁实在是太累了。一方面说佛喻世工作辛苦,;另一方面是把佛从神坛上请下来,说他也会疲劳,也会厌倦。这种略带讥讽的通俗评论在任何时代都能赢得读者。当然,这也许是庾亮有感而发,他和王导的执政理念不一,杀宗室、驱王氏、抚流民、主北伐,后半生急于事功,以``庾与马''取代``王与马共天下''的局面,估计有时也会心神交疲,不知所以,故有此叹。

\section{2.42}\label{section-88}

\begin{quote}
挚瞻曾作四郡太守、大将军户曹参军,复出作内史,年始二十九。尝别王敦,敦谓瞻曰:``卿年未三十,已为万石,亦太蚤。''瞻曰:``方于将军,少为太蚤;比之甘罗,已为太老。''
\end{quote}

\begin{itemize}
\tightlist
\item
  \emph{挚瞻}:字景游,陕西西安人。他先后历任安丰、新蔡、西阳等地太守,多年在俸禄二千石的中高级官位上,内史也是两千石,所以是平调,但权力有所下降。
\item
  \emph{大将军}:枭雄王敦。
\end{itemize}

挚瞻和王敦不和,有一次王敦把穿坏的皮裘送给老病的帐下都督,挚瞻觉得己所不欲,勿施于人,就说不适合。王敦问他为什么。挚瞻见王敦如此懵懂,干脆就曲解说:``上官的服饰都可以赐给部下,那么你官帽上的玉貂蝉也可以赐的。''王敦大怒,说:``这个比喻不恰当,你不配两千石的职位。''挚瞻说:``我把抛弃西阳太守看得就像脱双鞋一样。''于是王敦让挚瞻去当诸侯国的内史了。
-
\emph{万石}:四任太守,一任内史,都是二千石的职位,所以加起来说万石,但其中包含嘲笑,因为三公、大将军的俸禄才是万石,说早其实就是不配。
- \emph{方}:比也。 -
\emph{少为太蚤}:太为词缀,不用具体落实意义。少:稍稍。 -
\emph{甘罗}:秦相甘茂的孙子,他12岁说服张唐相燕,自己出使赵国,兵不血刃取得赵国河间5城,于是嬴政封甘罗为上卿,其事详见《史记》。据说甘罗后来帮吕不韦把嫪毐送进宫,估计被吕不韦灭了口。

挚瞻的答词也许有这样的意思:王敦你虽然是万石的大将军,但也没什么稀罕的,看看人家甘罗,12岁就和你一样了,你狂妄什么!挚瞻后来与王敦作对,被杀。虽然``无实力的愤怒无意义'',但``馒头蒸的就是这口气''!

\section{2.43}\label{section-89}

\begin{quote}
梁国杨氏子,九岁,甚聪惠。孔君平诣其父,父不在,乃呼儿出,为设果。果有杨梅,孔指以示儿曰:``此是君家果。''儿应声答曰:``未闻孔雀是夫子家禽。''
\end{quote}

\begin{itemize}
\tightlist
\item
  \emph{梁国}:东晋时的一个诸侯国,具体位置我不清楚,如果指的是梁州,就是陆游所说的``当年万里觅封侯,匹马赴梁州'',那么在川陕一带,但根据孔坦的生平,似乎没去过那里。那么这个梁国是指杨家以前住在梁国,现在住在绍兴,因为会稽孔家``其先世居梁国'',两家是世交,关系亲密,可以看玩笑的。这个故事版本不少,在南北朝梁元帝萧绎的《金楼子》一书中,小孩子落实为7岁的杨子州,在其他故事中,也有说是杨修和孔融。我们不必细究,反正是一个以子之矛、攻子之盾的小故事,靠姓杨姓孔来展开。
\item
  \emph{孔君平}:孔坦,字君平,孔子的后裔,绍兴人,孔家当时是会稽第一高门,孔坦被庾亮认为``足下才经于世'',历任太子舍人、尚书郎、廷尉等。和庾家交好,就和王家搞不来的。
\item
  \emph{乃呼儿出}:古代的礼节,家里有男子在,女子不必接待客人,所以小孩子出场,这也从小培养了男子的责任意思。
\item
  \emph{君家果}:那别人的姓氏开玩笑,言辞虽然客气,但在那时不算礼貌,所以要干脆利落地反驳。杨梅不姓杨,孔雀不姓孔,记得有个笑话里讲,孔明聪明,因为祖上是孔子,秦桧凶残,因为祖上是秦始皇。这种反诘训练到后来大概就是对对子了,培养孩子思维敏捷。也许仅从语文教育来说,古人的某些方法比现在的要好。
\end{itemize}

\section{2.44}\label{section-90}

\begin{quote}
孔廷尉以裘与从弟沈,沈辞不受。廷尉曰:``晏平仲之俭,祠其先人,豚肩不掩豆,犹狐裘数十年,卿复何辞此!''于是受而服之。
\end{quote}

\begin{itemize}
\tightlist
\item
  \emph{孔廷尉}:就是上面的孔坦。三国孙吴以后,会稽出现``孔魏虞谢''四大家族。
\item
  \emph{裘}:皮衣,古人最常用的冬服,棉衣可能是宋以后的事情。古人平时可能穿羊裘,上朝时穿狐裘,穿法可能是毛露在外边。有些裘衣很贵,譬如《史
  记 -
  孟尝君传》中说:``(秦昭王)囚孟尝君,谋欲杀之。孟尝君使人抵昭王幸姬求解。幸姬曰:`妾愿得君狐白裘。'此时孟尝君有一狐白
  裘,直千金,天下无双,入秦献之昭王,更无他裘。''后来依靠鸡鸣狗盗之徒偷回来再送。《论语
  - 公冶长》中说:``子路曰:`愿车马,衣
  轻裘,与朋友共,敝之而无憾。'\,''
\item
  \emph{孔沈}:也是孔家的杰出人物,在《世说 -
  赏誉》中被认为是``孔家金'',其生平缺乏比较详细的记录。
\item
  \emph{不受}:古人讲的是``长者赐,不敢辞'',但古代礼仪可能还有一条补充规定:过于贵重的东西不能接受。
\item
  \emph{晏平仲}:晏子,齐国晏婴,字仲,谥平。《晏子春秋》:``齐景公赐晏子狐白之裘,玄豹之纰,其赀千金。晏子辞而不受,三反。公曰:`寡人有此二,将欲服之,今夫子不受,寡人不敢服。与其闭藏之,岂如弊之身乎?'晏子曰:`君就赐,使婴修百官之政,君服之上,而使婴服之于下,不可以为教。'固辞而不受''(这件狐裘据说是一千只狐狸腋下的毛皮为身和黑豹的毛皮为领做成的。晏子说,如果我和你都穿这样珍贵的狐裘,上行下效,不好教化百姓),不过晏子有自己的狐裘衣服,尽管很旧,因为《礼记》中说:``晏平仲祀其先人,豚肩不揜豆,澣衣濯冠以朝,君子以為隘矣。\ldots{}\ldots{}晏子一
  狐裘三十年,遣车一乘,及墓而返,\ldots{}\ldots{}晏子焉知礼?''儒家认为当官就得有官的派头,祭祀祖先就得有祭祀的排场,儒家学说很重要的一条就是``讲的就是个面子''。
\item
  \emph{豚肩不掩豆}:豆,一种托盘,在博物馆里可见。晏子祭祀用的是乳猪(其他人可能用大猪),猪的肩膀还没有盘子宽。
\end{itemize}

古人可能熟读礼记,不大读《晏子》的,所以孔沈被孔坦说服了。

\section{2.45}\label{section-91}

\begin{quote}
佛图澄与诸石游,林公曰:``澄以石虎为海鸥鸟。''
\end{quote}

\begin{itemize}
\tightlist
\item
  \emph{佛图澄}:前面解释过,佛图就是``浮屠、浮图、
  佛陀''等意思,可能因为这个和尚有非凡的神通,所以别人称他为``佛''。在晋以前,和尚一般姓竺,表明其渊源从西方天竺来的;东晋道安和尚认为出家人就要断却与尘世间的一切渊源,就以印度佛教称出家人为``释子''的说法,取``释''为姓,表明继承了释迦牟尼的道统。佛图澄是个神秘人物,像《高僧传》这些书中,记载了不少他善于变幻,赢得大批信众的故事。他可能得到了``阿罗汉''的果位,非我等常人所能理解和解释,不过宗教本为夸饰,像小弟这样``不语怪力乱神''的凡
  夫在此就不讲那些故事了,有兴趣的读者可以自行查阅。佛图澄大约也是新疆龟兹人,7、80岁时来中国传教,在西、东晋年间``门下受业者常有数百,前后门徒近万,立佛寺八百余所'',其在中国佛教中的贡献可能无与伦比。
\item
  \emph{与诸石游}:与后赵皇帝石勒、石虎等人交游。佛图澄用种种神通赢得杀人魔王石勒、
  石虎等人的敬畏和皈依。石勒尊佛图澄为``大和尚'',军国大事都要请教他后才施行。石虎称佛图澄为``国之大宝''。
  佛图澄参加石虎的朝会时,官员们要给和尚抬轿子,太子们必须扶轿子而上,当司礼唱``大和尚到''时,全体官员起立。石虎还规定司空李农每天早晚两次亲自问候,太子五日去拜访一次。这些举动虽然极大地满足了后世佛教徒的虚荣心,但在中国社会往往只是昙花一现。
\item
  \emph{林公}:东晋另一个高僧支遁,河南人,俗姓关。字道林,后从师改姓,世称支道人、支道林,尊称为林公、支公等。他精通老庄,情趣亦有道家色彩,与东晋权贵交游甚广,主要活动在江浙一带,《世说》中多有其言行记录。
\item
  \emph{石虎}:石勒的弟弟,杀侄自立。石虎嗜杀,对战争中的俘虏一律坑杀(坑杀不能理解为活埋,而是杀害的意思);他后宫有5万汉女,平时极尽凌杀污辱之行。在他的王国中,沿途树上挂满死尸,城墙上挂满汉人的头颅。他极度满足于自己的武力,曾经说:``我家父子如是,自非天崩地陷,当复何愁!但抱子弄孙日为乐耳。''石虎的一系列民族奴役政策和极端无耻的生活方式,,最终导致了冉闵的``杀胡令'',北方汉族进行了凶猛的报复,几十万羯族人也差不多杀光了。
\item
  \emph{海鸥鸟}:《列子》中的故事:海上之人有好鸥鸟者,每旦之海上,从鸥鸟游,鸥鸟之至者百住而不止。其父曰:``吾闻鸥鸟皆从汝游,汝取来,吾玩之。''明日之海上,鸥鸟舞而不下也。有人说《列子》是东晋人的伪作,从支道林能熟练使用这个典故,其时代应该更靠前一些。
\end{itemize}

这个故事是说,佛图澄没有机心、真诚待人,所以换来了魔王石虎同样的对待。

\section{2.46}\label{section-92}

\begin{quote}
谢仁祖年八岁,谢豫章将送客。尔时语已神悟,自参上流。诸人咸共叹之,曰:``年少一坐之颜回。''仁祖曰:``坐无尼父,焉别颜回!''
\end{quote}

谢仁祖:谢尚,字仁祖,小名坚石,谢鲲的儿子,谢安的堂兄,历任镇西将军、豫州刺史等,谥号``简'',或称谢镇西。
谢豫章:谢鲲,曾任豫章太守,封咸亭侯。据说他才子风流,生活纵情随便,但从来不装腔作势。谢家从第三代谢鲲、谢裒这里逐渐兴盛,其父谢衡、祖父谢缵碌碌而已,在二、三流之间,到谢尚、谢安、谢奕第四代,谢家达到鼎盛。
将送客:带领谢尚送客。古代送别是非常重要的礼仪,有祭路神、喝酒、写诗、折柳等等,送完五里送十里,很大的场面。前些年国家领导人出访,我们常从新闻里看张三走,李四送,张三回,李四接,占据新闻联播大半的时间,古人更是有过之而无不及。谢鲲带谢尚参加这么重要的场面,这是从小栽培。
神悟:一种理解是大人讲话他已经能够心领神会;第二种理解是他讲话已经高明有悟性。
自参上流:此句费解,就教于方家。
年少一坐之颜回:一坐大概是满座的意思,也许这句话可以这样理解:谢尚是满座之中年少的颜回。颜回刚入孔门时13岁,是弟子中年龄最小的,也是最杰出的。
尼父:孔子,字仲尼,父是男子的美称。孔子去世,鲁哀公亲诔孔子,诔文说:``旻天不吊,不慭遗一老,俾屏余一人以在位,茕茕余在疚,呜呼哀哉!尼父!无自律。''(大意是说:老天爷不善良,不留下这位老人,扔下我一人在位,让我独自内疚。悲伤啊,孔先生啊,我再也没有学习的楷模了。)
别:识别,鉴别。颜回的好处只有孔子才可看清。《论语 -
为政》说:``子曰:`吾与回言终日,不违,如愚。退而省其私,亦足以发,回也不愚。'\,''开始时大家因为颜回比较内向,所以有人以为他``愚'',孔子经过长期观察,终于认为颜回``贤哉回也''。
这句话到底是谦虚还是傲慢,我看不懂,但至少说明谢尚8岁时已经比较熟悉《论语》了。谢尚15岁时,父亲谢鲲去世,``及遭父丧,温峤唁之,尚号叫极哀。既而收涕告诉,有异常童。峤奇之,由是知名。''所谓玄学家,往往儒道兼通,能放能收,由此可见一斑。

\section{2.47}\label{section-93}

\begin{quote}
陶公疾笃,都无献替之言,朝士以为恨。仁祖闻之,曰:``时无竖刁,故不贻陶公话言。''时贤以为德音。
\end{quote}

陶公:陶侃,字士行(士衡),江西人,陶渊明的祖先,出身寒族,甚至可能是当时人非常鄙视的溪族人。陶侃早年丧父,有一个非常杰出的母亲,在她的教导下逐渐做到小吏。西晋大乱后,陶侃得以以寒族出身登上重要的政治舞台,多次平叛,不但战功赫赫,而且为人质朴,当时人认为``陶公机神明鉴似魏武,忠顺勤劳似孔明''。他历任湘、广、荆州刺史等,后封长沙郡公,太尉,大司马,是两晋期间了不起的人物。他的战功无非就是东征西伐,现在看来皆付与流水,不过有很多小
故事反映出他高尚的品格和志向,譬如他``在州无事,辄朝运百甓于斋外,暮运于斋内。人问其故,答曰:`吾方致力中原,过尔优逸,恐不堪事。'\,''这种以天下
为己任,死而后已的精神吾辈虽不能至,心向往之。
献替之言:``献可替否''的省略语,《左传》:``君所谓可,而有否焉,臣献其否,以成其可;君所谓否,而有可焉,臣献其可,以去其否。''就是对君主劝规善过、建议兴革的言论(遗言)。``无献替之言''不确,《晋书》中记载陶侃病重时的上表:``伏愿遴
选代人,使必得良才,足以奉宣王猷,遵成志业。则虽死之日,犹生之年。陛下虽圣姿天纵,英奇日新,方事之殷,当赖群俊。司徒导鉴识经远,光辅三世;司空鉴简素贞正,内外惟允;平西将军亮雅量详明,器用周时,即陛下之周召也(指王导、郗鉴、庾亮等)。献替畴谘,敷融政道,地平天成,四海幸赖。\ldots{}\ldots{}''对后事交
代得很清楚。 恨:遗憾。 仁祖:谢尚。
竖刁:春秋时齐桓公最喜欢的一个宦官,姓刁,竖是小臣、宦官的意思。管仲病重时,齐桓公问管仲:``子如不讳(死亡的委婉语),谁代子相者?竖刁何如?''管仲曰:``自宫以事君,非人情,必不可用!''(先秦时候以宦官为相,可见不是赵
高一人。)历史上君王鲜有听快死臣子的遗言,后来竖刁、易牙等宦官果然出手不凡,把齐国搞得一塌糊涂。
贻:赠送、遗留。
德音:美好的言论。《诗经》:``德音不瑕''、``厌厌良人,秩秩德音''、``德音莫违,及尔同死''等。是啊,既捧了陶侃,又捧了朝廷,虽然不是事实,但谢尚这番美言,总比2004年沙祖康先生``中国人权比美国好5倍''更真实一些,谢尚可得公元334年最佳捧哏奖。

\section{2.48}\label{section-94}

\begin{quote}
竺法深在简文坐,刘尹问:``道人何以游朱门?''答曰:``君自见其朱门,贫道如游蓬户。''或云卞令。
\end{quote}

竺法深:晋代僧人,名潜,字法深,或称道潜、深公等,见1.30。竺法深当时住在绍兴新昌一带。
简文:简文帝司马昱,见1.37。司马昱是桓温废司马奕后立的,当时还没当上皇帝,为会稽王,虽然水平一般,``惠帝之流'',但就像今天一样,地位使他成为玄学界的名誉会长。当时东晋几个帝王大都信奉佛教,结交僧尼。
刘尹:刘惔,字真长,曾任丹阳尹等,见1.35。看样子刘惔好玄学,但不信佛道,故语带讥讽。
道人:和尚。见2.39.
朱门:豪门。古代色彩有贵贱之别,黄色最贵,红色是官僚才能使用的象征高贵的正色。汉诸侯王第宅以朱红漆门,故称朱邸。朱门还曾被纳入《礼记》中帝王赐重臣的``九锡''之列。明代朱元璋规定从宫门到九品官的府门依次为红门、绿门、黑门等。
``君自见''句:这里包含佛教的性空论,也契合了道家的齐物论。性空论认为``净土与秽处不二''、``真如与生灭不二''、``菩萨与众生不二''等;齐物论认为``言谈和鸟叫不同,但真有区别,还是没有什么区别?''刘惔抨击竺法深势利,
``翩然一只云间鹤,飞去飞来宰相衙'';竺法深抨击刘惔拘泥于``外相'',不能看到事情的本质。佛教在中国带到极大的发展,就是它的那种``方便法门''提供了一种既可以得到心灵解脱,又能够过奢侈生活的理论,造就了一大批亦仕亦隐、亦僧亦俗的``居士''、``狂禅'',导致了具有中国特色的知识分子的人格分裂,因为所有腐朽堕落的生活方式都可以得到很好的借口。
卞令:卞壸,字望之,山东菏泽人,历任东晋太子中庶子、吏部尚书、尚书令等,据《晋书》说,其为
人刚正不阿,以匡风正俗为已任,看不惯放浪形骸、群居淫乱的王澄、谢鲲等人,痛斥他们``悖礼伤教,罪莫斯甚,中朝倾覆,实由于此''。卞壸在平定苏峻之乱时
战死。由他来说也有可能。
竺法深虚伪。不过这个故事不由让人想起女人是老虎和苏轼骂佛印是屎的故事,禅机这个东西有时可以解闷和玩味。

\section{2.49}\label{section-95}

\begin{quote}
孙盛为庾公记室参军,从猎,将其二儿俱行。庾公不知,忽于猎场见齐庄,时年七八岁,庾谓曰:``君亦复来邪?''应声答曰:``所谓`无小无大,从公于迈。'''
\end{quote}

孙盛:字安国,山西平遥人,史学家,《世说》注中多次引用的《晋阳秋》就是他的作品,现散佚。曾任长沙太守、秘书监,封吴昌县侯等。孙盛是为良史,把桓温北伐时枋头之败如实记载于《晋阳秋》中,桓温大怒,就对孙盛的儿子孙放说:``枋头诚为失利,何至乃如尊君所说?若此史遂行,自是关君门户事(你们家的门就不会再开了)。''但孙盛就是不修改,孙放他们挡不住,自行修改,所以当时《晋阳秋》有两个版本。孙盛不修改也有历史渊源,当年他任长沙太守的时候,据说
颇置家私,州府派从事到郡里来调查,考虑到孙盛名望很高,不打算弹劾。而孙盛却给桓温写了一封信,把这些纪委干部大大揶揄了一番,说他们名义上是来地方督察,但来的时候一点也没有凤凰那样的威仪,走的时候也不像鹰隼那样敢于搏击,只是在湘江上飞来飞去,不是怪鸟又是什么?(``进无威凤来仪之美,退无鹰隼搏击之用,徘徊湘川,将为怪鸟''。)
桓温收到信后就抄了孙盛的家,抓了孙盛。后来大概双规后桓温发现,孙盛贪污查无实据,没有再处理。
庾公:庾亮,见1.31。 记室参军:文字幕僚长。
从猎:在古代有个田猎制度,国家每季度都举行一次大规模的田猎,由官吏率领,在猎事前后受武艺和战阵的训练。《周礼》中有田猎制度的详细要求,《左传》中说:``春蒐、夏苗、秋狝,冬狩,皆于农隙以讲武事也。''其作用如管仲所说:``(军事训练使)祭祀同福,死丧同恤,祸灾共之。人与人相畴,家与家相畴,世同居,少同游。故夜战声相闻,足以不乖;昼战目相见,足以相识。其欢欣足以相死。居同乐,行同和,死同哀。是故守则同固,战则同强。''
齐庄:孙盛的二儿子孙放,字齐庄。
君亦复来邪:您这次也又来了?对小孩子这样说话,庾亮的风度可知。
``无小''句:出自《诗经 - -
泮水》,该诗是鲁僖公打胜仗后,在泮宫祝捷庆功,宴请宾客的诗,首节为``思乐泮
水,薄采其芹。鲁侯戾止,言观其旂。其旂茷茷,鸾声哕哕。无小无大,从公于迈。''其大意是``在泮水边采摘水芹,多么快乐。看看旗帜,知道鲁公驾临。战旗猎猎,车铃叮叮。大小臣子,都随公出游。''孙放把庾亮比作打了胜仗的鲁公,把小孩子也参加来烘托围猎的盛况,这不仅仅是``捷悟'',更表现出很好的文化素养。
如今文明昌盛,但别说孩子,就是吾辈的文化修养却没有7岁小儿那样的底蕴了。在《世说》中,记载了许多儿童的捷对,而且长者有礼,童子识趣;长者无状,童子放肆。作者把这些小故事认真地收集、记录下来,这从一个侧面反映出晋代人平等意识的上升。

\section{2.50}\label{section-96}

\begin{quote}
孙齐由、齐庄二人小时诣庾公。公问齐由何字,答曰:``字齐由。''公曰:``欲何齐邪?''曰:``齐许由。''齐庄何字。答曰:``字齐庄。''公曰:``欲何齐?''曰:``齐庄周。''公曰:``何不慕仲尼而慕庄周?''对曰:``圣人生知,故难企慕。''庾公大喜小儿对。
\end{quote}

孙齐由:孙盛的大儿子孙潜。 齐庄:二儿子孙放。
庾公:庾亮。当时孙盛在庾亮帐下当参军,和庾亮有私交。据《晋书》上说,有一次王导要打压庾亮的势力,孙盛就劝庾亮:王导常有世外之怀,怎么会屈身做那些凡夫俗子的事情,必定有小人在挑拨离间。后来一查,是陶侃的儿子陶称在作祟,庾亮杀陶称。陶侃家的军事势力从此没落。
欲何齐邪:这里的``齐''是见贤思齐的意思,达到。
齐庄何字:似乎应该表述为``问齐庄'',不然回答的主人公是谁会产生歧义。
圣人生知:圣人生而知之的省略语。《论语 -
季氏》子曰:``生而知之者,上也;学而知之者,次也;困而学之,又其次也;困而不学,民
斯为下矣。''而孔子在《论语 -
述而》中说:``我非生而知之者,好古,敏以求之者也。''虽然孔子认为自己不是生而知之的圣人,但后人(特别是他的弟子)以为孔子是谦虚。子贡就认为自己的老师是``固天纵之将圣''(《论语
-
子罕》),``仲尼,日月也\ldots{}\ldots{}夫子之不可及也,犹天之不可阶而升也''(《论语
-
子张》)。学生把老师捧得很高,说他是``大师'',其实是有意无意抬高自己的方式,或者在其中捞取自己的利益。现在有些先生说自己不是``大师'',但学生这样说他总是笑,``大师''证书也是自己去领来的。
故难企慕:这句话也许透露出,无论魏晋时候儒家看似式微,老庄如何推崇,其根本地位和价值判断还是儒家胜于佛道,但内在人格意识的觉醒使魏晋人更觉得佛道亲切一些,儒家消除不了人生的感伤和对人生意义的追问。

\section{2.51}\label{section-97}

\begin{quote}
张玄之、顾敷是顾和中外孙,皆少而聪惠。和并知之,而常谓顾胜,亲重偏至,张颇不懕。于时,张年九岁,顾年七岁。和与俱至寺中,见佛般泥洹像,弟子有泣者,有不泣者。和以问二孙。玄谓:``被亲故泣,不被亲故不泣。''敷曰:``不然,当由忘情故不泣,不能忘情故泣。''
\end{quote}

张玄之:又名张玄,字祖希,历任吏部尚书、吴兴太守、会稽内史等,又称张吴兴。魏晋时期江东有朱张顾陆四大家,张玄之属于张家,其谱系不清,但和想吃鲈鱼脍辞官回家的张翰应有亲缘关系。从《世说》看,张玄之和顾敷不仅是表亲,可能他的妹妹还嫁给了顾敷,早早当了寡妇。张玄之和顾敷两人年纪在《世说》中就有抵牾,一说同岁,一说张玄之大顾敷两岁。张玄之和谢玄齐名,时称``南北二玄'',《世说》中有好几则他的故事。
顾敷:字祖根,23岁去世。 顾和:见2.33。
中外孙:中就是内的意思,孙子和外孙。 偏至:结合前文大概是偏颇的意思。
不懕:恹或厌。大概是不安、不满足的意思。
般泥洹:又叫般涅槃,般是佛教用语,直译为般若、波若等,意译为``智慧''。
涅槃的本意大约是消灭烦恼,获得真如永恒的意思,也代指僧人死亡。现在的寺庙中卧佛像是有的,但卧佛的壁画好像没有见过,或者我没留心。据《出三藏記集》中说释迦牟尼去世时,``天地震动,师子等百兽悉大哮吼,诸天人号啕,山林树木皆悉摧裂,天女人女无量百千喐咿交涕不能自胜。诸有学人(烦恼未断的学生)佥然不乐,诸无学人(断尽烦恼的学生)但念诸法一切无常。\ldots{}\ldots{}佛既灭度,诸大罗汉各各随意,於诸山林流泉谿谷,处处舍身而般涅槃,或有飞腾虛空雁行而去,現
种种神变令众人得信心清净。''
被亲故泣:平时被释迦牟尼亲厚的所以要哭。张玄之的出发点大约还是儒家学说的亲亲厚厚观点,看前文也许是趁机表露对顾和的不满。
忘情:《世说新语 -
伤逝》中说:``圣人忘情,最下不及情。情之所钟,正在我辈。''圣人忘情就是修养高的人能做到不为感情所动。《出
三藏記集》中说,释迦牟尼去世``唯阿难亲爱未除未离欲故,心没忧海不能自出'',佛教认为有情皆苦,就是着相,学佛的目的就是弄通苦、集、灭、道``四谛''。
而庄子认为,``至乐无乐,至誉无誉,\ldots{}\ldots{}今又变而之死,是相春秋冬夏四时行也。人且偃然寝于巨市,而我嗷嗷随而哭之,自以为不通乎命,故止也。''顾敷的回答从玄理出发,的确比张玄之深邃一些,这个故事也是为了印证``顾胜''。但从现在的眼光看,一个7、8岁的小孩子说这样的话,无非就是无体悟的鹦鹉学舌,不足为观。

\section{2.52}\label{section-98}

\begin{quote}
康法畅造庾太尉,握麈尾至佳。公曰:``此至佳,那得在?''法畅曰:``廉者不求,贪者不与,故得在耳。''
\end{quote}

康法畅:和尚名,《高僧传》里有他的简单介绍,他写过一本《人物始义论》,说自己``悟锐有神,才辞通辩。''
庾太尉:庾亮。
麈尾:一种扇子或拂尘,有点像如意,上面扎着驼鹿尾(驼鹿也叫``四不像'',魏晋时期在中国分布很广,现在很少),一些画里保留着它的形状。传说驼鹿行走
时,以前鹿之尾为方向。东汉名士手执麈尾清淡,实际就是以``主鹿''自命,表示引导潮流,后相习成俗,成为名流雅器,所以清谈也称麈谈。当时弘扬佛法也叫
``麈尾一振''或``捉麈尾'',如《高僧传 -
释慧通传》中说:``(释慧通)少而神情爽发,俊气虚玄,止于治城寺。每麈尾一振,辄轩盖盈
衢。''
那得在:怎么还能留得住?麈尾是当时一种重要的礼物,并具有象征意义。
贪者不与:廉者不求倒还罢了,贪者不与表示康法畅``我的就是我的'',不为外物所侵扰的意思。这句话和佛教思想有区别,也是其中国化的表现。

\section{2.53}\label{section-99}

\begin{quote}
庾稚恭为荆州,以毛扇上武帝,武帝疑是故物。侍中刘劭曰:``柏梁云构,工匠先居其下;管弦繁奏,钟、夔先听其音。稚恭上扇,以好不以新。''庾后闻之,曰:``此人宜在帝左右。''
解释:
此节有误,据《晋书》,庾稚恭当做庾怿庾叔预,荆州当做豫州,晋武帝当做晋成帝。
庾稚恭:庾翼,庾亮的弟弟,书法家。庾亮死后,代镇武昌,任都督江、荆、司、雍、梁、益六州诸军事、荆州刺史。其素胸有大志,以收复北方为已任,中道病死。
庾怿:庾亮的弟弟,历任辅国将军、豫州刺史等。庾亮死后,晋成帝和琅琊王家决定打压庾家,说``大舅(庾亮是成帝的舅舅)已乱天下,小舅(庾怿)复欲尔邪!''庾怿因故自杀。
毛扇:羽毛扇、白羽扇。傅咸《羽扇赋序》中说:``昔吴人直截鸟翼而摇之\ldots{}\ldots{}灭吴之后,翕然贵之,无人不用。''
故物:旧物。常理当然是送礼古董要旧,日用品要新。
刘劭:字彦祖,彭城刘氏,祖父刘讷是西晋``二十四友''之一。历任侍中、豫章太守等。
柏梁:汉武帝时期造的一个宫廷建筑,因以香柏为梁,故名。《三辅旧事》中说:``以香柏为梁也,帝尝置酒其上,诏群臣和诗,能七言诗者乃得上。''后泛指宫殿。
云构:高耸入云的建筑物,大厦。唐武宗废佛敕书中说,当时寺庙``寺宇招提,莫知纪极,皆云构藻饰,僭拟宫居''。
钟、夔:伯牙的知音钟子期,尧舜时的乐官夔,这里代指懂得鉴赏的音乐家。
该句出自《三国志 -
魏书》刘桢答曹丕语:``夫尊者所服,卑者所修也;贵者所御,贱者所先也。故夏(厦)屋初成而大匠先立其下,嘉禾始熟而农夫先尝其粒。''但语言更加简练优美。
此人宜在帝左右:有个笑话,一个大臣要离京赴任,临走时对皇帝说,希望陛下远小人亲忠臣。皇帝问他如何辨忠佞。大臣说,说臣好的就是忠臣,说臣坏话的就是小人。
\end{quote}

\section{2.54}\label{section-100}

\begin{quote}
何骠骑亡后,征褚公入。既至石头,王长史、刘尹同诣褚。褚曰:``真长,何以处我?''真长顾王曰:``此子能言。''褚因视王,王曰:``国自有周公。''
\end{quote}

何骠骑:何充,字次道,安徽次等士族出身。在晋成帝时由王导推荐任侍中,晋康帝时封骠骑将军。晋穆帝二岁即位,他任宰相,与庾亮的弟弟庾冰、会稽王司马昱等共辅幼主,处理朝政。《晋书》上说:``充虽无澄正改革之能,而强力有器局,临朝正色,以杜稷为己任。凡所选用提拔,皆以功臣为先,不以私恩树亲戚,谈者以此重之。''他在康帝死前即以褚裒出镇京口,主持军务。
褚公:褚裒,字季野,见1.34,他当时任徐州、兖州刺史,镇守京口。因为他是褚太后的父亲,当时朝廷根据庾家以外戚主政的先例,就征他入朝欲任尚书事,但他本人可能出于主掌兵权、守住南京咽喉之地更有利于家族发展和朝廷稳定的考虑,并不愿株守朝廷。
石头:石头城,现在南京清凉山一带,所谓``钟山龙蟠,石头虎踞'',三国孙权时就石壁筑城,依临长江,易守难攻,称石头城。后来人们也把南京城叫石头城。
王长史:王濛,字仲祖,太原王氏,当时任司徒左长史。他性格放旷,曾照镜子自夸说:想不到``想不到王讷能生这样好的儿子(王文开生如此儿邪)!'';死的时候又说:``像想不到我如此了不起的人还活不到四十岁(如此人曾不得四十也)!''他与刘惔、桓温、谢尚当时称为``四名士''。其女儿也是晋哀帝司马丕的皇后。
刘尹:刘惔刘真长,见1.35。他们都和会稽王司马昱交好,所以推荐司马昱主持朝政。
周公:周公旦,周武王的弟弟,周成王的叔父,成王年幼,周公辅佐朝政。这里比作司马昱。司马昱是当时清谈协会的名誉会长,这些名士围绕在司马昱身边,在其政治发展中起到了谋士和拥护者的作用,最后终于使司马昱当上了比较窝囊的简文帝。
当然,《世说》这样描写是对历史的简化,事实上,有一大批人在权衡褚裒要不要离开军队前来辅政,因为当时桓温的势力很大,需要有人在军事上加以牵制平衡,褚裒镇守京口就是为了起到这个作用。褚裒回去后以都督徐、兖二州又兼征北大将军,经略北伐军务,避免他人以北伐之名,挟北伐之功,形成对司马家的觊觎。

\section{2.55}\label{section-101}

\begin{quote}
桓公北征,经金城,见前为琅邪时种柳,皆已十围,慨然曰:``木犹如此,人何以堪!''攀枝执条,泫然流泪。
\end{quote}

曹操的老乡,18岁手刃杀父仇人之子;皇帝的女婿,一生戎马,功高震主,``政由桓氏,祭由寡人'',主持废立之事;满门的高官,名士的言谈,朝中人不敢仰视,他还准备刀斧手要杀王谢两家族长。枭雄桓温终于作为主人公登场了,我们已经见过多次他,但都是侧面,这次正面与他相遇,呀,原来他还这么多愁善感。我们至今还记
得他,也许并不是这一句话,而是另外一句``既不能流芳百世,亦不足复遗臭万载耶'',这句话就像曹操那句``宁教我负天下人,不叫天下人负我'',如此惊心动魄的赤裸,如此令人厌恶的狂妄,以至于站上道德底线的人都有理由谴责他们。不过只要我们读过他们的传记,不免会想起王尔德在《温德米尔夫人的扇子》所说的:``任何把人们分成好与坏,这委实荒唐,人们要么有趣,要么乏味。''桓温,决不是乏味的人。他的父亲桓彝和大名士温峤是好友,桓温出生时,温峤来祝贺,听其震天的哭声,赞叹道:``真英物也!''桓彝佩服温峤,就给孩子取了``温''的名,字元子,希望孩子长大后成为温峤这样一流的人物。结果物超所值,桓温成就了更大的名声。桓温的才能在年轻时代就独占鳌头,得到执政庾翼的赏识,向成帝推荐说:``桓温少有雄略,愿陛下勿以常人遇之、常婿畜之,宜委以方、召之任,托其弘济艰难之勋。''于是桓温永和元年(345年)34岁时出任荆州刺史,35岁率7000人直入四川,平定蜀地,擒获汉王李势。从此以后南征北伐,去过氐族人占据的长安,收复羌族人占据的洛阳,打到鲜卑人占据的枋头,纵观东晋一朝,谁也事业都不能和桓温比肩。但是,几乎所以人都认为,桓温是个丘八,桓温是曹操,桓温必反,桓温当皇帝了,大家日子不好过了。桓温晚年,成为门阀士族的公敌,最终病死,终年61岁。《晋书》中说:``桓温挺雄豪之逸气,
韫文武之奇才,见赏通人,夙标令誉。时既豺狼孔炽,疆场多虞,受寄捍城,用恢威略,乃逾越险阻,戡定岷峨,独克之功,有可称矣。''桓温是誓报父仇的少年,是高谈阔论的名士,是亲为士卒的将军,是泫然泪下的情种,是随时会颠覆晋朝的乱臣孽贼。
金城:现在山东诸城,原南琅邪郡郡治。桓温在咸康七年(公元341年)任琅邪国内史时镇守金城,这次是太和四年(公元369
年)伐燕国,已经相距28年。
十围:十掐,也就是胸围在1米5到2米左右。我们小时候听评书,说一员猛将往往是身高一丈,腰宽十围,这么大的柳树还真少见。有人说也许是快死了柳树才这么大,因为庾信有个名篇叫《枯树赋》,其中说:``桓大司马闻而叹曰:昔年种柳,依依汉南;今看摇落,凄怆江潭。树犹如此,人何以堪!''。但也有人说,十围的柳树正是郁郁葱葱之时,而桓温已经57岁,夕阳迟暮,所以特别感慨。
无论是这棵树快死了还是生命力十分旺盛,总之,这种引起人生的集体感伤是魏晋文学的一个重要主题。我们只要翻翻《古诗十九首》,那时人们对时间流逝、生命短暂的悲哀情绪几乎俯拾皆是:``白杨何萧萧,松柏夹广路'',``所遇无
故物,焉得不速老'',``人生天地间,忽如远行客''\ldots{}\ldots{}这种表面上的消极悲观,其实真正要表达的是它的方面,就是对人生、欢乐、事业的强烈追求和留恋。
后来桓温这场性情中人的痛哭成为一个著名的典故,辛弃疾《水龙吟》唱道:``把吴钩看了,阑干拍遍,无人会,登临意。\ldots{}\ldots{}可惜流年,忧愁风雨,树犹如此!倩何人、唤取红巾翠袖,揾英雄泪!''英雄,不仅在于他的功业,而且也在于他的情怀。不过``尧与舜,你让天子;我笑那汤与武,你夺天子;他道是没有个傍人儿觑,觑破了这意思儿,也不过是个十字街头小经纪''!
我们也不要以为桓温真的不可一世,365年,他已经都督中外诸军事、录尚书事、遥领扬州牧,但他依旧很担心朝廷突然对他下手,他当时兵驻安徽芜湖,唯恐朝廷驻扎在京口的徐、兖军队和寿春、合肥的豫州军包围攻击,《太平寰宇记》中说:``昔桓温
驻赭圻,恒惧掩袭。此圻宿鸟所栖,中宵鸣惊。温畏官军至,一时惊溃。既定,乃群鸟惊噪,故相传为战鸟山。''正是为了解决豫、徐军队,所以他假借北伐名义整肃军队,并在途中哭了一场,虽然遭遇枋头兵败,但取得了兼领徐、兖二州的职务,还诿过他人,以其子桓熙代为豫州刺史,终于扫清了进入建康的障碍。

\section{2.56}\label{section-102}

\begin{quote}
简文作抚军时,尝与桓宣武俱入朝,更相让在前。宣武不得已而先之,因曰:``伯也执殳,为王前驱。''简文曰:``所谓`无小无大,从公于迈'。''
\end{quote}

简文:简文帝司马昱,320-372。
抚军将军:魏晋南北朝设中军、镇军、抚军三将军,三品,地位仅次于骠骑将军、车骑将军、卫将军。
桓宣武:桓温,谥号宣武,312-373。
更相让:互相推让。司马昱是郡王,岁数小、官职低;桓温是大司马,官位高、岁数大,各有优势,座次不好排。
``伯也''句:引自《诗经 -
伯兮》:``伯兮朅兮,邦之桀兮。伯也执殳,为王前驱''。大意可能是妻子夸自己的先生高大英武,是邦国中的
杰出人物。他手拿着国王的仪仗,为国王作先锋开道。桓温引用这句诗是谦虚的说法,说自己拿着长殳为司马昱开道。桓温要周旋于士族名士之间,亦求附庸风雅。
``无小''句:见2.49。司马昱哪里敢当啊,就同样引用《诗经》的句子,说``我们这些大小官员,都紧紧围绕在以您桓大司马为核心的中央周围,在您的带领下,努力开创建设和谐社会的新局面''。这样的对答,两个人都倍文雅,倍有面子。
中国古代的诗歌以含蓄为美,中国古人讲话,也讲究文雅收敛。当今的温家宝先生频频在各种场合引用诗歌,也是为了体现自己的这一特质,不过平时讲话繁复,冲淡了其原始目的。
司马昱后来被桓温推上前台当皇帝,他死时留下遗嘱给桓温:``少子可辅者辅之,如不可,君自取之。''后来在王坦之的劝说下,改成了``家国事一禀大司马,如诸葛武侯、王丞相故事''。
司马昱一辈子对桓温都惧怕。

\section{2.57}\label{section-103}

\begin{quote}
顾悦与简文同年,而发蚤白。简文曰:``卿何以先白?''对曰:``蒲柳之姿,望秋而落;松柏之质,经霜弥茂。''
\end{quote}

顾悦:江东顾家子弟,字君叔,又名悦之(晋代人单名的话往往后面加个之字),大画家顾恺之的父亲。
简文:以后无特殊情况不解释了。
蚤白:大概是少白头,也许是用脑过度早生华发。前些年我见同事年纪轻轻,工作轻松,就纳闷怎么就发已二毛。最近自己去理发,工作人员一声惊叹:``怎么突然有白头发了!''我就笑对自己多情,估计小姐也听不懂。如果是女朋友,就简单多了:``想你想的呗!''
蒲柳之姿:有人说蒲柳分别是指水杨树和柳树,有人说就是仅指柳树,我们不必细究,总之是很容易凋落的树种,比喻体质早衰的人。后人也常常用蒲柳弱姿比作卑微的女子或者阿娜多姿的女子,这是种曲解和引用,但用多了也就固定下来。如《红楼梦》写那个嫁给中山狼的迎春,其中的诗歌就是:``觑着那,侯门艳质同蒲柳;作践的,公府千金似下流。''唐代诗人韩翃给分别多年的老情人写过一首诗:``章台柳,章台柳,往日依依今在否?纵使长条似旧垂,也应攀折他人手。''情人
回信说:``杨柳枝,芳菲节,可恨年年赠离别。一叶随风忽报秋,纵使君来岂堪折。''回信用了柳树未老先衰的典故,感慨和无奈,很让人怜惜。姿:通``资'',资
质,《论衡 - 本性》:``初禀天然之姿,受纯一之质。''
顾悦把自己比作经不起时节的蒲柳,,把简文帝喻为傲立霜寒的松柏。言辞对仗优美,深得赋体之妙,不仅奉承不着痕迹,而且言有尽而意无穷,可以激发读者更加广泛的联想,自然得到文学青年司马昱的高度赞叹,``王称善久之''。顾凯之为父传时就主要回忆了这件往事。

\section{2.58}\label{section-104}

\begin{quote}
桓公入峡,绝壁天悬,腾波迅急。乃叹曰:``既为忠臣,不得为孝子,如何?''
\end{quote}

桓公入峡:345年,庾翼去世,临死时推荐好友桓温接替荆州刺史。346年,桓温上表要求征伐成汉(也称后蜀),不等朝廷批复就带领7000人起兵,沿江而上。成汉,是晋惠帝时流民领袖李特的儿子李雄在304年四川建立的,现在是李势即位,其政治运作模式模仿石虎,以杀戮为能事,国内很乱。
``绝壁天悬''句:四川天险,蜀道艰难,地崩山摧,绝壁急流,心胆欲丧,剑阁峥嵘,,万夫莫开。我们小时候学过郦道元《水经注》中的名篇:``自三峡七百里中,两岸连山,略无阙处;重岩叠嶂,隐天蔽日,自非亭午夜不见曦月\ldots{}\ldots{}故渔者歌曰:`巴东三峡巫峡长,猿鸣三声泪沾裳。'\,''桓温经略荆州不久就西征,``诸将佐皆以为不可'',朝廷也不支持,``众少而
深入,多以为忧'',本来就是非常之人做非常之事。桓温的勇气无可置疑,但内心总忐忑不安。就像曹操北征乌桓大捷,回来后重赏那些谏阻的谋臣,老实地说:``我这次远征,实在是一次很冒险的举动,虽然徼幸取得成功,是老天帮忙,只能偶尔为之,不能经常如此。你们当初劝阻我,所贡献的是万全之策。这次奖励你们,是希望以后你们有什么想法,都可以依旧讲出来。(孤前行,乘危以徼幸,虽得之,天所佐也,故不可以为常。诸君之谏,万安之计,是以相赏,后勿难言
之。)''军事行动,即使是名将,每一个命令都是以战栗的心情作出的。
既为忠臣,不得为孝子:西汉王阳任益州刺史,来到宜宾邛僰九折阪,看到道路如此险峻,就感叹说:``父母生了我,我怎么能多次冒这种险(奉先人遗体,奈何数乘此险)!''于是托病辞官。后来王尊继任益州刺史,路过这里,就叫车夫赶
马前进,说:``王阳选择当孝子,我选择当忠臣(驱之,王阳为孝子,王尊为忠臣)!''
如何:本来是疑问句:怎么样?但这里似乎应该理解成反问句:就是这样!桓温为了鼓舞士气,说忠臣不能贪生怕死,忠大于孝。在儒家学说中,忠臣孝子有时候会形成悖论,如在法家的寓言故事中,孔子认为为了行孝也可以当逃兵。但这是韩非子的故意歪曲和讽刺。很多事例说明,儒家也认为忠君报国要重要一些。这句话也许反映出桓温伐蜀时犹豫的心情,同时,从一个侧面说明桓温的英雄气概------胜利代表一切。一般认为,随着桓温的巨大胜利,忠臣的概念在桓温心里也逐渐隐退。

\section{2.59}\label{section-105}

\begin{quote}
初,荧惑入太微;寻,废海西。简文登阼,复入太微,帝恶之。时郗超为中书,在直。引超入曰:``天命修短,故非所计。政当无复近日事不?''超曰:``大司马方将外固封疆,内镇社稷,必无若此之虑。臣为陛下以百口保之。''帝因诵庾仲初诗曰:``志士痛朝危,忠臣哀主辱。''声甚凄厉。郗受假还东,帝曰:``致意尊公,家国之事,遂至于此!由是身不能以道匡卫,思患预防。愧叹之深,言何能喻!''因泣下流襟。
\end{quote}

我们眼见桓温和司马昱互相推让行次,眼见桓温奋勇向前去打打胜仗,又眼见桓温把司马昱吓哭了。
荧惑:火星。太微:天空的某片区域。古代天象学我不懂,大约火星到太微区域比较少见,古人又讲天人合一,以为天象与政府人事有关,牵强附会,譬如白虹贯日、日食月食之类,要么是天子要死了,要么说明天子荒淫无道,要下罪己诏了。我们不必管它。反正这个故事我们可以看出,天象这个东西不靠谱,随你理解的。
寻:不久。
废海西:桓温第三次北伐兵败,声望下降,他有点担心罢免程序启动。他问谋士郗超有什么办法让大家再高看他一眼。郗超出主意叫他废了皇帝司马奕,另立简文帝。桓温正中下怀,废司马奕为东海王,后又降封为海西县公。
帝恶之:鉴于天象,司马昱害怕桓温再兴废立事。
郗超:字景兴,一字嘉宾,名臣郗鉴的孙子,都督徐兖青幽四州诸军事、平北将军、徐州刺史郗愔的儿子,桓温的谋主和好友。当时任中书侍郎,是桓温放在中央掌控政坛的重要人物。简文帝找郗超聊天,也许是希望他当好传声筒,表达自己是很听话的人,是很怕桓温的人。从史料看,桓温确实权势滔天,说他是霍光不以为过,但桓温为人也不见得飞扬跋扈,更看不出来他一定想当皇帝,只是当时的其他门阀实在不服气,用旁敲侧击的方法警告桓温或者心怀妒忌,反正中国人习惯把嫉
妒冒充成正义。桓温也稍微死的早了一些,如果活满70岁,估计现在的历史书上全是他伟岸的形象。
在直:当班值日。 天命修短:天命 是指皇室的命运。修就是长的意思。《庄子 -
齐物论》中说:``予恶乎知悦生之非惑邪?予恶乎知恶死之非弱丧而不知归者邪?\ldots{}\ldots{}予恶乎知夫死者不悔其始之蕲生乎?''大意是说:``我哪里知道贪生并不是迷误,我哪里知道人之怕死并不是像幼年流落在外面不知回归故乡呢?\ldots{}\ldots{}我哪里知道死了的人不会懊悔他从前求生呢?''庄子是通达,简文是无奈:``国家命运的长短,本来就不是我能关怀的。''
政当无复近日事不:``政当''它本有作``故当'',
可以理解为``总应该'',《史记 -
项羽本纪》``义帝虽无功,故当分其地而王之''。
无复:不会再有。近日事:指废立之事。不:疑问词。整句话的意思是:``总应该不会再发生废立之事了吧?''但也有说``政当''通``正当'',
如《梁书 -
范云传》中记载,齐武帝曰:``闻范云谄事汝,政当流之。''正当就是应当,意思差不多的。
大司马:指桓温。 封疆:封就是疆,疆界。
以百口保之:我以全家上百口人担保桓温不会废立。
庾仲初:庾阐,字仲初,庾亮一族,曾任彭城内史、散骑侍郎、零陵太守等。诗人。本诗叫《从征诗》,原诗散佚。
受假还东:指获准请假去会稽探望父亲郗愔。郗愔本来镇守徐、兖二州,忠于朝廷,是桓温头上的刀子。桓温借北伐为名,自领平北将军、徐兖二州刺史,调郗愔为会稽内史,都督浙江东五郡军事。
尊公:郗愔是郗鉴的儿子,王羲之的小舅子,袭爵南昌县公。郗愔和郗超政治立场不同,不喜欢桓温。
以道匡卫,思患预防:这里是说司马昱本来不想当皇帝,是桓温硬要他当的,他没有做到坚持正确的主张,防患于未然。简文帝的确不想当皇帝,在位8个月病死(愁死?),享年53岁,比桓温还小9岁。
司马昱三岁时即受封为琅邪王,后转会稽王,历经明、成、康、穆、哀与废帝六位皇帝,长期置身于帝位争夺之外,官位一路扶摇直上,当上皇帝后在很短的时间内大量援引会稽人士以及玄学同好进入中央,以制衡桓温,对东晋后期的政局颇有影响。
司马昱对桓温的死党郗超讲做人要当忠臣,自己现在很受欺负、很无能之类的话,是不是很妥当?他到底想达到一种什么目的?令人费解。难道起效果了,郗超幡然悔悟?

\section{2.60}\label{section-106}

\begin{quote}
简文在暗室中坐,召宣武,宣武至,问上何在。简文曰:``某在斯。''时人以为能。
\end{quote}

能:后面漏``言''字。
这个故事难理解,这么一句毫不起眼的话怎么就能算``善于言辞''呢?难道是引用恰当,不着痕迹?简文帝在暗室中召见桓温也比较奇怪。
``某在斯''出自《论语 -
卫灵公》,孔子喜欢和重视音乐,有一次他和宫廷乐师师冕会面,因为师冕是盲人,所以孔子给以引导,说:``台阶在这里、座位在这里'',落座后向他一一介绍与会人员:``某人在这里,某人在这里''。某本来是为了行文简便,不明确指出具体姓名的人。到了晋代,``某''也成
为自指代词,``我在这里''。 说这句话讲的好,也许是默契的快乐吧。

\section{2.61}\label{section-107}

\begin{quote}
简文入华林园,顾谓左右曰:``会心处不必在远,翳然林水,便处有濠、濮间想也。觉鸟兽禽鱼,自来亲人。''
\end{quote}

华林园:传说中弥勒佛在华林园成道,并在此园说法度化众生。东汉时洛阳有华林园,后东晋仿洛阳,在南京东吴皇家花园的基础上改建华林园。
顾谓:对周围的人说,见2.33,。
会心处不必在远:这是一个中国文化的著名论题,它给出了所谓仕隐的答案。做好官是很受拘束的,正如嵇康在与山涛的信中说,做官有``七不堪,二甚不可'',
明代康海有首曲子写道:``见如今狼虎相豭,眼睁睁许多愁怕。\ldots{}\ldots{}新催科旧拖欠乱如麻,碜可可穷百姓,恶狠狠甚刑法,你便是铁心肠,怎下的打?
\ldots{}\ldots{}少不的骑鞍压马,升堂佥画。甫能的出几纸批儿,勾几个人儿,怎当他命是人差,问着他口兀剌,虚惊诡怕。你便是炳灵神也百错千差。想着那官家礼法,都被这奸豪厮诈。他结交的吏典滑熟,人情通透,上下欢洽。既不曾拿住他,又怎么下重罚?他反关门,倒又把别人诓吓,怪不的整年家事无捉抐。\ldots{}\ldots{}少年时豪气天来大,动不动要做英雄俊雅。到如今行步紧钳拿,是多少海样波查。指望待春风禁柳行骢马,倒做了落日庭槐数暮鸦。愿的是风雨顺收禾稼,落一个民安事妥,说甚么紫绶黄麻。''当皇帝更危险,中国皇帝的平均寿命不过三十岁左右,东汉、两晋的皇帝多数还是傀儡皇帝,更何况批不完的奏折、捉襟见肘的国库、狼烟四起的反叛、居心叵测的兄弟等等。那么如何消解内心的无奈、恐惧,如何获得精神的自由和人格的完善呢?道家给了一条出路------寄情山水,从大自然中获得快乐,保持天
真率性的品质,进入自得其乐的佳境,惬意享受``会心''的乐趣。在两晋时期的文章诗歌中,保留了大量官僚游山玩水的记录,自然山水成为士人生命体验和人格主体的投射倒影,个体从中获得一种``独与天地精神往来''的愉悦。而且更重要的是,由于佛教理论的介入,它讲的是``一花一念无量劫,大千俱在一毫端''、``须弥纳芥子,芥子纳须弥'',你还不必舍近求远,随时随地可得乐境,这种不用脱离官职就能享受精神愉悦的理论对于具有享乐天性的人类来说是何等宝贵的说辞,从此
以后,我们可以心安理得地活在尘俗,享受尘俗。
翳然:遮蔽貌。据说园林设计中,讲究的是有水但不露水,就要依靠地势和树木进行恰当的遮蔽。
濠、濮:这是庄子关于鱼的两个故事。一个故事说,庄子与惠子游于濠梁之上。庄子曰:``儵鱼出游从容,是鱼之乐也!''惠子曰:``子非鱼,安知鱼之乐?''庄子曰:``子非我,安知我不知鱼之乐?''惠子曰:``我非子,固不知子矣;子固非鱼也,子之不知鱼之乐,全矣。''庄子曰:``请循其本。子曰`汝安知鱼乐'云者,既已知吾知之而问我。我知之濠上也。''我很快乐,所以觉得鱼也一定很快乐,这是在濠梁观鱼寄情式的快乐。另一个说庄子正在濮水之上垂钓。两个楚王派来
的使者说大王``愿以境内累矣'',要聘任庄子为楚国相。庄子``持竿不顾'',一边继续垂钓,一边向两位大夫讲起了楚国的典故:听说贵国有只神龟,已死了三千年,尸骨被大王珍藏于庙堂之上,尊荣无比。请问两位大夫,``此龟者,宁其死为留骨而贵乎?宁其生而曳尾于涂中乎?''两位大夫只好按常理回答:``宁生而曳尾涂中。''庄子于是拒绝说:``往矣。吾将曳尾于涂中。''这是做自由自在的鱼的快乐。后人就把``濠濮间想''比喻悠然自得的精神状态。
自来亲人:不
说人对鱼鸟的亲切,而说鸟主动与人相亲,这是移情拟人的手法,表达自己已经与自然融为一体,物我两忘,自然的人化与人的自然化得到统一,这种对自然的赞赏与亲近是以中国古代作品中所表现的农业文明的主要旋律之一,至今为止,我们依旧很难读到对钢筋水泥的城市丛林的赞美和对能够建功立业的官职岗位的歌颂之类
的作品。但事实是我们离不开它,我们向往的也许是在(《庄子 -
让王》)中所说的一个寓言故事:孔子问颜回为什么不去当官,颜回回答:
``回有郭外之田五十亩,足以给粥;郭内之田十亩,足以为丝麻;鼓琴足以自娱,所学夫子之道者,足以自乐也!回不愿仕。''白居易更是说:``大隐住朝市,小隐入丘樊。丘樊太冷落,朝市太嚣喧。不如作中隐,隐在留司官。似出复似处,非忙亦非闲。不劳心与力,又免饥与寒,终岁无公事,随月有俸钱。''(《中隐》)

\section{2.62}\label{section-108}

\begin{quote}
谢太傅语王右军曰:``中年伤于哀乐,与亲友别,辄作数日恶。''王曰:``年在桑榆,自然至此。正赖丝竹陶写,恒恐儿辈觉,损欣乐之趣。''
\end{quote}

谢太傅:谢安,见1.33,320年-385年,世称谢太傅、谢安石、谢相、谢公等。谢安、王羲之从小都客居绍兴,算是老乡,虽然相差有点年纪,但在名士眼里这显然不是问题,经常一起同游娱乐清谈,王羲之说:``比当与安石东游山海,并行田视地利,颐养闲暇。衣食之余,欲与亲知时共欢宴,虽不能兴言高咏,
衔杯引满,语田里所行,故以为抚掌之资,其为得意,可胜言邪!''。谢安出山很晚,是359年弟弟谢万兵败后,为挽救家族地位才走上政治舞台的。以上的对
话,谢安尚未出山和王羲之辞官后的对答。
王右军:王羲之,字逸少,大概是属老虎的,小字阿菟,303年---361年。他父亲淮南太守王旷也是书
法家,和王导、王敦是堂兄弟。王羲之做过临川太守、江州刺史、右将军、会稽内史等,所以称王右军、王会稽。他13岁时就得到名士周顗的赏识,并寄托着王导、王敦的希望。但他的爱好显然在山水游仙而不是做官,屡辞侍中、吏部尚书这样中枢显要的职位。但后来,平时他很看不惯的太原王述居然做了他的顶头上司,屡次刁难,他一怒之下发誓再也不当官了:``自今之后,敢渝此心,贪冒苟进,是有无尊之心而不子也。子而不子,天地所不覆载,名教所不得容。信誓之诚,有如皦日!''他去世后朝廷赠金紫光禄大夫,他的儿子们因此固让不受。
因为王羲之是最杰出的书法家,所以他的故事流传下来很多,像谢安、王导这样的名相现在知道的人不多,但不知道王羲之的人很少,这大概就是艺术的魅力,时间是最优秀的评论家。
中年伤于哀乐:哀乐,偏义复词,仅指哀。就像我们平常说的不知轻重,生死相许、忘记、质量这些词汇,前面对偏义复词也有介绍。这句话是说,中年以后常常容易感伤。就像明代钟惺写道:``子侄渐亲知老至,江山无故觉愁生。''
恶:不愉快。史书上说:``(谢安)又于土山营墅,楼馆林竹甚盛,每携中外子侄往来游集,肴馔亦屡费百金,世颇以此讥焉,而安殊不以屑意。''自东汉以后。
匡扶天下的志向已经逐渐摆在及时行乐之后,有意义的人生当是没有忧愁和减少忧愁的人生。谢安常与子侄聚会,晚年纵情妓乐,就是为了聊以销忧。
桑榆:《初学记》中说,``日垂西,影在树端,谓之桑榆''。太阳下山时,阳光只照到桑榆树的树梢,桑榆借指黄昏或人的晚年。姚燧:``十年燕月歌声,几点吴霜鬓影。西风吹起鲈鱼兴,已在桑榆晚景。''
丝竹陶写:用音乐歌舞来陶冶和抒发。
恒恐儿辈觉,损欣乐之趣:这里的断句有两种说法,一种认为觉损是同意词,就是减少的意识;一种认为应断开,解释为我们这种生活方式如果给子侄辈察觉效仿,是不合适的,大人、老人可以做的事情,年轻人不能做的。
王羲之在《兰亭序》中说:``夫人之相与,俯仰一世,或取诸怀抱,悟言一室之内,或因寄所托,放浪形骸之外。虽趣舍万殊,静躁不同,当其欣于所遇,暂得于己,快然自足,不知老之将至。''他在晚年显然没有选择静思而选择放浪形骸这条路,``既去官,与东土人士尽山水之游,弋钓为娱。又与道士许迈共修服食,采药
石不远千里,遍游东中诸郡,穷诸名山,泛沧海,叹曰:`我卒当以乐死。'\,''在欢乐中死去,就像维吾尔民歌唱的那样:``我是破烂王,篝火是我的宝座,窝棚是我的宫殿,世界在我眼里一如废墟。我的左脸已被情火烧伤,右脸仍在唱情歌。''

\section{2.63}\label{section-109}

\begin{quote}
支道林常养数匹马。或言道人畜马不韵。支曰:``贫道重其神骏。''
\end{quote}

支道林:见2.45。他在《世说》中出现有数十次,称呼很多,恐怕还是有必要再介绍一下,他又叫支氏、支公、林公、林道人、支法师、林法师等。我们从《世说》看,支道林和江东名士交往极为频繁,并对《庄子》有独特深刻的理解,是公认的玄学专家。他在宣讲佛经时,常常借助玄学来解说佛学,使繁琐复杂的佛学变的通俗易懂一些,推动了玄学的佛学化,也推动
印度佛教中国化,为佛教在东晋的发展开拓了道路,以至于郗超评价说:``林法师神理所通,玄拔独悟,数百年来,绍明大法,令真理不绝,一人而已!''他平时``既乐野室之寂,又有掘药之怀'',喜欢作诗写字、养马放鹤、吃药服食、优游山水,要打算买山自住,一派名士阔少的派头。他有两件事对后世还有点影响,一是和尚可不可以吃鸡蛋,一是和尚要不要跪皇帝。第一件事,支道林本来也认为吃无受精卵是没有关系的,他师傅辩不过他,支道林一直心安理得地吃鸡蛋。若干年后
他师傅死了,他怀念他,希望师傅能在梦中出现。于是心想事成,他果然梦见师傅又出现在眼前,而且手里拿了个鸡蛋,往地上一掷,小鸡就出来了,得到梦示后支道林不吃鸡蛋了。不过我想,支道林后来不吃鸡蛋,其实是怀念自己的师傅,并找出一个借口,叫全中国的僧人都不吃鸡蛋,------都有机会去怀念师傅的教导之恩,
用心可谓良苦。当然,佛教一般认为,如果是不受精卵可以吃,照此类推,那么活物不可食,死物即可食,这种说法不过是为口腹而炫己智罢了。第二件事叫``沙门不敬王者'',当时像支道林这样的僧人常常奔走于皇室和达官显要之间,就引发了和尚要不要遵守儒家纲常拜皇帝的问题,通过支道林等和执政何充的交涉,达成一定的共识:僧侣是方外之人,有自己的理想和行为模式,因此不只不该和世俗的权威沾上边,见了皇帝也不用行俗人之礼。``今一令其拜,遂坏其法,令修善之俗,
废于圣世。''
不韵:韵的本意是指合符节拍的的声音,这里引申为不合适,不和谐,不符合僧人的行为规范。为什么这样讲呢?一种可能是当时讲究风雅,养鹰养鹤可以,养马这是俗人、下人干的事,一种可能是作为僧人不畜私人物品,应该专心于修持与解脱之道。
贫道:佛教传入中国时,佛教僧人也自称贫道。当时的``道''不是道教的意思,而是指至高无上的真理,佛教也认为它掌握着真理。贫道是谦虚的说法,提醒自己自己道德和智慧不足。在当时,佛教还和道教争过``道教''这个称呼,认为自己才配叫做``道教''。后来佛教和道教逐渐清晰分开,道人成为道士的专利,僧人只好自称贫僧了。
神骏:神骏也许就是指马形体健美,行动矫捷,负重致远,在形、神、能方面都具备较高的审美品位,支道林认为清峻萧疏、势可万里的骏马正好与他的胸怀相契合,是理想人格的象征,所以喜欢养。后来喜欢写马的杜甫有诗歌说:``可怜九马争神骏,顾视清高气深稳。借问苦心爱者谁?后有韦讽前支遁。''

\section{2.64}\label{section-110}

\begin{quote}
刘尹与桓宣武共听讲《礼记》。桓云:``时有入心处,便觉咫尺玄门。''刘曰:``此未关至极,自是金华殿之语。''
\end{quote}

刘惔和桓温是连襟,都是晋明帝司马绍的女婿,见1.35。不过刘惔的年纪比桓温要小一些,因为刘惔称桓温为``老贼'',想来私交也是好的。两个人在《世说》中有多次对话,给人总的印象是刘惔虽然认为桓温有政治、军事才能,但自持是玄学大家,在这个领域他看不起桓温,多次出言讥讽,而桓温为附庸风雅,打进玄学清谈的圈子,谋求政治利益,也能忍则忍。
礼记:古人治国,依据的无非是儒、法、道三家,为什么儒家最后占据主导地位呢?其中很重要一个原因就是儒家有系统治国的制度和方法,这个制度就是《礼记》,古代君主专制社会的政府架构、法律条文、行为规范、道德准则基本就出自于《礼记》,应该说,儒家的核心经典不是《论语》而是《礼记》。即使在晋代,玄学是时尚之学,但我们依旧发现名士们都非常熟悉儒家经典,玄学始终建立在儒学的基础之上,我们也不会忘记前面刘惔``莫得淫祀''的故事。所以看到两大名人请教师讲《礼记》,我们不必为奇。
翻开三家的经典著作,就可以发现,《论语》、《老子》、《韩非子》都是一些原则性的讲话,它们只提供思路而不提供措施,没有操作性;即使每个人派系都分明,也可能因为理解不同在派系内就产生分歧;而且从《论语》、《老子》、《韩非子》中看,三者之间也不见得泾渭分明,都有相互吸收借鉴的地方,譬如《荀子
-
宥坐》中记载,孔子曰:``聪明圣知,守之以愚;功被天下,守之以让;勇力抚世,守之以怯;富有四海,守之以谦;此所谓挹而损之之道也。''这个论调和《老子》中的``知其雄,守其雌;知其白,守其黑;知其荣,守其辱''即是一个意思,韩非子更是苦心孤诣``喻老''、``解老'',司马迁将老子与韩非同入一传,也不是谬托。儒家的``仁者乐山,知者乐水'',``饮水曲肱、乘桴浮海''与老庄恬淡出世之旨也相契合。儒、法、道三家都非常默契地认同愚民政策,都搞阴谋文化,更不要说儒、道对《易经》的共同推崇。
咫尺玄门:八寸为咫。玄门:奥妙的门径,这里不是指玄学,而仅指高深的境界。桓温是政治家,一听到《礼记》里面有这么强的操作性,又体会到儒家的良苦用心,立刻心领神会,觉得统治起别人来会更加得心应手,于是``喜得他抓耳挠腮,眉花眼笑,忍不住手之舞之,足之蹈之。''
未关至极:没有涉及到最高境界。桓温,你买了本赤脚医生手册,还以为从此以后可以包治百病,你太浅薄了!癌症你听说过吧,心脏搭桥手术会做吗?
金华殿:汉代宫殿,在未央宫内。汉成帝时,郑宽中、张禹曾在金华殿给皇帝讲解《尚书》、《论语》。金华殿语比喻老生常谈或者普及教育。从这句话看,刘惔愿意听《礼记》的,像他这样的名士,如果觉的老师课讲不好,一般就是立刻走人,没有人敢免他的职,如果网络敢曝光,也是自取其辱。不过他认为,即使老师阐述《礼记》深刻圆密,离高深之境不远,但谈的总是制度和具体世界,并不涉及宇宙人生之奥秘,缺乏触及灵魂的震撼。百家论坛终归是百家论坛,不是鹅湖之会,
不是曲女城辩论会。估计后来桓温在刘惔的刺激和帮助下,玄学水平有所长进,以至于有一天他看不起清谈领袖人物殷浩了。

\section{2.65}\label{section-111}

\begin{quote}
羊秉为抚军参军,少亡,有令誉。夏侯孝若为之叙,极相赞悼。羊权为黄门侍郎,侍简文坐,帝问曰:``夏侯湛作《羊秉叙》,绝可,想是卿何物?有后不?''权潸然对曰:``亡伯令问夙彰,而无有继嗣;虽名播天听,然胤绝圣世。''帝嗟慨久之。
\end{quote}

羊秉:是名将羊祜的子侄辈,简文帝司马昱任过抚军将军这个职务,大约看过以前的名册或了解过历史,知道羊秉曾经当过该部门的参军。有些译注以为羊秉当过司马昱的参军,他们弄错了。
少亡:少大约通``早'',羊秉32岁去世。
令誉:与后面的令问是一个意思,美好的名声,见2.11。
夏侯孝若:夏侯湛,字孝若,夏侯渊的后代,西晋著名的文学家,和潘岳一起被称为``连璧'',以现在的眼光看,他的文字多为赋体,我们看起来很累,所以被时光淘汰了不少。他和羊秉有点亲戚关系,是外表兄弟,所以给羊秉写生平传记。夏侯湛有件事历史上很有名,他当时写了本大著作《魏书》,后来看到了陈寿写的《三国志》,自叹不如,就把自己的书烧了。这种勇气现代人是很难做到的,现代人反而会东拼西凑地撰写毫无意义的文章,托人出钱发表在刊物上。能够公允评价
自己和别人,本身就不易,而把自己苦心撰写的著作付之一炬,更是非常严肃认真的人生态度。
叙:叙事,写传记。原文是这样的:``秉字长达,太山平阳人。汉南阳太守续曾孙。大父魏郡府君,即车骑掾元子也。府君夫人郑氏无子,乃养秉。龆龀而佳,小心敬慎。十岁而郑夫人薨,秉思容尽哀,俄而公府掾及夫人并卒,秉群从父率礼相承,人不闲其亲,雍雍如也。仕参抚军将军事,将奋千里之足,挥冲天之翼,惜乎春秋三十有二而卒。昔罕虎死,子产以为无与为善,自夫
子之没,有子产之叹矣!亡后有子男又不育,是何行善而祸繁也?岂非司马生之所惑欤?''说这篇文章好,关键是后面几句立意深,就是作者发出了为什么善人短命无后的疑问,就像明代贾凫写的那样:``忠臣孝子是冤家,杀人放火的享荣华。太仓里的老鼠吃的撑撑饱,老牛耕地使死倒把皮来剥!河里的游鱼犯下什么罪?刮净鲜鳞还嫌刺扎。那老虎前生修下几般福?生嚼人肉不怕塞牙。野鸡兔子不敢惹祸,剁成肉酱还加上葱花。古剑杀人还称至宝,垫脚的草鞋丢在山洼。\ldots{}\ldots{}莫不是玉皇
爷受了张三的哄,黑洞洞的账簿哪里去查!''这就是生命的疑惑。
羊权:据道教的书籍讲,他认识一个仙女叫萼绿华,每月和羊权相会六次,后来度他成仙。他当时是简文帝的侍从,五品官,负责传达诏命。
绝可想:这里的断句有点麻烦,如果不分就是很让人想念的意思,如果分开,绝可大约是很好的意思。如何断句需要证据。
何物:古代何人何物何事不分,统称何物。
夙彰:夙,长期;彰,显著。一向出名,长期突出。
继嗣:嗣就是继,继承人,传宗接代。没有继承人就叫胤绝。
天听:皇帝的耳朵。
古人认为这句话好,大约是羊权哭的好,天听、圣世这些马屁词语用的好,或者是表扬司马昱是文学青年,有人文关怀精神,虽然他应该从《羊秉叙》知道羊秉无后,但还有侥幸心理。

\section{2.66}\label{section-112}

\begin{quote}
王长史与刘真长别后相见,王谓刘曰:``卿更长进。''答曰:``此若天之自高耳。''
\end{quote}

\begin{itemize}
\tightlist
\item
  \emph{王长史}:王濛,见2.54。顾影自怜的美男子,曾经在王导手下做过长史,和刘惔是好朋友,当时``凡称风流者,举濛、惔为宗'',所以在《世说》中,他的故事不下60则,有叫他王、王长史、长史、仲祖、王仲祖、阿奴等。根据史书,王濛、刘惔不积财物,以清贫自守,穷人能够保持风流比财主潇洒更加难得。王濛的孙子就是身无长物的王恭。
\item
  \emph{刘真长}:刘、刘惔、刘尹、刘丹阳,也是老相识了。
\item
  \emph{卿更长进}:这话可能有点开玩笑的轻薄意味,这样的用词一般是长辈、上司的评价,朋友之间这样说不很恰当,所以刘惔立刻反击。
\item
  \emph{天之自高}:《庄子》中说:``老聃曰:`至人之于德也,不修而物不能离焉。若天之至高,地之至厚,日月之自明,夫何修焉?'\,''意思就是圣人生而知之,天本来就很高,地本来就很厚,日月本来就很明亮,谈不上什么长进不长进的。
\end{itemize}

在刘孝标的注中,这个故事还要复杂一些:``仲祖语真长曰:`卿近大进。'刘曰`卿仰看邪?'王问何意?刘曰:`不尔,何由测天之高也。'\,''刘惔说:``你是仰视我吧,不然,你怎么去估测天的高度?''用现代的话说就是:``王濛,你终于知道天高地厚了。''

刘惔能自比为天,这是一种坦然而强烈的自信,我们再回想起他的``丘之祷久矣'',不是一般程度的尊重自我、肯定自我,它超越了``每个人都是自己的国王''的独立意识,确实是令人敬畏的高傲品格。而这种高傲品性,不是个别人的个别现象,而是一个时代的典型特征。我们再回想一下郭沫若的``我是一条天狗呀!我把月来吞了,我把日来吞了,我把一切的星球来吞了,我把全宇宙来吞了。我就是天狗!'',就会觉得郭沫若这首诗很特殊和嘈杂,再结合郭沫若的生平经历,这种独奏、呓语又觉的荒唐可笑;而魏晋人整体的自信和放旷,却让人深思回味。

\section{2.67}\label{section-113}

\begin{quote}
刘尹云:``人想王荆产佳,此想长松下当有清风耳。''
\end{quote}

  刘尹:刘惔。
  王荆产:王徽,字幼仁,小字荆产。他的父亲很有名,叫王澄王平子,见1.23。王澄自幼聪明,生活又非常放诞,``穷欢极娱'',``不以众务在意'',在当时很受推崇,经过哥哥王衍的吹捧,在当时``夙有盛名'',所以刘惔把王澄比喻成``长松''。
  长松下当有清风:这句话的出典大约是宋玉《风赋》:``夫风生于地,\ldots{}\ldots{}缘泰山之阿,舞于松柏之下。''此语换成现代的话,就是``虎父无犬子''的意思,不过刘惔这种比喻用词清远,有出尘之息,在修辞上更富有诗意和联想。松风想必不是庶民的雌风,后面的``赏誉''一节中,就说东汉大名士李膺``谡谡如劲松下风'',说嵇康````肃肃如松下风'',《梁书
- 处士传》中说山中宰相陶弘景``特爱松风,每闻其响,欣然为乐。''
  在《世说》中,我们可以经常看到把人比喻成松树,那么松树有什么特点的,《论语》中说:``岁寒,然后知松柏之后凋。''荀子解释说:``岁不寒,无以知松柏。事不难,无以知君子。''
不过这种坚贞不屈的节操好像和与猪狗抢食、纵酒乱交、傲然自得的王澄没有关系。但是松树还有一个品格,就是特立独行,表现出其与现实世界人群的距离,这点就符合王澄旁若无人的特点,就像嵇康``岩岩若孤松之独立'',陶渊明诗中说``景翳翳其将入,抚孤松而盘桓'',就是突出一个``孤''
字。也许松树还有一个特点,就是它古代往往种在坟地周围,就像现在的烈士陵园。王澄一生醉生梦死,比喻成松也是恰当的。当然,晋代人喜欢用树木、岩石作比,也是当时亲近山水自然、崇尚隐逸的表现。
  刘惔这句话有批评王徽的意思,``人们以为王徽好,这种想法就是以为高大的松树下会有清风罢了。''在《三国志》中,陈寿就说诸葛瞻``蜀人追思亮(诸葛亮),咸爱其才敏。每朝廷有一善政佳事,虽非瞻所建倡,百姓皆传相告曰:`葛侯之所为也。'是以美声溢誉,有过其实。''王徽在史书上没有什么记录,想必这是刘惔公允、优美、委婉的评价。 

\section{2.68}\label{section-114}

\begin{quote}
王仲祖闻蛮语不解,茫然曰:``若使介葛卢来朝,故当不昧此语。''
\end{quote}

  这个故事让人不舒服,王濛侮辱少数民族的语言为禽兽的语言,自大狭隘。其中反映出东晋时人的民族歧视心态,还不是个别人的个别观点,还是一种古代的风气,刘义庆把这个故事放在``言语''里,本身也说明了他的态度。所谓蛮,就是相对于中原地区以南的地方,其范围很大,在晋代可能长江以南都可以叫做``蛮''或``越''的,涉及的民族很多,称蛮不称越,本来就带有歧视色彩,说他们原始落后。不过语言的确和权力、地位、身份等相关,南北朝时,外来民族在北朝占据统治地位,除了统治者学习汉语,汉族人也主动学习胡语,就是所谓的``学得胡儿语,站墙骂汉人''。有时候方言也会占优势,如明清时,绍兴人把持幕府,在官场上绍兴师爷见同乡打官话,对外省人说绍兴话,以示幕业正宗。现代一些学者,学习西方后有的故意玩弄词汇,明明可以用原本汉语就能讲清楚的东西,用短句可以讲清楚的东西,故意把文字、句式搞得艰涩繁复,以示高深和先进。
  有时候朋友会问我:现代的方言就有这么多,那古人交流怎么办?这个题目很大,我们就仅以晋代为例,当时大约是以河南洛阳地区方言为正宗,可能东周以后就是这样。有个成语叫``洛生咏'',东汉初到西晋约三百年,洛阳为中国政治、文化的中心,读书人多操洛阳地区方言,而北人、南人混杂时``洛生咏''就具有炫耀正统地位的资本,被南方人所效仿,这就说明当时的官话就是洛阳话,是``雅言''。但是官话也要``共以帝王都邑,参校方俗,考核古今,为之折衷'',衣冠南渡后,洛阳话大抵还是标准语言,但也结合了吴越方言的一些读法,因为我们现在看到当时的诗歌格律,有些读音是根据吴越方言来的。而且史料中说,王导他们还有意识地学习吴越方言,表示对南方人的尊重和融合。谢安平时根本就是讲吴越方言。在而后隋唐,这种以中原洛阳话为基础的官话可能还是继续延续的,只是由于首都不同才在细部有变化。到了清代和新中国,北京方言才占据主导地位。因为语言障碍,古时候做官要求异地任职,就常常还要聘请专人做方言翻译,很不方便。当然,古时候地广人稀,语言不像现在这么复杂,方言除了消亡,还有产生的过程。
  介葛卢:春秋时有个小国叫介国,国君叫葛卢。据《左传 -
僖公二九年》载:``冬,介葛卢来,以未见公,故复来朝,礼之,加燕好。介葛卢闻牛鸣,曰:`是生三牺,皆用之矣,其音云。'问之而信。''介葛卢懂得牛的语言,听见牛叫声,就说这头牛叫的是说它生了三头小牛,都用作祭祀了。在传说中,古代有些人是听的懂动物说话的,我们比较熟悉的是孔子的学生公冶长的故事,这表明,古人认为动物具有和人一样复杂的语言系统,具有同等的灵性。
  故当:应当或许。 

\section{2.69}\label{section-115}

\begin{quote}
刘真长为丹阳尹,许玄度出都,就刘宿。床帷新丽,饮食丰甘。许曰:``若保全此处,殊胜东山。''刘曰:``卿若知吉凶由人,吾安得不保此!''王逸少在坐,曰:``令巢、许遇稷、契,当无此言。''二人并有愧色。
\end{quote}

  许玄度:许询,字玄度,河北人,自幼迁居绍兴,是当时的清谈名家,屡次拒绝做官,终身不仕,但交游十分广阔,这几点使他受到敬仰,于是给他的文学作品以不相称的高度评价,被誉为东晋中期的``文宗''。
许询和刘惔、王羲之都是好友,他这次来南京,就住在刘惔家里,受到款待。
  出:到。出席,出勤,都是到的意思。
  殊胜东山:东山本来指绍兴上虞谢安隐居时的一个小山,这里应该借指隐居。这句话的意思是长期过这样的生活,比隐居就强多了。从许询的生平经历看,这句话很奇怪,他一心一意当隐士,怎么会向往舒适的生活呢?他当年为了表示决不当官的决心,把家产都扔了,``家财珍异,悉皆是给'',自己跑到四明山区隐居。不过晋代这些名流是把当神仙比当官看的重,就像一个仙人出现在你面前,叫你选择,一是给你荣华富贵,一是让你成仙,一般我们都会选择后者。而且当时人对成仙的信心要深,热情要足,他们看不起当官,不是视富贵如敝履,而是想获得天大的富贵,天大的造化,是觉得当神仙权势更大,生活更好,同样出于享乐主义的目的。这个小故事,就揭露了这些玄学家的真实面目。
  吉凶由人:《左传》中说:``祸福(吉凶)无门,唯人所召。''意思是灾祸和幸福不是注定的,都是人们自己造成的。刘惔反意用之,是说祸福不由人。晋代人看惯了生离死别,看惯了祸福无常,对儒家的一些观点进行了矫正。风流倜傥的刘惔36岁就死了,也是命运无常的一个证明。
  巢、许:巢父、许由,有名的隐士,常被古人挂在嘴上。
  稷、契:后稷是周朝的祖先,就是传说中后土之神,在尧时任农业主管。契是商朝的祖先,在舜时辅助大禹治水,都是著名的贤臣。这里王羲之把许询比作巢、许,把稷、契比作刘惔,是以他们的身份地位比较,指出他们俩平时以为高洁和贤能,但思想境界、言行举止却差得很远。在历史上,王羲之虽然不慕功业,但对一些事物的评论往往一针见血,击中要害,是当时少有的清醒人士。在《世说》的后面一节中,就讲到许询死后,王羲之对他的评价转为很低,评论很刻薄,可能就是越来越清楚地认识到许询的庸俗不堪。
  后世南齐孔稚珪就写过一篇著名的《北山移文》,尖锐地批评那些假隐士:``其始至也,将欲排巢父,拉许由。傲百氏,薎王侯。风情张日,霜气横秋。或叹幽人长往,或怨王孙不游。谈空空于释部,核玄玄于道流。务光何足比,涓子不能俦!\ldots{}\ldots{}今又促装下邑,浪拽上京。虽情投于魏阙,或假步于山扃。岂可使芳杜厚颜,薜荔无耻。碧岭再辱,丹崖重滓。''(当他初来的时候,似乎把巢父、许由都不放在眼下;百家的学说、王侯的尊荣,他都瞧不起。风度之高胜于太阳,志气之凛超过秋霜。一忽儿慨叹当今没有幽居的隐士,一忽儿又怪王孙远游不归。他能谈佛家的``四大皆空'',也能谈道家的``玄之又玄'',自以为上古的务光、涓子之辈,都不如他。\ldots{}\ldots{}听说此人目前正在整理行装离乡,乘着船往京城。虽然他心中想的是朝廷,但或许还会到山里来借住。如果是这样,岂可让我们山里的芳草蒙厚颜之名,薜荔遭受羞耻,碧岭再次受侮辱,丹崖重新蒙污浊。)许询就是这样厚颜无耻的假隐士。

\section{2.70}\label{section-116}

\begin{quote}
王右军与谢太傅共登冶城,谢悠然远想,有高世之志。王谓谢曰:``夏禹勤王,手足胼胝;文王旰食,日不暇给。今四郊多垒,宜人人自效。而虚谈废务,浮文妨要,恐非当今所宜。''谢答曰:``秦任商鞅,二世而亡,岂清言致患邪?''
\end{quote}

  共登冶城:冶城在南京,是个花园。前面说过,王羲之和谢安岁数相差17岁,谢安出山时王羲之已经辞官,没有再去过南京,于是这段对话的时间就有问题。如果谢安还是没有执政的小年轻,王羲之和他讲这番话不着边际;如果谢安当官了,王羲之又没有条件讲这番话。所以这个故事要么出于虚构,要么地点有误。不过无论如何,这个故事反映出东晋士人对清谈的不同态度,令人警醒,具有代表性的意义,所以被后人多次引用。
  高世:超脱世俗。登高望远,汉末陈蕃有澄清天下之志,晋中谢安有超凡脱俗之情,两人都是当时最高级的官僚,而志向截然不同,这是时代的变迁。我们权当这是谢安执政后与王羲之的对话,谢安的志趣也不似作伪,谢安40多岁为维系家族,被迫出山,《晋书》中说:``安虽受朝寄,然东山之志始末不渝,每形于颜色。及镇新城,尽室而行,造海之装,欲须经略粗定,自江道还东。''
  夏禹勤王:勤王是指为王事辛勤。传说中尧任命大禹治水,禹在外九年,由于身先士卒,长期泡在水里,体毛褪尽,手脚都起了厚茧。《史记
-
李斯列传》:``禹凿龙门,通大夏,疏九河,曲九防,决渟水致之海,而股无胈,胫无毛,手足胼胝,面目黎黑。''
  文王旰食:旰是晚上的意思,旰食指天黑了才吃饭。《汉书 -
董仲舒传》``周文王至于日昃不暇食'',传说中周文王一心投入造反大业,常常处理事物忙碌得从早到晚,也没有功夫吃饭。他儿子周公执政后也是日理万机,所谓一餐三哺,一餐饭要分三次吃,老是要应付别人来请示汇报。丘吉尔先生说过:``权力是最好的春药'',信哉此言。
  四郊多垒:首都周围都是军营堡垒,指敌人逼近,国家不安全,也可指不安定因素很多。《礼记》中说:``四郊多垒,卿大夫之辱也。地广大,荒而不治,此亦士之辱也。''就是说高级官僚有保卫国家的责任,这比``天下兴亡,匹夫有责''要公允。而以现在的眼光看,古人把都城建在边境线附近或者地势险要之处,都不明智,像南京实在是不适合在古代建都。
  自效:自当效力,自觉地为国效命。
  当今所宜:王羲之本来也喜欢清谈,而且还是此道高手,这里他不说清谈不好,超凡脱俗不好,而是委婉地说现在不合适主次不分,沉迷于虚谈浮文。清谈误国的观点来源已久,当年王衍兵败被杀,临终忏悔说:``吾曹虽不如古人,向若不祖尚浮虚,戮力以匡天下,犹可不至今日。''东晋初庾翼也用王衍的例子劝告名士殷浩:``(王衍)高谈庄老,说空终日,虽云谈道,实长华竞。''庾翼这话说到点子上了,清谈本身未必误国,聊天、思辨关国家命运何事,关键是高官以仰慕老庄为借口,无心事功,提倡享乐人生,生活极尽奢华,权力斗争十分血腥,阶级剥削十分残酷。这种``道''不是老庄的``道'',而是东汉末年以后士人逃避现实、穷奢极欲的``道''。
南朝梁时有个叫鱼弘的,
历任南樵、盱眙、竞陵、永宁、新兴诸郡太守,曾恬不知耻地对人说:``我做郡守有一特色,即皆为四尽:水中鱼鳖尽,山中麋鹿尽,田中米谷尽,村里百姓尽。''两晋南北朝的门阀,大都就是这样敛财,我们在后面的《世说》故事里,能接触到一些这样的事例。而那时民族矛盾、阶级矛盾、门阀矛盾都很尖锐,如果所有高级官员都很超脱,都以个人、家族利益至上,难怪国阼不久。
  秦任商鞅:谢安的回答是偷换概念。王羲之并没有否定清谈的意义,即使王羲之说清谈会导致亡国,但不等于只有清谈会亡国,亡国的原因多种多样。我们说,君王耽迷酒色会亡国,任用小人会亡国,武备不修会亡国,政策不妥会亡国等,这些都只是亡国的一个因素,甚至很难说哪个是决定性因素。而谢安用个例去反驳,是没有说服力的。
  当年殷王请来傅悦做宰相,对他寄予厚望:``若金,用汝作砺;若济巨川,用汝作舟楫;若岁大旱,用汝作霖雨。(比如铁器,希望你来作磨刀石;比如渡大河,希望你来作船和桨;比如年岁大旱,希望你来作甘霖。)''``安石不出,如苍生何!''谢安也同样寄托着天下人的厚望,王羲之这番劝导,虽然不是什么惊人之语,但情真意切,可谢安没有正面回应,采用侧面反驳的诡辩之辞,令人失望。淝水之战,公一生功业多来于此,虽名满天下,誉贯古今,但实出侥幸,谢家之衰,由此可知。

\section{2.71}\label{section-117}

\begin{quote}
谢太傅寒雪日内集,与儿女讲论文义。俄而雪骤,公欣然曰:``白雪纷纷何所似?''兄子胡儿曰:``撤盐空中差可拟。''兄女曰:``未若柳絮因风起。''公大笑乐。即公大兄无奕女,左将军王凝之妻也。
\end{quote}

  这个故事应当发生在谢安尚未出山之时,他的哥哥谢奕在外边做官,谢据早卒,子侄岁数还小,谢安分担了教育任务,《世说》中就多次记载谢安出题询问子侄的故事,其侄女谢道韫表现了文学才能和委婉体贴的细腻情感。在我小时候,父母亲检查功课倒是有的,但似乎从来没有和我谈过文学、哲学之类的话题,尽管父亲是大学中文教师。只是高一的有一天,父亲突然叫我骑上自行车,两人一起逛了一次花港观鱼,好像我还卖弄书本,向父亲介绍了园林知识。二十年过去了,我和父亲聊天大约从不超过5分钟,只是有时候陪父亲上网打牌,对他的牌技指指点点过一番,常常恨其不长进。家里的叔伯似乎也喜好书画文艺的,连《四库全书》这样大而无当的书也买来堆着,但我们在一起也无话可说。
  内集:家庭聚会。
  雪骤:后人之所以认为谢道韫比谢朗说的好,主要是雪骤,用盐比喻雪,应该是雪子、小雪。
  谢安、谢朗、谢道韫所用的都是七言,都押韵,这是模仿汉武帝时的柏梁体歌句。当时柏梁台(见2.53)落成以后,汉武帝与大臣们宴会,联句赋诗,句句用``止''韵,这里也是一样。不过这些也不一定能算诗,虚词多,缺乏诗意。魏晋诗歌以四言、五言为正统,七言往往被认为是游戏之作,比较出色的有曹丕的《燕歌行》。
  胡儿:谢朗,谢据的儿子,字长度,小名胡儿,据说天资很好,从小善谈玄理,谢安常常带着他和别人谈玄。谢朗和他父亲一样,早卒。
  兄女:漏道韫二字。谢道韫是那个灌老翁酒的谢奕的女儿,她有个外号叫``咏絮才'',来源就是这里。说雪如柳絮,也许不仅是形象,而且也许包含着一定的情感和风姿。
  公大笑乐:子侄愿上进,是人生一乐;子侄有杰出的才能,是人生大乐。
  这节的写法有点意味深长的味道,作者没有介绍谢朗的来历,而特别点了谢道韫的父亲和丈夫,这并不是谢奕、王凝之官位高,而是作者也认为谢道韫回答好,要特别注明一下。王凝之是王羲之的小儿子,谢安和王羲之结亲家,也说明这两个人关系好。我曾经怀疑王羲之和谢安到底是不是差17岁,因为从《世说》记载来看,两人岁数不至于差的这么悬殊。谢道韫嫁给王凝之是桩不满意的婚姻,``甚不乐''。后来因为王凝之当了会稽内史,孙恩造反,王凝之居然不休武备,对大家说:``吾已请大道,许遣鬼兵相助,贼自破矣。''于是城破,王凝之和两个儿子都被杀了,而谢道韫``闻寇至,举措自若,命婢肩舆,抽刀出门,手杀数人,乃被执''。孙恩佩服谢道韫,没有杀老太太。不过只是传说而已。估计谢道韫平时思想工作也不会少做,但摊上这么一个神神叨叨的丈夫,有什么办法呢?从《世说》后面的故事看,谢道韫正如她《登山》诗写的那样``峨峨东岳高,秀极冲青天'',所以她是《晋书
- 列女传》中的一员。

\section{2.72}\label{section-118}

\begin{quote}
王中郎令伏玄度、习凿齿论青、楚人物。临成,以示韩康伯,康伯都无言。王曰:``何故不言?''韩曰:``无可无不可。''
\end{quote}

  王中郎:王坦之,字文度,太原王氏,前面个别地方提及过他和他的家人。太原王氏虽然自西晋以来就是大家族,但一直居于琅琊王氏、谢氏、郗氏、桓氏之后,而只有到了王坦之这一辈,终于进入朝廷中枢,位居三公,可以与其他世家平起平坐。王坦之人称``江东独步'',所谓独步,大概就是像博尔特在百米赛跑中那样在出尽风头。他一生最突出的成绩就是在当司马昱秘书长(侍中)的时候对司马昱说,``天下,宣元之天下,陛下何得专之'',打消了司马昱的怕心、抵制了桓温的野心、对司马曜有拥立之功,于是任中书令,领丹杨尹,兼任北中郎将,都督徐兖青三州诸军事、徐兖二州刺史。
  太原王氏的辉煌比其他世家要晚、要长久,而且并不在江南,而是南北朝和隋唐的北方,原因之一是太原王家以坚守儒学著称,一家万口群居,``一宗近万室,烟火连接,比屋而居'',树立了一个标榜忠孝礼义的门户风范,另一个原因是王坦之有个子孙叫王慧龙的入仕北朝,就像鲁迅的孙子去台湾了(程度当然还要深得多,起码比周作人仕日影响要大),成为北魏的一个可以炫耀的资本。之后王慧龙与北朝大家族进行了广泛的联姻,在北朝和隋唐时期与清河崔氏、范阳卢氏、陇西李氏并称``四姓'',``后魏定氏族,佥以太原王为天下首姓,故古今时谚有`鼎盖'之名,
盖谓盖海内甲族著姓也。''太原王氏地位之高,可以从唐初宰相薛元超的话中可以窥豹:``我富贵过人,但人生还有三恨,一是没有中过进士,二是没有娶五姓女(王李崔郑卢),三是没参与编撰国史。''宋人彭渊材说:``人生的恨事是鲥鱼多刺,海棠无香,曾固不能诗'',就是套用这个典故,查良镛先生在《鹿鼎记》中所谓``平生不识陈近南,自称英雄也枉然'',也是模仿``娶五姓女''的。
  伏玄度:伏滔,字玄度,山东平昌人,曾任大司马桓温参军,桓温``每宴集,必命滔从''。伏滔后来任著作郎,掌国史。
  习凿齿:字彦威,湖北襄阳人,他是桓温为荆州刺史时的别驾,史学家,曾著《汉晋春秋》。前秦苻坚攻陷襄阳,将习凿齿和道安法师二人接往长安,说:``朕以十万师取襄阳,所得唯一人半,安公一人,习凿齿半人。''王坦之曾任桓温的长史,和伏滔、习凿齿有交往,官职比他们要高一点。
  青、楚:青州和荆州。古中国分为九州,青州大约是山东齐鲁一代。《尚书 -
禹贡》中称``海岱惟青州'',海即渤海,岱即泰山,《周礼》:``正东曰青州\ldots{}\ldots{}盖以土居少阳,其色为青,故曰青州。''
楚国旧号荆,所以荆州也称楚,本指湖北一带,但因为楚国很大,也可包括两湖甚至整个南方,习凿齿笔下的``楚''超越了荆州。
  人物:偏义复词,当然是光讲各自家乡的名人。两人的文章现在还保留大意,伏滔列举了自春秋以来山东的一些名人,譬如管仲、孟子、荀子、东方朔等(居然没有孔子),习凿齿的楚地从神农打头,列举了伏羲、少昊这样神话中的人物,并把南方的名人与齐鲁名人一一比较,指出南方可没有赤眉、黄巾军这样的造反派。这种地域帖现在看来没多大意思,但物以类聚,人以群分,这个``群''在高人那里以学识志趣为标准,但在常人无非就是亲缘、学缘、乡缘之类。论青楚人物,是当时关注地域文化特点和结党之风的表现,也就是现在的乡土教育,增加地方自豪感和认同感,把这种情感扩大,就是爱国主义。
  临成:临是到的意思,临成大约是说草拟稿出来了。
  韩康伯:韩伯,字康伯,见1.38。   都:表程度深。
  无可无不可:无所谓一定对,无所谓一定不对。这句话出自《论语》,子曰:``君子之于天下也,无适也,无莫也,义之与比。''
就是说,君子对于世界的看法,没有什么非这样不可,也没有什么非不这样不可,只要符合``义''的原则。在《论语》中,孔子也对人物进行品评,他说:``逸民:伯夷、叔齐、虞仲、夷逸、朱张、柳下惠、少连。不降其志,不辱其身,伯夷、叔齐与!柳下惠、少连,降志辱身矣,言中论,行中虑,其斯而已矣。虞仲、夷逸,隐居放言,身中清,废中权。我则异于是,无可无不可。''孔子认为,做人要``毋必毋固'',不要拘泥,在坚持``义''的原则基础上灵活圆融。这种圆融特点在孟子那里得到进一步阐释:《孟子
-
离娄下》``大人者,言不必信,行不必果,惟义所在。''我们在阅读书籍时,常常会发现作者言语、思维先后有矛盾,我们在观察人物时,常常会发现他们说一套做一套,顶真的人以为抓住了他们的把柄。但事实上世界本身就是多声部的,思想混乱、生活混乱是生活的常态,韩康伯貌似平淡的回答,其实灵活使用儒家经典表明自己的态度:这种地域帖有意思,也没意思。我们不能以为所谓清谈一定是长篇大论,魏晋风度尚简,韩康伯回答做到简洁而又一语中的,所以本节列入``言语''一章。

\section{2.73}\label{section-119}

\begin{quote}
刘尹云:``清风朗月,辄思玄度。''
\end{quote}

  刘尹:刘惔。
  玄度:就是2.69的许询,不是指伏滔。魏晋人用玄作为名字,大约就像新中国建立后用卫东、建国、建新之类的,表达了一种社会风尚。
  许询极善清谈和玄言诗歌,在当时有很高的声誉。从现在文学史评价来看,玄言诗的地位不高,诗歌以形象语言来抒情和叙事的文学体裁,如果用来阐述哲理就不大让现代人有阅读的兴趣,不过玄言诗也有写的极好的,如阮籍、陶渊明的不少诗歌可以归入玄言诗,但他们的诗风是抒情叙事与哲理相交融,而其他大部分现存的玄言诗则不具备这种特点,一味用艰涩的词语说理,``理过其辞,淡乎寡味''。就像小说是讲故事的体裁,作家当然可以尝试意识流、象征主义、神秘主义等,但终究不是主流,很难获得读者的喜欢。
  2.69的故事讲过,许询到南京,就是刘惔来接待的。许询口才极好,可以达到覆雨翻云、信口雌黄的境界,极合刘惔心意。故事还有另一个版本,是说许询开始没有住在刘惔家里,而是``许询尝从会稽出都,船泊淮渚,刘真长为丹阳尹,数往船造之,人问刘尹:`玄度定称所闻否?'刘曰:`才情过于所闻。清风朗月,辄思玄度。'\,''大概刘惔喜爱许询,所以接着请他住自己家(当然也许许询不止一次到南京)。许询不当官,但朝中交友很广,他隐居萧山的时候每每接受高官的馈赠,于是有人讥讽他:``上古隐居箕山的许由不是你这个样子的。''许询回答道:``可他们送我这些东西也比不上尧送天子宝座啊!''由此可见他的机锋便利。
  美好的景色之下想起美好的人,刘惔的言下之意是说,听许询谈玄,给人的感觉就像清风拂面,明月当空,想来具有自然、温暖、明亮、灵动等特点。而刘惔不直接说许询谈玄像清风朗月,从修辞角度讲,就有若即若离的距离之美,给人想象的余地。当然,玄度这个词,在古文当中有月亮的意思,刘向《列仙传
-
关令尹赞》:``尹喜抱关,含德为务,挹漱日华,仰玩玄度。''这也是双关的修辞手段,也给人巧妙的联想之美。譬如很多人喜欢戴安娜,而戴安娜在古罗马神话中,就是月亮女神的名字,``原来月亮女神就是这个样子的啊!''    

\section{2.74}\label{section-120}

\begin{quote}
荀中郎在京口,登北固望海云:``虽未睹三山,便自使人有凌云意。若秦、汉之君,必当褰裳儒足。''
\end{quote}

  荀中郎:荀羡,字令则,荀彧的子孙。荀羡是驸马,受到殷浩的赏识,提拔为北中郎将,28岁出任徐、兗二州刺史,牵制桓温,38岁病死。他小时候体现出非常的胆气,当年苏峻叛乱,他全家被软禁,苏峻很喜欢荀羡,常抱着7岁的他在膝盖上玩耍。荀羡暗中告诉母亲:``给我一把匕首,我就能杀死苏峻''。我们小时候也可能看过一本连环画:他有个妹妹叫荀灌娘,是位小英雄,13岁时候有一次打仗,她杀出重围讨来救兵,帮助父亲荀崧击败叛军。
  北固:就是那座有名的小山坡------江苏镇江北固山,梁武帝曾经手写``天下第一江山''。我有一次慕名前往,该山大约5、60米,10分钟可以登上,这``天下第一''不知从何而来。不过沧海桑田,当年可能山下江海一体,视野非常开阔。唐代王湾有首著名的诗歌``潮平两岸阔,风正一帆悬。海日生残夜,江春入旧年'',描写的就是他新年登北固山所产生的浩渺胸怀。
  三山:指神话中的蓬莱、方丈、瀛洲三山。《史记》中说:``蓬莱、方丈、瀛洲,此三山世传在海中,去人不远。尝有至者,言诸仙人不死药在焉。黄金白银为宫阙,草物禽兽尽白,望之如云。''
  凌云意:超脱尘俗的愿想,出典应该是汉武帝读了司马相如的《大人赋》,``飘飘有凌云之气,似游天地之间意''。前面说过,有些人登山产生壮志,有些人登山有所思念,有些人登山悲从中来。这里荀羡登上小山坡,遥看大海,想起成仙来,这大概是当时道教氛围浓厚,不以世事为怀。
  秦、汉之君:秦始皇曾遣多人人海寻仙求药,东巡时又曾从坐船沿海北上,希望遇见神山;传说中他还``于海中作石桥,海神为之竖柱''。汉武帝在泰山祭天以后也曾到东海,希望能遇见蓬莱山,神仙没见着,于是将女儿许配给方士,想通过代言人和神仙拉上点关系。在旧社会,没有像现在这样的无神论,我们可以理解成古人生活在两层世界中,生活在永恒之中。
  褰裳濡足:《诗 -
褰裳》``子惠思我,褰裳涉溱;子不思我,岂无他人。''(你要是喜欢我,就赶快撩起衣裳渡河来见我;你要是不喜欢我,你以为就没有别人喜欢我?)《楚辞
-
思美人》``因芙蓉而为媒兮,惮褰裳而濡足。''(我想采来荷花作为我的聘礼,可是顾虑淌水撩起衣裳和污了脚。)
  荀羡是说:登上北固山,给人超尘出世的感觉,想来那成仙欲望更加强烈的秦皇汉武,更会不顾一切去寻找仙境。这是衬托悠远北固山、神秘东海的一种修辞手法。

\section{2.75}\label{section-121}

\begin{quote}
谢公云:``贤圣去人,其间亦迩。''子侄未之许。公叹曰:``若郗超闻此语,必不至河汉。''
\end{quote}

  去人:与普通人的距离。谢安用贤圣而不用圣贤,也许是强调贤。
  迩:近。《论语》:``迩之事父,远之事君。''谢安的意思大概是人人可以为尧舜的意思。孔子并不自认为圣人,也不承认他的时代有圣人,他说:``圣人,吾不得而见之矣;得见君子者,斯可矣。''不过后人认为孔子就是历史上最大的圣贤。而孟子是性善说的坚持者,就提出``人皆可以为尧舜''的说法,认为圣人与常人的区别也就是知道道理而去不去做的问题。
  未之许:谢安的子侄不认同谢安的说法,也许他们是体会到自己个人品质与圣人的差距。有个故事说,苏轼和佛印打禅机,佛印认为苏轼是佛,苏轼认为佛印是屎。这次辩论苏轼输了,因为对外在世界的看法其实就是自己内心的投影。
  郗超:见2.59。当时人认为,郗超精通玄理,见识高明。
  必不至河汉:《庄子》:``肩吾问于连叔曰:`吾闻言于接舆,大而无当,往而不反。怪怖其言,犹河汉而无极也。'\,''这里是说,谢安的子侄认为谢安的说法像银河一样没有边际,不实在。
  谢安也许是期望子侄们能向圣贤看齐,树立远大的理想,追慕圣贤的品德,这次教育好像失败了。

\section{2.76}\label{section-122}

\begin{quote}
支公好鹤,住剡东岇山。有人遗其双鹤。少时翅长欲飞,支意惜之,乃铩其翮。鹤轩翥不复能飞,乃反顾翅,垂头,视之如有懊丧意。林曰:``既有凌霄之姿,何肯为人作耳目近玩!''养令翮成,置使飞去。
\end{quote}

  支公:支遁支道林。《世说》中的主要人物之一,因为在当时他是``览通群妙,凝神玄冥,灵虚响应,感通无方''。
  剡:剡县,现绍兴嵊州、新昌一带。
  岇山:支道林要在剡县隐居,就想把这座山买下来,结果被人嘲笑了一番,此事可见排调章(25
-
28)。不过后来这山大约给及时雨、赛孟尝郗超买下来送给了支道林。前几年我去过岇山,东西岇山现在修了水库,景色佳美,山腰还有座小庙,不知道是不是与支道林有一点关系。
  遗:音wèi,赠送。《古诗十九首》``客从远方来,遗我双鲤鱼'',杜甫``客从西北来,遗我翠织成''等。
  少:不久。   铩其翮:剪去翅膀上的硬羽毛。
  轩翥:轩有高的意思,轩然大波。也有说轩有飞的意思,轩翥就是同义反复,指高飞貌。这句话似乎表述有点疑问,轩翥或飞可删。唐宋之问有首名诗《度大庾岭》:``魂随南翥鸟,泪尽北枝花。''
  反顾翅,垂头,视之如有懊丧意:细节描写,显得生动些。
  置:似赘语,这里大约是放到野外的意思。
  从《世说》中支道林生平事迹看,他表现出并不像一位僧人,而更接近与名士和玄学家。他这种推己及物的想法,是庄子在濠上的移情,并不是佛家的慈悲心和放生的意思。支公好鹤,本来也和养马是一个道理,好其神骏出尘,不过看到鹤因不能飞而神气懊丧,不由感同身受,不愿意做别人的玩物。
  周作人曾经三番四次提及笼养动物的话题:``如要赏鉴,在他自由飞鸣的时候,可以尽量的看或听:何必关在笼里,擎着走呢?我以为这同喜欢缠足一样的是痛苦的赏玩,是一种变态的残忍的心理。贤首于《梵网戒疏》盗戒下注云,`善见云,盗空中鸟,左翅至右翅,尾至头,上下亦尔,俱得重罪。准此戒,纵无主,鸟身自为主,盗皆重也。心鸟身自为主',------这句话的精神何等博大深厚,然而又岂是那些提鸟笼的朋友所能了解的呢?'',``我对于植物比动物还要喜欢,原因是因为我懒,不高兴为了区区视听之娱一日三餐地去饲养照顾,而且我也有点相信`鸟身自为主'的迂论,觉得把它们活物拿来做囚徒当奚奴,不是什么愉快的事,若是草木便没有这些麻烦,让它们直站在那里便好,不但并不感到不自由,并且还真是生了根地不肯再动一动哩。但是要看树木花草也不必一定种在自己的家里,关起门来独赏,让它们在野外路旁,或是在人家粉墙之内也并不妨,只要我偶然经过时能够看见两三眼,也就觉得欣然,很是满足的了。''他也引用清秦书田的《曝背余谈》:``盆花池鱼笼鸟,君子观之不乐,以囚锁之象寓目也。然三者不可概论。鸟之性情唯在林木,樊笼之与林木有天渊之隔,其为汗陛固无疑矣,至花之生也以土,鱼之养也以水,江湖之水水也,池中之水亦水也,园圃之上土也,盆中之上亦土也,不过如人生同此居第少有广狭之殊耳,似不为大拂其性。去笼鸟而存池鱼盆花,愿与体物之君子细商之。''

\section{2.77}\label{section-123}

\begin{quote}
谢中郎经曲阿后湖,问左右:``此是何水?''答曰:``曲阿湖。''谢曰:``故当渊注渟著,纳而不流。''
\end{quote}

  谢中郎:谢万,字万石,谢安的四弟,他出山比老三谢安早,曾任西中郎将,也正是因为他在任西中郎将、豫州刺史、淮南太守的时候奉命北伐,其个性放浪,不以世事经怀,出言无忌,侮辱下属,部下离心,未与敌军接仗而下令撤退,结果一发不可收拾,全军溃散,谢万被废为平民。于是谢安由远志变为小草,出山去维护谢家。
  曲阿:南京附近的丹阳湖,据说秦始皇认为金陵有王气,就把丹阳湖改道变为弯弯曲曲的河道,以破坏龙脉,故称曲阿湖。
  故当:故当在《世说新语》中比较常见,但好像不能想当然地认为是``所以应当''的意思,而是``本应该''的意思。
  渊注渟著,纳而不流:渊注渟著同义反复,就是兼容并蓄的意思。纳而不流是说曲阿湖只有注入没有流出。这句话从字面上不难读通,但意思却很费解,为什么取了曲阿的名字,就应该``纳而不流'',难道``阿''有``大''的意思?要真真弄懂这句话,大约要亲眼看到当时的曲阿湖,并有突然地感触。我们暂且不要管它,这句话有言外之意,大概是说名士应该像曲阿湖(也许恰恰相反,是不应该像曲阿湖),虽然经历可能坎坷,但最终不能放任,随波逐流,而是要用这种坎坷的经历来养成宏大的气度。谢万虽然当官比较失败,但从小以文采、言论出名,这话符合名士的身份和要求。  

\section{2.78}\label{section-124}

\begin{quote}
晋武帝每饷山涛,恒少。谢太傅以问子弟,车骑答曰:``当由欲者不多,而使与者忘少。''
\end{quote}

  车骑:车骑将军谢玄,谢安的侄儿,淝水之战的主帅,在世说中,他的回答比较机警。
  司马炎给山涛的赏赐少,似乎不符合史实,因为在晋书中说:``帝以涛清俭无以供养,
  特给日契,加赐床帐茵褥,礼秩崇重,时莫为比。''不是每月发工资,而是每每都有额外的表示,细水长流,应该是说不少的。不过我们在处理史料中,也常常会遇到``所见异辞,所闻异辞,所传闻异辞''的问题,如何选择判断就要依靠史识了。晋武帝也算是英雄人物,自然是杀人如麻、挥金如土,不太可能``恒少''的。也许我们可以这样理解,司马炎经常赏赐山涛,但每次数量不多,其中应该包含有表示特别恩宠的意思。当然,我老家有句谚语,大抵是见到别人贫穷,送一斗米是结亲家,送上一石米就可能结仇家。古代的统治者也表达过类似的意思,对人赏赐,不可以太高,不然以后就功高难赏了。
  谢玄的回答避开了统治术的问题,重点在于刻画、突出山涛的个人形象。这也是事实,传说中山涛``贞慎俭约,虽爵同千乘,而无嫔媵。禄赐俸秩,散之亲故'',这种形象,在现代的官吏中可谓几希了,这是为什么呢?大概是老庄学说没有多少官员看得懂了,在古代的个别官僚当中,清廉不是一种外在的道德要求和制度设计,而是一种快乐生活的原则,一种内心的生活方式。
  虽然与本节的故事主旨无关,但有句古诗是这样说的:笼鸡有食刀枪近,野鹤无粮天地宽。

\section{2.79}\label{section-125}

\begin{quote}
谢胡儿语庾道季:``诸人莫当就卿谈,可坚城垒。''庾曰:``若文度来,我以偏师待之;康伯来,济河焚舟。''
\end{quote}

  谢胡儿:谢朗,字长度,小名胡儿,见2.71,谢安的侄子,从小就以善于玄言而闻名。
  庾道季:庾龢(和),太尉庾亮的儿子,和谢朗一样,``风情率悟,以文谈致称于时''。
  莫:暮,黄昏。玄的本意是黑。当时清谈时间就一般放在黄昏,大家黑灯瞎火,鬼影重重,通宵达旦,以此至``玄''。记得有本书上说,后来闻一多讲楚辞,也要求学校把课安排在黄昏,自己先抽一根烟,然后唱一句``痛饮酒,熟读《离骚》,便可称名士'',接着开始上课。闻一多这样安排,就是模仿魏晋先贤的风范。说句题外话,闻一多可能是民国时期最具有名士风范的学者。
  坚城垒:加固城墙堡垒。谢朗认为庾龢要以一对多,首先应该以防守的姿态。这句话可能反映出当时清谈的特点,别人上门聊天,其实是找主人来挑衅的,辩论估计和战争差不多,攻难守易。
  文度、康伯:王坦之,字文度;韩伯,字康伯,我们见过他们多次了,玄学家兼名臣。
  偏师:一般认为是指侧翼部队,并非主力,但偏师还有个意思,指精锐之师,突袭之师。这里也许应该认为是精锐的意思,因为庾龢在《世说》中还有一句话:``要论思路条理清楚,我自愧不如康伯;要论志气坚强不屈,我自愧不如文度。除此以外,我都超过他们百倍。''
  济河焚舟:破釜沉舟,决一死战的意思,语出《左传》:``秦伯伐晋,济河焚舟。''过了黄河就烧掉渡船。这句话大概说明王坦之和庾龢玄学口才差不多,韩伯水平要高一点。
  玄谈讲究的是得兔忘蹄,得鱼忘荃,所以这些谈话记录我们看不到,只能通过当事人的著作来遥想一下。记得欧玛尔
-
海亚姆有首诗:``昨夜,我走近一家陶罐作坊,见两千陶罐有的沉默有的吵嚷,猛听一只陶罐大喝一声:烧罐的买罐的卖罐的现在何方?''

\section{2.80}\label{section-126}

\begin{quote}
李弘度常叹不被遇。殷扬州知其家贫,问:``君能屈志百里不?''李答曰:``《北门》之叹,久已上闻;穷猿奔林,岂暇择木!''遂授剡县。
\end{quote}

  李弘度:李充,字弘度,东晋文学家。小时候他父亲坟地的柏树被人砍伐,他就把砍树的人杀了。因为晋代以礼治天下,按《礼记》的要求,这种用私刑大约是不用被处罚的,所以李充反而以孝知名。关于中国私刑的合法性特点,是很好的研究题目,我们至今可以发现它对现代意识的影响。在《晋书》中,记录了李充研究老庄、儒学后,对道德、法律和社会发展的思考,他认为``圣教救其末,老庄明其本,本末之涂殊而为教一也'',也就是说,当时很多人认为,道、儒本是一家,道是根本,儒是表征。李充开始是王导的记室参军,后来是褚裒的参军,应该说都是要害职务,但工资收入不高,就像我听到省里的厅局长发牢骚说自己级别挺高,但收入比不上市里、县里的小官,李充大概也是这样向殷浩发牢骚的。
  遇:机会遇合,当官和个人的本领能力没有必然的关系,主要是得到机会。最近我复习各地的民间故事,很多故事认为,考中进士或者能当官,主要是阴德而无关公平。
  殷扬州:殷浩,扬州刺史。
  百里:指当县令。百里在秦以前是一个诸侯国的大小,《孟子 -
万章下》:``天子之制,地方千里,公侯皆方百里。''所以《论语》中说:``可以托六尺之孤,可以寄百里之命,临大节而不可夺也。''郡县制度以后,《汉书》中说:``凡县大率方百里,民稠则减,稀则旷。''《三国志》中诸葛亮评价蒋琬``社稷之器,非百里之才'',所以这里用``屈志''的说法。
  《北门》:《诗经》``出自北门,忧心殷殷。终窭且贫,莫知我艰。已焉哉!天实为之,谓之何哉!王事适我,政事一埤益我。我入自外,室人交徧谪我。已焉哉!天实为之,谓之何哉!\ldots{}\ldots{}''诗歌描写一个小官僚公事繁重,虽辛勤应付,但生活依然清贫。李充说自己发牢骚就是北门之叹。
  穷猿奔林:被猎人追赶的猿猴一心逃命,那在乎计较树林的好坏。这是《左传》``鸟则择木,木岂能择鸟''典故的反用。

\section{2.81}\label{section-127}

\begin{quote}
王司州至吴兴印渚中看。叹曰:``非唯使人情开涤,亦觉日月清朗。''
\end{quote}

王司州:琅琊王胡之,字修龄,和王羲之同辈,谢安的好友,曾任吴兴太守,后为司州刺史,未到任病逝世。王胡之我们可能不记得,但有个囊萤夜读的车胤,就是王胡之赏识提拔的。
  吴兴印渚:临安於潜镇的一个湖泊,要去一趟才能向大家汇报该地景色。
  使人情开涤:似乎缺``心''字,心情开涤,就是胸怀就像被洗过一样,清澈开朗。
  现代人认为晋人有风采,大抵就是很多名士性格清澈,没有尘埃。我十几岁的时候去过一趟温州泰顺,一路崇山峻岭,道路如带,下临不测之渊,在而后的生活中,自己一遇到不开心的时候,就回忆泰顺沿路风光,顿时心生云气,人生琐事大可排解。

\section{2.82}\label{section-128}

\begin{quote}
谢万作豫州都督,新拜,当西之,都邑相送累日,谢疲顿。于是高侍中往,径就谢坐,因问:``卿今仗节方州,当疆理西蕃,何以为政?''
谢粗道其意。高便为谢道形势,作数百语。谢遂起坐。高去后,谢追曰:``阿酃故粗有才具。''谢因此得终坐。
\end{quote}

  豫州都督:355年,谢尚出任豫州都督,357年病故;弟谢奕出任豫州都督,358年病故;弟谢万继任,359年,谢万未战而全军溃散。短短四年间,谢家在豫州接连遭遇不幸,也许算是死地。
  都邑:大概是借代,指京城的亲友大臣。谢万到前线重地出任一方诸侯,了解谢万名士做派的亲友十分担心,谢安、王羲之纷纷告诫要收敛性子,但没有什么效果。这里高崧也是这样提醒他,为他出谋划策,但谢万依旧不太上心。
  于是:或当作``于时'',这个时候。
  高侍中:高崧,字茂琰,小名阿酃,曾任抚军司马、吏部郎、侍中等,简文帝的亲信。
  仗节方州:拿着符节到州郡。   当:面对。
  西藩:豫州在今河南一带,长江以西,东晋的边疆,所以叫西蕃。
  谢遂起坐:谢万始终挺直坐着,大约是表示认真恭敬的意思。
  追:接着。或做``迺'',于是。
  因此得终坐:因为高崧讲得好,所以尽管谢万非常疲劳,但始终坚持严肃的坐姿。
  这个故事很平常,谢万的评价比较轻薄,似乎也没有什么典故,放在言语里有点费解。不过荀子说过,君子赠人以言,庶人赠人以财,大约就是这个意思,

\section{2.83}\label{section-129}

\begin{quote}
袁彦伯为谢安南司马,都下诸人送至濑乡。将别,既自凄惘,叹曰:``江山辽落,居然有万里之势!''
\end{quote}

  袁彦伯:袁宏,小名``虎'',文学家。《世说》后面有个故事叫``倚马可待'',大意与``七步成诗''相仿,主人公就是袁宏。他任桓温的掌书记(文字秘书)时,起草布告文书,靠在出发的战马前,片刻就写好了,不像我们,起草一个领导讲话,往往要花费一个星期的时间,而且千人一面,面面俱到,都是俗话、空话,自己还常常感到很羞愧。
在《世说》中,有不少袁宏的写作故事,的确才思敏捷,文字流利潇洒。
  谢安南:会稽谢家的谢奉,字弘道,历任安南将军、广州刺史、吏部尚书等。谢安一族叫阳夏谢氏,谢奉则是会稽一族,并不是一回事。
  司马:在先秦,司马与司徒、司空、司士、司寇并称五官,掌一国的军政和军赋。汉代大将军、三公可以开府(自选幕僚,建立自己的小政府),但到了魏晋,刺史往往兼管军事,带将军衔者,即可开府。司马成了将军府的属官,相当于将军的参谋长。谢奉当时是广州刺史、安南将军,
  都下:``下''应该是衬词,没有意义,譬如齐国有个``稷下学宫'',就是设在齐国稷门的学堂,唐代诗歌有首叫``都下寒食夜作'',其实描写的就是京都寒食节的情景。
  濑乡:江苏溧阳境内,离南京有上百里,古人送亲戚朋友的阵仗实在太大了,也许是因为去广东的缘故。
  既自:大约是既而自生的意思
  江山辽落,居然有万里之势:这句话不太好懂,主要是被``居然''难住了。《世说》里有很多了``居然'',好像不是出乎意料的意思,而是``显然''、``自然''的意思,但放在这里这句话就显得有豪情,与前面的``凄惘''有冲突。凄惘是伤感迷茫,结合语境,这句话大概是说河山万里辽阔,前途如此遥远难测,自己何其渺小。这样解释大约比较符合东晋人的普遍心态。
  袁宏是当时最著名的文学家(``一时文宗''),我们现在看他的作品,虽然没有什么触及灵魂的感动,但可以感受其文笔的自然流畅,名副其实。但史书上说,虽然大家公认他才华出众,不过``性强正亮直,\ldots{}\ldots{}每不阿屈,故荣任不至''。这也没什么大不了的,``文章憎命达,魑魅喜人过'',你走你高的路,我走我低的路,究竟谁走的远,后人总会知道。

\section{2.84}\label{section-130}

\begin{quote}
孙绰赋《遂初》,筑室畎川,自言见止足之分。斋前种一株松,恒自手壅治之。高世远时亦邻居,语孙曰:``松树子非不楚楚可怜,但永无栋梁用耳!''孙曰:``枫柳虽合抱,亦何所施?''
\end{quote}

  孙绰:字兴公,虽说文无第一,但孙绰被当时人认为是东晋中期最出色的文学家,冠绝一时,当时重臣殷浩、桓温、庾亮等的墓文都是由孙绰写的。
  遂初:实现初衷,所谓人的初衷,就是居于山林。孙绰在《遂初赋》中说,自己仰慕老庄,赞赏战国时齐国陈仲子隐居山林的节操,不愿当官而亲近自然。关于陈仲子我们多说几句,据说陈仲子多次辞让高官显位,他的``仇官''
心态比现在的网友还要猛烈得多,他在《於陵子》中多次抨击所有官僚:``我看你们做官的人,心胸就像高山大川那样神秘莫测;你们道貌岸然,用道德的枷索一样禁锢着人们的肢体;你们看人的眼神就像贪婪的凶鹰,不知怎样才能满足心中的欲望;你们说话从来不算数,这山说话得到那山去听。''以至于赵国的赵威后也对陈仲子的不合作非常愤怒,有一次对齐国的使臣说:``於陵子仲尚存乎?是其为人也,上不臣于王,下不致其家,中不索交诸侯。此率民而出于无用者,何为至今不杀乎?''我看过一些史料,诸如毛泽东、周恩来、陈毅,在平时讲话时也多次谈到自己的理想是当文学家,自己后来的经历大违初衷。邻居家的金正日,据说文学修养也很高,喜欢创作;前些年的萨达姆、老卡斯特罗,都有丰硕的创作成果或者期盼成为文学青年。这是一个很令人吃惊和值得研究的问题,也许事关``自由''。
  畎川:据《世说新语 - 轻诋》,或作伏川,地名,大约在绍兴上虞一代。
  自言:孙绰才华很高,但他的道德在当时并不被人看好,具体事例不详,也许是言行不一。后来孙绰当官很认真的,历任章安令、尚书郎、长史、太守、散骑常侍、著作郎、廷尉卿等,其中有些起伏,可能受到过处分。
  见止足之分:见,明白。止足,《老子》``知足不辱,知止不殆,可以长久'',知道停止和满足的道理。
  松:古人种植松树,一般是在墓地、园林,在家门口种松,总不会是迎客松的味道。现在估计还是有这些讲究的。明代陈琮、清初归庄,就直接把住宅建在墓地,这些都是特立独行的怪人,要表明自己看待世界的特别态度。
  自手:亲手。   高世远:高柔,曾为司空郗鉴的参军、安固令等。
  松树子:``子''有点费解,大约是指小松树。
  楚楚可怜:楚楚或者是指枝叶茂盛,或者指娇柔美丽。可怜,可爱。孙绰的祖父叫孙楚,高柔故意在孙绰表现出不礼貌的行为。古人的名生前曰名,死后曰讳,不能直呼(书)其名,其原因也许是表示尊敬,也许有巫术的因素,也许有权力意志的表现。至今在官场,还存在称职务而不呼名字的恶习。
  永无栋梁用:松树有很多种,不少是高级的木料,但江南一般的松树是马尾松,矮小、歪曲、似乎没有用处。但这不是高柔讲话的重点,我们都知道庄子关于``有用无用''曾经有长篇的论述,生活的高境界在于``有用无用之间''。
  枫柳:大约是指一种木质疏松的枫树。孙绰这样回答,要么高柔的长辈中有叫枫或柳的名,要么暗指高柔,柳树本来就暗含柔弱的意思。
  这仅仅是个关于微小聪明的故事,为何给古人如此的快感,也许要从破坏原始宗教禁忌角度去分析。

\section{2.85}\label{section-131}

\begin{quote}
桓征西治江陵城甚丽,会宾僚出江津望之。云:``若能目此城者,有赏。''顾长康时为客,在坐,目曰:``遥望层城,丹楼如霞。''桓即赏以二婢。
\end{quote}

  桓征西:大概是桓温,但也有人认为是弟弟桓豁,他们都当过征西将军。
  江陵:三国时候有名的地方荆州。桓温督荆州时曾经搞过大规模的城市整治。我没去过荆州,应该去去的。
  江津:指汉江的渡口。   目:题目,就是品题、评价的意思。《后汉书 -
许劭传》:``曹操微时,尝卑辞厚礼,求为己目。''月旦之风,不仅限于人。我猜想后来唐朝的行卷,送上的不仅仅是诗歌文集,估计也得有所物质表示,就是现在所谓的鉴定费。
  顾长康:顾恺之,字长康,东吴丞相顾雍之后,虽然他以画家知名,但文章诗歌在当时也是一流的的,人称``才绝、画绝、痴绝''。
  层城句:据《淮南子》说,昆仑山上有神仙之城,层城九重,这是指荆州似昆仑。丹楼:指官衙或高官的府邸,它们可以涮成朱红色。这两句话都是暗喻投入大量人力物力整修过的荆州有仙家的气派。
  顾恺之说的简洁,他的文风、画风现在看来也很明亮奢华,有道教享乐主义的味道,符合当时的文化氛围。宋代钱惟演曾写道:瘦云萧萧伊水头,风宜清夜露宜秋。更教仙骥旁边立,尽是人间第一流。

\section{2.86}\label{section-132}

\begin{quote}
王子敬语王孝伯曰:``羊叔子自复佳耳,然亦何与人事!故不如铜雀台上妓。''
\end{quote}

  这里的断句也许可以这样:
  王子敬语王孝伯曰:````羊叔子自复佳耳,然亦何与人事!''(王孝伯曰):``故不如铜雀台上妓。''在《论语》中,有些句子就是这样断的,但无论怎样,这句话放在言语里总有些费解,应该在《轻诋》中,因为言辞轻薄。
  王子敬:小王,王献之。在晋代,小王的字评价比大王要高,唐以后则相反。
  王孝伯:身无长物的王恭,痛饮酒、熟读离骚的王恭,神仙中人的王恭,又称王丞、王宁、阿宁等,见1.44。
  羊叔子:羊祜,字叔子,三国时在边境与东吴名将陆抗对垒,多采取怀柔政策。有儒将之风,东吴人尊称他为``羊公''。他以信义德行著称,譬如他岳父夏侯霸因怕司马懿迫害,投降蜀汉,羊祜对岳父的家人依然亲近,不以为忌。羊祜德行高,死后襄阳人给他在生前喜欢的岘山立了块碑纪念,据说人们一看见碑就会忍不住伤心落泪,羊祜的继任者杜预称之为``堕泪碑''。
  然亦何与人事:人事,己事。但与我们个人的人生有什么关系。与或作``预'',因为2.92也有类似的句式,``谢太傅问诸子侄:`子弟亦何预人事,而正欲使其佳?'\,''。
  铜雀台上妓:曹操有个遗嘱,说自己死后,他宫里的侍妾歌姬要按时唱歌跳舞。
  这句话是说,羊祜虽然德行高,但跟我们有什么关系,对我们来说,还不如铜雀台上的歌姬可以赏心悦目。王献之、王恭为什么这样说,大约是时代不同了,羊祜的德性并不受二王的推崇,因为人生贵在适意,无可无不可,秉烛夜游,以吃喝玩乐、随心所欲来挽留短暂的人生。当然,琅琊王氏和羊家有宿怨,当年王戎和王衍常攻击羊祜,时谚称:``二王当国,羊公无德。''      

\section{2.87}\label{section-133}

\begin{quote}
林公见东阳长山曰:``何其坦迤!''
\end{quote}

  林公:放鹤的支遁支道林,见2.45等。
  长山:金华的长山,山脉相连三百余里,在东阳、婺城、兰溪一带,属会稽山脉,支道林晚年主要活动在会稽郡。
  坦迤:平坦逶迤。
  这句话大概是个人的写照,心胸、思想开阔而细腻,支道林虽然是和尚,但是当时最出名的玄学家.

\section{2.88}\label{section-134}

\begin{quote}
顾长康从会稽还,人问山川之美,顾云:``千岩竞秀,万壑争流。草木蒙笼其上,若云兴霞蔚。''
\end{quote}

  顾长康:顾恺之。《世说》行文故意错落,一般不把同一个人放在一起写。
  若云兴霞蔚:后来一般作云蒸霞蔚,修辞上属于互文,大约是云霞华美的意思,``若''字比较费解,草木像云霞,不知从何说起,不过我看过平原整片的油菜花,山间整片的杜鹃,以它们比喻云霞就好理解一点,所以``草木''或当做花木。
  顾恺之的话很有名,后人反复引用,形容江南山水秀美。竞秀、争流都是拟人化的笔法,而且动静结合,相对热闹一些、生动一些,这种修辞看似简单,在行文中能自觉把握却很不简单。现在新昌有穿岩十九峰,诸暨有五泄瀑布,乐清有雁荡龙湫,都印证了顾恺之的赞誉。

\section{2.89}\label{section-135}

\begin{quote}
简文崩,孝武年十余岁立,至瞑不临。左右启:``依常应临。''帝曰:``哀至则哭,何常之有!''
\end{quote}

  孝武:孝武帝司马曜,见1.46等。司马曜出生的时天刚亮,所以叫曜,字昌明。他十一岁时候父亲简文帝去世。他对母亲孝顺,在位其间东晋击败了前秦苻坚,故曰``孝武''。古时候皇帝去世,有时候是丧事办完新皇继位,有时候是一死立刻继位,好像并不统一。
  临:哭丧叫临,聚众大哭叫临。按照古代的礼仪,人去世,亲人必须哭的,不哭就有不孝的嫌疑,不孝是极大的罪。但在魏晋以来,这个风俗受到了庄子思想的抵制,很有一些人要求按照自己的本性生活,甚至矫情,拒绝形式上的悲伤。
  司马曜为何不哭,也许是他母亲的原因。司马曜排行老三,但他的哥哥都夭折了,司马昱不由为千秋大业担心,就去算命。卜者说要找个下贱的宫女才能生出龙种,认为宫里洗衣的黑奴李陵容是合适的人选,后来司马曜果然养大了。也许是桓温的原因,当时桓温势力极大,司马昱在起初的遗诏中对桓温说,如果司马曜不堪扶持,桓温可以取而代之。这种话一定传到司马曜的耳朵里,内心一定是怨恨惊怕的。但这里成为名士之风,我作出了过度诠释。

\section{2.90}\label{section-136}

\begin{quote}
孝武将讲《孝经》,谢公兄弟与诸人私庭讲习。车武子难苦问谢,谓袁羊曰:``不问则德音有遗,多问则重劳二谢。''袁曰:``必无此嫌。''车曰:``何以知尔?''袁曰:``何尝见明镜疲于屡照,清流惮于惠风?''
\end{quote}

  孝经:据说是曾子的学生写的,记录了孔子和曾子关于孝的论述,是儒家十三经之一。但其内容有非常拘泥的地方,把君权放在一个非常突出的位置,与孔子在《论语》中平和的风格有所区别,也不太符合春秋时期的生活面貌,甚至有些内容孔子大约都违背了的,孔子想来不会是现在的领导干部,人格分裂,大玩两面派(四月的一天,办公室里还在传阅省纪委书记王华元的年度工作报告,转眼间有人爆料说王书记被拉下马来,而且在广东长期玩弄女性,我的耳边还袅绕着王书记对我省干部情妇问题铿锵有力的问责要求,当时真是又惊又佩),所以孝经的来历存疑。
  谢公兄弟:谢安和弟弟谢石。司马曜讲孝经,就和现在政治局开会,请人讲社会主义核心价值观一样,大家都要私下准备,期待出色发言。当时在座的有谢安、谢石、陆纳、卞耽、袁宏、车胤、王混。其讲经模式类似释迦牟尼讲经,有人通解、有人提问、有人辨析、有人记录。
  私庭:私宅。前面这些人聚在一起先补补功课,再表演给皇帝看,不像现在敝帚自珍,临场卖弄。
  车武子:车胤,年轻时囊萤读书,先捉一、二小时的萤火虫,然后拿出现今年轻人打网游的劲头埋头苦读,后来终于成为礼仪大家,当上太常、吏部尚书等显职。不过我小时候看过一篇,说康熙他当真了,有一天捉来上百只萤火虫,装进纱袋里,进行照明读书的测试,结果后来下旨对国子监的学生说:``简册所载,不可尽信。''这个故事的确有点奇怪,晋代的纸比较珍贵,穷人家不会有的,家里有书就有钱,砍点、买点木柴、或者油脂燃烧照明不就结了,何必作这番秀呢。所以后来对车胤这个故事多有讽刺,当做笑话讲。这个故事告诉我们,当领导必须善于来事,善于把简单问题复杂化,善于出奇思妙想。车胤在这次理论中心组学习会中担任捧哏的角色,``难苦''、``摘句'',就是懂装不懂,提出字句间的疑问。谢石回答,谢安总结阐述。
  德音:善言。   袁羊:当做袁宏,见2.83。
  ``何尝''句:何曾见过明镜会因连续照影而疲劳,清澈的流水会害怕和风吹拂?明镜、清流是很自然和高上的评价袁宏果然才思敏捷、词句清丽。史料上讲,谢石虽然很贪财,但词章之学很不错,``务存文刻''。

\section{2.91}\label{section-137}

\begin{quote}
王子敬云:``从山阴道上行,山川自相映发,使人应接不暇。若秋冬之际,尤难为怀。''
\end{quote}

  王子敬:王献之,王羲之第七子,因为他话少,所以被认为是王羲之最优秀的儿子。琅琊王家长期寓居会稽。
  山阴道:据我估计,可能指绍兴到诸暨枫桥的一条山路,中间经过鉴湖、兰亭、若耶溪、眉山、赵家。为什么这么说呢,我4月因为祖父葬礼的事,开车送吊唁的亲友回家,平原、丘陵、高山次第逼来,时时峰回路转,移步换景,眼前一路葱葱荣荣,时有溪水湖池,鹅鸭犬牛,心情一时非常柔和。于是一下子想起王献之的话来,想来人心皆有相通之处。不过这种突然的感悟也许是单相思,古代的路线和现在道路差别很大。古人旅行,大抵是乘舟前往,更加能缓缓欣赏,想必更加心旷神怡。王献之反复走这条路线,想来是对山水的热爱。
  老家有句谚语,叫会稽不收,山阴不管。大约绍兴城以大江桥为界,一边是会稽县,一边是山阴县,人在桥上无论干什么事都没人管的,有一次桥上冻死个乞丐,两个县就互相推诿,后来徐文长就收过桥费,卖桥,逼迫两县应对。
  为怀:本义应是记住,这里应该是忘怀的意思。肯定词语为什么会变成否定词语,汉语中有多少这样的用法,代考。
  秋冬之际大约是指万物清冷萧索,但江南山水不像北方,即使苍凉,也依旧深邃,更有生命的表现力。这种景色更能触发东晋人的共鸣,自渡江以来,虽有家国之悲,但由于以老庄思想为支撑,士人反复拷问生命的意义,``回家''不再局限于实在的处所。
  王献之有个书帖,其中写道:``镜湖澄澈,清流泻注。山川之美,使人应接不暇。''可以作为此条的佐证。后世诗人,对山阴道也多有歌咏。

\section{2.92}\label{section-138}

\begin{quote}
谢太傅问诸子侄:``子弟亦何预人事,而正欲使其佳?''诸人莫有言者。车骑答曰:``譬如芝兰玉树,欲使其生于阶庭耳。''
\end{quote}

谢太傅:谢安。
  何预人事:关自己什么事。前面介绍过,孔融曾经发表过父母与子女无恩的高论,``父之于子,当有何亲?论其本意,是为情欲发耳!子之于母,亦复奚为?譬如寄物瓶中,出则离矣!''魏晋时期个人主义的强势可以从谢安这话中得以体现。人生就是这样,``我就是我,别人与我无关''成为正论。
  正:总是。《世说新语 - 文学 -
24》:``何以正善人少而恶人多?''古人曾说,人生最愉快的事情就是一早起来听见子侄的朗读声。苏轼发牢骚说希望子女愚笨,但最终依旧希望孩子``无灾无难到公卿'',都是``欲使其佳''。就像我解释《世说》一样,主要为了孩子有点文化,有点情趣,胸怀开阔,理解他人。
  车骑:谢玄。曾任车骑将军,见2.78。
  芝兰玉树:芬芳的兰草,美丽的树木。《孔子家语 -
在厄》:``芝兰生于深林,不以无人而不芳。''宋之问《折杨柳》:``玉树朝日映,罗帐春风吹。''
  生于阶庭:长在自家庭院里。谢安提问是庄子的思想,谢玄回答带有儒家的内涵。谢安未必是真心话,只是一种考验,谢玄这次又胜出。不管事实如何,历史怎么评价,反正谢家的子侄都以自己为傲,谢道韫就很看不起婆家王羲之的子侄,以自己的的兄弟为高明,结果郁郁寡欢。

\section{2.93}\label{section-139}

\begin{quote}
道壹道人好整饰音辞。从都下还东山,经吴中。已而会雪下,未甚寒。诸道人问在道所经。壹公曰:``风霜固所不论,乃先集其惨澹;郊邑正自飘瞥,林岫便已皓然。''
\end{quote}

  道壹道人:据《高僧传》中说,道壹和尚本姓陆,江南人,可能还是会稽人,所以后文用``还东山''的字眼。道壹从竺法汰学习佛学,学理、口才出色,成为有名的高僧。竺法汰是前面介绍过有特异功能的活佛佛图澄的弟子,法汰的弟子有昙壹、道壹、昙二,法号都蛮有趣的。
  整饰音辞:修饰言辞。道壹后面讲雪景形式上都是六字、押韵,对仗工整。但虚词过多,并不能算六言诗。
  都下:就是指都,南京,下为衬词,没有意义,见2.83。
  吴中:指春秋时吴国旧都苏州。法汰圆寂后,道壹离开南京,据传记上说,他没有直接回会稽,而到了虎丘住下了。后来好友帛道猷写信约他回山阴读书、爬山、采药、服药,道壹这才回会稽。太守王荟建山阴嘉祥寺,道壹当了庙里的僧主。古希腊有首诗叫《幸福四要素》:人皆有死,最重要第一是健康,第二是天生性情温和,第三是有一份并非来之不义的财产,第四是有一批朋友欢度春光。道壹好像就是这样。
  未甚寒:有点费解,未或当作``天''。我们不要有下雪时候天不是很冷,雪化时气温才更低这种过于理性的思考。
  壹公:我小时候看书上讲,有人认为把莎士比亚称为莎翁,马克思称为马公都是无知的表现,事实上这个传统历来就有,在世说中,就称支道林为林公,竺法汰为汰公,约定俗成,不用煞有其事地去找他们的姓。不过像恩格斯这样我们就不称恩公了,得称斯公。前些天中央台有个主持人称别人的父亲为家父,惹来大家一片讥笑。事实上,古人称别人的父亲为家君也是有的;我去北方出差,有些人自来熟,称我家人为咱爸、咱叔等,我不大习惯,甚至别扭,但这种用法并不出奇。
  ``风霜''句:此句意会不难,准确解释却说不好。风霜应该是偏义复词,就指风,或者代指严寒。本句的意思大概是:暂且不说一路中的风霜严寒,(天空)一点点的暗淡堆积。
  飘瞥:飘落、消失。瞥有很快出现一下的意思。
  林岫:树林、山峰。我们当然记得祖咏的《咏雪》:终南阴岭秀,积雪浮云端。林表明霁色,城中增暮寒。祖咏的诗和道壹的言辞都指出下雪的气温问题,祖咏说的是城外要化雪,城中就会变冷一些。道壹说的是有人居住的平原气温高,雪积不了。风花雪月是古典诗歌最常见的主题之一,我曾经收集过一些关于雪的不同思考,可以在后面陆续介绍。
  按道理讲,佛教万物皆空,客观的山水并不成为审美对象,但晋代的和尚显然醉心自然,释玄合一,并不超越于时代。  

\section{2.94}\label{section-140}

\begin{quote}
张天锡为凉州刺史,称制西隅。既为苻坚所禽,用为侍中。后于寿阳俱败,至都,为孝武所器。每入言论,无不竟日。颇有嫉己者,于坐问张:``北方何物可贵?''张曰:``桑椹甘香,鸱鸮革响;淳酪养性,人无嫉心。''
\end{quote}

  张天锡:张天锡,字纯嘏。嘏是福气、福分的意思,《诗经》中说:``锡尔纯嘏,子孙甚湛。''张天锡是16国中前凉的最后一位君主,他在位时向东晋称臣,后来又归降了晋朝,是16国中唯一投靠晋朝的国王。
  称制:国王的诏令叫制,称制就是称王。凉州是现在的宁夏、甘肃等地区,所以叫西隅。
  寿阳:淝水之战主要在现在的安徽寿县一代。山之南水之北叫阳。苻坚大败的原因不是一声吼导致的,也不是什么东晋是正朔,兵法高超的缘故,
而是苻坚统一北方的形式和处理方式有问题,仗还没打,几个以前投降的诸侯王孙早就做好了各自回家,重新称霸的打算,苻坚居然让他们继续带族兵,还一味怀柔,大家都在等这一声吼,输得并不冤枉。
  每入言论:入后似应加``朝''字。当时张天锡以文采著名,言论高明。他当国王时,喜欢宴乐游玩,臣子们劝他,他回答说:``你们以为我喜好游乐,其实我用心良苦谁知道!我从中在体会人生的哲理:看到花开茂盛,我就敬重才华俊秀的士人;品玩芝兰,就更加敬重德行高洁的臣子。看到松竹,就想起忠贞节操的贤才;面对清流,就更加看重清澈廉洁。我一看到蔓草,就鄙薄贪婪的污吏;迎著风少,无比痛恨凶狠狡诈的奸徒。如果我们大家能把游乐引申出去,触类旁通,那么朝政就不会有什么遗漏了。''这样华丽优美的词藻,举一反三的言论,找他整天聊天应该是愉快的事。
  嫉己者:``己''他本作``之''。晋书上讲是会稽王司马道子,人品恶劣。
  北方何物可贵:这种问法在现在看来很正常,但在当时是无礼的行为,我们在《世说》中还可以陆续看到一些类似的挑衅问题,如2.26。
  桑椹:古代的水果不多,像西瓜是宋代以后才有的,苹果可能是清末才有的,猕猴桃根本不能吃。在晋代,估计也就是橘子、桃子、枣李之类。虽然个别品种当时已经种植了,像荔枝、桂圆之类,由于保鲜的原因。除产地外其他人很难吃到。所以像桑果这样不入流的小吃件,在古代也当做珍贵的果品。
  鸱鸮革响:鸱鸮,一种猫头鹰;革,变换。《诗经》中说:``翩彼飞鸮,集于泮林;食我桑椹,怀我好音。''这是说,像猫头鹰这样叫声很难听的鸟,吃了桑果以后声音也很好听。言下之意就是司马道子这只猫头鹰没吃过北方的桑果,讲话真难听。
  人无嫉心:大约是说吃奶酪的人淳朴,胸襟开阔。前面2.26讲过,北方人以奶酪为美食。张天锡的回答整齐、押韵,暗藏典故用来反讽,有言辞之美。司马道子吃了亏,孝武帝死后道子执政,于是张天锡``后形神昏丧,虽处列位,不复被齿遇(平等对待)''。不过生活的问题就是:我为什么要咽下这口气?泰戈尔说,燃烧的木块熊熊地生出火焰,叫道:``这是我的花朵,我的死亡。''

\section{2.95}\label{section-141}

\begin{quote}
顾长康拜桓宣武墓,作诗云:``山崩溟海竭,鱼鸟将何依!''人问之曰:``卿凭重桓乃尔,哭之状其可见乎?''顾曰:``鼻如广莫长风,眼如悬河决溜。''或曰:``声如震雷破山,泪如倾河注海。''
\end{quote}

桓温的墓地在现在的安徽当涂县。 溟海:就是海。
凭重:倚重。顾恺之得到桓温的赏识,曾经担任他的参军。 见:表现。
鼻、眼:抽泣、眼泪。
广莫:广漠。《淮南子》:``穷奇,广莫风之所生也。''传说穷奇这种怪兽是北风所生。《說文解字
- 风部》:``风,八风也。东方曰明庶
风,东南曰清明风,南方曰景风,西南曰凉风,西方曰闾阖风,西北曰不周风,北方曰广莫风,东北曰融风。风动虫生,故虫八日而化。''
决溜:急流.,不是决堤。
这种佛教中的夸饰语气我不太喜欢,显得比较做作,而不是生动形象。《庄子》中的夸饰是宏大叙事,为了阐释哲理,佛教中的夸饰是为了招摇,用肤浅来招揽信徒。以我的个人感觉,夸张的修辞手法很难用好,往往只能用于讽刺,对于认真描写事物的本来面目并不恰当。顾恺之采用这样的说法,是内心实在佩服桓温而来的,反而显得荒诞,谁在悲伤的时候注意到自己那时的表现?``不敢高声语,恐惊天上人''比前句``危楼高百尺,手可摘星辰''要含蓄生动得多。

\section{2.96}\label{section-142}

\begin{quote}
毛伯成既负其才气,常称:``宁为兰摧玉折,不作萧敷艾荣。''
\end{quote}

毛伯成:毛玄,河南颍川人,当过征西行军参军(大概就是桓温的参谋),缺乏传记资料,大概就是因为言行多有锋芒,去世又较早。钟嵘在《诗品》中说毛玄``文不全佳,亦多惆怅'',判定为``下''。不过曹操也在``下''中,能上富豪榜,不管是排名倒数多少位,也算个款爷。
萧敷艾荣:同义互文,杂草茂盛。
《离骚》:``人好恶其不同兮,惟此党人其独异。户服艾以盈要兮,谓幽兰其不可佩。览察草木其犹未得兮,岂珵美之能当?\ldots{}\ldots{}何琼佩之偃蹇兮,众薆然而蔽之。惟此党人之不谅兮,恐嫉妒而折之。时缤纷其变易兮,又何可以淹留?兰芷变而不芳兮,荃蕙化而为茅。兰芷变而不芳兮,荃蕙化而为茅。何昔日之芳草兮,今直为此萧艾也。''
人们的好恶本来就各不相同,只是那些党人特别奇怪,他们个个都将杂草系满腰间,反而说幽兰不可佩在身边。香花恶草他们都不会
鉴别,那美玉他们又怎能认识?\ldots{}\ldots{}为什么我的玉佩光彩,人们却要将它的光辉遮掩?这些党徒不能信赖,担心他们会出于嫉妒而将玉佩折断!时世纷乱无常,我怎能在这里久住。兰芷都消尽了芬芳,荃蕙都化作了草蔓,为什么过去那些香草,今日竟变成了蒿艾?
据说看书有三个好处,一是虽然自己现在看似风光,但知道学识上还比不上很多人,没有值得满足的;二是虽然自己不得志,但那些才华很高超的人比自己还落魄,内心就能够平静;三是世界上像自己这样的人还有很多,同行的路上并不寂寞。

\section{2.97}\label{section-143}

\begin{quote}
范宁作豫章,八日请佛,有板。众僧疑,或欲作答。有小沙弥在坐末,曰:``世尊默然,则为许可。''众从其义。
\end{quote}

范宁:字武子,河南南阳人,《后汉书》作者范晔之祖父,曾任中书郎,豫章太守。
八日请佛:释迦牟尼的生日有说是二月初八,有说四月八日等,反正是个传说,不必当真。就我个人而言,对佛教的一些理论可以理解和共鸣,但对佛教的繁复、排场、虚伪和神秘主义反感,想来那也不是释迦牟尼的本意。我曾经写过简短的佛教批判,有信徒就说怎么可以狂妄到用``批判''呢?他们连批判是什么意思都不清楚,这样就缺乏对话的基础。佛诞日又称``浴佛节''、``龙华会''、``华严会''等,有各式各样的仪式,在晋代,大概是要请出佛像来供奉。
有板:像现在请示类公文,必须有回复。当时用木简请示,所以叫有板。
沙弥:初出家的(小)和尚,受十戒(不杀生、不盗、不淫、不妄语、不饮酒、不着香华鬘不香涂身、不歌舞倡妓不往观听、不坐高广大床、不非时食、不捉持生像金银宝物),比在家的居士多两戒。据一些佛经讲,小和尚也有年龄限制,七岁以上,十三岁以下,称驱乌(鸟)沙弥。十四岁以上,十九岁以下,因其已能顺应沙弥行法,所以称为应法沙弥。二十岁以上,七十岁以下,岁数大了,但因为各种原因不能受足戒,称为名字沙弥。
默然:不做回答就是默认。像《法华经》、《大乘造像功德经》中,常有``世尊默然而不制止''、``世尊默然许可''的字眼。在生活中,我们可以认为,我们提问,对方不置可否,一般就是认可的意思。当然这也有特例,当年王敦叛乱,抓获名士周顗和戴渊,问王导的意见:让他们继续做官?王导不说话。杀了?王导还是不说话,等于是认可死刑立即执行。

\section{2.98}\label{section-144}

?司马太傅斋中夜坐,于时天月明净,都无纤翳,太博叹以为佳。谢景重在坐,答曰:``意谓乃不如微云点缀。''太傅因戏谢曰:``卿居心不净,乃复强欲滓秽太清邪?''

司马太傅:司马道子,孝武帝司马曜的弟弟,哥哥喜欢酒色,谢安死后,道子执政,任太傅。他在文学、音乐等方面很有些天赋,同样喜欢酒色享受,工作并不努力,所以在``醉相''执政期间,东晋没有多大起色。司马家族在淝水之战后,在试图振兴皇权的过程中,中央和地方的矛盾不断,诱发了门阀之间不断的军事斗争,
司马道子被称为``乱臣''。 纤翳:翳,遮蔽。微小的遮蔽指云彩。
谢景重:谢重,字景重,谢安最喜欢的侄子谢朗的儿子,他在司马道子手下任骠骑长史。
乃复:竟然。``孔北海乃复知天下有刘备耶?''
滓秽太清:污秽天空。古人认为天系清而轻的气构成,汉刘向《远游》:``譬如王乔之乘云兮,载赤云而陵太清''。道教认为,``上升四十里,名为太清,太清之中,其气甚罡,能胜人也''(《
抱朴子 - 杂应》)。
就修辞艺术而言,谢重的说法是有道理的,点缀、闲笔、衬托往往能使所要表现的事物更加生动。画月亮,不免加上一些云彩;画花朵,不免加上蜜蜂蝴蝶;在描写人物中,作者也往往会用与主要故事本身无关的细节来表现其性格特点。即使从老庄哲学出发考量表现创作手法,微云点缀也恰恰说明其创作在``有意无意之间''。而我们按照司马道子的说法推理,岂不是月亮也是天空中滓秽?当然,月亮使天空光明,人能做到胸无点尘,一片清亮,是很好的境界。后来苏轼说:``参横
斗转欲三更,苦雨终风也解晴。云散月明谁点缀,天容海色本澄清。''

\section{2.99}\label{section-145}

\begin{quote}
王中郎甚爱张天锡,问之曰:``卿观过江诸人,经纬江左轨辙,有何伟异?后来之彦,复何如中原?''张曰:``研求幽邃,自王、何以还;
因时修制,荀、乐之风。''王曰:``卿知见有余,何故为苻坚所制?''答曰:``阳消阴息,故天步屯蹇;否剥成象,岂足多讥?''
\end{quote}

王中郎:北中郎将王坦之。但王坦之375年去世,张天锡383年以后归晋,时间对不上,因为是小说,我们不必深究。张天锡虽然世居凉州,似为蛮荒之人,但有很高的学术修养,精于易理,辩才无碍。
``经纬江左轨辙''句:这里的标点有些困难,因为经纬究竟是什么意思,是否可以作动词缺乏证据,但大意总是:过江的人治理江南的法度,有什么特别突出的地方?有没有后起之秀?王坦之希望著名的归国华侨张天锡评价一下自己那旮旯的干部,这种心情并不难理解。
``研求幽邃''句:张天锡对东晋时具体的某个人避而不谈,表扬的是一种现象。说当时的学术研究继承了王弼、何晏的传统,精研玄理;朝廷适时调整礼仪法制,有荀顗、乐广的遗风。张天锡这样回答,是不是谨慎的表现?这个回答王坦之不会满意,后面的提问就不留余地。
息:不是停止,而是生长。``休养生息''。 天步:国家的命运。《诗经 -
白华》:``天步艰难,之子不犹。''
屯、蹇、否、剥:《周易》中卦名。屯卦,刚柔始交而难生,动乎险中。蹇卦,难也,险在前也,见险而能止,知矣哉。否卦,天地不交而万物不通也,上下不交而天下无邦也。\ldots{}\ldots{}小人道长,君子道消也。剥卦,不利有攸往,五阴在下,一阳在上,阴盛而阳孤。
张天锡说,有盛必有衰,有生必有死,,阴阳盛衰相互依存;在艰难困苦的时候人也不必灰心丧气,坚守正道就能走出困境;时运不好,不值得大加讥笑。张天锡走的是《易学》的易理派而不是象数派,不具体占卜算命而讲形而上的哲理。这样的回答完全是教育的口气。

\section{2.100}\label{section-146}

\begin{quote}
谢景重女适王孝伯儿,二门公甚相爱美。谢为太傅长史,被弹,王即取作长史,带晋陵郡。太傅已构嫌孝伯,不欲使其得谢,还取作咨议;外示絷维,而实以乖间之。及孝伯败后,太傅绕东府城行散,僚属悉在南门,要望候拜。时谓谢曰:``王宁异谋,云是卿为其计。''谢曾无惧色,敛笏对曰:``乐彦辅有言:`岂以五男易一女'。''太傅善其对,因举酒劝之,曰:``故自佳!故自佳!''。
\end{quote}

谢重的女儿谢月镜嫁给了王恭的儿子王愔之。
门公:家公,两个子女的父亲。古代称家为门。《周礼 - 春官 -
小宗伯》:``其正室皆谓之门子,掌其政令。''《逸 周书 -
皇门》:``乃维其有大门宗子。''谢重和王恭交好,王恭看不起司马道子,矛盾很深。王恭为人好像很正直,对坏人坏事都看不惯,
``读《左传》至`奉王命讨不庭',每辍卷而叹'',是个孤臣。
带晋陵郡:王恭担任司马道子的长史,但同时管辖晋陵郡。但在《晋书》上,并没有王恭担任太傅长史的记录,而是督兖、青、冀、幽、并徐州、晋陵诸军事,兖、青二州刺史、镇京口。
构嫌:王恭杀司马道子的亲信:``陈郡袁悦之以倾巧事会稽王道子,恭言之于帝,遂诛之'';王恭批判司马道子亲近巫婆:``裴氏有服食之术,常衣黄衣,状如天
师,道子甚悦之,令与宾客谈论,时人皆为降节。恭抗言曰:未闻宰相之坐,有失行妇人。坐宾莫不反侧,道子甚愧之'';王恭拒绝和解:``道子亦欲辑和内
外,\ldots{}\ldots{}冀除旧恶。恭多不顺,每言及时政,辄厉声色。''
絷维:《诗 - 小雅 -
白驹》``皎皎白驹,食我场苗,絷之维之,以永今朝'',拉住马缰,引申为挽留人才。
乖间:乖不是现在的小孩子乖,而是不乖。乖,相背也。乖间:隔阂,疏远。``乖''词义会完全转到相反的意思上去,这种现象并非个例,譬如前面我们介绍过``去'',本义是前往、达到,后来变为离开。
孝伯败后:王恭起兵讨伐司马道子,结果大将刘牢之叛变,兵败。见1.44。
东府城:司马睿当东晋元帝以前,曾任镇东大将军领扬州刺史。其官衙就叫东府。司马道子当时也领扬州刺史,同一个官衙。
行散:服用寒食散后暴走。
要望候拜:要望大概是请示的意思。这句话大概是反映道子的排场极大,气焰极盛,一副意得志满的样子。
时:这句话不要理解,``时''似为衍字,没有必要。司马道子既然在行散,为什么后面又敬酒,难道像刘伶一样走一路喝一路?谢重的回答就是乐广当年的回答,见2.25。
王宁:王恭,字孝伯,小字宁。
故自佳:自应为衬字,没有意义。故自佳就是的确是这样的意思。司马道子在《世说》中的表现好像比较糊涂,后面紧接着的一则故事也是酒后失言,把自己不能说出来的心思暴露出来。这么大的事情要讲证据的,一出手就应该是杀手,而不是趁着酒意发泄情绪。

\section{2.101}\label{section-147}

\begin{quote}
桓玄义兴还后,见司马太傅。太傅已醉,坐上多客,问人云:``桓温来欲作贼,如何?''桓玄伏不得起。谢景重时为长史,举板答曰:``故宣武公黜昏暗,登圣明,功超伊、霍。纷坛之议,裁之圣鉴。''太傅曰:``我知!我知!''即举酒云:``桓义兴,劝卿酒!''桓出谢过。
\end{quote}

义兴:现宜兴,周处的老家,当年山上有猛虎,河中有鳄鱼,村里有恶霸,端的是山清水秀、人杰地灵。周处的儿子周纪屡起义兵,帮助晋惠帝平定江南。为表彰周纪的功劳,晋惠帝改阳羡郡县为义兴郡。桓玄曾出任义兴郡太守,守卫太湖,他心不甘,就说:老子执掌九州(桓温``为九州伯''),儿子倒成了湖河的看守
(``为五湖长''),弃职而去,并上书表示不满:``至于先帝龙飞九五,陛下之所以继明南面,请问谈者,谁之由邪?谁之德邪?岂惟晋室永安,祖宗血食,于陛下一门,实奇功也。''你司马曜父子本是司马家旁支,能当皇帝,全是我老头子桓温的功劳,``若陛下述遵先旨,追录旧勋,窃望少垂恺悌覆盖之恩。''。
来:或作``本'',或作``未''。本来或者晚年。贼:强盗,指桓温晚年有篡位之心。
``故宣武公''句:无论桓温有没有谋位之心,但始终没有事实。而且桓温对司马昱、司马曜家的功劳实实在在,其他人或许可以非议桓温,但司马道子没有资格直呼其名,更不应该在大庭广众中说桓温要造反。当时桓家依然把握一方之要害,桓温的弟弟桓冲为荆州刺史,其故旧部曲遍布天下,皆掌兵权、处要津。道子酒后随性乱说,毫无政治、军事准备,可能直接导致天下大乱。谢重慎重其事地聚笏直谏,是恰当的。
伊、霍:伊尹、霍光。伊尹是商汤的宰相,又辅佐商汤的孙子太甲,曾一度废去太甲,自任``假王''。霍光受汉武帝遗诏辅佐汉昭帝,昭帝死,本立刘贺,霍光废去刘贺,迎立汉宣帝刘病已。伊尹、霍光当然是权臣,但他们废除皇帝,一般认为是对天下、社稷忠诚的表现。桓温废除司马弈,立司马昱,形同两者。
圣鉴:看样子不是皇帝的专有名词,皇家直系都可以使用的。这里指司马道子的的鉴察。
桓义兴:对桓温直呼其名,这时候又表示歉意,称桓玄的官职了。
桓出谢过:出应当做``起'',当时桓玄还跪在地上,没有被架出去就刑。
司马道子这出戏演的十分糟糕,要么是酒精上脑,要么这个人实在太单纯了,不谙政事。有了这一遭后,桓玄必生反心。我想,没有人从小立志一定要造反的,而是生活一步步推动起来的。这次酒会对东晋来说,是重大的转折点,一席锦袍已经撕去,桓家彻底和司马家决裂。

\section{2.102}\label{section-148}

\begin{quote}
宣武移镇南州,制街衢平直。人谓王东亭曰:``丞相初营建康,无所因承,而制置纡曲,方此为劣。''东亭曰:``此丞相乃所以为巧。江左地促,不如中国。若使阡陌条畅,则一览而尽;故纡余委曲,若不可测。''
\end{quote}

桓温治荆州搞城市建设,``丹楼如霞'',治当涂则街道平直。就像现在杭州市的王国平先生一样,无论市民、网友对他的经济收入、个人品行和执政方法有多少非议,后任将为弥补巨大的财政亏空付出多大的努力,但在王的治理之下,杭州市的确发生了前所未有的城市面貌变化。大搞城市建设,成绩一目了然,本来就表明执政者事功的决心,虽然儒家对大兴土木历来有些看法,现在看来是他们不懂经济,``竖子不足与谋''。
桓温当时已经两次北伐,官居都督中外诸军事,录尚书事,又牧扬州,一时风光无二;他移镇当涂,遏制南京的南边门户,重现了王敦、苏峻所造成的局势,也引起了朝廷的惊惧。不过在这个形势之下,自然有一些大臣有抬轿子、拍马屁的想法,这里有人贬低王导,赞扬桓温,大抵出于这种思想,于是引起了王导孙子王珣的强烈反弹,王珣直言不讳地说王导心思技巧,深不
可测,桓温比不过他,似乎在传递一种信息:琅琊王家并不打算和桓温合作。
南京曾经毁于苏峻之乱,当时有人提出是否可以迁都绍兴,而王导执反对意见,认为南京有帝王之气,是王者之宅,贸然迁都,容易引起思想混乱,百姓疲乏,海内动荡。于是重修南京,而东晋的南京,不像是现在的大都市格局。假如现在的乌衣巷是保持了一点原貌的话,我们可以想见,当时应该是江南小镇的风格,类似于二十世纪90年代前苏州、绍兴等地,小巷转折,幽深变换。
南京素有帝王之气的传说,但那些故事估计都是孙权以后编造的。秦始皇从来没有到过南京,在三国以前,南京根本没有城市,而是孙权为了迁都制造舆论。后来南京屡屡成为中国的首都,但国怍都不长,似乎又表明南京不适合建都,这可能与南京的地理位置有关。在冷兵器时代,南京地势优良,虎踞龙盘,长江天堑将它与北国隔断,易守难攻,统治者会产生偏安苟且思想。江南素为鱼米之乡,有比较富庶的经济条件,在另一方面给统治者提供了享乐的温床。古代多靠马战,定都南京也
导致士兵平时的训练与北方世纪情况脱节等。
王东亭:王珣,字元琳,王导之孙,桓温的主簿,累迁尚书左仆射,封东亭侯。
丞相:王导。 无所因承:没有什么可以借鉴继承的。
江左地促:一般而言,在城市格局方面南方比较曲折,北方比较规正,这主要是地理环境决定的。因为北方一马平川,多数街道形成棋盘格状,南北清楚,南方多为丘陵水泊,因势而建,曲折弯绕。王珣就提出,南京城的建设,是讲究``曲则全''、``少则多'',在有限的空间内,把握好有无、虚实、藏露、疏密,这样才能
``曲尽其态''、``曲尽其妙''。 条畅:条就是长,条直通畅。
纡余委曲:纡余、委曲同义反复,就是曲折。

\section{2.103}\label{section-149}

\begin{quote}
桓玄诣殷荆州,殷在妾房昼眠,左右辞不之通。桓后言及此事,殷云:``初不眠,纵有此,岂不以`贤贤易色'也!''
\end{quote}

殷仲堪前任王坦之的儿子王忱打压袭爵南郡公的桓玄(我们应该记得他们父辈的恩怨),殷出任荆州后,桓玄终于松了口气,和殷仲堪攀附交好,不过后来殷成了绊脚石,桓玄杀了殷,夺了荆州。
辞不之通:``之''费解,也许是``辞之不通'',下人谢绝给他通报。也有说,``之''同``与'',``辞,不与通''。这句话看上去显得有些罗嗦,但语气平缓是世说文字特点。
初:本来。这句话大意是说,我平时不大睡午觉的。 贤贤易色:语出《论语 -
学而》,子夏曰:``贤贤易色,事父母能竭其力,事君能致其身,与朋友交言而有信,虽曰未学,吾必谓之学
矣。''子夏的话,事父母,事君,与朋友交都好理解,就这一句不大看得懂,东汉的孔安国认为是``贤妻易色'',就是像看中妻子的外貌一样看重别人的品德品德,也有说是看重妻子的品德,不在乎妻子的外貌等。抛开子夏的原意,这里殷仲堪的意思应该是:我平时一向不睡午觉的,即使那天睡了,不接待你也不是我的本意,那样做岂不是没有做到``像看中妻子的外貌一样看重贤人吗''?
小时候我们学过一篇叫《报刘一丈书》,其中有一段描写小人拜见权贵时受到的屈辱和自己的无耻:``日夕策马候权者之门,门者故不入,则甘言媚妇人状,袖金以私之。即门者持刺入,而主人又不即出见;立厩中仆马之间,恶气袭衣袖,即饥寒毒热
不可忍,不去也\ldots{}\ldots{}相公又稍稍语人曰:`某也贤!某也贤!'\,''读一遍《报刘一丈书》,就会加深对这个小故事的理解。
我有时候很佩服有些同事和上级,他们对领导比对自己父母要孝顺得多,而且举止非常自然,端是了得。不过估计桓玄不会忍受这样的遭遇,充其量也就是黛玉晚上去找宝玉,被丫鬟吃了个闭门羹,宝黛产生误会。

\section{2.104}\label{section-150}

\begin{quote}
桓玄问羊孚:``何以共重吴声?''羊曰:``当以其妖而浮。''
\end{quote}

羊孚:字子道,羊祜的子孙,当时是桓玄的记室参军,当时他的文字很出名,后面有介绍。
吴声:在东晋,官方语、士大夫的交际语言应该是河南话、洛阳话,而江浙的老百姓平时使用的自然是吴方言。但听歌无论官庶都听吴歌,就像上世纪80年代流行歌曲一般都是粤语。我想,吴歌的唱法大约相当于《茉莉花》或苏州评弹,声音娇媚细腻,靡靡动听。吴歌不光声音好听,内容也多表达情爱,意味缠绵,直指人心。我们现在还可以看到乐府诗集中的《吴声歌曲》,譬如《子夜歌》:``宿昔不梳头,丝发被两肩。婉伸郎膝上,何处不可怜'',``今夕已欢别,合会在何时?明
灯照空局,悠然未有期'',``盛暑非游节,百虑相缠绵。泛舟芙蓉湖,散思莲子间'',有的亲切写真,有的谐音机巧,有的艳丽放荡,适合以声色忘忧;而当时的北
方民歌大约像《古诗十九首》,比较深沉感慨,唱着唱着就伤心了。
妖而浮:娇柔清丽,在古代,这些词不一定是贬义词。

\section{2.105}\label{section-151}

\begin{quote}
谢混问羊孚:``何以器举瑚琏?''羊曰:``故当以为接神之器。''
\end{quote}

谢混:谢安的孙子,被孙恩打败的谢琰的儿子,他极有风采,被誉为``风华江左第一'',后因投靠错阵营被刘裕杀死,后来刘裕说:``可惜后辈不能再见谢混的风流了。''
器举瑚琏:《论语 -
公冶长》中,子贡问孔子:``我这个人怎么样?''孔子说:``你像一个器具。''子贡当然记得``君子不器''的典故,就
紧张地追问是什么器具,孔子说:``是瑚琏。''这时候子贡松了口气。瑚琏是一种用玉石装饰的器皿,朱熹在《论语集注》中说:``夏曰瑚,商曰琏,周曰簠簋,皆宗庙盛黍稷之器,而饰以玉,器之贵重而华美者也。''商周时期,鼎是祭祀时放肉的,瑚琏是祭祀时放谷物的,孔子看样子把子贡比喻为国家中最为重要的人才之一,《魏书
-
李平传》:``实廊庙之瑚琏,社稷之桢干。''谢混想更加准确地把握孔子的原意,就进一步追问:瑚琏到底有什么特性,孔子这样评价子贡的根据是什么。羊孚认为,瑚琏是用来交接神灵的工具,应该是很高的评价。
说这话讲得好,我不大懂。

\section{2.106}\label{section-152}

\begin{quote}
桓玄既篡位,后御床微陷,群臣失色。侍中殷仲文进曰:``当由圣德渊重,厚地所以不能载。''时人善之。
\end{quote}

403年12月,桓玄逼迫晋安帝司马德宗禅位(说逼迫有点过了,在史书上,晋安帝弱智,可能连皇帝未何物也不知道,可能因为父祖辈也有名士的习性,服用寒食散出了意外),并急切地即帝位于安徽当涂,国号楚,完成了桓家的心愿。
御床:在晋代应该是指没有靠背的宝座,座垫可能是用竹藤之类的东西做的,桓玄可能体胖,所以出了这样的意外。我们应该记得,当年司马炎即帝位的时候,占卜国运长久得``一'',都很让人不快。
殷仲文:桓玄的姐夫,当时以文章好、为人贪婪著称。谢灵运曾经说殷仲文``若读书半袁豹,则文才不减班固'',就是说殷仲文如果读书有袁豹(当时以博闻著称)一半多,他的文采就可以和班固媲美。在唐古文运动以前,司马迁并不出名,班固更有地位一些,大概赋体是当时的主流文体,班固善于写赋。他多次劝桓玄早日当晋帝,私下为桓玄写了九锡文与册命,是桓玄的得力助手,当时桓玄对``王谧见礼而不亲,卞范之被亲而少礼。其宠遇隆重,兼于王卞矣''。
渊
重:深重。《易》中说:``地势坤,君子以厚德载物。''成语就简写成厚土载德。殷仲文夸桓玄德行深重,以致大地也承受不了。桓玄的德行的确不错,他这么大的官,甚至当了皇帝之后,见到别人有好字画、大园子,也不好意思直接索要,而是大家一起掷骰子打扑克,用各自喜欢的东西作赌注,用苦练出来的赌技光明正大地赢来。桓玄被刘裕杀死以后,殷仲文又大义灭亲,投靠刘裕。

\section{2.107}\label{section-153}

\begin{quote}
桓玄既篡位,将改置直馆,问左右:``虎贲中郎省,应在何处?''有人答曰:``无省。''当时殊忤旨。问:``何以知无?''答曰:``潘岳秋兴赋叙曰:`余兼虎贲中郎将,寓直散骑之省'。''玄咨嗟称善。
\end{quote}

桓玄403年底当皇帝,一共过了80天的瘾,准确地说,80天是没有的,因为刘裕在404年2月开始就一直在追打他,桓玄一路逃。桓玄为人虚浮,性格急
躁,名士气足,在皇帝的考察期间就没有通过群众的评议,但其胆略、气度有过人之处,才华也突出,对文学、书法、周易都有一定的研究和造诣。
他
在打败仗的路上,最喜欢的事是坚持写《起居注》,叙述讨伐刘裕的事,自称自己英明神武,所使用的战略战术没有失策的地方,只是手下人违背自己的指挥调遣,所以才落得这样的境地。待《起居注》写完之后,便公开展示给手下人阅读,叫他们发扬批评与自我批评,激发他们的自惭之心,结果自然是众叛亲离。
改置直馆:直馆大概是指值班机构,改置直馆就是局部调整政府机构。
虎贲中郎省:虎贲中郎将的官署。桓玄要恢复虎贲中郎将,问左右虎贲中郎将应归口哪个部门,官署置于何处。《孟子
- 尽心下》:``武王
之伐殷也,革车三百两,虎贲三千人。''虎贲就是像老虎一样的猛士,后指守卫王宫、护卫君主的专职人员,一般选用功臣之子弟,贴身宿卫君王。
殊
忤旨:桓玄英明神武天下人都是知道的,一向都是交口称赞,他当陛下没几天,今天问你官署放在哪里,你回答说没有官署,这怎么行啊。当年我朝太祖在开国大典中规定礼炮放28响,尽管按礼仪21响为最多,但太祖说28下就是28下,你提醒他不就是找刺激吗呢?准确地回答当然是陛下说放哪里就是哪里,陛下放了以后我们再领会深刻圣意。这里的有人是指参军刘简之。
秋兴赋:潘郎的文章确实是好的,我有时候想,能写出这么优雅文章的人,这么会对王侯望尘而拜呢?他是那样忧伤,那样深沉,是文学王国中的国王,他有饭吃,有衣穿,何必倾倒于世间的权势?潘岳在秋兴赋叙中说:``晋十有四年,余年三十二,始见二
毛,以太尉掾兼虎贲中郎将,寓直散骑之省。高阁连云,阳景罕曜。仆野人也,猥厕朝列,譬犹池鱼笼鸟!有江湖山薮之思。于是染翰操纸,慨然而赋。于时秋至,
故以秋兴命篇。''说得很好啊,我的才能不能发挥,皇帝不知道我,我本来就是个山野之人,想回到江湖之中,``逍遥乎山川之阿,放旷乎人间之世''。潘郎啊潘郎,你貌是潘安,才是潘安,爱妻子是潘安,你是全好男人,怎么就掉进官缸中去了呢,最后横死在小人的手里。
寓直散骑之省:寄宿于散骑的署衙当值,散骑之皇帝的咨询顾问,散骑之省应该隶属于门下省吧。

\section{2.108}\label{section-154}

\begin{quote}
谢灵运好戴曲柄笠,孔隐士谓曰:``卿欲希心高远,何不能遗曲盖之貌''?谢答曰:``将不畏影者未能忘怀!''
\end{quote}

解释: 谢灵
运:才有一斗的谢客、谢康乐、谢公义出场了。说谢灵运才有一斗,是他的自评:古往今来,天下诗才,曹植分了八斗,我得一斗,其他的就只能在另外的一斗里匀匀了。谢灵运是谢玄的孙子,小时候就表现出一等聪明,以至于谢玄感慨说,我怎么就生了个笨孩子谢瑍,又怎么想得到谢瑍还能生出灵运来!
一般认
为,谢灵运是古代山水诗的最主要开拓者,诗歌格式一般是述行------写景------悟理,尽管句子漂亮,看多了、看快了我们自然像看自家的妻子,直接选择无视。不过我们拎出来几句,偶尔背背,心思就会远一些,``野旷沙岸净,天高秋月明'',``春晚绿野秀,岩高白云屯'',``异音同至听,殊响俱清越'',他的好诗据说是这样得来的:谢灵运常携众游山玩水,``常自始宁南山伐木开径,直至临海,从者数百人''。
说他隐居,其实还是在做太守这样的``小官'',从来不管政事,而是动用劳役建设装修自己的庄园,``穿池植援,种竹树堇,驱课公役,无复期度'',过着挥斥方遒、生活极其浮华的大地主生活,他有一篇著名的《山居赋》,说自己的庄园``阡陌纵横,塍埒交经。导渠引流,脉散沟并。蔚蔚丰秫,荚荚香粳。送夏蚤秀,迎秋晚成。兼有陵陆,麻麦粟菽。侯时觇节,递艺递熟'',他
的果园``北山二园,南山三苑。百果备列,乍近乍远'',``植物既载,动类亦繁'',鱼塘里的鱼我都读不出来,打不出字,只能搜索来拷贝,``鱿鳢鲋鱮,鳟鲩鲢鳊,鲂鲔鲨鳜,鲿鲤鲻鳣'',肉食甚至有``熊罴豺虎'',他还很谦虚,告诫自己和子孙,自给自足就可以了,不用交易,``供粒食与浆饮,谢工商与衡牧。生何待于多资,理取足于满腹''。是啊,你老人家都这样半官半隐的神仙中人了,还要妄图去做什么宰相?可是谢灵运回答非常闪烁,有时表现出牢骚满腹,``自谓才能宜参
权要'',``自以名辈,才能应参时政,初被召,便以此自许'',``在郡一周,称疾去职'',``出郭游行,或一日百六七十里,经旬不归,既无表闻,又不请急'',正是谢灵运这种过于现实的狂傲,最终因为一首莫名其妙的反诗,被宋文帝刘义隆赐死,终年49岁。
曲柄笠:大概是一种像帝王出巡时仪仗的缩小版,只是他是自己带着而不是别人举着,所以后面孔淳之讽刺谢灵运搞这种山寨版的东西,是不是没有忘掉富贵啊?
孔隐士:孔淳之,会稽孔家,在上虞隐居,``茅室蓬户,庭草芜径,唯床上有数卷书''。
希心:希就是仰,希心就是仰慕,希古就是仰慕古人。
将不:恐怕,表示测度而意思偏于肯定。
畏影者:害怕自己影子的人。《庄子》:``人有畏影恶迹而去之走者,举足愈数而迹愈多,走愈疾而影不离身,
自以为尚迟,疾走不休,绝力而死。不知处阴以休影,处静以息迹,愚亦甚矣。''一个人害怕自己的影子,想甩开它,就拼命逃跑,可是影子仍然跟着,结果气绝身死。这个人不知道在阴暗的地停下来,就不会有影子;静止不动,就不会有脚印,可悲呀,真是太愚蠢了!谢灵运是说,只有畏影者心里才有个影子,看来是你孔淳之不能忘怀于富贵,所以认为我的曲柄笠像帝王的曲盖。
谢灵运在这次辩论中获胜了,可是他真的胜了吗?红塔集团原董事长储时健出狱后种田去了,
他养了条狗,取名叫``放下'',朋友去看他,就老远听到储时健在喊:``放下!放下!''储时健真的放下了吗?我们每一个人,曾经放下过什么东西?往往就是放下
了自己的快乐,而把沙子放在自己的眼里,痛、伤害放在自己的心里。

\section{\texorpdfstring{``言语''小结}{言语小结}}

孔子说:``不学诗,无以言''。在这一章节中,主人公的话往往都有诗歌的特点,工整有韵律,含蓄藏典故,形象用比喻。有韵律则言辞
简洁、声音铿锵,有典故则相会于心、多有情趣,用比喻则意味深远、明灭不定。我们当然记得,``言语''中有个典故被两次使用,一是司马昱和桓温入朝,桓温使用了《诗经》中``无小无大,从公于迈'',来表明自己没有政治野心,愿意尊重皇室,这种偶尔的细节表现出来的心态,比当年曹操发布长篇政治宣言《让县自明本志令》要生动,更是瞬间真实思想的流露,更容易让人接受;二是孙盛的天才儿子孙齐庄跟随庾亮去打猎,解释自己是``无小无大,从公于迈'',一个七八岁的小孩
子在这么重大庄严的场合能应答便给,恰如其分,充分烘托了众人、甚至是孩子对庾亮的推崇,这正是孔子所谓的``诗可以兴,可以观,可以群,可以怨。迩之事父,远之事君,多识于鸟兽草木之名''。谢安召集子弟论文义,谢道韫从此以``咏絮才''名称后世,我们且不说柳絮之比喻的好坏,但这种把文学当做高雅之事的风气造就了晋人的潇洒高远;顾悦说``蒲柳之姿,望秋而落;松柏之质,经霜弥茂'',这样随口表现出来的才华也是今人很难企及的。古人倚马可待、七步成章正是建
立在他们长期的训练之中。
在``言语''中有大量辩论或者回答问题的场面,虽然庄子说辩论是没有意义的,但《庄子》本身就是无数辩论组成的,《世
说》保留下来大量的辩论技巧和方法给后人很大的影响,中国人后来的很多辩论故事都可以从``言语''中找到母典,乐广``岂以五男易一女''一语中的,因为其简洁明了,在``言语''中也两次被使用;孔融``想君小时,必当了了''以彼之道还施彼身,这两天我反复在听《刘三姐》,其中一个秀才唱道:``见你种田受奔波,长年四季打赤脚,不如嫁到莫家去,穿金戴银住楼阁。''
刘三姐毫不客气地回答:``你爱莫家钱财多,穿金戴银住楼阁;何不劝你亲妹子,嫁到莫家当小婆。''这不由让人想起前几天讨论的白毛女要不要嫁给黄世仁的争论,在女权主义的今天,居然还能出现这样的问题,实在是对现今中国所谓的``解放''和建设``和谐社会''极大的讽刺。无论建国后人们对曹操有多少形象的纠正和褒赞,但是孔融孩子的``岂见覆巢之下,复有完卵乎''和``如果死者有知,得见父母,岂非至愿''始终使我对曹操的``网目不疏''有巨大的憎恨和厌恶,在石头和鸡
蛋之间,我们应该始终站在鸡蛋的一边。在``言语''中,就保留了不少士人在大人物面前不卑不亢、针锋相对的斗智场面,这看似是自不量力和逞口舌之快,但作为有自我意识的``人'',为什么放弃抗争要让赢家通吃?``宁为兰摧玉折,不作萧敷艾荣'',我在祢衡的故事中就说过:``向那个浑浊不堪的社会和同样污浊的人群投出自己鄙夷的目光------我绝不和你们站在一起!''
当然,``言语''中的绝大多数故事并不是这样剑拔弩张,他们讲话多为委婉含蓄,意在言外常常反映魏晋人物的高雅情致,他们对山水进行歌咏,留恋在云兴霞蔚、日月清朗的大自然中,正所谓``何尝见明镜疲于屡照,清流惮于惠风''。总的来说,魏晋时
期是个非常混乱的时代,死亡的阴影始终笼罩在社会的上空,许多士人留恋山水、玄学,对生命的意义进行拷问,以自己的作风、自己的生活方式来选择自己的命运,这本身就是个人意识觉醒的特点,他们大多数人浮华的生活方式虽然可以讨论,但这种追求本身是很让人心仪的。``四顾何茫茫,东风摇百草'',``人生天地间,忽如远行客'',``良无磐石固,虚名复何益''。
卡斯帕罗夫见过:``思考一个故事为什么被讲述的原因,比我们从故事本身得到的要多得多。''
``言语''一章我整整一年才完成,有的写得还比较草率,不少网友还依旧关注一个不起眼的帖子,给我勉励,实在让我有点不好意思。嗯,一年时间,``天翻地覆慨而慷'',现实中三大战役已经打完,游戏中我三国也统一了N遍,希望今后我能``宜将剩勇追穷寇'',不辜负网友的情义。

\chapter{政事第三}

\section{3.1}\label{section-155}

\begin{quote}
陈仲弓为太丘长,时吏有诈称母病求假,事觉,收之,令吏杀焉。主簿请付狱考众奸,仲弓曰:``欺君不忠,病母不孝;不忠不孝,其罪莫大。考求众奸,岂复过此!''
\end{quote}

解释:
考众奸:查问小吏的其他罪状。这话很有意思,陈寔这样杀官吏在当时肯定是不合法的,那么长官一定要弄死他,作为副手的主簿就得使事情圆转以服众,觉得应该搜罗其他罪名。是啊,就像现在,你``越级''上访告市委书记建白宫搞腐败,就把你抓进去判个几年,上访是判不了刑的,那么公安、检察院就得深刻领会党委意见,罗织其他一些罪名,让社会舆论无话可说。
莫大:莫同漠,极大。莫大的讽刺。
以德行著称的陈寔在这里显得是那样杀气腾腾,令今人不寒而栗,与之相似的还有那个水滴石穿的故事,宋代张咏当年也是一样把一个贪污了一块钱的小吏当堂斩于剑下。在这种故事的背后,我们也许可以读出,长官
是不能欺骗的,权威是不容挑战的。即使是在封建社会,陈寔、张咏的做法在法理上同样是站不住脚的,但这样的行为却得到了社会甚至皇帝的肯定,全是因为托了
专制社会的福。
陈寔啊陈寔,你是正人君子,你可曾想过,你拿母亲的高帽杀了他的儿子,那个母亲是怎样的痛苦流涕,还得背上教育无方的道德谴责?不过我们还可以欣慰地想,也许这个小吏平时就罪恶多段,又极为忤逆,陈寔不过是借机为社会清除一害,这样我们的内心或许会平静一些。
自汉代以来,统治者实行的基本是``以孝治天下'',``孝''在我们的脑海里,是那样母慈子孝,脉脉温情,但是孝一旦诉诸政治活动,上升到国家行为,立刻被权力充分利用,异化为一种对统治者极为有利的武器,不
少人就死在``孝''的屠刀之下。于是孔融、嵇康就这样被曹操等人杀死了,鲁迅先生说:``但实在曹操、司马昭又何尝是著名的孝子,不过将这个名义,加罪于反对自己的人罢了。''古时候的满门抄斩,株连九族,也包含着统治者恐惧于死者的子女可能以``孝''为挡箭牌,以此进行血仇活动。
在唐以后,由于中央集权的加强,孝又被忠击败了,官员也可以由皇帝``夺情'',不用守孝三年。前些天国庆阅兵,大概是一个士兵,虽然父亲去世,但他没有抽出时间去奔丧,坚持队列训练,成为``忠''的典型,我们可以估计,他因此踩着父亲的尸体,获得一定的荣誉和上升空间。这令人恶心。

\section{3.2}\label{section-156}

\begin{quote}
陈仲弓为太丘长,有劫贼杀财主,主者捕之。未至发所,道闻民有在草不起子者,回车往治之。主簿曰:``贼大,宜先按讨。''仲弓曰:``盗杀财主,何如骨肉相残!''
\end{quote}

解释: 劫:名词,强盗。《世说 -
自新》陆机谓戴渊曰:``卿才如此,亦复作劫邪?'' 贼杀:抢劫杀害。
财主:物主。《唐律 -
疏议十九》:``既得阑遗(阑遗费解,大概就是指遗失)之物,财主来认。''
主者捕之:大概就是指有关部门正在追捕。主者,主事之人。古代机构极为简便,县令行政、司法统揽,身兼多职。因为这句话表述上存在缺陷,有人认为应该是``有劫贼杀财主者,捕之'',或者是``有贼劫杀财主者,捕之''。
发所:发案地点。
在草不起子者:``在草''就是``坐蓐'',巴金的《家》里面有个场景:女人生孩子是在偏屋,分娩时身下铺草席,所以临产叫``坐蓐'',晋代叫``在草''。``起
子''难理解,不过后汉书讲贾彪的故事,情节与陈寔的基本一样,我们可以应证,得出答案。《后汉书
-
党锢传》:``贾彪字伟节,补新息长。小民贫困,多不养子。彪严为其制,与杀人同罪。城南有盗劫害人者,北有妇人杀子者。彪出,案发,而掾吏欲引南。彪怒曰:`贼寇害人,此则常理;母子相残,逆天违道。'遂驱车北行,案验其罪。城南贼闻之,亦面缚自首。''所以``在草不起子''估计就是``在草弃子'',生子而丢弃孩子或杀害孩子。
贼大:抢劫杀人案重大。 按讨:查问追讨。
在中国的法律中,一般来说保护的是长者,杀父母或者打骂父母远比杀子女、打骂子女的惩罚要重得多。但这个故事有体现出了另一种法则,古人也认为,侵害亲属不仅伤害人伦,亦伤害了天伦、天理,亲属间相残的社会危害程度比普通人相残更为恶劣,因而要有特重刑罚以惩戒。但是这种法则随着中央集权的加强,陈寔这种执法思想逐渐向维护尊长者靠拢。不过在西方社会,许多国家的法律只是强调对亲属关系的维护,并不体现对尊长者的偏斜。在现代中国的法律体系中,已经把亲属间的犯罪混同于普通犯罪,并不体现出它的特殊性,判决结果常常是亲疏无别,一视同仁,甚至``以清官难断家务事''为理由,往往更轻于常人相犯,这是人类道德和社会理性的倒退。
余嘉熙认为,这个故事与贾彪的故事相同,应该是陈寔子孙美化其祖先的表达。不过后面紧接着的故事又传递出另一个信息:生活中完全可能出现类似的事例。

\section{3.3}\label{section-157}

\begin{quote}
陈元方年十一时,候袁公。袁公问曰:``贤家君在太丘,远近称之,何所履行?''元方曰:``老父在太丘,彊者绥之以德,弱者抚之以仁。恣其所安,久而益敬。''袁公曰:``孤往者尝为邺令,正行此事。不知卿家君法孤,孤法卿父?''元方曰:``周公、孔子,异世而出,周旋动静,万里如一。周公不师孔子,孔子亦不师周公。''
\end{quote}

解释:
陈元方:陈纪,陈寔的大儿子,见1.6等。难弟陈谌在1.7中表现过了,把父亲陈寔比喻为泰山,陈纪在这里把父
亲与周公旦或孔子比较,都表现出他们对父亲强烈的自信。这种自信比当今毛三要求把毛伟人的生卒日当成全国的假日更具有合法性,陈寔死后,``海内赴吊者三万余人,制蓑麻者以百数'';三陈在世时,据说已经``豫州百城,皆图书寔、纪、谌形像焉'',三陈虽不见赫赫之功,但比当时稍后的曹操、刘备等名角更让人尊敬。
袁公:据说是袁绍,因为在后文中袁公自称``孤'',曾封邺侯。但袁绍的岁数(202年卒)估计和陈纪(128-199年)还小20年左右【袁绍岁数估计和
曹操(155-220年)差不多,看《世说》的资料,他们年青时候一起玩。啊呀,这全是被西蜀兄逼的,花大量的笔墨解释年代】,年青时只当过濮阳令,未见资料说当过邺令。按《世说》的习惯,也未见称袁绍为袁公,不过袁家五世三公,也许是指袁绍的父辈袁逢、袁隗等。
何所履行:倒装句,所履行者何。履行:践行。
``彊者绥之以德''句:彊就是强。《吴子》:``绥之以道,理之以义,动之以礼,抚之以仁,此四德者,修之则兴,废之则衰。''对待豪强以德服人,对待弱势群体以仁爱安抚。陈寔对小偷说:``夫人不可不自勉。不善之人未必本恶,习以性成,遂至于此。''史书上说,陈寔当太丘令时
``修德清静,百姓以安'',`` 无讼'',``
道训天下''。陈寔后来因故辞官,声望却越来越大,以至于当朝三公认为自己抢了陈寔的位置。
恣其所安:激发人民的道德感,放手让人民做安心的事。谁没有羞耻感,愿意一心犯罪呢?现在有人对外来务工人员的犯罪问题言辞表现得很激烈,包括西方社会对华人进入他们的生活领域很感冒,他们看待问题,没有从``仁''的角度出发。没有互相理解。我始终认为,由于贫困,人就有突破法律、走出困境的权力。
邺:我们不会忘记西门豹治邺的故事吧,曹操也在此造铜雀台,今在河北南,不过风流尽去,盛时难再还。这次在河北一个星期,狼牙山开阔壮丽,隆兴寺沉着悠远;最可笑的是柏林禅寺,和尚故弄玄虚,口吐莲花,搞得同行的一大帮子党员拜服于泥塑土胚之下,令人哂之。
法:效仿。王安石:``天命不足畏,祖宗不足法,人言不足恤。''
``周旋动静''句:举止进退,虽然相隔遥远,如出一辙。周旋一般认为是交际应酬的意思,《孟子
- 尽心下》:``动容周旋中礼者,盛德之至也。''
孔子亦不师周公:师,学习。孔子当然是师周公的,孔子还经常在梦里与周公交流。但陈纪的意思是说,环境不同,不能单纯地认为孔子效法周公,只不过是他们同心罢了。你们两个都做得一样,同样不能单纯认为是师法周公、孔子,而是你们都有仁德之心。袁公在挑衅,而陈纪在抬高对方的同时,也毫不贬低父亲,这个故事也许更应该放在``言语''中,放在这里,作者也许是为了突出陈寔的行政能力,也许是解释上一则故事与贾彪雷同的原因。
《世说》的意义并不在于故事本身,而是读者接触《世说》文本所产生的体会之中。这个故事跳出了学习、效仿的意义,而是在于理解人的行为。

\section{3.4}\label{section-158}

\begin{quote}
贺太傅作吴郡,初不出门。吴中诸强族轻之,乃题府门云:``会稽鸡,不能啼。''贺闻,故出行,至门反顾,索笔足之曰:``不可啼,杀吴儿。''于是至诸屯邸,检校诸顾、陆役使官兵及藏逋亡,悉以事言上,罪者甚众。陆抗时为江陵都督,故下请孙皓,然后得释。
\end{quote}

贺太傅:贺邵字兴伯,山阴人,曾任吴国吴郡大守、中书令,领太子太傅等,后因直谏被孙皓杀害。贺邵的儿子是2.34中的``海内之秀''贺循,在西晋时候也当过中书令,领太子太傅。所以说,魏晋时会稽有孔魏虞谢四家,除此之外,贺家也是一个望族。
吴中诸强族:江苏的诸豪门大族。我们可以注意到,孙策占据江东,司马炎吞并东吴,司马睿衣冠南渡,摆在他们面前的重大问题就是得到江东地方豪强的支持。
这个问题其实在汉光武帝刘光那里就埋下了伏笔,东汉的政权就建立在豪强之上,豪强们大养门客、佃户、私兵,所谓``仕于家者,二世则主之,三世而君之'',严重影响了中央集权。孙策、孙权没有显赫的家世,在征服会稽时更是遭到了激烈的反抗,内乱不断,后来依靠江苏的顾陆朱张(``吴四姓,多出仕郡,郡吏常以干
数。'')击败了浙江的孔魏虞谢等。这次会稽郡的贺邵当上了首都的太守,江苏的手下败将当上了江苏的领导,豪强们根本不买账,侮辱贺邵。
会稽鸡:说贺邵是鸡,一则押韵,二则说贺邵胆小,三则说他上任以后没有立即发表重大施政纲领,``初不出门''。谁料到贺邵其实在衙内整理黑材料,准备烧一把大火。后来陆游写诗道:``场中啄粟树头栖,三唱明星出尚低。关路元无孟尝客,吴儿莫议会稽鸡。''我们会稽的鸡啊,吃的是粟,在枝头栖息,唱歌的时候星星还没有高挂天空。孟尝君如果没有门客学鸡叫,连关门都叩不开,那些江苏人啊,不要看不起会稽的鸡。
故出行:借故出门。 反顾:反即返即顾。 诸:之于。 屯邸:军屯的物资仓库。
检校:检查。
顾、陆:吴郡大家族以顾、陆为首,顾家就是顾雍家族,陆家就是陆逊家族,他们长期担任宰相,满朝的亲眷。虽然史书上说顾雍、陆逊都很谦虚谨慎,但贾府除了门口的石狮子,其他哪有干净的?当今两任的首辅,据说个人品行都不错,不过我们只要看看他们子女的职务,就知道``主席也是人嘛''。
役使官兵及藏逋亡:逋即亡,逃亡。奴役官兵和窝藏逃亡户口。王莽改革失败,东汉以来,豪强多``起坞壁''、``作营堑'',
挟藏户口,拥众自守,以为私附,有时候一个大族的附庸民可以达到数万人,这些人不服徭役,不交租税。据记载,陆逊死后,子陆抗代``领逊众'';抗死,五个儿子又``分领抗兵'',这样的情况十分普遍,已经成为不符合君主意愿的客观事实,所以贺邵就拿这个开刀。
逃亡。战乱之时,赋役繁重,贫民多逃亡到士族大家中藏匿,给他们做苦工,官府也不敢查处。
陆抗:陆逊之子,字幼节,与羊祜对抗的名将。
江陵都督:误,应为西陵都督。因为从《三国志》看,贺邵在孙休时期任吴郡太守了,孙皓时期入朝。
下:当时陆抗在湖北,孙皓在南京,故说``下''。 孙皓:误,应为前任孙休。
然后得释:陆抗亲自向孙休请罪,才使顾、陆等家的犯人获释。
孙休任命会稽人当首都吴郡的太守,本来就意味深长,可能已经对贺邵交代了整治户口的任务;即使没有交代,我新任的官员,你们竟敢挑衅,也必定大力打压,
以示支持。可是那些豪强子弟不看风色,吃几天牢饭自在情理之中。这个故事告诉我们,谁挑战新任领导的权威,谁就必定倒霉,古往今来,概莫能外。诸君谨记!
谨记!

\section{3.5}\label{section-159}

\begin{quote}
山公以器重朝望,年逾七十,犹知管时任。贵胜年少若和、裴、王之徒,并共言咏。有署阁柱曰:``阁东有大牛,和峤鞅,裴楷鞦,王济剔嬲不得休。''或云潘尼作之。
\end{quote}

解释:
山公:山涛山巨源,河南人,竹林七贤之一,48岁开始当官,因为和司马懿有点亲戚关系,又有名士的声誉,所以后来当官直至司徒,位在``三公''。他一生的主要成绩是选拔干部,在曹魏时就是吏部郎,晚年西晋初时更是长期担任吏部尚书,当时人认为其有识人之能。当年他在隐居的时候,就非常自信自己是天生的高官,曾经对妻子说:``忍饥寒,我后当作三公,但不知卿堪公夫人不耳!''山涛品德好,声誉高,城府深,处事敏锐,所以一直当官到77岁才退休(《礼记》中要求官员70岁退休)。以:因为;器重:像瑚琏一样的重器,指才能,或指气度,《晋书》中说,山涛``少有器量'';
朝望:望,有名,``望族'',在朝廷中有名望,司马炎评价山涛``以德素为朝之望'';
知管时任:主管当时的重任,应该是指山涛任右仆射,领吏部尚书,主持官吏选拔考核。
贵胜:即胜贵,权贵。《孔雀东南飞》``磐石方且厚,可以卒千年。蒲苇一时纫,便作旦夕间。卿当日胜贵,吾独向黄泉!''
和、裴、王:后文所指和峤、裴楷、王济诸人。和峤和长舆,见1.17;裴楷裴叔则,见1.18;王济王武之,见2.24。他们是少年才俊,颇得司马炎的欢心。他们是文学青年,山涛的下属,他们的提拔基本都来自山涛的举荐。山涛即为名士,在朝亦不能废吟啸,年轻人就捧场唱和。
潘尼:据《晋书》载,当做潘岳。潘岳少年成名,``才名冠世'',入职尚书省,但``为众所疾,遂栖迟十年郎'',
而山涛主持吏部,潘岳迟迟不得进,这样的人最见不得别人受宠,故写押韵歌谣讽刺。
大牛:在《世说 -
轻诋》中记载,刘表``有大牛重千斤,啖刍豆十倍于常牛,负重致远曾不若一羸牸。魏武入荆州,烹以飨士卒,于时莫不
称快'',这里大概是讽刺山涛``器重''是很重,但重而无用。不过也许不含贬义,在晋代牛地位比马高,是主要交通工具,据说王济就赢过一头``八百里駮''。牛代替马,大概是因为牛饲养比较简单省事,力气又大,也许更符合玄学名士的身份。
``和峤鞅''句:大意是说和峤在前面拉牛脖子上的皮带,裴楷在后面操控牛屁股上的皮带,王济在其中纠缠不休。这句话可能是讽刺三个年轻人常常逗山涛开心的行为,也可能是说这三个人其实引导着山涛的思想,构成朝廷的一个帮派。那么山涛究竟是那一党呢?他和司马氏的关系是不是水乳交融呢?从史料上看,他应该是代表比较低层的士族,对曹魏其实抱有留恋、对司马氏篡位保持中立或者退避的名士派,山涛在晋代一直比较被重用,也许是司马氏对曹魏旧臣的一种高级姿态,山涛推荐嵇康和他的儿子嵇绍,推荐阮籍的侄子阮咸,不能理解为念旧,而是是他政治抱负的选择。他赏识和峤等人,没有提拔潘岳,也许是因为看穿了潘岳少年轻狂和趋炎附势,不堪大用。
这个故事奇怪,因为在``政事''一章本身就应该是褒义的故事,而潘岳的歌谣怎么看都像是讽刺,这难道是说作者从反面来认可山涛任用得人?

\section{3.6}\label{section-160}

\begin{quote}
贾充初定律令,与羊祜共咨太傅郑冲。冲曰:``皋陶严明之旨,非仆暗懦所探。''羊曰:``上意欲令小加弘润。''冲乃粗下意。
\end{quote}

解释
贾充:字公闾,西晋的开国元勋,他父亲贾逵是曹魏的三朝元老,自己在魏时也深受重用,但他和司马家族关系极好,曾指挥手下杀死了曹髦,又是征伐东吴统一天下名义上的统帅。他晚年舆论评价不高,时人认为他媚上傲下,专权怙宠,当然更主要是他刺杀曹髦是无论如何都无法逃脱罪责的;他的女儿贾南风,更对西晋的灭亡又不可推卸的责任。贾充死的时候大臣们甚至认为他的谥号是``荒'',``外内从乱曰荒,好乐怠政曰荒,昏乱纪度曰荒''。
裴楷提出,司马炎``陛下所以未比德于尧舜者,但以贾充之徒尚在朝耳'',
河南尹、关内侯庾纯更是大骂贾充:``天下凶凶,由尔一人!
律令:《晋律》是部非常重要的法典,一般认为它``宽简、周备'',自《晋律》融儒、法一体后,中国古代法律的体例、宗旨和内容基本都在其框架下作细部的调整,其基本内容一直沿用在清代。
郑冲:字文和,河南开封人,魏时是司徒、太保,晋时是太傅。他``以儒雅为德,莅职无干局之誉'',``虽位阶台辅,而不预世事'',他当大官却不干事,主要是因为道德和学问高明,``耽玩经史,遂博究儒术及百家之言''。
皋陶:舜时的法官,据说制定了中国的第一部律法《狱典》,其主要思想应该是法律用以教化而不是惩罚,见于《尚书
- 大禹谟》``临下以
简,御众以宽;罚弗及嗣,赏延于世。宥过无大,刑故无小;罪疑惟轻,功疑惟重;与其杀不辜,宁失不经;好生之德,洽于民心,兹用不犯于有司''。郑冲认为这部《狱典》``严明'',严肃而公正。当然在传说中,皋陶养了头叫``獬豸''的怪兽,皋陶判案的时候,不用审问,獬豸出场即可,如果獬豸顶撞犯罪嫌疑人,就说明他有罪;如果獬豸不触碰犯罪嫌疑人,就说明他无罪。这种方法还算人道,我小时候看过一篇文章,大约是西方的中世纪,某个民族断案,把人绑上石头扔进河里,
能浮起来的是罪人,烧死;不浮起来是好人,好人灵魂进天堂。是啊,好人留个全尸,坏人挫骨扬灰。
小加弘润:一般理解就是``稍加补充润色'',但这样的回答毫无出彩和意义,那么弘润应该理解成``宽宏温和'',这样就衬托出郑冲的性格为人宽宏温和,为老百姓也做一点宽宏温和的事情。《世
说 - 品藻》:
阮思旷何如?曰:弘润通长。世说的注中也提到《晋律》``蠲除密网''。
下意:提意见,出主意。

\section{3.7}\label{section-161}

\begin{quote}
山司徒前后选,殆周遍百官,举无失才;凡所题目,皆如其言。唯用陆亮,是诏所用,与公意异,争之,不从。亮亦寻为贿败。
\end{quote}

解释: 山司徒:山涛。
前后选:山涛56岁时,魏国司马昭杀高贵乡公曹髦,第二年(261年)山涛出任魏国吏部郎,两年后转任其他职务。70岁时(274年)出任晋朝吏部尚书,而后长期负责吏部,因为有两次,所以叫前后选。
殆周遍百官:山涛在吏部选拔官员极为广泛和谨慎,``甄拔隐屈,搜访贤才'',``每一官缺,辄启拟数人,诏旨有所向,然後显奏''。他任用官员,往往加以自己的品评,说明任用的理由,这些文章汇编成为《山公启事》,当时人认为是``周遍内外,而并得其才''。
唯用陆亮:在司马炎一朝中,羊祜、山涛是一党,贾充是一党,陆亮是贾充的人。278年,吏部郎空缺,山涛推荐竹林七贤之一、阮籍的侄子阮咸,认为阮咸是``真素寡欲,深识清浊,万物不能移''的典型,是``妙绝于时''的``整风俗理人伦''的人选。但是贾充认为自己在朝中选用事``每不得其所欲'',于是推出了``公忠无私''的陆亮,以牵制山涛。后来山涛推荐阮咸,``三上帝弗能用'',因为司马炎认为阮咸``酖酒虚浮'',启用陆亮。这次斗争的结果是山涛辞去尚书左仆射和领吏部的职务,而陆亮不久被揭发受贿,司马炎用舆车把山涛请回。
这个故事是说山涛执政有识人之能。

\section{3.8}\label{section-162}

\begin{quote}
嵇康被诛后,山公举康子绍为秘书丞。绍咨公出处,公曰:``为君思之久矣。大地四时,犹有消息,而况人乎!''
\end{quote}

解释: 利用这个故事,我们说说山涛。
山涛的父亲当过县令,但去世很早,剩下孤儿寡母,于是山涛打小就要负责起家庭的社会交往,所以在他年幼的时候就显示出老成和沉静,以至于师长们都平等对待他,``宿士不敢慢'',而且正是因为没有了父亲,山涛还肩负着振兴家族的重任,年青时候就立下要成功,要出人头地的志向。人世间所谓的成功,一般就是官职或金钱。汉末没有考试,要当官只有关系和举荐。举荐是要名气的,所谓的名气也无非就是打造团队,借光造势。于是山涛组织了``竹林七贤''这个响当当的隐士团
体。现在当然认为``七贤''的代表是嵇康和阮籍,但在当时,山涛的年纪比他们要大十几、几十岁,为人又厚道有气度,自然是当仁不让的领袖。富贵人当隐士,一眼望去就是潇洒脱俗,当得也优哉游哉,穷人当隐士,无非就是取得富贵的一个途径。通过山涛的这般运作,他积累了一些名头,终于在40岁出任老家河内郡的主簿、功曹、上计椽,举孝廉,他能干,一年一个进步,照这个势头60岁估计也就实现``三公''的志向了。但这时候朝廷大有风云,曹爽和司马懿两个大佬严重对
立,马蹄纷沓,山涛一看风色不对,就不敢站队,辞官又去当隐士了。这首先使得作为亲戚的司马懿对他不满,在司马炎战胜曹爽之后,山涛的族人向他推荐山涛是栋梁之才时,以至于司马懿嘲笑说:``卿小族,哪得这令人快意之人!''在司马懿执政期间,山涛始终在隐居。其次祸福相依,在后来朝廷中的曹魏一派,始终对山涛抱有好感,山涛在后来的官员生涯中,多次担任与曹魏遗族沟通协调的职务,他处理得好,官职也就渐渐高大丰满起来。
杀死曹爽两年后,司马懿也跟着去了。年青时候的志向挠着47岁山涛的心脏,他再次走出竹林,腆着脸向表外甥司马师(山涛和司马师的母亲是中表亲)要官做。司马师开玩笑说:``吕尚欲仕乎?''80岁的姜太公在渭水垂钓是在等待机遇,而47的山涛是在争取机遇。司马师调侃完山涛后,给了他一个``秀才''的名头和从事中郎的职务。一番宦海起
伏,在山涛生命中的最后一年(79岁),司马炎终于授予山涛``司徒'',``三公''的理想终于达成,幸亏山涛寿命长啊。这时候山涛上辞表说:``臣闻德薄位高,力少任重,上有折足之凶,下有庙门之咎,愿陛下垂累世之恩,乞臣骸骨。''我有时候在想,像山涛这样的私房话,怎么会流传记载下来呢,是不是山涛其实想当``三公''而故意造舆论呢?这也未免太进取了吧。是的,大家一起学老庄,有人得到了豁达潇洒,有人得到了阴险狡诈,有人得到了神仙方术。在那年,他还做了另
外一件事,就是举荐嵇康的儿子嵇绍到杀父仇人的儿子那里当官,而这个儿子还真替司马家卖命,甚至为了保护司马昭的孙子晋惠帝而被叛军杀害。古人的心理还真不好懂,人的心理还真不好懂,正如《庄子》所说的那样:``凡人心险于山川,难于知天。天犹有春秋冬夏旦暮之期,人者厚貌深情。''
出处:《周 易 -
系辞上》:``君子之道,或出或处。''又叫用舍行藏,指出仕和隐退。山涛估计是私下举荐,嵇绍原不知情。因为嵇康被杀害时曾把其家人托付于山涛照顾,所以嵇绍上门请教。其实这有什么好请教的,父仇不共戴天,而且嵇康是被枉杀的,报不了已令子女人痛心疾首,夙夜悲愤,哪有还到仇家那里效命的!估计嵇家那时候实在是穷,贫穷使人堕落。
``大地四时''句:《易 - 丰卦 -
彖》:``日中则昊,月盈则食。天地盈虚,与时消息。而况于人乎!况于鬼神乎!''消息指生灭消长,这句话的意思是说,天地间的一年四季,也还有交替变化的时候,世界万物也都在变化,你就不必拘泥于恩仇。山涛学老庄、学周易,不见他有什么论述,倒得到了其中的无原则,还用它劝诫晚辈,正是``卑鄙是卑鄙者的通行证'',晋代的道德沦丧可见一斑。我同事有时候品评领导,我从来不置一词,乌鸦也许有白的,高官里难道还有善人?山涛绝非善类。

\section{3.9}\label{section-163}

\begin{quote}
王安期为东海郡,小吏盗池中鱼,纲纪推之。王曰:``文王之囿,与众共之。池鱼复何足惜!''
\end{quote}

解释:
王安期:王承王安期,太原王氏,蓝田侯,前面所说对山涛纠缠不休、多次挑衅别人的王济的堂兄弟,他儿子王述更是王羲之的仇人。王承被认为是东晋初名士第一,``众咸亲爱焉,渡江名臣王导、卫玠、周顗、庾亮之徒皆出其下''。他当时是东海郡内史,相当于太守。
池中鱼:官府池塘的鱼,按古代的规定,属于王承的私产。不向领导家里送东西,反而偷领导家里的东西,事情虽小,可也忒不懂规矩。
纲纪:综理府事者也。魏晋时对公府、州、郡中高级属吏称呼,如功曹、五官掾、主簿等。
推:用词费解,或当作``究'',追查。 文王之囿:周文王的园林。《孟子 -
梁惠王下》:``文王之囿,方七十里,刍荛者往焉,雉兔者往焉,与民同之。民以为小,不亦宜乎?臣始至于境,问国之大禁,然后敢入。臣闻郊关之内有囿方四十里,杀其麋鹿者,如杀人之罪,则是方四十里为阱于国中。民以为大,不亦宜乎?''孟子是说,周文王的园林70里,虽然大,但不禁老百姓砍柴打猎;你齐宣王的园林40里,杀你一头鹿就同杀人一样的罪,简直就是超大的陷阱
王承清廉寡欲,以至于他死后家中没有财产,儿子王述虽然袭爵蓝田侯,但也向王导讨县令当,上台后疯狂受贿,以补贴家用。

\section{3.10}\label{section-164}

\begin{quote}
王安期作东海郡,吏录一犯夜人来。王问:``何处来?''云:``从师家受书还,不觉日晚。''王曰:``鞭挞宁越以立威名,恐非致理之本。''使吏送令归家。
\end{quote}

录:拘捕。
犯夜人:违反宵禁的人。一般来说,古代的城市在晚上都是要实行宵禁的,昼刻(白天结束)之后,黎明之前,除特殊情节人员和特定节日,禁止城市人员通行,违者肉刑。所以一个元宵节,大街通亮,搞得城市里的人兴奋非常,``东风夜放花千树,更吹落、星如雨''。那么有人会问,逛妓院怎么办?那就白天去嘛,古代没有讲钟点这一说,都是包夜。再说了,正因为要求严格,执行者就可以谋私利饱私囊。豆腐都凭票,副食品店的人日子就好过了;办事要身份证,我一张卡片收你
20元工本费、60元加急费,你还得看我脸色! 受书:接受教育。束发受书。
宁越:刻苦的读书人。战国时人,传说他怕种田劳苦,发愤读书。别人休息他不休息,别人睡觉他不睡觉。学了十五年终于学成,成为诸侯周威公的老师。
致理:``理''或当作``治'', 致理:致治。
这两则故事似乎是说王承执政时不按法律办事而随心所欲地搞德治、人治。但法律本身就有经与权的关系,``情有万殊,事有万变,法岂能尽情、人之事哉?''。
经就是法律规定好的准则,权就是灵活变通。一般而言,古代社会也要求宁屈道德而不枉法律,但古人同时认为,``圣达节,次守节,下失节''(《左传》),孟子认为:``常谓之经,变谓之权。怀其常道而挟其变,乃得为贤。''就是说在尊重法律、执行法律的大背景下,对于具体案例不能过分拘泥,而是要处理好合法性和合理性的关系,处理好普遍正义和具体正义的关系,可以做出新的选择,藉以完善法律。在法官认为``刑名不可威,律法不可用''的时候,他可以采取另一种话语系统
为他的合情合理的判决作出辩护,而在中国古代社会,这套语言系统必须是不在法律语言中的儒家学说,以儒家的经义为标准,引礼入法,德法并重。王承对偷盗行为和违反宵禁的人不做出处理,所依据的就是儒家经义,并得到了社会的赞赏。但是这种权变在古代社会并不构成案例引用,它往往不能重复。
现代中国社会,由于社会管理层已经不需要从儒家经典获得话语权,就过于灵活,``权''已经被大大扩大化了,许多群众不走司法途径解决问题,而走信访途径``小闹小解决、大闹大解决'',成为司法工作者非常无奈的事。譬如邓玉娇案,法院屈服于舆论压力,做出了既违背普遍正义、又违背具体正义的裁决,这种裁决从根本上讲,
它更加否定了政府的执政合法性。近日在海地牺牲的8名工作人员,按有关规定只能算是``因公牺牲'',但中央立刻做出了``烈士''的判断,这只能说明``权''已经大大损害了中国的现代化进程。

\section{3.11}\label{section-165}

\begin{quote}
成帝在石头,任让在帝前戮侍中钟雅、右卫将军刘超。帝泣曰:``还我侍中!''让不奉诏,遂斩超、雅。事平之后,陶公与让有旧,欲宥之。许柳儿思妣者至佳,诸公欲全之。若全思妣,则不得不为陶全让,于是欲并宥之。事奏,帝曰:``让是杀我侍中者,不可宥!''诸公以少主不可违,并斩二人。
\end{quote}

解释:
东晋初司马家的寿命都不长,开国君主晋元帝司马睿47岁,儿子晋明帝司马绍27岁,孙子晋成帝司马衍22岁,司马衍的兄弟晋康帝司马岳23岁,司马岳的儿子晋穆帝司马聃19岁,而后又是司马衍的儿子晋哀帝司马丕25岁。如此低下的平均寿命,大约就是服药纵欲、缺乏锻炼的问题。而且这些人当皇帝基本都在童年,大族执政,估计压力很大,加速了他们的生命消耗。
晋成帝司马衍7岁的时候,发生了苏峻兵变。起因是司马衍舅舅庾亮想加强中央集权,召回历阳内史、流民统帅苏峻,夺其兵权。苏峻和另一个流民统帅祖约(祖逖的弟弟)起兵攻击庾亮,庾亮兵败投奔陶侃,司马衍被苏峻安置南京郊外的石头城。这时候侍
中钟雅、右卫将军刘超想把成帝救出,事泄,苏部大司马任让入宫杀了钟、刘。最后东晋依靠陶侃、郗鉴等寒族和流民帅击败苏峻诸人。苏峻、祖约之乱影响比较深远,此后东晋重整户籍,流民归土,70年内士族内争有所收敛,不敢轻动干戈。
陶公:陶侃。
许柳:祖逖的妻弟,淮南太守。苏峻攻陷建康后,任许柳为丹阳尹。苏峻失败后许柳被杀。
全:保全。
少主不可违:主上越是年幼,他的命令越不好违背。在门阀制度之下,东晋的皇帝当得不怎么随心所欲,但我们也能从那些材料中看到,即使是幼帝的话,大臣们在表面上还是非常尊敬,这可能主要是那时的人有君臣名分和历史敬畏感。当年贾充如此权倾朝野,庾纯在酒会上大骂:``贾充你是乱臣,高贵乡公现在在哪里!''
即使庾纯这样揭司马家的老底,在处分庾纯的时候,司马炎也不敢拿这事做理由。

\section{3.12}\label{section-166}

\begin{quote}
王丞相拜扬州,宾客数百人并加霑接,人人有悦色。唯有临海一客姓任及数胡人为未洽。公因便还,到过任边云:``君出,临海便无复人。''任大喜悦。因过胡人前,弹指云:``兰阇,兰阇!''群胡同笑,四坐并欢。
\end{quote}

解释:
王丞相:王导。东晋建立,王导旋即任右将军、扬州刺史、监江南诸军事。
霑接:接待。主人有招呼好每个客人的义务,这个故事体现出王导善于调和,待人接物八面玲珑的特点,也因为他有这个特点,所以在他的执政期间,有时候也讲南方话,以此消减北人和南人隔阂。能和任何人说上话,是一种很好的本领,苏东坡说自己``上可陪玉皇大帝,下可以陪卑田院乞儿'',有时候客人比较拘谨,苏轼甚至说,那你就随便讲讲鬼故事吧,你姑妄言之,我姑妄听之,于是宾主皆欢。这样善于引入话题的人才,现在可能应该在接待办或者信访局工作吧。不过后来的王士禛却认为苏轼其实是``料应厌作人间语,爱听秋坟鬼唱诗''。
临海:临海郡,现浙江台州、温州、丽水一带,隶属扬州。 因:于是。
便还:便就是旋。 到过:到疑为衍字。
弹指:搓指,古印度的一种风俗,用以表示欢喜、赞叹等含义。隋朝西域和尚吉藏解释《法华经》说:``弹指者,表觉悟众生。''现在有些魔术师还使用弹指来表明魔术场面开始,这大概是古印度的传统遗韵吧。弹指在佛教中意义还有很多,不赘述。
兰阇:西域语。现在外国人到中国来演出,总要说几句音调不准的中国话``你好''、``谢谢''之类的,为讨观众欢心,可能还得来一两首中文歌曲;中国人接待外国客人,也会来一两句外语,以烘托气氛,拉近距离。王导的外语大约也音调不准,所以胡人大笑,彼此亲近了不少。因为兰阇的意思不好懂,学者多有论述,似无准确答案,我想总不外是问候感谢之类的吧。

\section{3.13}\label{section-167}

\begin{quote}
陆太尉诣王丞相咨事,过后辄翻异。王公怪其如此。后以问陆,陆曰:``公长民短,临时不知所言,既后觉其不可耳。''
\end{quote}

解释:
陆太尉:陆玩陆士瑶,江南陆家。估计古代人玩品高雅,喜欢玉,故字瑶,他和陆机、陆云是堂兄弟,``士''字辈。去年我省有个局长,喜欢玩玉,有人送他一个玉佛,据说价值60万,他供与家宅,日夜上香,保佑他升官发财、遍览春色。后事发自首,上交佛祖,纪委一查,竟然是几千元的工艺品,其刑期大大减短。去年鄙省纪委书记、监狱局长几乎同时双规,真是所谓反腐警示教育的美谈。陆玩后来是司空、太尉,当时他为尚书左仆射(相当于现在的部长助理),而王导为录尚书事,总揽朝政,所以陆玩是王导的下级,常需请示。
翻异:这种做法不可思议。这似乎有两个可能,一是在常委会这样的大庭广众之下,陆玩维护王导的权威,因为在《晋书》中有个故事,``导每发言,一坐莫不赞美'',只有王述当面说:``人非尧舜,何得每事尽善!'',于是王导``改容谢之'',可见平时工作中除了王述外没人提出不同意见。二是陆玩压根儿看不起王导,在《晋书》中有个故事:时王导初至江左,思结人情,请婚于陆家。陆玩却说:``培塿无松柏,薰莸不同器(小土堆不长大树木,香草和杂草不放在同一个器物里)。玩虽不才,义不能为乱伦之始。''陆玩有一次拜访王导,在他那里吃了点奶酪,结果生了病。陆玩就给王导写了个便条:``仆虽吴人,几为伧鬼(北方鬼佬)。''当时南方人和北方人有国仇家恨,现在你北方人打了败仗逃到南方,还指望我陆家给你看好脸色?
公长民短:长短即尊卑。当时王导兼任扬州刺史,陆玩是扬州吴郡人,所以说自己是王导的治下之民。我请示的时候你是尊长,我怕你怕得不敢发表意见,可是过后我觉你说的不对,所以按照我的意见办了,(你拿我怎么样?)
这个故事大概是表扬陆玩的直耿和王导的宽容,但我们可以从中体会到东晋初南方士族与北方士族的冲突,所谓和谐总是要一方做出一定的让步,现在甚嚣尘上的``双赢'',无非是麻醉一下自我罢了。

\section{3.14}\label{section-168}

\begin{quote}
丞相尝夏月至石头看庾公。庾公正料事,丞相云:``暑,可小简之。''庾公曰:``公之遗事,天下亦未以为允!''
\end{quote}

解释: 前面介
绍过,晋明帝司马绍在位不过4年,当时王导辅政,司马绍病死,即位的司马衍是个孩子,按照规矩这时候王导不能单独执政,而是与司马衍的舅舅庾亮一起辅政,皇太后庾文君摄政。在这种二比一的情况下,王导选择了退让。庾亮和王导的执政风格不太一样,比较进取,当年明帝病重,司马宗族封锁宫门,就是庾亮直入宫中痛陈厉害,然后打压司马宗族,维持门阀制度;而后处理流民问题,进行北伐,都体现了他事功的决心。这个故事大概发生在他想处理流民统帅的时候,因为他在南
京郊外建石头城,就是防备流民帅陶侃,他也看到另一个流民帅苏峻恃功骄恣,藏纳见逼于庾亮的司马宗室诸王,打算把苏峻召回任大司农,架空他的兵权,这时候想必公事十分繁忙。
小简:稍微简单一点。据说王导执政比较宽松,譬如前面我们提到王述当县令后疯狂贪贿,王导也只是轻描淡写地说了一句:``这样不太好吧?''王述倒一点不羞愧:``我捞够了就不贪污受贿了!''后来王述照样升官,直到``三公''。
在世说中我们可以看到他平时大量的时间花在清谈上面,特别在他的晚年,公事基本就是画画圈罢了。从积极方面理解,我们可以认为王导试图以宽松怀柔的治理方针获取和维护东晋初稳定的局面,但这种无原则的做法还有其学术来源,老子说``治大国如烹小鲜'',庄子``齐万物''、``无是非'',孔子说:``无可无不可'',《维摩诘经》中说:``如我意者,于一切法无言、无说、无示、无识,离诸问答,是为入不二法门'',王导的执政思路就是以这样的理论为支撑。
遗事:弃置不顾世事。
《世说》在``政事''中似乎对事功进取和无为而治都抱有肯定的态度,是啊,在生活中,我们任何行动也都能找到自圆其说的理由,正所谓``此亦一是非,彼亦一是非'',我也经常在想,什么是正确的生活道路?在魏晋的时候有一个比较公认的答案,就是适性,最好的选择是放弃认识活动,放弃了对是非善恶的认识和执着。
当时最杰出的作者之一的郭象就提出:``人皆以天为父,故昼夜之变、寒暑之节,犹不敢恶,随天安之,况乎卓尔独化,至于玄冥之境,又安得而不任之哉?''谁去指责天的残暴和宽仁?天犹如此,除人类以外,世间万物都自生自化,那么人就要按照天道,只要不违性,无论怎样做都可以。

\section{3.15}\label{section-169}

\begin{quote}
丞相末年,略不复省事,正封箓诺之。自叹曰:``人言我愦愦,后人当思此愦愦。''
\end{quote}

解释: 末年:王导(276-339),他
拜丞相更是在338年,但世说的称谓并不根据实录,而往往已该人最后的官职或最出名的官职为限,所以末年一般理解可以是330年以后吧。那时王敦之乱结束,而后晋明帝去世,庾亮与王导辅政司马衍。王导早年执政就是抓住关键一以贯之,``务存大纲,不拘细目'',``为政宽和得众,事从简易'',到这个时候更是锐气尽销,知雄守雌,当年``戮力王室,克复神州''的口号或者理想早被东晋初反反复复的内乱和阀争磨灭。
略:两种理解似乎都可以,大致,差不多、基本上或者简略。
省事:省是看的意思,引申为管、顾、处理。 正:止,只,仅仅。《世说新语 -
文学》:阮乃叹曰:``非但能言人不可得,正索解人亦不可得!''《世说新语 -
规箴》:殷覬病困,看人政见半面。
封箓诺之:在公文上写同意。封:上奏的文书,《汉书 -
魏相传》:``故事,诸上书者,皆为二封,署其一曰`副',领尚书者,先发副
封,所言不善,屏去不奏,相复因许伯,自去副封,以防雍蔽。''箓:下发的命令。一般指古代帝王自称其受命于天的文书,张衡《东京赋》:``高祖膺箓受图,顺天行诛。''在明代进一步确定格式,基本就是``奉天承运,皇帝诏曰''。``箓''在《世说》的个别版本中作``篆'',因为古代官员的印章多为篆书,也可代指命令。
王导这样做当然有深的用心,不过即使没有含义,自魏以来一直有这个风气,官员以清静为名,不屑公事,``魏正始及晋之中朝,时俗尚于玄虚,贵为放诞。尚书丞郎以上,簿领文案,不复经怀,皆成于令史。逮乎江左,此道弥扇'',南朝时则更甚,官僚们只署文牍不问政务,所谓``望白署空,是称清贵;恪勤匪解,终滞鄙俗''(《梁书
-
列传31》)。清谈没落后,既然当上了高级官僚,一般处理公务照呈照转也不失为荣名贵身之道。北宋有个宰相王珪,``以
其上殿进呈,曰取圣旨;上可否讫,云领圣旨,退谕禀事者,曰已得圣旨也'',照样深得帝心。清代有句歌谣说:``大臣经济在从容,莫显奇功,莫说精忠。万般人事要朦胧,驳也无庸,议也无庸。''长期在多种衙门辗转,我也渐渐发现,即使避开王导期望内部秩序相对稳定融洽的具体情况不说,他的这种执政方式也有可取之道。现在当领导的大多喜欢来事,无事生非,真把自己当成红日北斗,英明神武,今天一个想法,明天一个花样,结果往往是鸡飞狗跳,徒劳无功,为害匪浅。
愦愦:糊涂。班固《咏史》:``百男何愦愦,不如一緹縈。''王导以愦愦为自傲,这即是唐代宗的``不痴不聋,做不得阿家翁'',苏轼``人皆养子望聪明,我被聪明误一生'',曾国藩写心得体会:``余昔在军营不妄保举,不乱用钱,是以人心不附,至今以为诟病'',``弟当以我为戒,一味浑厚,绝不发露'',郑板桥总结为
``难得糊涂''。

\section{3.16}\label{section-170}

\begin{quote}
陶公性检厉,勤干事。作荆州时,敕船官悉录锯木屑,不限多少,咸不解此意。后正会,值积雪始晴,听事前除雪后犹湿,于是悉用木屑覆之,都无所妨。官用竹,皆令录厚头,积之如山。后桓宣武伐蜀装船,悉以作钉。又云,尝发所在竹篙,有一官长连根取之,仍当足,乃超两阶用之。
\end{quote}

检厉:费解,疑误,或当做``俭吝'',《世说新语 -
俭啬》中作``陶性俭吝''。当然我们也可以强作解人,把检理解成全面,厉理解成严肃,就是《晋书》中说陶侃``综理微密''。
作荆州:325年,晋明帝司马绍启用陶侃为都督荆、湘、雍、梁四州军事、荆州刺史,以寒族而登高位要害之津,牵制门阀。陶侃``终日敛膝危坐,阃外多事,千绪万端,罔有遗漏。远近书疏,莫不手答,笔翰如流,未尝壅滞。引接疏远,门无停客''。不过当年司马绍去世,陶侃成了庾亮的眼中钉。
录:收集。
正会:新春团拜会,正月初一长官大会群僚听事前除雪后犹湿:《晋书 -
陶侃传》作``听事前余雪犹湿'',干净利落,不然我们又得解释除是台阶的意思。听事,办公楼。
都无:一点也没有。见(2.47)``陶公疾笃,都无献替之言''等。
厚头:竹子的最下面几节往往很短,又突然增大,一般的建造工作大概派不上用处。
桓宣武:345年桓温任荆州刺史,346年伐蜀。
又云,尝发所在竹篙:疑为``又尝发所在竹篙''。第三个故事表述存在问题,所以在我们小时候的教育课本上就把它给去掉了。我猜想,陶侃征收竹篙,有个下级费了大力气把竹鞭也挖出来上缴,竹鞭坚硬,刚好可以省去竹篙下部打铁尖,陶侃就提拔了他两级。
在两晋时期,我们可以发现士人动不动就是``独步、无双、第一'',但考其德功言行,多名不副实,而陶侃起于寒庶,德性贞固,远志图南,勇于任事,功业彪炳,说他是``第一''并不为过。当时就有人说他``机神明鉴似魏武,忠顺勤劳似孔明,陆抗诸人不能及也''。他在东晋初始终受到门阀猜忌,便给他编造故事:``陶侃梦生八翼,飞而上天,见天门九重,已登其八,唯一门不得入。阍者以杖击之,因坠地,折其左翼。''王导、庾亮之流以``阍者''自诩,诋毁陶侃觊觎帝位,而以
当时的形势论,以陶侃的明智不可能妄图推翻东晋。他告老还乡的时候,``军资器仗牛马舟船皆有定簿,封印仓库,自加管钥以付王愆期(人名),然后登舟'',陶
侃是当时的光亮。 后来,这个故事中
``竹头木屑''的典故成为我们童年时常接触的勤俭节约,变废为宝的教材。竹头木屑还有一个意思,就是告诉我们平时要及时收集资料,以备他日之用,一旦错过就再难找寻了。但从根本上说,陶侃当时收集竹头木屑,其原意未必局限于此,而是如李斯所说:``泰山不让土壤,故能成其大;河海不择细流,故能就其深。''

\section{3.17}\label{section-171}

\begin{quote}
何骠骑作会稽,虞存弟謇作郡主簿,以何见客劳损,欲断常客,使家人节量,择可通者。作白事成,以见存。存时为何上佐,正与謇共食,语云:``白事甚好,待我食毕作教。''食竟,取笔题白事后云:``若得门亭长如郭林宗者,当如所白。汝何处得此人!''謇于是止。
\end{quote}

何骠骑:何充,见2.54,当时任会稽内史,晋时诸侯国设内史,相当于太守。
虞存:虞家是会稽四族之一,《世说 -
赏誉》中说:``会稽孔沈、魏顗、虞球、虞存、谢奉,并是四族之俊,于时之杰。''
使家人节量:前面说过,魏晋时期的官僚都比较好客,连杀幕僚不手软的曹操也说:``周公吐哺,天下归心''。何充也很好客,虞謇作为何充的大管家(主簿),认为何充这样不分三六九等接见客人,劳心伤神,没有必要,而且当时舆论认为``充所昵庸杂,以此损名'',门下应该有筛选地通报。
白事:这里指报告这一建议的文书。
存时为何上佐:虞存、虞謇当时都在何充手下当官,虞謇是主簿,虞存是治中,治中职位次于内史,高于主簿,所以虞謇先要把报告给哥哥虞存批签。
正与謇共食:何充正在和虞存一起吃饭,所以何充直接把报告拿了过来批签,跳过了虞存这一关。这段``白事甚好,待我食毕作教''的细节似乎没有必要,大概作者想表述何充是经过考虑才作答的。说到吃饭,前些天我看过一段资料,说先秦古人是吃两餐的,后来逐渐改为三餐。
门亭长:晋《职官志》中说:州有主簿、门亭长等。郡有主簿,不设门亭长,而别有门下及门下吏。门亭长大约相当于现在的接待办,负责交际通禀等事宜。
郭林宗:1.3中的东汉末名人郭泰,以德性高超、善于发现和培养人才著称。
何充的意思是,找不到像郭泰那样一眼就能识别人才的人,所以我就必须礼贤下士,通过交往选拔人才。事实上,所谓会稽四族之俊,都是何充提拔的。

\section{3.18}\label{section-172}

\begin{quote}
王、刘与林公共看何骠骑,骠骑看文书,不顾之。王谓何曰:``我今故与林公来相看,望卿摆拨常务,应对玄言,那得方低头看此邪!''何曰:``我不看此,卿等何以得存!''诸人以为佳。
\end{quote}

王、刘与林公:顾镜自怜的王濛、天之自高的刘惔、放鹤养马的支道林。
望卿摆拨常务:当时何充已经为宰执政,王濛称``卿''不很礼貌,也许故而何充反击。摆拨:摆脱。陆游《晓登千峰榭》:``只道文书无了日,也能摆拨上层台。''常务:日常事务。
方:还。
晋人以见识高明、勇于探求生命的意义为首要,而把处理具体的日常事务认为是低下的。嵇康说:``素不便书,又不喜作书,而人间多事,堆案盈机,不相酬答,则犯教伤义,欲自勉强,则不能久,四不堪也。''明代袁宏道也说:``弟作令,备极丑态,不可名状。大约遇上官则奴,候过客则妓,治钱谷则仓老人,谕百姓则保山婆,一日之间,百暖百寒,乍阴乍阳,人间恶趣,令一身尝尽矣。苦哉!毒哉!''何充不是不能谈玄,从其他故事看他还是聊天的一把好手,而且``笃信经典'',
佞佛过分,是佛教史上著名的``护法''之士,但当时何充执政,``有万夫之望'',必须在政事花极大地心思。与这个故事类似的是《世说
- 排 调》中桓温说的:``我若不为此,卿辈亦那得坐谈''。
王濛等认为何充回答的很好,即使是这样,当时舆论评价是``何充于王濛、刘惔好尚不同,由此见讥于世'',见讥于世的不是王、刘等,而是何充。``你忙是忙,但你干的不过是蝇营狗苟的小事,转眼即逝;我们闲是闲,但我们整天记挂在心上的是天地间的道
理。''也许有人可以反驳,你们整日无所事事的生活不是建立在另一帮子人的辛勤工作之上吗?不是因为他们的保护吗?但从较高的境界来讲,个体的生还是死并不是问题,南宋末的无学祖元对着元兵的刀剑神色不变,诵偈说:``乾坤无地卓孤筇,喜得人空法亦空。珍重大元三尺剑,电光影里斩春风。''

\section{3.19}\label{section-173}

\begin{quote}
桓公在荆州,全欲以德被江、汉,耻以威刑肃物。令史受杖,正从朱衣上过。桓式年少,从外来,云:``向从阁下过,见令史受杖,上捎云根,下拂地足。''意讥不著。桓公云:``我犹患其重。''
\end{quote}

解释: 全欲:都欲,一心想。
威刑肃物:以严刑使人恭顺。袁世凯曾经对张之洞说:练兵之事,看起来似乎很复杂,其实很简单,主要是练成绝对服从命令。我们一手拿着钱,一手拿着刀,服从命令就有官有钱,不从就吃一刀。
正:只。
朱衣:古代官员衣服的颜色有级别,唐代中高级官员才能着朱衣,令史大约是中级官员。
桓式:桓温的三儿子桓歆,小字式。
``上捎云根''句:杖举得高入云霄,可打下来只是拂过地面。巧用对仗句。
这个故事是说桓温在当荆州刺史的时候以宽政收买人心,我们想想可能是有这么几层原因,一是晋代道教盛行,晋人崇尚老庄学说,当年张鲁在汉中就是就是搞忏悔来惩戒下属,一些人执政讲雅量,讲无为而治,不大对下属行霹雳手段。二是桓温府下往往都是门阀子弟,一大堆名士都在他那里历练,犯了错桓温也不愿激起门阀冲突,只是小施惩戒,甚至不施惩戒,不伤和气,桓温这样的故事在《世说》中不少。三是桓温在打造独立小王国,不按朝廷法令,结果是荆州人只知桓温,不知皇帝,``八州士众资调,殆不为国家所用''。
当然,在中古时期,朝廷讲的是刑不上大夫,士大夫不接受肉刑,杜甫有诗说:``脱身簿尉中,始与捶楚辞。''就是只要官当到簿尉以上,就不加捶楚了。这种局面只有到了仇官情结严重的朱太祖手中才发生重大转变,明朝皇帝可以用廷杖对付大臣,但在州府还是不能
对有功名的人使用肉刑。即使在口号``人民是主人''的现在中国,官员依旧是特权阶级,有一次我学习反腐内部讲话,大约是尉健行还是吴官正,其中有句话印象很深,大意是说我们党培养一个干部不容易,你只要现在把钱吐出来(廉政账户),无论曾经你贪污受贿多少,事情就算过去了。现在搞什么领导问责制,即使出了什
么事,罢官所丧失的只是某种官位的行使职权,而个人身份权利则不丧失。

\section{3.20}\label{section-174}

\begin{quote}
简文为相,事动经年,然后得过。桓公甚患其迟,常加劝勉。太宗曰:``一日万机,那得速!''
\end{quote}

解释: 简文:简文帝司马昱
366年任丞相,371年桓温废晋废帝司马奕,立司马昱。司马昱的谥号是简文,庙号是太宗。``壹德不解曰简,平易不疵曰简;道德博闻曰文,学勤好问曰文。''本来按道理司马昱是东晋的第八位皇帝,庙号是不能用太宗的,但因为司马奕被废,其中的谱系断了,他是东晋开国皇帝司马睿的儿子,所以另起庙号。
``事动经年''句:动辄年把才有批复下来。这样的行政不作为的确不可思议,想必他手里的公文又不全是搞车改、大小非解禁、稳定房地产这种大事,犯得着这样吗?不过我们想想,他老人家连凳子上的灰尘也不要人抹,平时以``拱默守道'',游山玩水,估计就是``自由化''理论的拥趸。我们也可以往险恶里想,当时桓温平蜀、北伐,官至大司马,党羽满朝、权势滔天,司马昱在玩弄权术,不得不韬晦。不过谢安有句评语可能是真相:``帝虽神识恬畅,而无济世大略,故谢安称为惠帝之流,清谈差胜耳。''------你书读得多,口才好,端的是是好华彩,但你不做事,做不了事,就和惠帝那样的白痴没有什么区别!
一日万机:小时候我以为周总理日理万机,就是一天要办很多很多大事、要事,真是为全国人民操透了心。后来读大学了,才发现``机''不是重要的意思,而是微小的意思。
这个故事想表述司马昱处理政务非常谨慎,但我们仔细想想,也许这仅仅是司马昱偷懒的一个借口。不过小时候我读过一则伊索寓言,大意是狼挤进山壁去吃羊,吃饱了就挤不去来了,急得不得了,搞得满身伤痕,后来饿了几天,咦,我出来了!伊索语重心长地说:时间是解决大多数问题的良药。

\section{3.21}\label{section-175}

\begin{quote}
山遐去东阳,王长史就简文索东阳,云:``承藉猛政,故可以和静致治。''
\end{quote}

解释: 山遐:字彦林,山涛的孙子,死在东阳太守的任上。
王长史:王濛。当时名人、大家族的子弟伸手要官当是常事,当然这个官不能太大、太重要。阮籍想喝酒,去藏有好酒的步兵曹当校尉;陶潜家里穷,伸手要了个彭泽县令。不过这次辅政的司马昱没有如王濛之愿,拒绝了,原因不明。不过后来王濛病重时司马昱觉的对不起王濛,又答应了,事见《方正49》。
承藉:凭借。也有认为藉是门第的意思,那么可以分开解释,``依靠自己高贵的门户'',不过没有必要。
猛政:山遐施政严厉,他在当余姚县令的时候,搞土断,发觉会稽虞家等大族藏匿户口万余人,就想把虞家的老大、大隐士虞喜给做了。王导、何充不答应,以山遐大建楼堂馆所要他回去述职。山遐给何充写信说:``乞留百日,穷翦捕逃,退而就罪,无恨也。'',一派李膺的风范,但王导还是免了山遐的职。王导死后,山遐重新启用,出任东阳太守,作风依旧,成千地抓人,监狱里顿时人满为患。以至于晋康帝司马岳批示说:``岂郡多罪人,将捶楚所求,莫能自固邪!''(你那里哪里
有这么多罪人,想来是刑讯逼供的产物,搞得大家人人不能自保!)山遐看了后毫不在意,依旧故我。
刘义庆和王濛大约对苛政也不以为然的。

\section{3.22}\label{section-176}

\begin{quote}
殷浩始作扬州,刘尹行,日小欲晚,便使左右取襆。人问其故,答曰:``刺史严,不敢夜行。''
\end{quote}

解释:
殷浩:名士殷浩殷渊源,简文帝提拔为杨州刺史,以牵制桓温。他一上任就励精图治,严格执行宵禁。何充曾说:``桓温、褚裒为方伯,殷浩居门(下南京属于扬州地区,故称门下),我可无劳矣。''古时候没有路灯,半夜里出来大都就是作奸犯科,一般是不允许的。
刘尹:名士刘惔。前面已经有他的很多故事了。殷浩的朋友,清谈对手,他当时大概住在扬州。
日小欲晚:此句费解,小、欲不知何意,黄昏太阳应该显得大而不是小。不过联系前后文,总是天色将晚的意思。
襆:包东西的布,代指包袱行李。
刘惔这番言谈不是夸殷浩,而是在调侃、讽刺,我们想想周星驰的``我好怕啊,小心肝被你吓的扑通扑通跳''就知道了。一般认为,殷浩有清名而无干才,而刘惔虽也是名士,但执政理念和殷浩截然不同。他见识高明,行政宽纵,``居官无官官之事,处事无事事之心'',着眼于维持大局的稳定,所以看不惯殷浩的严刻。今天在书话里看到一篇讽刺上海管理菜刀的妙文,我们就可以知道俞正声、韩正草木皆兵,不过是殷浩之流,竖子不足与谋也!

\section{3.23}\label{section-177}

\begin{quote}
谢公时,兵厮逋亡,多近窜南塘下诸舫中。或欲求一时搜索,谢公不许,云:``若不容置此辈,何以为京都!''
\end{quote}

  谢公时:谢安时代的意思,就像我们说的毛泽东时代、邓小平时代一样。谢安的鼎盛时期是376年任中书令,随后都督扬、豫、徐、兖,青五州诸军事,到384年淝水之战结束后出镇新城。不过我们不必把时间拘泥于这9年,因为在谢安快去世的时候,他说:``我16年前曾经做梦,坐着桓温的车走了16里,见一白鸡而停止。乘桓温的车驾,预示我替代他的职位;16里,意味着我代居宰相16年而止;鸡是酉,如今是酉年,我恐怕要死了。''这是从371年桓温立简文帝开始算,386年谢安去世。
  兵厮逋亡:厮,杂役;兵厮,偏义复词;兵卒逃亡。
  南塘下:下,无意义,见2.83``都下''等。南塘,秦淮河南岸,又称``横塘''、``横栅'',是当时南京最繁华的区域。南塘很有名,因为南朝的《西洲曲》唱到:``采莲南塘秋,莲花过人头。低头弄莲子,莲子清如水。''当时南塘是码头,又有驻军,又停靠了许多商船,估计鱼龙混杂,民族英雄祖逖缺钱花了,就在南塘抢劫一番。
  一时:同时。《口技》:妇手拍儿声,口中呜声,儿含乳啼声,大儿初醒声,夫叱大儿声,一时齐发,众妙毕备。
  京都:京就是大的意思,海纳百川,有容则大。兵卒逃亡,有些是被迫的,所以要逃,但主要的问题是当时门阀豪强收容,做他们的私兵、佃户、奴婢,充实自己的实力。谢安和王导一样,执政``不存小察,宏以大纲'',他虽然在浙江地区搞人口普查,但并不支持在南京也一样搞。一来首都地位敏感,由于流民问题的历史原因和军阀背景,一严打就会触动各方利益,引起大混乱。二来首都是个商业城市,一严打就容易萧条。《隋书
-
地理志》中说:``京兆王都所在,俗具五方,人物混淆,华戎杂错,去农从商,争朝夕之利,游手为事,竞锤刀之末,贵者崇侈靡,贱者薄仁义,豪强者纵横,贫窭窭者窘蹙,桴鼓屡惊,盗贼不禁,此乃古今之所同焉。''后来曾国藩打下南京,``秦淮画舫,亦复麇聚,盖如承平时矣。官吏溺游,江宁知府某欲禁绝之,言于国藩,国藩欣然曰:`有是哉!明日试治具,吾亦欲约诸公一游,领略其风趣。'某君因不敢治。''有人评论说:``曾国藩正师谢安意''

\section{3.24}\label{section-178}

\begin{quote}
王大为吏部郎,尝作选草,临当奏,王僧弥来,聊出示之。僧弥得便以己意改易所选者近半,王大甚以为佳,更写即奏。
\end{quote}

王大:王忱,王坦之的儿子,见1.44。王僧弥,王珉,王导的孙子,书法家,中书令,从兄王献之人称``大令'',王珉``小令''。王忱是太原王,王珉是琅琊王,不同宗,家族间免不了有矛盾,所以说这个故事可以见识两人的风范。
选草:选拔官员的草稿。
聊出示之:按理说这是朝廷机密,不过王忱生性随便,随手拿人东西,裸体去丈人家吊丧,这也许就是所谓晋人的``真'',消灭了物我之分。没有物我之分,是很高的境界。
我们从正面理解,王珉越俎代庖,是勇于任事;王忱雅量高致,是从善如流,但也许另有原因,只是我们不了解故事的背景。这种伯乐相马式选拔干部危害很大,
采用田忌赛马倒还好一些,但赛马弊端也很多。选官制度到明代索性搞抽签了,大家都觉得公平。古人往往把当官任职作为运气、阴德,就连现今的张副总理,当年
有一次开会时调侃自己是祖坟上冒青烟,当众说这样的话,搞得我们都很开心:原来你张书记也有自知之明啊,我官没你大,也许主要是祖上阴德不够。内心的紧张
就消解了。

\section{3.25}\label{section-179}

\begin{quote}
王东亭与张冠军善。王既作吴郡,人问小令曰:``东亭作郡,风政何似?''答曰:``不知治化何如,唯与张祖希情好日隆耳。''
\end{quote}

解释:
王东亭:王珣,字元琳,小字法护,谢安的前侄女婿,王珉的哥哥,书法家,三希堂中《伯远帖》的作者,东亭侯。王珣书法潇洒,看不出与王献之的距离,但说实话,晋代流传下来的各书法信件,我没一个读得通,读得懂的,惭愧。
张冠军:张玄之,字祖希,顾和的外孙,见2.51,曾任冠军将军。
小令:王珉王僧弥。 风政:德政、政绩。《论语 -
颜渊》:``君子之德风,小人之德草。草上之风,必偃。''意思是说君子的品德像风,小人的品性如随风俯伏的草。古人讲德治,要求长官要像风教育草、引导草一样。
治化:就是风政,法治教化。荀子《性恶篇》:``明礼义以化之,起法正以治之,重刑罚以禁之,使天下皆出于治,合于善也。''荀子说,如果没有治化,那么``强者害弱而夺之,众者暴寡而哗之,天下之悖乱而相亡不待顷矣。''
王珉回答避实就虚,我哥哥整日与大名士张玄之为伍,你们认为他怎么样?谚语说:物以类聚,人以群分;西方人也说:欲知其人,常可观其所读之书,恰如观其
所交之友。这两年反腐教育,上级的讲话稿里总期盼官员多和专家学者交朋友,不要和商人在私生活上有太多交集。不过现今学校衙门化,学者官员化,专家商人
化,上次请郎咸平讲课,据财务说一堂扯天扯地的课居然花了十七万,妈妈的!而且在材料中,那些独立知识分子都成了敌人、异见分子,总书记啊,这朋友怎么交
啊?王珉这话也许不老实,因为谢安认为王珣这人不厚道,是桓温的心腹秘书,谋划造反大业,后来桓温死了,王珣没事人一样的回朝做官了,所以叫侄女和王珣离婚,并在官场上打压王珣。

\section{3.26}\label{section-180}

\begin{quote}
殷仲堪当之荆州,王东亭问曰:``德以居全为称,仁以不害物为名。方今宰牧华夏,处杀戮之职,与本操将不乖乎?''殷答曰:``皋陶造刑辟之制,不为不贤;孔丘居司寇之任,未为不仁。''
\end{quote}

解释:
殷仲堪:见1.40,他是当时的大名士,朝野的厚望,孝武帝时任荆州刺史,镇江陵。
``德以居全为称'':这句话不知何典,大意是说所谓德要求面面俱到,完美无缺,所谓仁就是要求不去伤害任何事物。王珣这个问题比较古怪,大概平时殷仲堪长
期在上级机关工作,颇受重用,稿子里老是和谐社会、公平正义、仁义道德之类的,``永为廊庙之宝'',老朋友王珣这时就调侃他,老兄你现在是一州之长,``忽为
荆楚之珍'',操杀戮之柄,看你是不是继续说``德音一发,必声振沙漠,二寇之党,将靡然向风,何忧黄河之不济,函谷之不开哉!''
宰牧华夏:华夏本指中原,荆州算不上,这里可能代指疆域的中部地区,不过《三国志》中说关羽擒于禁,诛庞德,``威震华夏'',荆州一带大概也是可以用华夏的。
乖:违背,见2.100。 皋陶:见3.6。古之圣贤,中国原始社会第一任司法部长。
司寇:孔子52岁的时候在鲁国当过3年司法部长,可能还摄相位。据《孔子家语》里说,孔子做大司寇的时候,他总是和同事们一起认真地讨论案情,然后鼓励
大家提出处理方案,听取处理意见,择善而从。据《吕氏春秋》说,孔子被任用三年后,男子行路沿右侧,女子行路沿左侧(这个有意思,孔子鼓励男女途中相遇,
擦肩而过而不是执手同行),财物有丢失的,民众都不伸手拾取。大家认为孔子的大智慧是难以逾越的。不过3年后孔子就下台了。当然也有故事说孔子手段狠辣,
对意识形态工作特别重视,上任7天诛杀少正卯。
殷仲堪在荆州的所为,大体是``好行小惠,夷夏颇安附之'',``佐史咸服之''。

\section{政事 小结}\label{-}

刘义庆的执政观点首先是宽、恕,26则故事中,谢安不惩逃兵,刘惔对殷浩宵禁不以为然,王承不追究小偷小摸和违反宵禁,王濛主张和静致治,羊祜请求郑冲修订法律时``小加弘润'',桓温欲``德被江汉'',这些执政宽恕的故事几乎占了《政事》中一半的篇幅。当然,这种宽恕当然有一定的时代背景和特殊性,东晋的门阀制度要求,有些现在看来必须严肃处理的事情,在特定的环境中可能还是``愦愦''更符合实际,王导就以此为自豪,而庾亮凡事``察察'',``颇失人心'',诱发了苏峻之乱。但其主要原因,可能还是晋代的意识形态起着决定性作用。陆贾在《新语》中说``君子之为治也,块然若无事,寂然若无声,官府若无吏,亭落若无民,闾里不讼于巷,老幼不愁于庭,近者无所议,远者无所听,邮无夜行之卒,乡无夜召之征,犬不夜吠,鸡不夜鸣,耆老甘味于堂,丁男耕耘于野,在朝者忠于君,在家者孝于亲;于是赏善罚恶而润色之,兴辟雍庠序而教诲之,然后贤愚异议,廉鄙异科,长幼异节,上下有差,强弱相扶,大小相怀,尊卑相承,雁行相随,不言而信,不怒而威,岂待坚甲利兵、深牢刻令、朝夕切切而后行哉?''我们不能说这种宽恕是道家学说的专利还是儒家学说的主张,而是汉初政治家对政治追求的教训总结和美好愿景,宽恕之道在一般情况下更容易获取人心,维护社会稳定,这就是最大的政治。晋代和汉初的政治有很大的相似性,汉初搞封建制,晋代也搞封建制;汉初崇尚黄老之道,晋代大批官员无所事事,而且变本加厉,耻谈具体政务。不过古代的一些政治家和学者也认为,宽恕并不能排除礼法,刘义庆把陈寔杀小吏和立即追查不养育孩子的故事放在一、二则,本身就是对宽恕政治的一个重要补充------用霹雳手段来推行教化。我们把《世说
- 政事》中的故事对照陆贾的《新语》,就会有较实际的理解。
  刘义庆也主张勤政和进取。陶侃勤于事,何充为政事放弃玄谈,勤于会见客人,王珉、殷仲堪当仁不让,他都持肯定态度,而把士人那些特立独行的行为放在``任诞''等处,显然是不知其可的。《颜氏家训》中说:``吾见世中文学之士,品藻古今,若指诸掌,及有试用,多无所堪。居承平之世,不知有丧乱之祸;处庙堂之下,不知有战陈之急;保俸禄之资,不知有耕稼之苦;肆吏民之上,不知有劳役之勤,故难可以应世经务也。\ldots{}\ldots{}其余文义之士,多迂诞浮华,不涉世务;纤微过失,又惜行捶楚,所以处于清高,盖护其短也。''这是当时有见地的人对两晋士人故作清高、不思应世经务的批评。不过这种批评在今天可能已经没有多少意义了,现今的领导们恐怕还是太想干``事业'',太勤于进取,一心考虑帽太小、钱太少,于是话太多、手太长。
  《政事》中还有几则是赞赏王导善于待人接物,山涛有识人之明,陆玩敢于``翻异''等,我们都可以想见前辈的风范。细读《政事》,我们平心而论,里面故事归类有些有点问题,也许放在语言、识鉴、排调等可能更恰当一些,不过我们在读《论语》等的时候,也没读出子路、冉有有什么政事之长,不必苛责。

\chapter{文学}

\section{4.1}\label{section-181}

\begin{quote}
郑玄在马融门下,三年不得相见,高足弟子传授而已。尝算浑天不合,诸弟子莫能解。或言玄能者,融召令算,一转便决,众咸骇服。及玄业成辞归,既而融有`礼乐皆东'之叹。恐玄擅名而心忌焉,玄亦疑有追,乃坐桥下,在水上据屐。融果转式逐之,告左右曰:``玄在土下水上而据木,此必死矣。''遂罢追。玄竟以得免。
\end{quote}

解释:
郑玄:字康成,山东高密人,他的学问在东汉末几乎到了``无所不知''的地步,我们现在读的《诗经》是毛笺郑注。更重要的是,《礼记》就是出自郑玄之手,经过他编辑注释,《礼记》成型,并且迅速成为显学。在前面介绍过,古代统治者亲睐儒家,其实就是《礼记》在发挥根本作用。儒家最重要的著作并不是《论语》而是《礼记》,《礼记》就是朝廷的行为规则和政治秩序,是普通中国人行为和思想的指南,过去说中国是礼仪之邦,这不是空话,我们只要抛出一本《礼记》就可以作为最关键的证明。郑玄一生好学,他33岁时已是著名的学者,``关东独步'',还千里求学于陕西的学者马融,一学7年。这里就讲述了他求学于马融的故事。郑玄的著作非常丰富,不过大多都散佚了,但丝毫不影响他伟大学者的声誉。
马融:字季长,大学者,他年轻时不欲出仕,有一次差点饿死,他以此为耻(刘孝标的注解释起来很麻烦,大意是说有很大才能的人却被人、事所制,贪图虚妄的名声而断送自己的生命,都是很愚蠢的),做官去了。马融做官时曾耿直不肯附炎趋势而吃过亏、坐过牢,后来辞官授徒,学生上千。他是古文经学(原始文本派)大师,``举大义''、``不为章句'',``才高博洽,为世通儒'',经过磨难后随波逐流,``达生任性,不拘儒者之节'',
他上课坐高堂,施纱帐,前授生徒,后列女乐,开魏晋士人任诞风气。
``三年''句:马融上课是单个辅导,``弟子以次相传,鲜有入其室者''。《太平广记》还有个传说:郑玄学了三年,没有什么成就,师门鄙视他将他赶走。郑玄在回去路上的一棵大树下休息,梦见来了一位老翁用刀割开他的心脏(开窍),说:``这样你就可以学习了。''于是郑玄醒来后回去学习,很快什么都懂了。
尝算浑天不合:天文推算出了错。
礼乐皆东:郑玄告别时,马融拉着郑玄的手说曰:``大道东矣,子勉之!''意思是说郑玄已经获得儒家文化的道统,他将是当时的文化中心。后来郑玄的确有很高的声望,有教化之德,《后汉书》中说,郑玄遇到黄巾军起义军,他们都听说过郑玄,``见玄皆拜,相约不敢入(高密)县境''。
据屐:据有依靠、按住的意思,这里引申为坐。或``据''通``踞'',《说文解字》``居,蹲也。''还是坐的意思。
转式逐之:转动木盘占卜推算追查。《广雅》:``曲道,栻梮也。梮有天地,所以推阴阳,占吉凶,以枫子枣心木为之。''《唐六典
-
十四》:``抱式以知天时也。今其局以枫木为天,枣心为地。刻十二辰,下布十二辰,以加占为常,以月将加卜时,视日辰阴阳,以立四课。''汉代占卜的风气很盛,周文王、孔子就搞这一套,西汉儒学大师董仲舒就是谶纬派的代表人物,他的天人感应学说就是来自占卜迷信,也为占卜迷信提供了理论来源和观念支撑,这种儒家学说上相信谶纬占卜的现象,可能到南宋理学昌盛后才被打压下去,只保留了其中天灾异变的一部分,但在人没有得到解放、有神论还有市场的情况下,谶纬的各种变种将长期存在。
土下水上而据木:郑玄的这种做法叫禳解。他在木头上,代表棺材。头上有土,边上有水,代表被安葬了。
竟:最终。
追杀的故事荒诞、不合情理、破绽明显,但颇有趣,是后来很多民间故事的母体。《世说》把这则故事放在《文学》第一,我们也许可以想象刘义庆对传奇故事的着迷,保持着人类童年的好奇和趣味,抱有文学的抱负。康
-
巴乌斯托夫说过:``对生活、对我们周围一切的诗意的理解,是童年时代给我们最伟大的馈赠,如果一个人在严肃而悠长的岁月中,没有失去这个馈赠,那他就是诗人或者作家。''
在各种书上,马融、郑玄的有不少谶纬故事,他们两个是汉末著名的谶纬专家。我们暂且按照维特根斯坦所说的:对于不可言说的,应该保持沉默。

\section{4.2}\label{section-182}

\begin{quote}
郑玄欲注《春秋传》,尚未成。时行与服子慎遇,宿客舍,先未相识。服在外车上,与人说己注《传》意,玄听之良久,多与己同。玄就车与语曰:``吾久欲注,尚未了。听君向言,多与吾同,今当尽以所注与君。''遂为《服氏注》。
\end{quote}

《春秋传》:《左传》,关羽喜欢看《春秋》,其实也是《春秋左氏传》。
服子慎:服虔字子慎,河南荥阳人,经学家,曾任九江太守,《服氏注》即《春秋左氏传解谊》。
郑玄``但念述先圣之元意,思整百家之不齐'',对儒家的经典基本都做过注,《后汉书》中说:``郑玄囊括大典,网罗众说,删裁繁芜,刊改漏失,择善而从,自是学者略知所归。''他一言知人,主动把自己尚未完成的作品交给服虔,以学术为公器,诚恳无私,境界高远,是真正向学之人,其风范许多人都比不上。当然,文化环境不同也是一个因素,曹操因``古人与己意同''烧了《孟德新书》,夏侯湛因看了《三国志》而烧了自己写的《魏书》,他们谦虚谨慎,律己很严。如果要我们再举后世类似的例子来,还真难找。当然我们也许会反问,曹操《短歌行》大段引用《诗经》中的句子,也不见得他烧掉自己的作品,但诗歌中的化用和移植是比较复杂的问题,黄庭坚说杜诗``无一字无来历'',提倡诗歌创作中的``点石化金术'',我倾向于认为,诗歌创作中的用典、含典和学术研究中的抄袭有本质区别,不能混为一谈;在现今写作市场化的环境中,作者比前人自然更加重视自己的劳动、创造、版权和名誉,因为现今的作者往往是以此谋生而不是为了兴趣,所以我们理应在引用中体现自己的光明磊落和对前辈的尊重。

\section{4.3}\label{section-183}

\begin{quote}
郑玄家奴婢皆读书,尝使一婢不称旨,将挞之,方自陈说,玄怒使人曵箸泥中。须臾,复有一婢来问曰:``胡为乎泥中?''答曰:``薄言往愬,逢彼之怒。''
\end{quote}

称旨:合乎心意。 ``胡为乎泥中?''是《诗经 -
式微》``式微,式微,胡不归?微君之躬,胡为乎泥中'',其本意不好懂。据《毛诗》说,泥中是地名,主旨是``微黎侯寓于卫,其臣劝以归也'';但也可以认为,这是首恋爱诗歌,男孩子在露水中、在泥泞中等女孩子;也有人说是农民的歌,``我为什么在露水中、在泥地里努力劳作?''
``薄言往愬,逢彼之怒''是《诗经 -
柏舟》``我心匪鉴,不可以茹。亦有兄弟,不可以据。薄言往诉,逢彼之怒'',其主旨是兄弟不可依靠,但自己坚贞不屈。
在这个故事中,《诗经》的本意不用细查,而在于两个女孩子讲话都来自《诗经》中有名的诗歌。她们讲话既很晓畅地道明眼前之事,感叹``仁人不遇'',又用典贴切自然,不见斧凿,通过文学保持了心灵的自由,显示出很好的心态和学养,所以显得风雅,实现了孔子所说的``诗可以群,可以怨''。
故事用词简洁,从一个侧面赞扬了郑玄家家学深厚。但也有人因此指责,你郑玄是大儒,怎么可以打骂体罚,虐待奴婢!不过《世说新语》本有小说成分,郑玄又是古人,我们不必买椟还珠,用现代的眼光要求故事的完美。
说句题外话,书话版主雍容有篇《式微》,是我近年来看过最好的散文之一,佳人难得。

\section{4.4}\label{section-184}

\begin{quote}
服虔既善《春秋》,将为注,欲参考同异。闻崔烈集门生讲传,遂匿姓名,为烈门人赁作食。每当至讲时,辄窃听户壁间。既知不能逾己,稍共诸生叙其短长。烈闻,不测何人,然素闻虔名,意疑之。明蚤往,及未寤,便呼:``子慎!子慎!''虔不觉惊应,遂相与友善。
\end{quote}

善《春秋》:对《春秋左氏传》很有研究。
崔烈:字威考,大学者,河北高阳(博陵)人。但他后来为了当高官,花了500万钱买了个司徒,舆论不佳。有一次他问儿子崔钧:``我现在位列三公,社会上有什么议论?''崔钧直言不讳:``大人您从来有很好的名声,您当九卿、太守的时候,舆论认为即使当三公也在情理之中。但现在这样当上三公,大家都掩起鼻子,说你身上有铜臭味!''崔钧对老子也毫不留情面,不过他这话其实过了,因为买官是当时的制度,举世滔滔,莫不如此。
为了解决财政问题,东汉灵帝时朝廷开始建立和完善卖官鬻爵制度。三公一千万,九卿五百万;郡守因为比较实惠,虽然品级低,但值两千万。但如果岗位热门,那就需要竞标拍卖。官职买完后,汉灵帝卖爵位,甚至卖``孝子'',所以当时有谚语说:``举秀才,不知书;察孝廉,父别居。寒清素白浊如泥,高第良将怯如鸡。''再后来,官位还可以抵押、按揭;任期一到,要留任、调动也得缴费。反正当时买官就像现在买房子一样,而且可以打折。据说崔烈一开始要面子,还不愿买,经过汉灵帝奶妈程夫人做思想工作,才交了钱:``崔公,冀州名士,岂肯买官!赖我得是,反不知姝邪!''因为崔烈搞的是内部价,有腐败嫌疑,所以舆论认为他违背游戏规则,太小气吝啬了,骂他是守财奴,一身铜臭。崔烈还有个儿子叫崔州平,大家可能有印象,因为诸葛亮把自己比作管仲、乐毅,``时人莫之许也。惟博陵崔州平、颍川徐庶元直与亮友善,谓为信然。''博陵崔氏是南北朝以后中国的第一等望族。
讲传:讲授。 赁作食::应聘当厨师。 听户壁间:听墙角。
崔烈趁服虔梦中叫他的名字,已有几分查问间谍的味道。梦是不加掩饰的,记得有个少数民族,把梦当做真实,而把生活当做虚幻。仔细想想,还真是这样,我曾经以为把有些人和事都忘记了,可是在梦里,在梦醒的时候,才了解自己的内心。
古时候书籍少,得到不易,古人的求学精神也从中得到体现。服虔给《左转》作注,光郑玄的资料是不够的,他到处求学受业,甚至怕崔烈听说他的名声后有顾虑,不辞``贱业'',很好地回答了学习为何的问题,正如一本小说中所说的:``过得一山又一山,不是巅峰不肯攀。''

\section{4.5}\label{section-185}

\begin{quote}
钟会撰《四本论》始毕,甚欲使嵇公一见。置怀中,既定,畏其难,怀不敢出,于户外遥掷,便回急走。
\end{quote}

解释:
四本论:钟会天资卓越,小时候就和玄学代表人物王弼齐名,在《三国志》中,《王弼传》就附在《钟会传》之后,不过钟会少年得志,19岁就当官了,所以学术上没有取得王弼那样的成就。四本论研究人的才干与人的内在本质(性可以理解为道德品质)的关系,是为政府选拔人才提供理论上依据。在当时,四本论显然是对曹操``唯才是举''政策和九品中正制的思考辨析,是玄学长期研究讨论的一个话题。钟会主张``才性合'',想来比``同''要退一步,但不赞成``离''和``异''。据说主张``离''和``异''的夏侯玄、李丰、王广后来都被司马家杀了,而主张同、合的在司马朝中春风得意。
嵇公:嵇康有很高的声望,其人品被誉为``方中之美范,人伦之胜业'',其学术著作有《声无哀乐论》、《养生论》、《难自然好学论》、《明胆论》、《释私论》等,前两个著作一直是魏晋时重要的玄谈题目,而且他被杀时,``太学生三千人请以为师'',应为当时的学术大家。从嵇康的文字看,他在理论上讲本性,主张清心寡欲、不为外物所累、不为名利伤德,``其善于不善,虽遭浊乱,其体自若而无变也'',认为才能不过是智慧,``下逮德衰,大道沉沦。智惠日用,渐私其亲'',``故世之所患,祸之所由,常在于智用,不在于性动'',对才并不看重,自己``含垢藏瑕''。我们没有材料具体证明嵇康对才性的看法,但钟会向嵇康请教,怕他批评,估计观点不是``合''。
既定:或当做``诣宅''或``既见''。
钟会和嵇康年纪相近,名声相仿,而且钟会的官职比嵇康要高一些和重要得多,``闻所闻而来,见所见而去''不卑不亢,可怎么在这里如此讨好和敬畏嵇康?这个故事我们不必当真,但嵇康的高傲总是有的,大大得罪了钟会。所以嵇康被杀,司隶校尉钟会廷论之曰:``今皇道开明,四海风靡,边鄙无诡随之民,街巷无异口之议。而康上不臣天子,下不事王侯,轻时傲世,不为物用,无益于今,有败于俗。昔太公诛华士,孔子戮少正卯,以其负才乱群惑众也。今不诛康,无以清洁王道。''
嵇康入狱后,以为自己不至于死,他写诗回顾人生旅程,畅谈人生愿景,最后一句涉及了``性'':``庶勖将来,无馨无臭。采薇山阿,散发岩岫。永啸长吟,颐性养寿。''

\section{4.6}\label{section-186}

\begin{quote}
何晏为吏部尚书,有位望,时谈客盈坐。王弼未弱冠,往见之。晏闻弼名(来),(乃倒屐迎之),因条向者胜理语弼曰:``此理仆以为极,可得复难不?''弼便作难。一坐人便以为屈。于是弼自为客主数番,皆一坐所不及。
\end{quote}

解释:
何晏:字平叔,汉末外戚、大将军、屠户何进的孙子。他的母亲因为貌美,被曹操收用(曹操的后宫有多少美人啊,啧啧),何晏小时候也就住在魏宫。《世说》中说,因为何晏相貌可爱,聪明伶俐,曹操想收为义子,7岁的何晏毫不客气地拒绝了,出宫,后来当了曹操的女婿。因为他是晚期曹魏的核心人物之一,被司马懿父子杀害,所以在史书上的表现并不好,连相貌好、喜欢女孩子都成了罪,对他的记录多有抵牾之处,譬如一方面说他在当吏部尚书的时候浮华专权,``宿旧多得济拔'',``内外望风,莫敢忤旨'',``为恶日甚'',一方面又在其他文章中说``内外之众职各得其才,粲然之美于斯可观''。我们撇开他私生活和政治立场不谈,何晏是玄学的祖师、清谈的领袖,他``少有异才'',学问广博,``魏武帝读兵书,有所不解,试以问晏,晏分散所疑,无不冰释'',他引导学术潮流,疏释《论语》、《孝经》、《周易》、《老子》,把老庄学说放在与儒家学说同样重要的位置,进行阐述,辨析``有无之境'',``自何氏始有玄虚之言'',``当时权势,天下谈士,多宗尚之''
。
王弼:字辅嗣,山东高平人。他的著作《老子注》《周易注》《论语释疑》等都保留下来了,虽然王弼去世时只有24岁,但靠作品说话,成为玄学创始人之一。汉代儒家学说研究到郑玄这里,已经基本结束,而何晏把兴趣逐渐转移到道家学说。因为儒家学说主要是指导如何处理人与社会的关系,讲的是人情世故,没有很强的哲学思辨性,而道家学说主要研究的是人与世界的关系,``天人之际'',直指世界的本源,显然更适合作为研究的课题。何晏开始把儒学和道学综合,提出``贵无''的本体论,而王弼更抱着超越圣贤的雄心,大大构建和完善了何晏的论点。玄学来源于儒道,但超越了儒道,是一门新的学说,而后几百年的学术研究,基本都在王弼的学说框架之中,魏晋很多士人的生活方式,也以何、王的学说为依托。在后面,我们要接触很多简短但复杂的玄学理论。
弱冠:《礼记 - 曲礼》说``男子二十冠而字'',《礼记 -
内则》``女子十有五年而笄。''周制男子20岁成人,举行仪式纪念,取字。弱冠,因为男子20岁左右,没有达到壮年,所以叫``弱''。冠礼的仪式比较复杂,有个成语叫沐猴而冠,就指出在行冠礼前还要洗澡的。
晏闻弼名:王弼的叔祖是王粲,外曾祖父是刘表,他少年时在荆州生活学习,当时刘表治荆州,没有战火,``关西、兖、豫学士归者盖有千数'',王弼受到很好的教育,``幼而察惠,通辩能言''。
因条向者胜理语:成语``理不胜辞'',胜理语就是文辞和义理都很高妙的话。于是列出从前的那些精妙的义理语辞。
复难不:难是论说、批评的意思。再进行辩论吗?
王弼先批驳何晏的理论,再批驳自己的理论,这样虚拟双方,通过往往复复的辨析,使自己的观点得到更加清晰的阐述。当时人认为,何晏虽然文辞比王弼好,但王弼理论比何晏精深,``唯王辅嗣妙得虚无之旨'',谁也比不上他。何晏很佩服王弼,说``后生可畏'',老老实实把自己的《老子注》烧了。
何晏自信而谦虚,王弼``弼天才卓出,当其所得,莫能夺也。''

\section{4.7}\label{section-187}

\begin{quote}
何平叔注《老子》始成,诣王辅嗣。见王注精奇,乃神伏,曰:``若斯人,可与论天人之际矣!''因以所注为《道》、《德》二论。
\end{quote}

解释: 神伏:倾心佩服。
天人之际:际是边缘、分界的意思,司马迁在《报任安书》里说,他写《史记》目的是``欲以究天人之际,通古今之变'',
在《太史公自序》中又说:``礼乐损益,律历改易,兵权、山川、鬼神,天人之际,承敝通变,作八书''。我们看《史记》,往往注意的是历史故事,但《史记》有《书》八篇,讲的是信仰、社会、历法、天文、经济、文化等问题,这些都在解释世界是什么,星空的意义,自然有什么特征和规律,社会怎么演变等。也许我们可以这样理解,对于人来说,一切都是客体,一切客体都包含着自然运行的规律,董仲舒提出``天人之际,合而为一'',认为在天人对立的基础上,最终人要顺应自然。
《道》、《德》:当时的《老子》版本就是现在通用的王弼版,道为上篇,德为下篇。但随着考古的收获,《老子》的版本不断被发现,1973年马王堆出土的汉代帛书《老子》,德为上篇,道为下篇,与定型的《老子》差别极大,在1998年郭店楚简版的《老子》中,就基本不涉及道篇形而上的内容。我们可以这样认为,标准版《老子》,是西汉中后期后某个人整理改编的作品。
这个故事和《世说》4.10基本一致。昨天何晏被王弼教训了一番,今天上门讨教,看了人家的《老子注》,更加自惭形秽,只能乖乖地把注释中观点整理成论文保存。何晏比王弼大20多岁,是早已成名的学术权威,当朝显贵,但有很强的学术规则自觉,不过这两篇论文,后来散佚了。

\section{4.9}\label{section-188}

\begin{quote}
傅嘏善言虚胜,荀粲谈尚玄远。每至共语,有争而不相喻。裴冀州释二家之义,通彼我之怀,常使两情皆得,彼此俱畅。
解释:
傅嘏:字兰石,正始年间的玄学家。四本论中他持``才性同'',是司马家的要员。
虚胜:难理解,《三国志 - 注》中说:``傅嘏善名理,荀粲尚玄远''
,虚胜或者就是名理的意思。``名理''是``辩名析理''的简称,譬如讨论才性、声无哀乐等。
荀粲:字奉倩,荀彧的儿子。他最著名的是爱妻子,《世说》后面有个故事:他妻子发热病,他自己在露天把衣服脱光,冻得冰凉后回家抱妻子,给她降温。不过妻子还是死了,他很伤心。有人劝他,他说:``我也不是爱妻子,我也不在乎女人的品德,而是爱美色,这样的绝色死了,再也没有了。''荀粲居然伤心而死。
``相喻''句:傅嘏清谈讲究逻辑思辨,而荀粲清谈取其精神,有时候讨论不到一起,相互不能明白理解。
裴冀州:裴徽,冀州刺史。清谈往往有攻、守方,同时也需要引导员、裁判员和评论员,裴徽善于理解、分析、提炼、总结,沟通两人的意见。
彼我:彼此。 《三国志 -
荀粲别传》中记录了一次他们清谈的内容。荀粲说:``你傅嘏今后在功业名位上一定比我强,但识度是我高明。''傅嘏说:``功业名位本就是胸怀见识的体现,不可能有胸怀见识强而功名弱的。''荀粲说:``功名主要取决于人的志向器量,志向器量与胸怀见识有关,但不等同。我的`识'能看到你们将来的显贵,但我的`识'并不以为你们那样的人生追求为对。''
荀粲为人放旷,他不守礼法地议论自己的父亲荀彧、叔父荀攸,认为父亲装腔作势,有意庄重,而叔父随便低调,所以荀彧不如荀攸。
荀粲29岁就死了,不然《世说》中将有关于的他更多故事。
\end{quote}

\section{4.10}\label{section-189}

\begin{quote}
何晏注《老子》未毕,见王弼自说注《老子》旨,何意多所短,不复(得)作声,但应诺诺(之)。遂不复注,因作《道德论》。
\end{quote}

此条与4.7雷同,不过是一个没写完,一个写完了。刘义庆为什么连写两次,不好懂。
达者为师,于是何晏只能听着王弼的滔滔不绝,嘴里连连说``对、对''。何晏胸怀难得,也许就是真正掌握《老子》的缘故,``居善地,心善渊,与善仁'',``知人者智,自知者明。胜人者有力,自胜者强''。他是高官、学术权威,50多岁的老头子,在不到20岁的小伙子面前恭敬谦虚,并且大力提携,奖掖后进。
何晏是权臣曹爽的长辈,司马懿诛杀曹爽的时候,据说叫吏部尚书何晏整理曹爽党羽的名单。何晏尽力收集,列出了七家。司马懿不满意:``要杀八家的。''何晏问:``难道是我家吗?''司马懿说:``对。''何晏的母亲被曹操宠信,自己被曹丕骂,被曹叡羞辱,只有到曹芳的时候被辅政曹爽启用,``晏有重名,与魏姻戚,内虽怀忧,而无复返也'',只是由于才华出众,声望很高,故为司马懿所忌杀,杀了以后还故意编个故事丑化何晏。司马懿的确是戏子中实力演技派。

\section{4.11}\label{section-190}

\begin{quote}
中朝时有怀道之流,有诣王夷甫咨疑者,值王昨已语多,小极,不复相酬答,乃谓客曰:``身今少恶,裴逸民亦近在此,君可往问。''
\end{quote}

中朝:见2.27,西晋。 怀道之流:心系大道的人。
王夷甫:清谈误国的王衍,字夷甫,琅琊王氏,王戎的堂兄弟,14、15岁的时候就人才非常突出,司马炎曾问王戎,:``当代哪个人和王衍差不多?''。戎说:``当世没见到谁能跟王衍相比,应该从古人中去寻求。''王衍气量宏大、人才俊秀,但没有抱负、节操,他后来当上了尚书令、太尉,但西晋也灭亡了,自己被石勒杀掉。一般认为,西晋的灭亡和首辅王衍有一定的关系。
小极:极,疲劳,见2.33。我们可以在《世说》中看到,清谈时大家往往通宵达旦,不知东方欲晓,但绝大多数内容并没有留下来,因为写下来就不是``无'',就堕入执着。他们看重的是思辨的过程,而不是思辨的结果,就像旅行,在于沿途的风景,而不是目的地,何况真理没有彼岸。
少恶:既``小极''。
裴逸民:山西裴頠,八王之乱中被赵王伦杀害,才34岁。有趣的是,裴頠的看法和王衍不同,《晋书
-
王衍传》中说王衍甚重王弼、何晏之``贵无''思想,``惟裴以为非,著论以讥之,而衍处之自若'',
《晋书 - 裴頠传》
中说:``頠深患时俗放荡,不尊儒术'',不但如此,王衍还很推崇裴頠,向人介绍裴頠的学说。
从《世说》中看,玄学家一般都胸怀广阔,也许是当时的学术风气或特征。

\section{4.12}\label{section-191}

\begin{quote}
裴成公作《崇有论》,时人攻难之,莫能折。唯王夷甫来,如小屈。时人即以王理难裴,理还复申。
\end{quote}

裴成公:裴頠,他是侍中、尚书、武昌侯,谥号``成'',
安民立政曰成,德见于行曰成。 崇有论:《晋书 -
裴頠传》全文照录,大意是说事物因为存在才有,不存在的事物没有意义。``无''不是事物的特征,研究事物就因为它有。如果说它的根本属性是无,那还研究干什么,你也不可能加深对事物运行规律的把握理解。因为事物的规律不容易讲。如果讲,就必须有在实际中经受考验。可是讲``无''不用在实际中经受考验,所以可以非常随便,``形器之故有征,空无之义难检''。所以讲贵无论的,看上去是玄妙、雅远、、旷达,但实际上是``或悖吉凶之礼,而忽容止之表;渎弃长幼之序,混漫贵贱之级,其甚者至于裸裎,言笑忘宜,以不惜为弘,士行又亏矣。''
裴頠当面和王衍辩论,似乎处于下风,但别人用王衍的话和他理论,却又说不过他。裴頠的话听起来是不错的,儒家有些话听起来都不错,但做起来却不是那回事。严复曾经说中华文明之弊,就是``始于作伪,终于无耻''。儒家文化的集体主义,社会主义的集体主义,归根到底败于西方文化的个人主义,也许其中的原因就是集体主义中的集体和个人更加无耻。马克思说:``羞耻本身就是一种革命,羞耻是一种内向愤怒,如果整个国家真正感到羞耻,它就会像一只蜷伏的狮子,准备向前扑去。''

\section{4.13}\label{section-192}

\begin{quote}
诸葛宏年少,不肯学问,始与王夷甫谈,便已超诣。王叹曰:``卿天才卓出,若复小加研寻,一无所愧。''宏后看《庄》、《老》,更与王语,便足相抗衡。
\end{quote}

诸葛宏:字茂远,山东琅琊诸葛氏。据说诸葛亮家族本姓葛,因为人口迁移,他们为了纪念从前住在琅琊诸城,所以在姓面前加了``诸''。诸葛氏在晋代是大族,但不是显族,因为诸葛亮在蜀,诸葛瑾在吴,诸葛诞虽然在魏国也是高官,但被司马懿所杀。诸葛氏在司马家族面前都是失败者。《世说》中记载,诸葛宏虽然名声不小,后来被莫名其妙地流放。
学问:动词。 超诣:超出一般的境界。
一无所愧:在任何人面前都不用为自己学问不够而惭愧。世上本来就有一些人天资卓越,悟性高超。宋代严羽论诗歌,他在《沧浪诗话》中说,``大抵禅道惟在妙悟,诗道亦在妙悟'',``夫诗有别材,非关书也'',
因为天资在起作用,唐诗自然而然地``尚意兴而理在其中''。但天资好,再读点书,就更妙了,能达到极致,``然非多读书,多穷理,则不能极其至''。王衍对诸葛宏的劝勉,就是严羽的意思。

\section{4.14}\label{section-193}

\begin{quote}
卫玠总角时问乐令梦,乐云是想。卫曰:``形神所不接而梦,岂是想邪?''乐云:``因也。未尝梦乘车入鼠穴,捣虀啖铁杵,皆无想无因故也。''
卫思因,经月不得,遂成病。乐闻,故命驾为剖析之,卫既小差。乐叹曰:``此儿胸中当必无膏肓之疾。''
\end{quote}

``玉人''卫玠是乐广的女婿。 总角:语出《诗经 -
氓》``总角之宴,言笑晏晏。信誓旦旦,不思其反。''古时候小男孩梳朝天辫,叫总角,借指少年时代。《氓》这句话的意思是我们发小的时候经常一起言谈嬉笑,是多么平静欢乐。你当时对我海誓山盟,谁想得到后来一切都会反转。
想:做梦是因为有所思。这个说法是比较笼统的,所以卫阶紧接着问,可是有些梦的内容自己自己从来没接触过,也没想过,怎么可以说有所思呢?
因:事物发生前已具备的条件。钱钟书说,心中情欲忆念盖得之``想'',体中之感觉受触可名为``因''。乐广说人不会做坐车进老鼠洞的梦,不会做捣蒜泥喂铁棒的梦,这是因为人不会这么想,就没有做这种梦的条件。
乐广的这种说法比较简单,我们仔细分析一下,有些梦是人愿望的反映,表达人在潜意识中对未来的期待和希望,按佛洛依德的说法,因为有些愿望不符合社会道德和本人的理智,无法进入意识被个体所觉察,只有在梦境中得到反映,而卫阶显然没有潜意思的理论,所以想不通。当然,梦还要复杂一些。有些梦是对睡觉时周围感觉的反映,譬如我们设定闹钟,可你不愿意醒来,在你听到闹钟的时候就会在梦中对这个声音做出解释。小时候我和表弟看过一本书,说如果在睡觉时往嘴里滴水,人就会做游泳的梦,我俩就互相提醒做实验,不过总是失败。有些梦是对应对恐惧的训练,譬如我们会做一些追逐、坠落的梦,据有些人解释是因为在人类的原始时期,往往会被追杀或者遭遇不幸,而这种记忆隐藏在基因之中,所以通过做梦来训练人的应对能力。有些梦是对记忆的筛选,有些重要的事情自己觉得应该记住,所以我们会梦见考试,死去的人,以前的恋人等等。
宗教对梦有特殊的感情。道教的始祖之一的庄周,就是通过梦蝶,提出了对世界真实性的质疑,``不知周之梦为蝴蝶与,蝴蝶之梦为周与''?进而提出``至人无梦''。伊斯兰教的穆罕默德,传说中就是在梦中被真主的使者带到天堂七天,终于开悟。在基督教中的《创世纪》中,更是保留了众多先知在梦中接受神谕,发出预言的故事,以至于看了《劝世良言》的洪秀全后来也自称做梦到天堂,耶和华说洪其实是耶稣的弟弟,还给他一把龙泉宝剑,叫他在人间斩妖除魔。连不言怪力乱神的孔子,也感叹``吾不梦复周公久矣'',言下之意就是周公就是孔子的神灵,他得到了周公的衣钵,两者经常神交。在《史记》以至于《二十四史》中,我们可以看到作者认认真真记录下达官显贵们做的各式各样的梦,并作出貌似合理的解释。在浩如烟海的佛教典籍中,更是充斥着对梦的各种分析和参悟。梦是佛教描述万有实相的最恰切的比喻,为大乘观察诸法实相的``十喻''和``十缘生句''之一。《金刚经》中有句名言:  ``一切有为法,如梦幻泡影,如露亦如电,应作如是观'',再进一步,把梦分成许多种类和成因,如四因缘、五因缘、六因缘等说,同时指出,唯佛``诸漏已尽,远离一切颠倒''。在藏密修炼中,更把梦作为一种修炼法门,提出梦观法和光明法,认为通过修炼禅定于梦境的实相,在梦境中明心见性,破除一切幻境。
命驾:命人驾车马,借指立即动身。 差:病愈。
必无膏肓之疾:膏肓之疾本来应该指药力无法达到的心脏以下的绝症,但这里大概是说心病。因为卫阶能善于思考,就不会有化解不了的心病。
乐广的解释想必过于粗陋,卫阶如果想的时间再长一些,内容深一些,也许《梦的解析》的作者就不是佛洛依德了。
我有时候会梦见自己在做梦,正是佛经中所说的梦中说梦。

\section{4.15}\label{section-194}

\begin{quote}
庾子嵩读《庄子》,开卷一尺许便放去。曰:``了不异人意。''
\end{quote}

庾子嵩:庾敳庾子嵩,庾亮的叔叔,当过吏部郎,他是个大胖子,肚量极大,成语``小人之心度君子之腹''说的就是他。
一尺许:当时没有印刷术,都是手抄本,一尺许就是是20几厘米,估计是厚度而不是长度,不然《逍遥游》才开了个头。了:全。
人:我。
汉代虽然学者也读《庄子》,但学者们好像多引用其中的典故文字,对它的研究却没有见到证据。但也有人认为《庄子》就是在汉代成书的,汉代庄子研究的成就就是学者们在书中夹杂了许多自己的文章,除了《内篇》以外,其他的可能都是汉代人写的。古代人没有稿费和版税,要使自己的文章流传,就要假借名人之名,把自己的私活也放进去,从此流芳后世。至于自己成不成名,会不会被后人记住,是不必要的,所以四大名著的作者,我们要靠蛛丝马迹来破案。《庄子》到了三国之后,在何晏、王弼的推崇下,在向秀、郭象的解释下,《庄子》的地位逐渐上升,到了西晋,成为一门显学。苏东坡曾经说:``吾昔有见于中,口未能言。今见《庄子》,得吾心矣。''这也许说明庾敳的天资比苏轼还要高明,也许说明庾敳在学问上比较疏阔。比如像我,算是完完整整看过几遍《庄子》的,却说不好庄子究竟主张是混世还是避世,是寡欲还是纵欲,是求知还是无知,等等。可能还是庾敳说得好:``在有意无意之间。''

\section{4.16}\label{section-195}

\begin{quote}
客问乐令``旨不至''者,乐亦不复剖析文句,直以麈尾柄确几曰:``至不?''客曰:``至。''乐因又举麈尾曰:``若至者,那得去?''于是客乃悟服。乐辞约而旨达,皆此类。
\end{quote}

乐令:乐广。 旨不至:语出《庄子 -
天下篇》,这是战国名家的一个辩论题目,原文是``惠施以此为大,观于天下而晓辩者,天下之辩者相与乐之:卵有毛。鸡三足。郢有天下。犬可以为羊。马有卵。丁子有尾。火不热。山出口。轮不碾地。目不见。指不至,至不绝。龟长于蛇。矩不方,规不可以为圆。凿不围枘。飞鸟之景未尝动也。镞矢之疾,而有不行、不止之时。狗非犬。黄马骊牛三。白狗黑。孤驹未尝有母。一尺之捶,日取其半,万世不竭。辫者以此与惠施相应,终身无穷。''以上二十一个题目主要是讨论概念与事物的关系,我们就举``白马非马''为例子,马是一个抽象概念,白马是个具体事物,假如黑马是马,那么白马不是黑马,所以白马不是马;再深入一步,其实白马也是一个抽象概念,它和马的抽象概念也不是一回事,所以白马还不是马。再譬如``犬可以为羊'',
物事的名称由人而定,与实际物事并非一体,郑国人将未曾雕琢的玉叫``璞'',周人却将没有风干的老鼠肉叫做``璞'',换言之,玉石也可以为老鼠肉,所以犬可以为羊。当然,这种辩论有诡辩的味道,但学者们已经深入考虑名与实的关系,开始研究逻辑推理,是很发人深省的。据说,战国时辩论成风,但凡聚会,学子们便会选择一个命题,相互驳论以为乐事。市井之人,也会在亲朋遇合之时津津乐道于辩驳卵究竟有没有毛,鸡究竟是两脚还是三脚,不管结论如何,人们都会感受到辩论的快乐。
``指不至,至不绝''的意思大概是:人接触了某件物事,不能完全知道这件物事;即使为某件物事定下了名称,也不能完全知道这件物事的全部。《公孙龙子
-
指物论》中说,``物莫非指,而指非指。天下无指,物无可以谓物。非指者天下,而物可谓指乎?\ldots{}\ldots{}''这篇文章理解起来比较困难,不过大意总是具体是宇宙万物的真正核心,抽象概念从属于具体,而不是具体从属于抽象。
确:同``触'',碰撞、敲击。
乐广用麈尾柄敲击几案,说明对具体事物可以描述,能够达到;随后又拿走,表示描述不等于与事物溶为一体,实质还是不能真正达到。玄学家辨析``旨不至''的意思,其实质恐怕还是在讨论``言与意''的关系,这是玄学讨论的一个重要话题,乐广显然是持言不能尽意的观点。他这样脱离词句,用实际行动来解释理论,已经大有禅宗机锋的味道。
有个古代笑话,说一个渔夫划破了皮肤,找大夫看,医了又医,总是时好时坏,他就经常看,诊费自然是鱼。一天大夫外出,儿子在,一看病况,也开了一副药,渔夫病就好了,不再来了。大夫后来吃不到鱼,就奇怪,嘴里念叨,儿子得意地说自己把病人给治好了。大夫大怒:``你以为你老子不会治啊,老子是想吃鱼!''大夫治病,可谓深得``旨不至''的精髓。现在治癌症,医院里也是先化疗,再手术,接着中药,医个不绝,非把病人搞得倾家荡产,最后说:``回去想吃点什么玩点什么,就满足他把'',这也是``指不至,至不绝''。近些年来励志书籍、管理书籍遍布书肆,但你以为一书在手,天下我有,那就大错特错。北大许智宏大唱《隐形的翅膀》,固然当上了国子监祭酒,我辈唱个一万遍,还是照样``旨不至''。

\section{4.17}\label{section-196}

\begin{quote}
初,注《庄子》者数十家,莫能究其旨要。向秀于旧注外为解义,妙析奇致,大畅玄风。唯《秋水》、《至乐》二篇未竟而秀卒。秀子幼,义遂零落,然犹有别本。郭象者,为人薄行,有俊才。见秀义不传于世,遂窃以为己注;乃自注《秋水》、《至乐》二篇,又易《马蹄》一篇,其余众篇,或定点文句而已。后秀义别本出,故今有向、郭二《庄》,其义一也。
\end{quote}

这个故事说郭象剽窃向秀的《庄子注》,因为向秀注没有流传下来,我们只有在其他典籍中看到向秀注的片言只语,所以些学者对此有争议。我想,郭象参考向秀是必然的,但是不是像这个故事中说的那样``其余众篇,或定点文句而已'',还是像《晋书
- 向秀传》说的那样``述而广之'',恐怕要取决于文物出土了。
刘义庆说魏晋前注《庄子》就有数十家,不免夸张,不过魏晋以后,注庄子的上百家总是有的。我个人的感觉,《庄子》比《老子》还难懂,因为长,用词古奥,句式奇特,思想混乱,每篇文章开头都还好懂,看着看着就不明白了。当年学者刘文典就说过,:``古今真懂庄子者,两个半人而已。第一个是庄子本人,第二个就是我刘某人,其余半个\ldots{}\ldots{}还不晓得!''他在昆明跑警报,看沈从文也跑,就十分刻薄地说:``我刘某人跑为了庄子,学生跑是为了未来,你沈从文替谁跑啊!''我们看刘文典的《庄子补正》,主要是贯通字句来历上,没没有进行翻译解释,他老先生学识深厚,理解上没有问题,像我们这样的后学,即使买了《庄子补正》,还是看不明白。
我在《风行雁鸣录》的跋中讲过,有一天我突然想,庄子这样的写法,也许并不追求文章明确的中心思想,而是想到什么写什么。大凡读过《庄子》的都会对他敬佩不已,《庄子》确实天下无双,通篇都是神话、空话、大话、假话、重复的话,需要作者非凡的想象力,无比的胆魄和自由恣肆的胸襟,这是无法学的,所以后来再也没有见过像《庄子》这样的文章。即使是《庄子》中的《杂篇》,仅文章布局而言,已严整拘谨了许多。
按这个故事说,似乎向秀、郭象真正掌握了《庄子》的要领精神。但我们看郭象的书,似乎不是那么回事,而是郭象用《庄子》来阐发自己的思想,按朱熹的说法,《庄子》不过是郭象表达意见的一个由头和依托:``自晋以来,解经者却改变得不同,如王弼、郭象辈是也。汉儒解经,依经演绎。晋人则不然,舍经而自作文。''而钱穆讲:``必至郭象注庄,乃始于此独造新论畅阐自然主义,转用以解决宇宙创始,天地万物一切所从来之最大问题,澈始澈终,高举自然一义,以建立一首尾完整之哲学系统。就此一端言,郭象之说自然实有远为超越于庄老旧谊之外者。''

\section{4.18}\label{section-197}

\begin{quote}
阮宣子有令闻,太尉王夷甫见而问曰:``老庄与圣教同异?''对曰:``将无同。''太尉善其言,辟之为掾。世谓``三语掾''。卫玠嘲之曰:``一言可辟,何假于三!''宣子曰:``苟是天下人望,亦可无言而辟,复何假一!''遂相与为友。
\end{quote}

《世说》中这个故事的主人公是阮修和王衍,《晋书》中说是阮瞻和王戎,也许后者更可信一点,因为阮瞻的爹是阮咸,阮咸和王戎是竹林七友,提拔起来顺手。王衍和王戎是族兄,阮修和阮瞻是族兄。阮修的无鬼论很有名,后面我们会接触到。
圣教:圣人的教化之道,儒教。
将无同:大概是一样的吧,差不多吧。这种说法当然不准确,因为庄子经常编故事拿孔子开涮,怎么会一样呢?但是阮瞻的口气本身就不坚定,比较委婉,说明区别还是存在的,只是大道本来就有多重性,儒道也许殊途同归。当然,孔子和老庄本身有共同点,他们的复古主义思想,``道不行,乘桴浮于海''的出世想法基本是一致的,但在魏晋时期,这个同一性并不是重点,而是当时一些人在对待礼法秩序上已经出现严重的对立做法。在魏晋时期,儒学落实为名教,老庄落实为自然,像嵇康、阮籍认为名教是``君立而虐兴,臣设而贼生。坐制礼法,束缚下民。欺愚诳拙,藏智自神。强者睽视而凌暴,弱者憔悴而事人。假廉而成贪,内险而外仁'',于是采取惊世骇俗的``越名教而任自然''作风,而葛洪抨击这种名士作风是``宾则入门而呼奴,主则望客而唤狗。其或不尔,不成亲至,而弃之不与为党。\ldots{}\ldots{}终日无及义之言,彻夜无箴规之益。诬引老庄,贵於率任,
大行不顾细礼,至人不拘检括,啸傲纵逸,谓之体道'',于是也出现山涛、王戎这样的人。他们采取揣着明白做糊涂、偶尔露峥嵘的作风,试图调和这种矛盾,那么在理论上就是以道释儒,解释为``名教出于自然''、``名教即自然''。像郭象所说的``任其天性而动,则人理亦自全矣''。三国时候的王昶就教导他的孩子要``尊儒者之教,履道家之言'',南北朝时像13岁就``通《孝经》、《论语》、《毛诗》、《尚书》''的李谧隐居不出,并作《神士赋》直接说:``周孔重儒教,庄老贵无为。二途虽如异,一是买声儿(都是博取声誉的手段)。''甚至做官其实也是隐士,东晋邓粲说:``夫隐之为道,朝亦可隐,市亦可隐,隐如在我,不在于物。''这些都是``差不多''理论的实践者,所以王戎一听阮瞻的话,立刻结合自己的人生态度,``咨嗟良久''。
掾:属吏佐官。``正曰掾,副曰属。''
一言可辟:卫玠的意思或者是阮修应该说``同''就可以了,或者说王衍说``调''就可以了,大家不必演找到知音这场戏。
阮修回答卫玠的话可能是奉承:像我这样靠言辞得当而当了个小官,可你是``天下人望'',不用言辞照样是达官显贵。

\section{4.19}\label{section-198}

\begin{quote}
裴散骑娶王太尉女。婚后三日,诸婿大会,当时名士,王、裴子弟悉集。郭子玄在坐,挑与裴谈。子玄才甚丰赡,始数交,未快。郭陈张甚盛;裴徐理前语,理致甚微,四坐咨嗟称快。王亦以为奇,谓诸人曰:``君辈勿为尔,将受困寡人女婿。''
\end{quote}

裴散骑:裴遐,字叔道,任散骑郎。河东裴氏裴楷的侄子,王衍的四女婿,在八王之乱中遇害。豪门之间联姻,意义颇大,升官发财时可以互相照顾支援,有矛盾时可以缓和彼此的关系,遇事时可以利用子女走动暗通款曲,大难时可以保留火种,凡此这种,都可以上抗皇权,下凌百姓。所以在唐高宗时,为避免前朝门阀大姓为家族利益而目无国家的教训,禁止七大姓之间的内部通婚,不过这一击并非有力,也不可能得到彻底贯彻。真正使世家豪门走向末路的是宋代科举制度的全面展开,官员的上升渠道主要来自于蟾宫折桂,还有就是唐末农民起义的``天街踏尽公卿骨'',``兰省诸郎皆鼠魅'',豪门的大庄园经济时代结束,他们不再具有经济上的独立性。
婚后三日:按风俗,婚后三日出嫁的女儿要回娘家住几天,俗称``回门'';不过各地风俗并不一样,像唐代诗人王建说:``三日入厨下,洗手作羹汤。
未谙姑食性,先遣小姑尝。''说明在唐代的一些地方,三日后新妇开始承担家庭主妇的责任,入厨做饭了。
诸婿大会:王衍的大女婿是太尉贾充的外孙、皇后贾南风的外甥贾谧,二女婿是惠帝司马衷的太子司马遹,据说大女儿比二女儿美貌,因为贾后仇视司马遹,故意让贾谧娶王家大姐。司马遹被废后,王衍就要求女儿和司马遹离婚,以避免祸害。王衍的三女婿不详,四女婿裴遐。我国婚丧嫁娶节庆的繁复礼仪活动,是增加彼此联系的重要纽带。
王、裴:西晋的第一等豪门是王、裴两家,刘义庆在《品藻》中称之为``八王、八裴''。在座的可能有我们认识的一大堆熟人:借钱不还的裴楷,崇有论的裴頠,王戎、王导、王敦等等。
郭子玄:郭象。 丰赡:赡就是丰,丰富。
未快:这句话不好理解,或当作``未决'',没有决出胜负。
徐理前语:慢条斯理地阐述前面的论点。清谈不仅仅是知识、思维、口才的竞赛,而是一种比较全面的审美活动,相貌好、风度好、语气好等都是让人赞美的法宝,裴遐清谈很有诱惑性,``辞气清畅,冷然若琴瑟。闻其言者,知与不知,无不叹服。''所以像岳父王衍这样信口雌黄的小人,只要``神情明秀,风姿详雅'',即使``矜高浮诞'',大家照样是``敬慕仿效''。微:精妙。《尚书?大禹谟》:``人心惟危,道心惟微,惟精惟一,允执厥中。''
``君辈勿为尔''句:你们不要再继续辩论和赞叹了,不然你们都会被我女婿所难倒的(困是困扰、难倒的意思)的。从这里看,``寡人''在西晋的时候,还不是君王的专有名词,像王衍这样的朝廷大臣也可以自称的。  

\section{4.20}\label{section-199}

\begin{quote}
卫玠始度江,见王大将军。因夜坐,大将军命谢幼舆。玠见谢,甚悦之,都不复顾王,遂达旦微言,王永夕不得豫。玠体素羸,恒为母所禁.尔夕忽极,于此病笃,遂不起。
\end{quote}

卫玠的母亲出身太原王氏,卫玠渡江以后,就去投靠琅琊王敦。当时有句俗语``王家三子,不如卫家一儿'',就是说太原王氏,琅琊王氏有三个杰出的儿子,但还是比不过卫家的一个卫玠。``中兴名士第一''的卫玠光临,自以为也是清谈好手的王敦连夜讨教,还拉来了谢鲲作陪,彻夜谈玄。
命:这里应该是招来的意思。
谢幼舆:谢鲲,字幼舆,见2.46谢豫章,他的名和字的关系不好理解。谢家本来是个小户士族,因为谢鲲,谢裒两兄弟后来取得高位,终于逐渐成为东晋第一流的门阀。谢鲲也是因为躲避战乱到南昌,当时在王敦手下任长史。谢鲲生活比较放浪,常常参加所谓中朝``八达''的``无遮大会'',这是当时的名士做派,风气使然。但他的学问佳,修养深,宠辱不惊,很得当时的好评。
豫:通``与'',参与。两位客人一讨论精妙的玄理,立刻目无余物,自顾自地滔滔不绝,通宵达旦,主人王敦虽然插不上话,但看来他是能听懂的,不然也撑不下去。后来王敦就评价这次聚会是``想不到在永嘉末年,我仿佛穿越到了正始年间。倘若何晏还在的话,也会因此而叫绝佩服''。
极:疲劳。
夜坐、达旦、永夕、尔夕,这同义变化的几个词反复强调玄谈的劳动强度大,终于导致古往今来第一美男子死在自己的嘴上,不知道历史上还有没有``说死''的其他事例。
卫玠看到中原大乱,就决意举家南迁以避祸;到了王敦这里,就发觉王敦野心勃勃,于是即使生病也要继续迁居南京,可以称得上智慧高,见识远。可惜他从小身体羸弱,加上生活颠簸流离,南京还没去成就去世了,年仅27岁。10多年后,根据他生前的愿望,朝廷组织把坟墓迁到南京,丞相王导说:``此君风流名士,海内所瞻,可修薄祭,以敦旧好。''

\section{4.21}\label{section-200}

\begin{quote}
旧云,王丞相过江左,止道声无哀乐、养生、言尽意三理而已,然宛转关生,无所不入。
\end{quote}

三理:主要见于嵇康写的三篇文章:《声无哀乐论》、《养生论》和《言不尽意论》,前两论现存,后一论散佚,但晋代欧阳建著有《言尽意论》。如果王导私淑嵇康,那么``言尽意''少一``不'',还有种可能就是王导不赞成``言不尽意'',故略``不''字。
声无哀乐不仅仅局限于音乐,从嵇康的文章看,``声''包括一切音响,嵇康认为声音是自然界的客观属性,本身并无情感,就像泪水一样,本身并不代表欢乐或悲伤,而只是来自人眼睛中含有盐分的水罢了,它可以是欢笑的泪水,也可以是悲伤的泪水。据现在的理解,嵇康的观点多为诡辩,但它有个背景,儒家认为礼乐可以治国,把音乐的教化作用无限放大,嵇康则不赞同这种观点,秉承老子说的``五音令人耳聋'',所以说的比较尖锐一些,他本人其实是优秀的音乐家,不然也不会在临死时奏曲广陵散。但王导赞同声无哀乐,恐怕有另一番含义,只是他的言论书中并没有记载。也许他赞同嵇康在《声无哀乐论》中提到的不刻意教化,``崇简易之教,御无为之治'',社会自然就达到``不期而信,不谋而成,穆然相爱''。
在嵇康的养生论中,他的主要意思是人有``性动之欲''和``智用之欲'',养生要顺应``饮食男女''这些生理本能和法则,而不要去追求名利功业富贵荣辱等外在的价值,出于健康和长寿的愿望而自愿节制欲望可取,而通过礼教法律等遏制人的欲望其实不能除去人的嗜欲。用通俗的话说就是,你宣扬烟酒的危害来要求人们禁欲很难实现,只有当人们真正从内心感受中认为平淡是美味才获得了健康。晋代是大庄园经济,豪门士族往往穷奢极欲,可能王导想提倡清心寡欲来引导社会风尚,他自己也做到了``仓无储谷,衣不重帛''。嵇康还提出,养生要``爱憎不栖于情,忧喜不留于意'',做到性情平和,精神安泰,这需要很高的修养。
言意之辨是一个重要的语言学论题,言辞确实无力和有很大的局限性,但事物和看法必须主要依靠言辞来表达。黑格尔在《哲学史讲演录》里说:``语言在本质上只是表示那一般的普遍观念;而人们所指谓的东西却是特殊者、个别者。因此人们对于自己所指谓的东西,是不能在语言中来说的。''《庄子》中有大量的故事来说明言与意的隔膜,``道不可言,言而非也'',就连孔子也说:``书不尽言,言不尽意'',子贡也说:``夫子之文章,可得而闻也;夫子之言性与天道,不可得而闻也。''有人认为,言辞不能真正反映事物,是因为我们对客体和主体的认识能力还不够。随着理性知识的发展,主体认识能力的提高,终能揭示出客体的真理,至少会向真理靠近了一步,我们必须言说。有人认为,客体实在超出了我们主体理性认知所能把握的能力之外,并不具备表达客体特点的能力,我们对自身主体也不真正掌握,客体真理就永远在彼岸,无法真正言说。
宛转关生:``关生''不好理解。我想,这三个问题涵盖面很广,王导的观点曲折暧昧,言辞东拉西扯,态度变化多端,好像是这个意思,好像未必是这个意思,有时候是这个意思,有时候又不是这个意思,所以无孔不入,无所不包。
文章中``旧云''很有趣,为什么加上这个词,作者拿不定主意?

\section{4.22}\label{section-201}

\begin{quote}
殷中军为庾公长史,下都,王丞相为之集,桓公、王长史、王蓝田、谢镇西并在。丞相自起解帐带麈尾,语殷曰:``身今日当与君共谈析理。''既共清言,遂达三更。丞相与殷共相往反,其余诸贤略无所关。既彼我相尽,丞相乃叹曰:``向来语,乃竟未知理源所归。至于辞喻不相负,正始之音,正当尔耳!''明旦,桓宣武语人曰:``昨夜听殷王清言,甚佳。仁祖亦不寂寞,我亦时复造心;顾看两王掾,辄翣如生母狗馨。''
\end{quote}

殷中军:殷浩殷扬州,前面已经多次谈到这个东晋清谈的名士,他后来当过中军将军、都督扬豫徐兖青五州诸军事,是简文帝用来牵制桓温的重要棋子。他早年受到执政庾亮的赏识,任命为记室参军,累迁司徒左长史(参谋长)。王导和庾亮接受遗诏共同辅佐晋成帝司马衍,庾亮是司马衍的舅舅,排挤了王导,后因庾亮执政引发了苏峻、王敦等叛乱,庾亮外放,任江、荆、豫三州刺史等,坐镇武昌。这次殷浩代表庾亮回朝廷述职,因为武昌在南京的上游,所以叫下都。殷浩小王导有20多岁,但从小以善玄学而知名,王导见猎心喜,举办集会邀请他清谈。在座的还有几个与殷浩年纪相仿的青年,像枭雄桓温、览镜自夸的王濛(简文帝时任司徒左长史),不会剥鸡蛋的王述(后封蓝田侯),谢安的哥哥谢尚(字仁祖,后封镇西将军),他们大约都30来岁。
``自起解帐''句:余嘉锡的解释是,麈尾悬于帐带,故自起解之。《御览》七百三引《世说》曰:王丞相常悬一麈尾,着帐中。及殷中军来,乃取之曰:``今以遗汝。''也就是说,谈天的房间里有帷幕,而麈尾挂在帷幕上。王导很赏识殷浩,就把心爱的麈尾送给了殷浩。
身:我。老身。 往反:问答。后面的4.45中``往反多时,林公遂屈。''
略无所关:基本上没插上话。是啊,不然王导早把麈尾送给在座的下属了。
彼我相尽:彼此尽兴,畅所欲言。
王导的感叹大意是说:``从前我和别人谈玄理,往往云山雾罩,不知所踪。和殷浩谈玄,做到了逻辑清晰,言辞理论不会出现矛盾。正始年间的谈玄,正是这样的。''这是把自己和殷浩比喻为何晏、王弼。前面说过,王导精于``三理'',而东晋初年名士谈玄,其实不如正始年间何王的思辨能力和逻辑严密。西晋的裴頠也对此有所阐述,他认为大多数名士谈玄,只有形式没有内容,没有什么思想的激荡和火花,``唱而有和,多往弗反(只有提出观点和别人迎合,没有实质性的问答辨析)''。
``辄翣如生母狗馨''疑有误文,其文不可解,其意当属贬损。桓温认为,殷浩的玄学水平在王导之上,故称殷王。谢尚和他自己有会于心,欣然自得,而王导的两个下属王濛、王述,根本就不是那块料。
这次聚会给桓温、王濛、王述三人留下了深刻的影响,以至于在日后,他们一想起王导对殷浩的推崇,就一个劲地说殷浩的好话,这些誉美之词,我们在《世说新语?赏誉》中经常可以看到。

\section{4.24}\label{section-202}

\begin{quote}
谢安年少时,请阮光禄道《白马论》,为论以示谢。于时谢不即解阮语,重相咨尽。阮乃叹曰:``非但能言人不可得,正索解人亦不可得!''
\end{quote}

阮光禄:焚车的阮裕。
白马论:就是战国时的名家公孙龙的白马非马论。公孙龙认为,马是一般属性,白马是个体属性,白马是一切马所共有属性加上个体颜色的属性,所以白马非马。白马非马的大意不难领会,可谢安当时还是孩子,看不懂公孙龙拗口的文章,也一时看不懂阮裕的阐述,所以反反复复询问以求完全掌握。
人之患在于好为人师,得英才而教育之,进而成为忘年之交,就更加喜悦了,在阮裕心中,有一个小朋友知己(索解人)很难得。可在后面的一个故事却耐人寻味。阮裕比谢安和王羲之岁数要大很多,有一天王羲之和谢安去看望他,王羲之一向尊重阮裕,就跟谢安强调,我们要推重阮裕。谢安却说,推重别人是最难的。

\section{4.25}\label{section-203}

\begin{quote}
诸季野语孙安国云:``北人学问,渊综广博。''孙答曰:``南人学问,清通简要。''支道林闻之,曰:``圣贤固所忘言。自中人以还,北人看书,如显处视月;南人学问,如牖中窥日。''
\end{quote}

诸季野:皮里春秋的褚裒,国丈,见1.34.
孙安国:孙齐由、孙齐庄的老爹孙盛,历史学家,见2.49。
这条是东晋人谈南方人和北方人的区别,不过褚裒是河南禹县人,孙盛是山西平遥人,按现在的观念,他们都是北方人,这个话题有点莫名其妙。唐长孺说:这里的南北应该是指黄河的两岸,不是以长江为界限的。褚裒是在夸孙盛,孙盛回应赞扬褚裒,其实就是两个北方人在彼此推重,和现代意义上的南北论无关。那么养马神骏的支道林是哪里人呢,他也是河南人,他接着阐述了中原地区黄河两岸人做学问的区别。所以这个话题和后来经常讨论的南方北方人没有关系。不过令人纳闷的是,山西河南两省相邻,同属黄土高原,风土相近,难道一个追求广博,一个追求精深,做学问有这么大的区别?不过当时没有省份的说法,而是分州郡县三级管理制度,河南禹县属于豫州颍川郡,山西平遥属并州太原郡,两地相距1000多里地,古代交通不发达,我们也可以理解这种小范围的南北之分。我们记得在1.26介绍的中,有另一则反映黄河两岸人士互相斗嘴的故事:``祖纳和钟雅聊天,钟雅说:我们河南人,锋利如锥;你们河北人,愚钝如槌。''其实这句话的意思不过是渊综广博和清通简要的另一种说法,锥自然专一,槌自然大气。
渊综:像深渊一样汇聚河水,就是博采众家的意思。有人说是精深的意思,这种解释含糊不确。
清通:在《世说新语?赏誉》中,钟会说:``裴楷清通,王戎简要。''从《赏誉》中理解,清通和简要应该有所区别;但从这一则中,清通就应该是简要。广博则繁杂,简要则清澈。
支道林的解释很有意思,他并完全不认可两人的说法,他认为圣贤的学问,都是广博和简洁的综合体,但圣贤以下的人(中人应该是掌握中庸之道的圣贤的意思),确实有南北的区别(他当然不在这个行列),河北的人是看月亮,河南的人是看太阳,看太阳的人当然高了一筹,不过两人站得地方不一样啊,河北的人看月亮在显阔之处,而河南的人看太阳在窗户里面。一开阔,一狭隘,这样大家又拉平了。
我们应该注意到,在《世说新语》中,关于人物评价的原则就是不分高下,各看其所长,言辞圆滑。在《世说新语?品藻》中,我们可以见识到这种大量的和稀泥的例子,这是口不臧否人物的另一种变通方法,其中也许受到老庄是非泯灭观的影响和残酷政治斗争的教训。
利用这个故事,其实我们还可以引申讨论南方人和北方人的区别,不过决不批评人是古老的生存智慧啊。

\section{4.26}\label{section-204}

\begin{quote}
刘真长与殷渊源谈,刘理如小屈,殷曰:``恶卿不欲作将善云梯仰攻?''
\end{quote}

刘真长:不祭祀淫祠的刘惔,天之自高的刘惔,见1.35。
殷渊源:殷浩,他在东晋初玄理第一,刘惔不是对手。殷浩的话不可解,疑有错字衍文,但总的意思估计是,你只有造云梯来强攻了,但也会徒劳无功。不过在后来,刘惔做了驸马,气场上升,后来居上,``我辈才是第一流人物''(《世说?品藻》),目中再无殷浩,在4.33中,一切都反了过来:``殷中军尝至刘尹所清言。良久,殷理小屈,游辞不已,刘亦不复答。''

\section{4.27}\label{section-205}

\begin{quote}
殷中军云:``康伯未得我牙后慧。''
\end{quote}

  韩伯韩康伯是殷浩的外甥,他的故事我们在前面已经接触过一些,有时也叫他韩豫章。
  拾人牙慧的出典就是这里,但牙慧这种说法比较奇怪,有人说是语意的错字。也有人说牙慧就是片言只语的意思,未得牙后慧就是一窍不通。这里殷浩大概是说韩伯的玄谈水平不高,没有真正领会殷浩玄谈的精髓(我倾向于牙慧是语意的错字,而不是一窍不通,因为前面我们知道,韩伯的玄学水准是好的,以至于庾道季说,韩伯来和我谈玄的话,我就要济河焚舟来对付他)。

\section{4.28}\label{section-206}

\begin{quote}
谢镇西少时,闻殷浩能清言,故往造之。殷未过有所通,为谢标榜诸义。作数百语,既有佳致,兼辞条丰蔚,甚足以动心骇听。谢注神倾意,不觉流汗交面。殷徐语左右:``取手巾与谢郎拭面。''
\end{quote}

谢镇西:谢尚谢仁祖,谢鲲的儿子,谢安的哥哥,当过镇西将军。谢尚三十几岁以后长期任高官,最后出镇一方,曾经都督豫、冀、幽、并四州军事,谢家由此成为第一流的门阀。谢安能在东山隐居20年,主要就因为有谢尚这个顶梁柱,不用他为家族劳心。
谢尚早慧,多才多艺,音乐、舞蹈、诗歌、书法都名噪一时。世说新语后面有个故事,他8岁的时候被家中的客人恭维为颜回,他应声回答:在座的没有孔子,哪里能说我就是颜回啊!显得机敏而实在。
未过有所通:此句很难理解,删去也不损害文意。如果硬要解释,过也许是衍文,``未有所通'',没有精讲(通大概是做一次通彻的演讲,因为后面多次用到这个``通''字。``谢看题,便各使四坐通'',``支道林先通''之类的)。大意是说殷浩没有全面阐述自己的玄学思想,介绍各个知识点(殷浩最拿手``四本论'',
但估计没有全面辨析才与性的合同离异),而是了了讲了几百字。(这里的断句可以讨论,``未''是管到``通''还是管到``标榜诸义''?我认为可以理解为殷没有标榜诸义。但``作数百语''前最好加个``而''字。)
佳致:情趣高雅。 辞条:不是指词条,而是指辞藻。
甚足以动心骇听:甚足以的用法比较少,但不等于没有这样用的。骇听就是振聋发聩的意思。
流汗交面:暑假期间听专题辅导课,殷老师讲得好,谢尚听得入神,陷入了长久的思考,满头大汗而不自觉。
谢尚由此与殷浩交好,他后来说:``殷浩不起,当于苍生何!''俨然把殷浩的沉浮作为东晋强盛的关键。不过从史实上看,殷浩辜负了谢尚这种厚望。东晋的命运其实是寄托在小他12岁的弟弟谢安身上,而不是大他3岁的殷扬州。

\section{4.29}\label{section-207}

\begin{quote}
宣武集诸名胜讲《易》,日说一卦。简文欲听,闻此便还,曰:``义自当有难易,其以一卦为限邪!''
\end{quote}

宣武:桓温。按中国人的看法,领导一定是全能型选手,他玄学水平不高,老是被名士看不起,始终是一块心病,于是知耻而后勇开始补课,结果还是被清谈中二三流人士简文帝耻笑:一看你的教学进度就是糊弄人的,平均使力,难易不分,毫无重点。
名胜:名士胜士。南宋末训诂学家胡三省说:``江东人士,其名位通显於时者,率谓之佳胜、名胜。''
易:《易经》,三玄之一。易经是一本占卜算命的书,64卦卦辞很简朴含混,简朴含混则多义。我们知道,清谈是从评价个人品行开始的,后来逐渐发展到试图认识命运(易),认识宇宙和选择人类正当生活的方式(老庄),这是一个从具体到一般,感性到抽象的过程,显然是一种进步,是理性觉醒的体现,它反映了人试图把握自己的命运,做出正确选择的一种努力。易经虽然是打卦和占卜,用现代的眼光看当不得真,但古人非常确信老天爷总会网开一面,给当下投下未来的阴影,指引前进的道路;但这片小小的未来阴翳,是那么飘忽不定、难以琢磨,往往只有事后印证才能恍然大悟,所以要广泛考察、深入研究和反复论证,这样才能把握人生。譬如说,八卦开始无非代表天地山泽风雷水火,但人心中的疑问有那么多,这简单的八个事物不能涵盖,于是代表男人或者天的乾卦也象征金啊玉啊冰啊这些坚硬的东西,甚至代表旗子、马匹、果子这些重要的东西,这还远远不够,汉代人还补充了很多``逸象'',什么高大威猛,乐善好施,神龙老虎等等,用这种方式来论证占卜的正确,但这种象征要自圆其说难度确实大,大家就得白首穷经,闲掷光阴。
其:同``岂''。

\section{4.30}\label{section-208}

\begin{quote}
有北来道人好才理,与林公相遇于瓦官寺,讲《小品》。于时竺法深、孙兴公悉共听。此道人语屡设疑难,林公辩答清析,辞气俱爽,此道人每辄摧屈。孙问深公:``上人当是逆风家,向来何以都不言?''深公笑而不答。林公曰:``白旃檀非不馥,焉能逆风!''深公得此义,夷然不屑。
\end{quote}

道人:僧人。他是故事中的配角,辩论多次被``摧屈'',结果连名字都没机会留下来。
才理:才思。``楼玄清白节操,才理条畅'',``钟士季精有才理''。
林公:支遁支道林。竺法深:竺道潜。孙兴公:孙绰。
小品:《小品般若波罗密经》是最早传入中国的大乘佛教典籍,因为是节略本,所以叫``小品''。
刘孝标在后面的故事中注释说:``释氏辨空经有详者焉,有略者焉,详者为《大品》,略者为《小品》''。363年,竺法深应皇帝邀,到南京设御筵,开讲《大品》。这个故事大概就发生在那个时候。
此道人语:``此''或当作``北'',不然文章两次使用``此道人'',用词显得累赘。``语''可以省略。``北道人屡设疑难'',``北道人每辄摧屈''形成对比。
爽:豪迈。
逆风家:名彻十方之人,``家''或当``香''。佛典《出曜经》中说:夫世间诸华香,尽顺风香,不逆风香。戒德之香亦逆风香,亦顺风香。世间华香齐熏欲界,不熏色界。或直熏一方,不熏三方。持戒之香,香彻十方。华香逼近乃别。持戒之香上彻一究竟天。是故说曰:华香不逆风,德人遍闻香。这段话大意是说一般的香是顺风香,而德行之香(持戒之香)是顺风香,也是逆风香。《俱舍论》中说:``城外东北有圆生树,是三十三天受欲乐胜所。盘根深广,五逾缮那。耸干上升。枝条傍布,高广量等百逾缮那。挺叶开花,妙香芬馥。顺风熏满百逾缮那。若逆风时,犹遍五十。''《成实论》中说:``波利质多天树,其香则逆风而闻。''《别译杂阿含经》中说:``一切树中波利质多罗为第一。''所以这句话大意是说,孙绰对法深说:大和尚你也是识见高超,声名远播之人(辩论中每每获胜的人),这场辩论以来怎么一句话都没有说?
白旃檀:檀香木的一种,这种香木``唯能随风,不能逆风''。支道林非常自信而直率,我才是像三十三天上独一无二的波利质多罗树那样的逆风香,法深不过是俗世中的顺风而香的檀香木罢了。
竺法深听懂了支道林的潜台词,他比支遁大近30岁,平静而不屑于反驳。
4年后,支道林去世,11年后,竺法深去世。《世说新语》中,记录竺法深的言行六、七条,记录支道林的言行有五十几条,大约就是因为支的锋芒。

\section{4.31}\label{section-209}

\begin{quote}
孙安国往殷中军许共论,往反精苦,客主无间。左右进食,冷而复暖者数四。彼我奋掷麈尾,悉脱落满餐饭中。宾主遂至莫,忘食。殷乃语孙曰:``卿莫作强口马,我当穿卿鼻!''孙曰:``卿不见决鼻牛,人当穿卿颊!''
\end{quote}

许:地方、场所。《五柳先生传》:``先生不知何许人也,亦不详其姓字。''
往反精苦:往反即往返,精苦费解,``精''或作``相'',
4.22中说,``丞相与殷共相往反'',与``无间''对应:来来回回互相辩难使之困苦。
客主无间:清谈中,一般就像苏格拉底一样,你问我答,层层逼问。而这次是互相折难,时为主客。到后来大家恼羞成怒,不但互相称之为``卿'',而且搞物理攻击,投掷麈尾。(也有人推测,麈尾代表这清谈中的提问方,谁拿着麈尾,谁就发问,所以大家要枪。)刘孝标注释说:当时殷浩是第一流的辩手,前面介绍过,``能与剧谈相抗者,为盛(孙盛)而已。''
莫:即暮。
强口马:比喻嘴硬。殷浩一时情急说错了,马不犟,牛犟。马穿鼻子没用,牛才要穿鼻子。所以孙盛乘机反击:你知道的也有限啊,没见过牛才要穿鼻。你这头犟牛,穿鼻子没用的话,就当心你的脸颊被我也穿了。
殷浩是东晋是第一流的名士,他曾隐居近十年,当时人就已经把他视为管仲、诸葛一样的人物。他也确实学问精深,思致杰出,``朝野推服'',
足以``领袖群伦'',《世说新语》和刘孝标的注中,记录殷浩的足足有八十来条,都能和王导、谢安相媲美,甚至王导把心爱的麈尾送给了殷浩,以示推许之意。但殷浩没有学到王导政治上的和稀泥,他玄理学问好,个人操行高,责任感强,这样的人政治手腕往往过于理想主义。他出镇一方后简单地按照简文帝的要求,急于北伐,以打压桓温,结果兵败名裂,常常以书写``咄咄怪事''发泄心中的不甘,没有一个智者应有的结局。

\section{4.32}\label{section-210}

\begin{quote}
《庄子》``逍遥篇'',旧是难处,诸名贤所可钻味,而不能拔理于郭、向之外。支道林在白马寺中,将冯太常共语,因及《逍遥》。支卓然标新理于二家之表,立异义于众贤之外,皆是诸名贤寻味之所不得。后遂用支理。
\end{quote}

旧是难处:从来就是难以理解的篇章。是啊,我到现在还不敢说懂``逍遥游''的意思。逍遥游中没有逍遥两字出现,游倒是有的:``若夫乘天地之正,而御六气之辩,以游无穷者,彼且恶乎待哉!故曰:至人无己,神人无功,圣人无名。'',``游于无穷''大约是浮游于宇宙天地之间的意思。从构词法来看,逍遥的本义应该是渐渐远去的意思,如王夫之就说:``逍者,向于消也,过而忘也;遥者,引而远也''。那么我们可不可以理解为,``逍遥游''就是忘我地融合在天地之间的意思。不过《庄子》外篇``天运''中对逍遥有个解释:逍遥,无为也。结合老子和庄子的综述,无为总是顺应天道,一切顺其自然,不以自我为中心,无事功之心。那么我们按照这种解释,逍遥游就是无为而至于无几、无功、无名,顺应天地而沉浮。
但后人对逍遥的理解就是自由。我想,无为和自由总有点区别。无为主要是一种生活方式,自由是一种精神状态。庄子的自由也不是常人理解的自由,不是你想干什么就干什么,而是干什么都不要被所做的事物束缚住,而是要``忘''。用个《挪威的森林》里的比方,你想抽烟的时候能抽个烟得到精神上的放松,这并不是真正地自由,所以渡边彻戒烟了,理由就是``我不情愿被某种东西束缚住''。不过按照庄子的认识,真人根本不会被烟所束缚,他根本就不在乎抽不抽烟,因为他没有一般人的情感,但意志自由的时候,一切都不在话下。
据说郭象就抄袭了向秀对逍遥游的理解。在他看来,逍遥就是自由,就是各按本性。他说:无论是鲲鹏还是小鸟,他们各任其性,都得到了自己的自由,没有谁更胜一筹的意思。``夫小大虽殊,而放于自得之场,则物任其性,事称其能,各当其分,逍遥一也,岂容胜负于其间哉!''郭象的解释中,确实包涵着无为和齐物论的内涵,顺其自然,你有什么就是什么,事物没有高下之分。但这里有个问题:什么是自然,什么是顺其自然?饿了要吃饱是自然,饿了想吃鱼翅是不是自然呢?譬如在《高僧传》中记载了另一个故事,有一次还在白马寺,支道林与刘系之谈起逍遥游,刘以郭象的意思说:``万物各适其性,便是逍遥。''支道林立即反驳说:``不对。夏桀、盗跖都以残害生命为本性,照你的说法,他们也算是逍遥了。''刘系之哑口无言。
老庄一般认为``无''才是人的本性,但到了魏晋时候,士大夫人格觉醒,并不简单地赞同老庄。向秀认为人比草木鸟兽高明,就是有生命,有智慧,有情感,如果去掉这一切,那人就不成其为人了。``有生则有情,称情则自然。得若绝而外之,则与无生同,何贵于有生哉?且夫嗜欲:好荣恶辱,好逸恶劳,皆生于自然。''不过按《庄子?外篇?骈拇》的原意,桀跖不能算是随性,他们和尧舜伯夷一样,``残生伤性均也''。
英国诗人马洛说:世界上只有一种成功,那就是按自己的意愿度过一生。我想,这个意愿一定不包括无尽的嗜欲。所以为了批驳向秀的``自然说'',嵇康就把欲望分为``性动之欲''和``智用之欲'',吃饱是性动,吃鱼翅是智用,``故世之所患,祸之所由,常在于智用,不在于性动。''而支道林没有做嵇康那样的思考,直接把智用与性动划上等号,把恶当做桀、跖的本性。于是他对逍遥有另外一番解释。
他认为大鹏和知了、学鸠都没有做到逍遥,因为它们还``有待于物'',逍遥是指心的自由而不是某个具体事物的自由。大鹏不逍遥,因为它高飞要靠风靠翅膀,小鸟不自由因为它坐井观天,内心骄傲。真正的自由是能领悟并驾驭天地的自然之道,凭着自己的直觉生活,那么你就看着不怎么快,其实你已经自然而然达到彼岸了。``夫逍遥者,明至人之心也。庄生建言大道,而寄指鹏鷃。鹏以营生之路旷,故失适于体外,鷃以在近而笑远,有矜伐于心内。至人乘天正而高兴,游无穷于放浪,物物而不物于物,则遥然不我得;玄感不为,不疾而速,则逍然靡不至,此所以为逍遥也。''
支道林的解释是不是符合庄子的本意另当别论,但他纠正了向秀、郭象认为大鹏和小鸟都达到自由的观点,被各方接受了。
唉,像庄子这样的伟大的作家也许没有什么确凿的本意,他是想写到哪里就到哪里,其创作的随便和思想的矛盾处处可见,而我们这些平庸的人却以辞害意,过度阐释。
我想,庄子的文章应该主要是描绘圣人、真人的生活状况,他画了一个蓝图,并没有提出具体画图的技巧或者方法,而后人努力想达到圣人的境界,这是不是被庄子耍了?就像有个教练说,我见过一个人,他100米才5秒,他的肌肉状态如何,他起跑的姿势如何,他冲刺表现如何,等等。我们听了羡慕不已,就拼命打听训练方法。可是庄子他老人家就是不说。妈妈的,这会不会是个老骗子?

\section{4.33}\label{section-211}

\begin{quote}
殷中军尝至刘尹所清言。良久,殷理小屈,游辞不已,刘亦不复答。殷去后,乃云:``田舍儿,强学人作尔馨语!''
\end{quote}

殷浩和刘惔是东晋永和年间最有名的玄学家,他们交往密切,相互之间嘴上都不饶人。4.26中,殷浩自诩是墨翟,刘惔是鲁班;4.33中,刘惔自诩是高门贵族,殷浩成了土包子。刘惔祖上肯定比殷浩阔,他好歹是汉家宗室。我们也记得,刘惔路过扬州的时候,故意装出不敢夜行的样子,讽刺殷浩为政严厉。
田舍儿:农民或者是下等人。辛弃疾《水龙吟》:``求田问舍,怕应羞见,刘郎才气。''
尔馨:这样,这般。馨是当时的形容词、副词语尾,如今之``般、样''。
殷浩说自己是墨翟,刘惔是公输班,总带有开玩笑的意思。而刘惔直接讥讽殷浩是乡巴佬,就有点伤人了。刘惔性格直接,以门第看人是当时人的共识,《世说新语》中就记载时人对他的看法就是``性至峭'',就是高高在上,不容易接近。后来他想当会稽的太守,托谢尚向殷浩说关系。殷浩就直接拒绝了,说刘惔党同伐异,性格狭隘到了宽阔的地步(标同伐异,侠之大者)。是啊,刘惔才兼文武,见识高明,但以自己的聪明才华姿态高居于人上,自然不为常人之所喜。
你水平高,但不可出口伤人,更没必要和左右之人炫耀卖弄。

\section{4.34}\label{section-212}

\begin{quote}
殷中军虽思虑通长,然于``才性''偏精。忽言及``四本'',便若汤池铁城,无可攻之势。
\end{quote}

思虑通长:刘勰《文心雕龙》:``寂然凝虑,思接千载;悄焉动容,视通万里。''意思是说学者静静地入神思考,就会想象到千年以来的林林种种,就好像看见遥远以外的情景。
偏:徐震堮说偏就是特别的意思。现在我们说偏爱,也是特别。
殷浩特别精通``才性四本''之说,为当时之公论。一般认为,才就是才能,性就是道德品质。东汉的官员选拔制度本身就是认为才能是道德的体现,只有道德高的人才有行政的能力,而从曹操唯才是举令来看,他当然是主张离和异的。我们现在赞成的才性是两回事,不过要辩论起来可不像我们想象的那样简单,这个问题牵涉到文化、社会和词义背景等。譬如三国的卢毓认为,才是用以行善的,大才能为大善,小才能为小善。如果说某人有才,但他不能行善,只是说他的才不中用,也就是无才。再譬如,告子说食色性也,孟子说仁义性也,我们一下子哪里说的清楚。不过殷浩可能想的很透彻。
忽:当做``如''。

\section{4.35}\label{section-213}

\begin{quote}
支道林造《即色论》,论成,示王中郎,中郎都无言。支曰:``默而识之乎?''王曰:``既无文殊,谁能见赏!''
\end{quote}

《即色论》:大意是说物质现象的本质不在于它本身,因为物质现象没有本质,所以它的本质是没有自性,而是因缘聚会。所以物质现象实质是空,但又不是简单的空。这种东西我们没有必要搞的很清楚,只要简单了解一下佛教的缘起性空之说就可以了。
王中郎:北中郎将王坦之、王文度,曾与谢安齐名,同为晋孝武帝司马曜的辅政大臣。前面已经多处出现。但最有名的还是他出洋相的样子,当年他和谢安共赴桓温的鸿门宴,吓得汗出如浆,风度尽失。记忆就是这么古怪,家父顾我育我的往事我差不多都忘了,但他的一次洋相我却偶尔还拿来开玩笑。上世纪80年代,雪碧刚流行,有一次老头子带我吃西餐,服务员问要什么饮料。家父思索片刻,就问:``碧雪有吗?''我大笑:``爸,是雪碧!''
默而识之:《论语》孔子自谓:``默而识之,学而不厌,诲人不倦。何有于我哉?''据朱熹说,识就是志,记下来的意思。默而识之比后面两句要难理解一些,但大意总是只看不说,暗自记下,慢慢领悟。这里的意思是:你不发表意见,是不是很佩服我,已经在暗暗记下来了?
既无文殊:你不是文殊菩萨!这是一个佛教典故,源自佛经《维摩诘经》,释迦牟尼的学生维摩诘居士``辩才无双'',是``在家菩萨''。有一次维摩诘称病,佛祖欲派弟子探访,弟子们竟然畏惧维摩诘的辩才而不敢往。释迦牟尼只好派出``智慧第一''的文殊菩萨。会面后两人大谈了一通``色空之辩''等,最后文殊请教维摩诘什么是``不二法门''。维摩诘一句话不说,文殊就对其他菩萨赞叹说:``高明啊,高明啊,这才真是不二之门。不用一点文字、声音和观念才能解释不二法门。''
支道林是佛教信徒,用儒家的典故来提问,其中隐含的意思又很自傲,把自己的著作当做孔子都要学习的经典。王坦之主要是儒家学者,用佛教的典故来讽刺支道林:你没有文殊的智慧,怎么能理解我不发表意见的真实含义!我懂的比你多,我不说话;你写的狗屁文章来解释``色空之辩'',你根本就得不到修行的正果!
在这场较量中,王坦之完胜。

\section{4.36}\label{section-214}

\begin{quote}
王逸少作会稽,初至,支道林在焉。孙兴公谓王曰:``支道林拔新领异,胸怀所及,乃自佳,卿欲见不?''王本自有一往隽气,殊自轻之。后孙与支共载往王许,王都领域,(当删)不与交言。须臾支退。后正值王当行,车已在门,支语王曰:``君未可去,贫道与君小语。''因论《庄子?逍遥游》。支作数千言,才藻新奇,花烂映发。王遂披襟解带,留连不能已。
\end{quote}

王逸少:王羲之的字,``羲之''与``逸少''有什么关系,似不可解。因为王羲之太有名了,逸少后来借指杰出的年轻人。王羲之当过会稽内史,就是会稽太守。
孙兴公:孙绰,文学家,见2.84。王羲之比孙绰和支道林大10来岁,孙绰称王羲之为``卿'',也许说明在东晋和南北朝时期,卿不再是上对下的称呼,要么``乃自佳卿欲见不''的断句应该为``乃自佳卿,欲见不?''
拔新领异:创立新意,提出独特的见解。
本自和一往好像是同义反复,本来,一向。高适《燕歌行》:男儿本自重横行,天子非常赐颜色;《新唐书?狄仁杰传》:
山东之人重气,一往死不为悔;《世说新语?任诞》子野可谓一往有深情,都是一向的意思。但两个``一向''的同义反复,用词不合语法,``向来有一向的才气?''在《高僧传》中,王羲之开始也看不起支道林,就随口说:``一往之气,何足可言?''那么``一往''究竟是什么意思呢?``一往隽气''不好理解,但``一往之气,何足可言''却可以让人联想到曹刿论战的``一鼓作气'',也许``一往''就是``一股子''的意思。``王本自有一往隽气'',王羲之本来就有一股子才气。``一往之气,何足可言'',``支道林故作惊人之语,就是为了一下子夺人眼球罢了,就是个三板斧,我没有什么想和他说的!''
殊自:特别。 许:处,地方。
王都领域:领域不可解,应为衍文,当做``王都不与交言''。
于丹以心灵鸡汤解释《论语》而一炮打响,支道林是以佛教的空无学说讲解《逍遥游》而名噪一时。支道林讲的和当时对《庄子》的主流认识究竟有什么区别呢?
关于支道林眼里的``逍遥游''我在4.32中解释了一下,庄子写《逍遥游》本来就比较随意,不像《管子》、《韩非子》、《墨子》等那样思路清晰,而且内容是谈``三观''的,自不免有自相矛盾之处,给后世的阅读造成很大的障碍。当时大家比较接受向秀、郭象的解释,认为万物虽小大之别,但只有顺其自然,大鹏和小鸟逍遥皆同,推而及之,世间万物只要适性,无不逍遥。《逍遥游》中还提出万物``无所待''才逍遥,而向、郭提出虽然``无待''是自由的最高境界,但``有待者''只要``所待不失'',可以达到自由。像庖丁与尸祝,尧与许由,虽然职责不同,行为各异,但他们能各安所司,各静所遇,各得其实,都是自由的。像这样的说法,比较符合当时的社会环境,能够让人在严酷变乱的生活压力下安身立命,自适其乐。
而支道林认为,大鹏因躯体庞大,非海运不能举其翼,非到九万里高不能往南飞,非到南冥不能休息,哪里有什么逍遥可言?鴳鸟不能远飞而嘲笑大鹏飞得那么远,骄傲自满,是为内心所累,同样得不到逍遥。向秀、郭象所谓的适性逍遥,不过是追求一种低级的满足,而这种欲望实际上又是永远得不到满足的,因为当其所足之时,似乎已经得到天真快乐,但这好比饥者一饱、渴者一盈之时,并不能忘掉美酒佳肴呢!所以所谓的足性,远不是一种逍遥的境界。他进而指出,``夫桀、跖以残害为性,若适性为得者,彼亦逍遥矣。''那些历史上的王八蛋以残害社会为乐,如果适性就是逍遥,他们肯定是最逍遥了。庄子所谓的``适性'',是指圣人、至人的``适性'',你总不能把普通老百姓和那些王八蛋的``适性''当做
``逍遥''吧。
支道林接着抛出佛教的理论鸡汤,世界的本质是寂静空无,色就是空,不要纠缠于语言和追求庄子的真意,而是从本心和世界``无''的真相中认识庄子,这样你就看不到庄子理论的矛盾了,不在意自由,才是真正的自由啊。
烂:烂漫。支道林口吐莲花,天花乱坠。
披襟解带:解开衣裳,心情舒畅。支道林的逍遥学说可谓善解人衣啊。

\section{4.37}\label{section-215}

\begin{quote}
三乘佛家滞义,支道林分判,使三乘炳然。诸人在下坐听,皆云可通。支下坐,自共说,正当得两,入三便乱。今义弟子虽传,犹不尽得。
\end{quote}

三乘:佛教用语。佛教宣称,人因为根机不同,有三种成道的修行途径:声闻乘,听了释迦牟尼的宣讲而悟道。缘觉乘,因理解了人生的因缘而悟道。菩萨乘,不仅要自己了悟,而且又愿度一切众生,这个是大道。佛教的思想体系非常庞大,教人觉悟的法门就很多了,像禅宗针对知识分子,他们有人生的基本觉悟,又舍不下身边的一些事物,就不讲究出家的概念,而是直指人心,以顿悟入道;像净土宗,主要针对没有什么文化知识的人,他们很难领悟长篇大论,就可以执持一句佛号,念念不忘,久而能所两忘,证入一心不乱;律宗追求纪律中的觉悟,要求精持禁戒,举止语默就合乎规律,久之就妄心不生,事理圆融\ldots{}\ldots{}随着时代发展,佛教各宗派大体逐渐融合。不过我对佛教的前景并不看好,也许宗教的前景都不好,我也并不认为中国现今的最大的问题是信仰缺失问题。曾经在一次会议上,我怒斥一些同事的观点:你们说共产主义信仰虚无缥缈,是对少数人的要求;难道其他宗教信仰不虚无缥缈了?你们说有宗教信仰的国家道德水准高,那个王八蛋阿育王道德水准高到哪里去了?中世纪的宗教裁判所道德水准很高?十日谈里哪个神父是圣人?一战,二战,各国军队、日本人搞各式各样的迫害和屠杀,他们这些宗教信徒的觉悟在哪里?就看我们中国人,解放前总大体有各式各样的宗教信仰,谁敢说当时的道德比我们高?有人说改革开放前中国人道德水准高,你们凭良心说,这个是不是真的?现在世界上的哪个强国,你敢说它是因为有宗教信仰而强大起来的?我知识水平有限,只能跟大家提示一下,现代社会的道德水准上升,主要是因为二战的关系和利益的分配制度。从此以后,我们这个从事意识形态工作的部门再也没人在我耳边鼓噪信仰问题了。
佛家:佛教似乎是外来词,以前我国称佛教大概就是佛家。古文中的佛教,指的是释迦牟尼的发言,佛宗指的是释迦牟尼弟子的发言。
滞义:疑难。杨慎《云龙风虎》:``古人多倒语成文,后人不达,便成滞义。''
分判:分析。
正:或当做``止'',仅仅。《高僧传?支遁传》``凡在听者,咸谓审得遁旨,回令自说,得两,三反便乱。''听的时候大家都说已经听懂了,但是支道林讲完后,大家各自谈学习体会,前两个意思已经疏通,到了第三个,就讲不清楚了。支道林讲课没有教材,不发讲义,平时演讲据说``每至讲肆,善标宗会,而章句或有所遗,时为守文者所陋'',学生用毛笔,也没学过速记,当然做不了数千言的笔记;做了点笔记,和书籍一对,还居然对不上,自然有些搞不明白了。
前些天看过一个佛教徒``燃指供佛''的视频和典籍,说实话,这样的佛教让人恶心,某些信奉佛教的,下拔舌地狱绰绰有余。

\section{4.38}\label{section-216}

\begin{quote}
许掾年少时,人以比王苟子,许大不平。时诸人士及于法师并在会稽西寺讲,王亦在焉。许意甚忿,便往西寺与王论理,共决优劣,苦相折挫,王遂大屈。许复执王理,王执许理,更相覆疏,王复屈。许谓支法师曰:``弟子向语何似?''支从容曰:``君语佳则佳矣,何至相苦邪?岂是求理中之谈哉!''
\end{quote}

许掾:许询许玄度,曾被召为司徒掾,不过他是隐士,没上任,其人见前《言语》69、73等。他是《世说新语》中的大人物,关于他的故事大约有20几条,也许是没有拿得出手的作品和成就,后世没有很大的名声。当年浙江的萧山不叫萧山,叫永兴,改名萧山有两个说法,一个是越王勾践兵败,逃到绍兴边上,驻扎在一座山边喘口气,四顾残兵败将,心中萧然,故称此山为萧然山。一说是许询隐居永兴边上的一座山,``凭林筑室,萧然自适'',取该山名为萧然山,后世歌咏许询说:``高栖不受鹤书招,北干家园久寂寥,明月室怀人姓许,故山犹是岫名萧。''勾践的萧然是萧索凄然,许询的萧然是洒脱自然,就像网友给外国人出题一样:``冬天,有多少穿多少;夏天,有多少穿多少'',没有语言背景,汉语有些意思就不太好懂。
王苟子:王修王敬仁小名狗子,太原王氏,他父亲是顾影自怜的王濛,王修据说早慧,才气逼人,所以把他和许询并称,但许询不满意。是啊,单位评选优秀工作者,居然把我和一帮子无所事事的人都选上了,难道是说我和他们一样空闲吗?我宁可不要!``你薄瓜瓜居然和陈美、傅家俊这样的娱乐界人士并称为英国十大杰出华人青年,你好意思吗?''也许还像那英,``一句话,我与她做人态度、表演风格不一样'';田震:``这个人我不熟悉''。王修24岁就死了,家人比之于王弼,来安慰内心的伤痛。
于法师:支法师,支道林和尚。
苦相折挫:据说许询那年16岁,王修那年12岁。折挫,现为挫折,逼迫挫败。许苦苦相逼王。
何似:如何。《北史?崔伯谦传》:``朝贵行过郡境,问人太守政何似?''
岂是求理中之谈哉:我并不赞成晋代为辩论而辩论的风气,许询和王修辩论,许正说也有理,反说也有理,充斥着诡辩、偷换概念和浅层次的斗嘴的快感。这不是研究学问的方式,而是没有原则,混淆真理。前面所说的王弼,也是这样搞双手互搏,``自为客主数番,皆一坐所不及''。记得当年的大专辩论赛,题目居然是抽签决定正反方的,我就很不喜欢,这样培养出来的学生,恐怕会怀疑和玩弄人生吧。东晋如此风气所趋,只有口才风度的观赏性,无道义支撑,不追求真理,玄学自然而然走向末路。这个丰富多彩的魏晋时期,由于充满了怀疑主义论者,它给予后人更多的是苦涩的教训而不是辉煌的成绩。

\section{4.39}\label{section-217}

林道人诣谢公,东阳时始总角,新病起,体未堪劳,与林公讲论,遂至相苦。母王夫人在壁后听之,再遣信令还,而太傅留之。王夫人因自出,云:``新妇少遭家难,一生所寄,唯在此儿。''因流涕抱儿以归。谢公语同坐曰:``家嫂辞情慷慨,致可传述,恨不使朝士见!''

支道林去拜访谢安。
东阳:谢朗,官至东阳太守,谢据的儿子,谢安的侄子。谢家子侄甚多,最优秀的正如侄女女谢道韫所说的``封、胡、羯、末'',谢朗的小名就是``胡''。总角就是童年,谢安就叫这么小的谢朗与支道林谈玄(讲论),可见言传身教之切,谢安老婆刘夫人怎么还不满意啊?在2.92中,谢安曾问子侄辈:``你们将来的好坏和我何干,为什么我偏偏盼望着你们都很优秀呢?''谢安表现出来的人生态度和人情心理是很微妙的:我知道一切皆为空幻,我为什么还有爱恨情仇?我为什么希望你好呢?
相苦:苦苦逼问。 再:二次,第三次就自己出马了。
新妇少遭家难:谢据早死,妻子王绥守寡多年。在当时新妇不是新娘子的意思,而是妇人的谦称,我。《世说新语》后面也有类似的用法:``昔夫人临终,以小郎嘱新妇,不以新妇嘱小郎''。《孔雀东南飞》``却与小姑别,泪落连珠子:新妇初来时,小姑始扶床''。但观察《孔雀东南飞》新妇的用法,有时候也不是谦称,``鸡鸣外欲曙,新妇起严妆'',有年轻妇女的意思。学者王利器说:``汉魏六朝人通称妇为新妇,\ldots{}\ldots{}同义词。''南方方言中至今保留``新妇''的发音。《碧玉簪》著名唱词:``阿林是我的手心肉,新妇大娘侬是我的手背肉。''
致可传述:``致''或当做``至''、``此''或``故''。
女人宝贝儿子,为什么要让朝廷知道,难道那个时候已经有贞洁牌坊了?但爱孩子和不改嫁也没多大关系啊!难道东晋丢失的领土,就像王绥``一生所寄''的谢朗一样,要把它夺回来?哈哈,``请叫儿的乳名,叫我一声`阿胡'!母亲!我要回来,母亲!''

\section{4.40}\label{section-218}

\begin{quote}
支道林、许掾诸人共在会稽王斋头,支为法师,许为都讲。支通一义,四坐莫不厌心;许送一难,众人莫不抃舞。但共嗟咏二家之美,不辩其理之所在。
\end{quote}

会稽王:``我在这里啊''的简文帝司马昱,曾受封会稽王。他最好玄谈,交接了一批玄学家和旅居会稽的名士。其历经晋明、成、康、穆、哀与废帝6个时代,最后当了2年皇帝。司马昱进入中央决策层后,他把很多老朋友提拔起来,以此牵制桓温。
斋头:书房,头为词缀,无意义。像魏晋时候用``馨''做词缀,现在是见不到的。世界上不少语言中都有意义不大或者没有意义的词缀,这种空虚的调节语气,也许说明了人类思维方式的某个特点。
法师、都讲:法师是佛教的一种学位的称号,通达佛法,能为人讲解的叫法师,精通经藏的叫经师,精通律藏的叫律师,精通论藏的叫论师。都讲是主持人,负责发问的岗位,引导听众思路,便于听众理解法师的论述。法师和都讲,我们想想从前看过的《聪明的一休》,就了其大概了。许询负责提出疑难,支道林负责阐述义理。
厌心:满意。
抃舞:拍手跳跃,手舞足蹈。《列子?汤问》:绕梁三日的``(韩)娥还,复为曼声长歌,一里老幼喜跃抃舞,弗能自禁,忘向之悲也。''
佛教很看重辩论,佛经多用辩论的形式来阐述自己的理论,也许由于此,他们在后来和道教的辩论中,也多次胜出;庄子也看重辩论,有时候也用对答的形式谈自己的观点,但庄子又看不起辩论,因为他是``无是非''的齐物论者,在他眼里,真理是不可知的,在他的文章里,经常使用诡辩的手段。譬如他说辩论:我有我的看法,你有你的看法,你我本来就不一样,就没资格评判对方的对错。我们请人来评判,请意见与你相同,既然与你相同了,怎能判定呢?请意见与我相同的人来评判,又怎么能判定呢?请意见与我你都不同的人来评判,既与你我都不同,又怎能断定你我究竟谁是谁非呢?那么我与你与人都不能确定谁是谁非,再又靠谁来知道真理呢?这种说法立论错误,不值一驳,但你不能不说他语气充沛,句式工整,能自圆其说,一下子就忽略了他避开客观实际的毛病,``不辩其理之所在''。古希腊人也好辩论,法国丹纳在《艺术哲学》中说:``(希腊)值得注意的是他们对辩证法本身的爱好,他们不因为长途迂回而感到厌烦;他们喜欢行猎并不亚于行猎的收获,喜欢旅途不亚于喜欢到达终点。\ldots{}\ldots{}微妙的甄别,精细而冗长的分析,似是而非的难以分清的论点,最能吸引他们,使他们流连忘返。他们以辩证法,玄妙的辞令,怪僻的议论为游戏,乐此不疲;他们不够严肃;作某种研究决不是只求一个固定的确切的收获\ldots{}\ldots{}真理是他们在行猎中间常常捉到的野禽;但从他们推理的方式上看,他们虽不明言,实际上是爱行猎甚于收获,爱行猎的技巧,机智,迂回,冲刺,以及在猎人的幻想中与神经上引起的行动自由与轰轰烈烈的感觉。''魏晋时候的的清谈基本也是这样。辩论本为真理而奋斗,但路途过于遥远,人们就在途中戏耍,忘记了出发的目的。
两晋以后,中央集权和专制体制日趋严密,辩论之风在中国就衰减了很多,人们更多的讲究微言大义,含沙射影,阴谋诡计和场外功夫,至今如此。

\section{4.41}\label{section-219}

\begin{quote}
谢车骑在安西艰中,林道人往就语,将夕乃退。有人道上见者,问云:``公何处来?''答云:``今日与谢孝剧谈一出来。''
逼老头子喝酒以示惩罚的谢奕(曾任安西将军)去世,儿子谢玄(曾追封车骑将军,这已经是武官中的最高阶别了,因为大将军和骠骑将军基本是空着的)服丧。支道林找他聊天。当时谢安还在绍兴隐居,谢玄15、6岁,支道林40多岁,这个故事含义是谢玄早慧。
见者:者或是衍文,可删。 谢孝:服孝的谢玄。
剧谈:畅谈。《汉书?扬雄传上》:``口吃不能剧谈,默而好深湛之思。''
一出来:费解,可删,或留一个``来''字。
也许是古人学科单一,所以早慧。许询和王修辩论,都还是娃娃;李白说``十五观奇书''、``十五好剑术''、``十五游神仙'';王勃写《滕王阁序》,据说也就15岁。像陈元方和孔融孩子等偶尔说一两句惊人之语,或者写一两首小诗,这种机智其实并不稀奇,但思维缜密,能做长篇大论或者陪学问家畅谈的,我至今没有见过。陈寅恪先生曾经说过,中国书虽然多,但值得读的不过几十种罢了。其他无非是抄来抄去,不用看的;大经济学家谢林也讲过,经济学里真正重要的道理,屈一手的手指而可尽数。也许大道至简,聪明的人闻一知十,牢牢把握住学问的原则和思维的方法,就势如破竹了。如我辈中人,盲人摸象,又苦无良师,一辈子在迷途中混世。
书话网友蒋方舟、虞慧彤和前些年名声大噪的韩2等,据说也是神童,但其实都是假冒伪劣。
\end{quote}

\section{4.42}\label{section-220}

\begin{quote}
支道林初从东出,住东安寺中。王长史宿构精理,并撰其才藻,往与支语,不大当对。王叙致作数百语,自谓是名理奇藻。支徐徐谓曰:``身与君别多年,君义、言了不长进。''王大惭而退。
\end{quote}

王长史:王濛,前面出场过很多次了,支道林出道时,他狠狠提携过,比之于王弼。但支道林这次从会稽重返建康,王濛已经去世了,所以此则有误。东晋姓王的官僚太多,看不起支道林和支道林看不起的大有人在,文中看不出是谁。
宿构:提前构思。《三国志?王粲传》:``(粲
)善属文,举笔便成,无所改定,时人常以为宿构。''
撰其才藻:写好了华丽的词藻。``支作数千言,才藻新奇,花烂映发''
不大当对:``当''就是``对'',``对酒当歌'',``对花当月'',``门当户对'',就是说王某人说不过支道林。
叙致:致有说讲的意思,致辞,致谢。叙致就是叙说陈述。《文学》``支道林先通,作七百许语,叙致精丽'',
李慈铭《越缦堂读书记》:``叙致错综,笔墨疎秀,萧然出町畦之外''。有人说致是道理的意思,缺证据,不通。
身:我。前面有``身今少恶'',``身今日当与君共谈析理''。
了:完全。了无惧色。
王濛和支道林交情很好,在世说里面多次记录两人互相吹捧的话,王濛多次说支道林是王弼、何晏这样的大学问家,支道林夸王濛``轩轩韶举''(高大优美)。王濛在临死前还见过支道林,《容止?31》说:王长史尝病,亲疏不通。林公来,守门人遽启之曰:``一异人在门,不敢不启。''王笑曰:``此必林公。''两人一般不会口出恶言。
吕蒙是``士别三日,当刮目相待'',支道林讽刺王某人``经别多年,君了不长进''这样骂人够狠。前些天我逛书店,看着看着,突然被自己的一个想法吓住了了:我居然比别人懂的更多,见识更高,写的更好。人处壮年,当然可以自信;随着马齿渐老,当然应该谦虚。支道林不久后就被当年推崇他的孙绰讽刺:林公理每欲小屈,孙兴公曰:``法师今日如著弊絮在荆棘中''。``和尚你已经没有当年的风采了,你的旧衣服已经被语言森林的荆棘撕扯得支离破碎。''

\section{4.43}\label{section-221}

\begin{quote}
殷中军读《小品》,下二百签,皆是精微,世之幽滞。尝欲与支道林辩之,竟不得。今《小品》犹存。
\end{quote}

殷中军:殷浩。当过中军将军、扬州刺史等,我们的老朋友。
《小品》:《小品般若波罗蜜经》,又叫《道行般若经》,宣扬大乘佛教的思想,东汉末支谶和竺佛朔翻译,因为是``小品''缩写本,文句简略,义理艰涩,前后不能贯通,和尚都看不懂,以至于三国时候中国第一个受戒僧人朱士行前往西域找来《大品般若波罗蜜经》。般若波罗蜜的意思就是以``大智慧到达彼岸''。
下二百签:记下两百处标记。他记下这么多疑难,差人邀请支道林去辨析解释。当时王羲之和支道林应该在一起在绍兴,王就劝阻说:殷浩不懂的东西,你也未必全懂。你名声在外,赶过去解释,出洋相就有辱你十来年积累下的名声,即使解释通了,也不会增加你的名声,千万别去!谢安、王羲之和支道林交往很多,但他们都认为支道林的口才还不如殷浩。世之幽滞:停留在世界的深处。隐藏的道理。
支道林后来受晋哀帝的邀请出山,据说``高步天邑'',这那个时候,他是否想起殷浩的200个问题?

\section{4.44}\label{section-222}

\begin{quote}
佛经以为袪练神明,则圣人可致。简文云:``不知便可登峰造极不?然陶练之功,尚不可诬。''
\end{quote}

袪练神明:袪练就是澡练,修炼。《魏书?释老志》:``渐积胜业,陶冶粗鄙,经无数形,澡练神明,乃致无生,而得佛道。''《明史?杨最列传》:``神仙乃山栖澡练者所为,岂有高居黄屋紫闼,兖衣玉食,而能白日翀举者?''神明就是精神,《荀子?劝学》``积善成德,而神明自得,圣心备焉。''《贤媛?31》:``发白齿落,属于形骸;至于眼耳,关于神明,那可便与人隔!''袪练神明,修炼精神。后面的陶练也是修炼的意思。
圣人可致:孟子说``人人皆可为尧舜'',但是在魏晋时期,一般认为圣人是人难以企及的,因为他们眼里的圣人,要生而知之,``圣人生知,故难企慕'';要``乘天地之正,而御六气之辩,以游无穷者'',
尧舜还谈不上是圣人,``圣人无名''。儒家和道家并没有途径,没有明确的成圣教科书,只有圣人的概念和跳跃性的想法。而佛教提出了达到圣人的具体步骤,``修智慧,断烦恼,万行具足,便成佛也'',你只要``戒定慧'',看完几百万、几千万乃至几亿字的佛教书籍,受三五百个戒,修``十善道、四谛法、十二因缘、六度'',然后``修进万行,拯度亿流,弥长远乃可登佛境矣'',按照一大帮子和尚说的那套去做,最后拯救完亿万民众,``地狱一空'',终能成佛。
登峰造极不:不就是否。琅琊王、会稽王、抚军将军、简文帝司马昱对佛教徒能成圣抱有疑虑,但认为它的修炼途径不能说错,``尚不可诬''。

\section{4.45}\label{section-223}

\begin{quote}
于法开始与支公争名,后情渐归支,意甚不忿。遂遁迹剡下。遣弟子出都,语使过会稽,于时支公正讲《小品》。开戒弟子:``道林讲,比汝至,当在某品中。''
因示语攻难数十番,云:``旧此中不可复通。''弟子如言诣支公,正值讲,因谨述开意。往反多时,林公遂屈。厉声曰:``君何足复!受人寄载来!''
\end{quote}

于法开:东晋名僧于法兰弟子兼医生。谢安、王坦之都和他友善,就问他:法师你见识高明,性格刚强简约,怎么经心于医术呢?于法开回答说:我是``明六度以除四魔之病,调九候以疗风寒之疾'',可自利,又可利人,有什么不好呢?
婆罗门教本来就教授``五明'',就是语言音韵知识,工艺、算历知识,医学知识,逻辑论辩知识,人生哲学知识。释迦牟尼据说也学过``医方明'',当过医生,``分别病相,晓了药性,随病授药,令众乐服'',也用医学知识来解释他的世界观,更擅长用心灵鸡汤来治疗众生各种不同的心理疾病。
自汉以后,僧人万里来华,学点医学知识既是养生保命之道,又可以以此吸引、教化民众。而佛教思想其实和医学知识有些抵触的,所以有些僧人就看不惯同仁以医术交接权贵,``数见朝贵门首,多有疗病僧尼,或有行医针灸求贪名利,\ldots{}\ldots{}致使秽响盈路,污染俗情'',就提出:``若比丘尼诵治病经方,堕地狱;若比丘尼为人治病以为生业,堕地狱。''说到宗教用医学来招揽民众,首推基督教,中国清末以前基本都是一个大夫坐诊,比较正规的医院基本是基督教宣教士来华建设的。时至今日,仍有成千上万的基督徒宣教士在第三世界提供基本的医疗服务,其他宗教就没有这种宏愿和力量。
情渐归支:支道林本来是研究玄学的,又善于交际,而于法开性格``刚简'',不讨人喜欢,于是人众的感情就渐渐趋向于支道林了。
意甚不忿:一般来说,应该是``意甚忿'',但汉语中有种特殊的用法,``不''表示加强语气或调整音节,``好不吓人'',``不尴不尬''。《诗?小雅》:``徒御不惊,大庖不盈。''《毛传》中说``不惊,惊也;不盈,盈也。''
遁迹:隐居。
剡下:下为语助词,无意,就是指剡(剡溪,新昌)。前面有``都下''。
出都:到建康。
语使过会稽:拗口费解,《高僧传?支道林传》中说:``开有弟子法威,清悟有枢辩。开尝使威出都,经过山阴,支遁正讲小品。''``语''可删。
开戒弟子:法开告诫弟子说。
比:比及,等到。《论语?先进》:``比及三年,可使有勇。''比肩。绍兴和新昌县也不过6、70公里,2天的行程,所以法开算得到他讲到哪里了。
攻难:质疑诘难。《文学?12》:``裴成公作《崇有论》,时人攻难之,莫能折。''
旧此中不可复通:复,回答。``从前这一部分不能解答通畅。''
君何足复:你有什么值得我回答。
受人寄载来:你不过是接受别人思想的托运工具罢了。来,无意义,语气词。``归去来兮'',``嗟来之食''。《高僧传》文字为:``君何足复受人寄载来耶''。寄载:托运,坐别人的交通工具。《搜神记》:``(糜竺)尝从洛归,未至家数十里,见路次有一好新妇,从竺求寄载,行可二十余里,新妇谢去。''
支道林恼羞成怒。

\section{4.46}\label{section-224}

\begin{quote}
殷中军问:``自然无心于禀受,何以正善人少,恶人多?''诸人莫有言者。刘尹答曰:``譬如写水著地,正自纵横流漫,略无正方圆者。''一时绝叹,以为名通。
\end{quote}

殷浩提出了著名的问题,为什么世界上好人少而坏人多。这个就像八股文考试一样,抽出《庄子》里的一句话,作为清谈的论题,原文出自于《庄子?胠箧》:``天下之善人少而不善人多,则圣人之利天下也少而害天下也多。''庄子原文诸家各有解释,但大多似是而非。观察《胠箧》前后文,其实不难理解:圣人讲究``圣、勇、义、知、仁'',圣人之道可为善人所用,在做好事上有所建树,也可为恶人所用,在做坏事上通行无阻。但这个世界上好人少而坏人多,坏人把圣人的``圣、勇、义、知、仁''不用在正道上,所以圣人的那套对世界的益处小,对世界的坏处大。故而我们可以得出结论,圣人的那套理论不死的话,靠圣人学说纵横于世的大贼盗就永远是社会的主流。
殷浩在善人少恶人多前加了个前缀``自然无心于禀受'',自然在中国古代并不是指大自然,一般就是天然,老天爷,天生之道的意思。禀受就是承受的意思,《淮南子?修务训》:``各有其自然之势,无稟受於外。''
王充《论衡?气寿》:``非天有长短之命,而人各有稟受也。''意思应该就是:``每个人品性的承受都处于老天爷的漫无目的''。为什么这么解释呢,我们只要翻阅《庄子》即可,庄子本来就持人天生没有善恶之分的观点,他和老子一样崇尚原始,认为原始人无知无识,没有机心,有机心然后有善恶。
  正善人少,正方圆者:在《世说新语》中,正一般可以解释为恰恰,不用解释为总是。正自,恰是,只是。``风景不殊,正自有山河之异'',``郊邑正自飘瞥,林岫便已皓然'',``推人正自难''、``身正自调畅''等。
  名通:名言通论。
  刘惔的名言什么意思,我们只要看看辛弃疾的诗歌《偶作》就了解了:``至性由来禀太和,善人何少恶人多。君看泻水着平地,正作方圆有几何?''方圆就是善人,纵横流漫就是恶人。在世上流窜的都是没有规矩的恶人,哪有方方正正、圆圆润润的善人啊!
  我小时候看过科普书籍,有个概念叫``熵''。其大意就是在微观世界中,物质都是向着混乱度增大的趋势运动。在物理反应中,系统总是有趋于最大混乱度的倾向,系统的混乱度越大,越有利于反应的自发进行。这种系统总是力图自发地从熵值较小的状态向熵值较大(即从有序走向无序)的状态转变的概念,后来运用到社会学里面。刘惔讲的就是社会学中比较直观的``熵''的概念。
  什么叫做善,在《庄子》中没有具体的解释,老子和孔子都谈过自己的看法。老子说:``上善若水。水善利万物而不争。处众人之所恶,故几于道。居善地,心善渊,与善仁,言善信,政善治,事善能,动善时。夫唯不争,故无尤。''老子的话向来不太好理解,但中心思想``善就是不争''总是清楚的,老子进而提出很有趣的一句话:``处众人之所恶故几于道'',善是大家所厌恶的(大家都是恶人呗),所以善近乎于道(道的主流是善)。我们可以这样理解,老子、庄子即使认为人之初虽然品性没有好坏之分,但现实社会中众人大都是恶的。
  在《论语》中,孔子没有讨论过人性的善恶,他恐怕还是认为人是善恶夹杂的,希望人向善的一面而发展,所以说``择其善者而从之,其不善而改之''。孔子眼里的善标准也不低,恐怕不是简单的好人、坏人这么简单:``善人,吾不得而见之矣,得见有恒者斯可矣。''孔子进而指出现实中没见过善人,而人往往是``亡而为有,虚而为盈,约而为泰'',人啊,就是``没有却装作有,空虚却装作充足,内在贫乏却生活骄奢''。
  我不敢说这个世界究竟是好人多还是坏人多。我朝太祖曾经有过批示,他在《资治通鉴》``崔挺上书谏曰:`天下善人少,恶人多。若一人有罪,延及阖门,则司马牛受桓魋之罚,柳下惠婴盗跖之诛,岂不哀哉!'\,''中用笔狠狠在``天下善人少,恶人多''上划了条线,然后写道:``此古人一贯谬论!''太祖批示是``太上老君急急如律令'',当年多少人死在他轻轻的几个字上,``彭德怀在历史上一贯反动'';太祖批示又是``铁卷丹书'',又多少人因为他的几个字而免于迫害毒打或者鸡犬升天,``你办事我放心'';太祖的批示也是金钱无数,一字千金,绵绵不绝,最近浙江省要大搞什么``枫桥经验''纪念会,不铲掉一座金山不会罢休,原因就是太祖60年前:``要各地仿效,经过试点,推广去做。''
  太祖读书不少,但很少看马克思主义典籍,当年恩格斯在《反杜林论》中很实在地指出:``与其说人性善,不如说人性恶更加接近于唯物主义。''(黑格尔:``人性本恶这一基督教的教义,比其他教义说人性本善要高明些'',《圣经?传道书》中说:``时常行善而不犯罪的义人、世上实在没有'',``看哪,一千男子中,我找到一个义人;但众女子中,没有找到一个。我将这事一一比较、要寻求其理,我心仍要寻找,却未曾找到。'')恩格斯如果生活在太祖统治之下,十有八九会因为这句话而被打成牛鬼蛇神,不跳楼就得坐牢,体会不到太祖眼里的世界上``善人多,恶人少'',享受不到革命同志之间的``从容不迫,文质彬彬和温良恭俭让''。
  比喻是天才的象征,刘惔的回答意思就是环境险恶,世上``根本没有''善人,但这种比喻具有多义性,他好像把人性的恶比作形形色色的不规则;又好像指出善恶由环境所定,地势使然,还在不断地流变;又好像表明自己的生活态度又如水之任其自然。语言生发,故而一时叹绝,辛弃疾引用了其正义,而鲍照又有了自己的看法,《拟行路难》``泻水至平地,各自东西南北流;人生亦有命,安能行叹复坐愁?''无常的命运啊,谁也不能强求,我只能听凭它贵者自贵,贱者自贱。这是鲍照的认命,也是鲍照对命运的控诉。
  我们比古人站的更高,我们现在一般认为,利己主义才是社会不断前进的动力,商业社会的一切活动都可归结为自利行为,只要把它置于完善的法律和公平的市场经济制度之下,社会才能不断进步。就像中黑格尔指出的那样:``有人以为,当他说人本性是善的这句话时,是说出了一种很伟大的思想:但是他忘记了,当人们说人本性是恶的这句话时,是说出了一种更伟大得多的思想'',``没有情欲,世界上任何伟大的事业都不会成功'',``恶是历史发展的动力的表现形式''。
  是啊,对于我们这个世界来说,普遍的爱和普遍的利益是毫无意义的概念。

\section{4.47}\label{section-225}

\begin{quote}
康僧渊初过江,未有知者,恒周旋市肆,乞索以自营。忽往殷渊源许,值盛有宾客,殷使坐,粗与寒温,遂及义理。语言辞旨,曾无愧色;领略粗举,一往参诣。由是知之。
\end{quote}

  康僧渊:出生在中国的西域僧人。衣冠南渡的时候,他也来到了江南,长期在建康的商业区以乞讨为生。
  殷渊源:殷浩,他当时估计还没出山,所以康僧渊还能随便登门。但这个故事有费解之处,因为殷浩据说晚年罢黜时才开始看佛经,《文学?50》:``殷中军被废东阳,始看佛经'',《高僧传》中说:``後因分卫之次(在讨饭途中),遇陈郡殷浩,浩始问佛经深远之理''。而事实是,在殷浩下台时,康僧渊早已出名,和王导、庾亮等大官僚多有交往,而且名声很大,``响附成群''。不过也有可能,殷浩其实不是晚年才看佛经的,早年隐居已在做功课,晚年被贬庶民,宣称自己由玄转佛,不过是政客的一种对外宣言。
  说句良心话,和和尚打交道,声称自己信佛的人,或者说佛教徒,讲话大都不可信。历史学家、藏学家王力雄在他的《天葬》中意味深长地说:西藏有是一个具有神话传统的地方。那里特定的生活环境形成信息传递的口头性,在口口相传中加进每个人的想象,常常是传不了几个人,一个消息就成了故事。\ldots{}\ldots{}这么近的距离,消息就能变形到如此程度,充分说明藏民族创造神话的天赋。西藏足以让文学家折服,也实在是一个能让考据者发疯的地方。\ldots{}\ldots{}关于这一点,一位英国妇女也有同感。她对1959年西藏叛乱后流亡在外的西藏难民抱有极大同情。本来她被邀请写一本报道中共``暴行''的小册子,但是当她在西藏难民中间做了大量采访以后,不得不谢绝了写作。她说:``\ldots{}\ldots{}凭良心说,没有收集到一个我认为是`真实的故事'。从本质讲,普通的西藏人都是可信的、诚实的,这是无可辩驳的事实。但同时也要认识到西藏人心目中的`事实'与西方人所认为的确凿证据是不同的。认识不到这点是危险的。西藏农民从生到死都习惯于把传说和神话当作事实接受\ldots{}\ldots{}''我长期从事意识形态的工作,对于宗教徒的讲话,对,我尊重他们的信仰,尊重他们执着的目光,但他们的事实观,确实不是事实,而是``大头天话''。
  值盛:费解,删去``盛''字,值,刚好。   粗与寒温:草草寒暄了几句。
  语言辞旨:言谈志趣。
  曾无:毫无。《史记?魏公子列传》:``今吾且死,而侯生曾无一言半词送我,我岂有所失哉?''《三国志?庞统法正传?裴松之注》:``备宴酣失时,事同乐祸,自比武王,曾无愧色。''
  领略粗举,一往参诣:这句话有点费解,也许意思是(康僧渊)领略(殷浩的)粗举(随口提出的问题),(显示出其)一往(一向)参诣(深入专研)。
  殷浩不耻下问,康僧渊自信有料。

\section{4.48}\label{section-226}

\begin{quote}
殷、谢诸人共集。谢因问殷:``眼往属万形,万形来入眼不?''
\end{quote}

  谢安向殷浩莫名其妙地问什么啊?我们要从佛经中去了解他的意思。佛教中把眼、耳、鼻、舌、身、意叫六根。色、声、香、味、触、法叫六尘。我们用眼睛去看物质世界,但是你看久了,眼睛就会发花,看到很多乱七八糟的东西,这些东西不是实有的。佛教认为,即使眼睛不发花,看到的其实和发花的时候是一样的,因为这个世界本来就不是实有的,不过是因缘聚合。那么世界的真相是什么呢?事物生起、变化和坏灭的主要条件称为因,把辅助条件称为缘,世间一切都是因缘和合而成,不同事物之间只有发生联系才得以存在,存在的状态不是孤立的,而是共生的。``眼根''和``色尘''就是由两种一起生出来的。但``眼根''和``色尘''不是一回事,之间还是有个界限,这个界限是什么呢?佛教说,就是``识'',``六识''发动``六根''而接触``六尘''。有了``识'',就可以``根''不看到``色''就知道``色'',并且认识到世界``空''的本质,这就是``知''。如果眼和色融为一体,中间没有离的话,你就没有知了。
  所以,谢安在问殷浩:眼睛看到万物,万物是否就进入眼睛里呢?往属就是``往瞩'',看的意思,周邦彦《薛侯马》:``往属嫖姚探虎穴''(看霍去病直捣虎穴)。万形就是万物。来,语助词,无意。谢安就是在问``色尘能否与眼根融为一体''的问题。殷浩如果回答,就要从``识''和``空明''的概念下手了,但这个故事阙文。
  苏轼曾经写过一首《琴诗》:``若言琴上有琴声,放在匣中何不鸣?若言声在指头上,何不于君指上听?''这也是谈佛理的。以我们现在的知识看,佛教的解释都是错误的,它没有树立主体、客体和媒介的观点,没有物理学知识,而是以云山雾罩,似是而非的抽象观念去理解世界,它只是煮了一锅又一锅心灵鸡汤。

\section{4.49}\label{section-227}

\begin{quote}
人有问殷中军:``何以将得位而梦棺器,将得财而梦矢秽?''殷曰:``官本是臭腐,所以将得而梦棺尸;财本是粪土,所以将得而梦秽污。''时人以为名通。
\end{quote}

起码晋代人已经认为,梦见棺材预示着升官,梦见粪便(矢秽)要得财,这种说法在《周公解梦》等中多有体现,``将棺入宅禄位至'',``粪土堆积主得财''。一般的认为是棺谐音官,粪便是黄色的,象征着黄金。但殷浩的解释更加意味深长,带有抨击社会和表明人生观的态度,辛辣而机智,所以当时人们认为是名言。
  梦是人们的另一个世界,弗洛伊德说,``梦是欲望的满足'',梦不是事件的预示而是反映自己的内心活动,而平时这种欲望受到理智的束缚无法观察到,于是转化为潜意识而在梦中得到释放,但在梦的状态下,理智检查系统仍要发挥作用,使欲望不能赤裸裸地表现自己,因此欲望只能采取象征的、曲折隐晦的手法来求得自我表现,以逃避检查。弗洛伊德进而指出,那些不愉快的梦,其实也是愿望满足的一种``变相的改装'',一个愿望的未能满足,其实象征着另一愿望的满足。
  那些更加曲折隐晦的梦,其实是人使用了隐喻和借代的手法。我们梦见棺材,梦见粪便,按弗洛伊德的看法,就是隐喻和借代。我们只要能正确地找出梦中``取代物''之所指,即可正确地找出梦的``隐意''。如果大家都承认棺材意味着升官,粪便意味着发财,那么这种物象就得到人潜意识的认可,它虽然不表示就会升官发财,但起码反映了人的这种愿望。
  弗洛伊德的理论比较复杂,譬如他指出,人的原动力在于性的本能,而这种本能只遵从快乐原则,像是一锅沸腾的水,根本无法控制。性欲受着社会规则(超我)的限制,一旦不能满足,只能直接或迂回地通过升华(超我)的调整,来实现本能的突围,于是便出现了政治家、军事家、科学家、文学家,甚或变态者等。譬如幽默,弗洛伊德学派指出,人们以幽默的语言或行为来应付紧张的情境或表达潜意识的欲望,通过幽默来表达攻击性或性欲望,可以不必担心自我或超我的抵制。在人类的幽默表现(如笑话)中关于性爱、死亡、淘汰、攻击等话题是最受人欢迎的,它们包含着大量的受压抑的思想。
  哈哈,殷浩很幽默,殷浩其实很压抑。大众欢笑于殷浩的幽默,大众其实很压抑。

\section{4.50}\label{section-228}

\begin{quote}
殷中军被废东阳,始看佛经。初视《维摩诘》,疑``般若波罗密''太多;后见《小品》,恨此语少。
\end{quote}

  殷中军被废东阳:
349年,后赵杀人``天王''石虎登皇帝位,同年去世。随后老婆、几个儿子、养子、养孙等互相砍杀,汉人、石虎养孙、后赵大将冉闵连杀石虎多个儿孙,后赵灭。冉闵颁``杀胡令'',并说``吾属故晋人也,今晋室犹存,\ldots{}\ldots{}欲望奉表迎晋天子还都洛阳''。受此影响,东晋决定收复中原,但为遏制桓温,任殷浩为中军将军、假节、都督扬豫徐兖青五州诸军事。而殷浩志大才疏,陷于复杂的人事纠葛和门阀集团利益之中,``受专征之重,无雪耻之志,坐自封植,妄生风尘'',属下将领离心背德,起兵反叛,殷浩未入中原而大败鼠窜。而冉闵兵败后,属下几十万汉人不甘受胡人之辱,纷纷逃往江南,投奔东晋。殷浩自顾不暇,未能及时接应,使得几十万汉人中途受到截击,死亡殆尽。殷浩罪责太大,``神怒人怨,众之所弃'',354年,桓温上书罢黜殷浩为庶人。殷浩被放逐到当时属于金华(东阳)的衢州居住。
  《维摩诘》:维摩诘经,就是4.35解释过的那部经书,记录了维摩诘称病,释迦牟尼欲派弟子探访,弟子们畏惧维摩诘的辩才而不敢往,释迦牟尼只好派出弟子中``智慧第一''的文殊与维摩诘较量的故事。这是本大乘佛经,讲的主要内容就是如何``般若波罗密''(以大智慧达到彼岸),但流传下来的《维摩诘经》中,并没有出现``般若波罗密''的字眼,估计晋代的译文和现在主要使用的鸠摩罗什翻译本有比较大的出入。当然,我们还可以有另外的理解,殷浩看了《维摩诘经》,觉得介绍成佛的法门太多了,无所适从;看了《小品般若波罗密经》,又觉得《小品》介绍成佛的法门有太少了。结合4.43中``殷中军读《小品》,下二百签''(不理解的地方有200处),这篇短文还有可能是在说,殷浩拿《维摩诘经》和《小品般若波罗密经》互相映照,觉得《小品》成佛之道讲得不够透彻,搞不清楚,感到很遗憾。
  用这个故事去对照4.47康僧渊的故事,说殷浩晚年才看佛经,不免相互抵触。

\section{4.51}\label{section-229}

\begin{quote}
支道林、殷渊源俱在相王许。而`才性'殆是渊源崤函之固。君其慎焉!''支初作,改辙远之;数四交,不觉入其玄中。相王抚肩笑曰:``此自是其胜场,安可争锋!''
\end{quote}

  殷渊源:殷浩。
  相王:清谈皇帝简文帝司马昱以会稽王身份任丞相,所以称相王。他有多少叫法啊,太宗,相王,晋简文,简文,抚军,会稽王等。
  二人:应为``支'',仅仅是对支道林说的。当然断句还可为:相王谓:``二人可试一交言\ldots{}\ldots{}''
  交言:交谈。
  才性:前面介绍过钟会拿着《四本论》去拜访嵇康,殷浩谈及``才性''
就如``汤池铁城'',现在又说殷浩谈及``才性''如``崤函之固'',可见在殷浩这里,关于人的本质与才能的关系已经想得比前人透彻多了。我们结合殷浩的生平经历,其本质、学问也许不坏,但政治、军事才能却不能恭维。我有时候也看关于李天一的新闻,其中有句话印象比较深,大意是其家人和朋友认为李天一本质是不坏的,才能也好,就是道德品质不行,道德品质不行是因为父母溺爱和社会环境的问题。这句话很有意思,就涉及性、德、才的关系了。
  依我看,如果支道林真的精通佛教理论的话,关于才性``离合同异''的问题,不可能辩不过殷浩,因为佛教有比较新颖的``空无''、``因缘''理论可作为辩论的立足点,根本不用纠缠于事物的表象,但这次支道林还是输了,也许说明了支道林对于佛教理论掌握其实并不熟练。后来殷浩约支道林讲解《小品》,王羲之劝支道林不要去,支道林还算有自知之明,没去(4.43)。支道林学问不够广大,镇不住人的马脚也就被大家熟知了,所以谢安说:``支道林不如王濛'',``支道林不如刘惔'',``支道林不如王羲之''。

\section{4.52}\label{section-230}

\begin{quote}
谢公因子弟集聚,问:``《毛诗》何句最佳?''遏称曰:``昔我往矣,杨柳依依;今我来思,雨雪霏霏。''公曰:``訏谟定命,远猷辰告。''谓此句偏有雅人深致。
\end{quote}

  因:趁着。隐居在绍兴的谢安常常召集子侄们谈文学,谈哲理等,这个应该是当时大家族的一个教育特点,门阀子弟就学,多半在家庭环境中,由长辈负责教学,继承家学。有客人来谈天的时候,家长也会把孩子带出来旁听,也允许他插话、提问甚至辩论。谢家当时的老师,可能是谢安的妻子刘夫人,所以在《德行?36》中记录:``谢公夫人教儿,问太傅:`那得初不见君教儿?'\ldots{}\ldots{}''但是看《世说新语》的记录,谢安也常常通过另外一种方式教孩子,就是聊天座谈,用情景教学因势利导,循循善诱。
  毛诗:《诗经》,西汉毛亨作传,流传最广,故称《毛诗》,其主要特点大概就是认为诗经的作者基本是王公贵族,他们写的诗歌都是带有政治教化作用的。譬如我们小时候学过著名的《伐檀》,看其大意,是劳动人民谴责剥削阶级不劳而获的诗歌。但毛亨认为,诗里面有微言大义,就是官员指责朝廷任人不贤,居高位者无功受禄:``刺贪也。在位贪鄙,无功而受禄,君子不得仕进耳。''这种说法当然讲得通,而且想得深远,但我们普通读者就不免觉得怪怪的。这种猜谜式的智力游戏,发展到后来,朝廷可以借此搞文字狱了。
  遏:谢玄。他的小名叫羯,但羯字不雅,就像王修小名叫``狗子'',后来书面就写成``苟子'',羯也照此写成遏。谢家杰出的子侄,小名要么是蛮族(胡、羯),要么是野兽(封,并封,传说中的怪兽),要么是下人(末,谢安儿子谢琰小名末婢)。
  ``昔我往矣:《诗经?小雅?采薇》的名句,雨是动词,下,所以说雨雪霏霏,霏,飘扬。这是诗经中少有的写景佳句,做到了情景交融。这首诗毛亨的解释很别扭,他说是周文王写的,是出征前的歌,目的是鼓励将士们抵抗外族侵略:``《采薇》,遣戍役也。文王之时,西有昆夷之患,北有玁狁之难。以天子之命,命将率遣戍役,以守卫中国。故歌《采薇》以遣之''。但你要说作者是一个退伍回家的老兵,也不可思议,他的文化程度太高了,居然能够驾驭这么长的一首诗。写作其实是个技术活,短诗可以妙手偶得,不需要多少写作技巧,民歌有很长的,但风格一般总是质朴大方。《采薇》很长,层次复杂,修辞技巧很高,难道一个普通战士在部队的大学校里自学成才了?该诗反映出战士出征前是春天,一派欣欣向荣;回来时是冬天,心事百折千回。清代王夫之说:``以乐景写哀,以哀景写乐,一倍增其哀乐。''
  《采薇》中的``昔我往矣,杨柳依依。今我来思,雨雪霏霏。行道迟迟,载渴载饥。我心伤悲,莫知我哀!'',《黍离》中的``彼黍离离,彼稷之实。行迈靡靡,中心如噎。知我者谓我心忧,不知我者谓我何求。悠悠苍天,此何人哉!''应该说是整个《诗经》中最优秀的句子。我们走在生活的道路上,世界景色变幻,我们脚步迟缓,常常会产生深沉的感伤。大自然什么也不说,按照它自己的规律生生灭灭,而我们的心灵感触,却因为起起伏伏的生活,那么激荡,那么长久,那么百般滋味。
  ``訏谟定命'':《诗经?大雅?荡之什?抑》:``訏谟定命,远猷辰告。敬慎威仪,维民之则'',大意是胸怀谋略来确定法令,宏伟的计划及时告诉百姓。君子恭敬谨慎,举止庄重,这才是百姓仿效的典范。谢安背诵这首诗歌,也许是在表明自己的志向:君子处世当有天下之虑,做长久之规,行万人之效。这个故事显然发生在谢安隐居时期,而他发布这样的宣言,就等于是做了出山的准备。
  谢安居然说诗经里面``訏谟定命,远猷辰告''写的最好,他有没有搞错啊?它有诗意吗?除了音节响亮意外,根本没有什么美感。当然,晋代流行玄言诗,南朝钟嵘在《诗品》中说:``永嘉时,贵黄老,稍尚虚谈。于时篇什,理过其辞,淡乎寡味。爰及江表,微波尚传。\ldots{}\ldots{}皆平典似道德论,建安风力尽矣。''当时的诗歌主流审美品味就是讲哲理和道德,谢安的说法有其时代基础。但是,这个故事还有另外一个版本。《晋书?谢道韫传》第一个故事就是:``叔父安尝问:`毛诗何句最佳?'道韫称:`吉甫作颂,穆如清风。仲山甫永怀,以慰其心。'安谓有雅人深致。''显然,《谢道韫传》和本文其实讲的是同一件事,原文也许应该是:谢公因子弟集聚,问:``《毛诗》何句最佳?''遏称曰:``昔我往矣,杨柳依依;今我来思,雨雪霏霏。''道韫称:``吉甫作颂,穆如清风。仲山甫永怀,以慰其心。''公曰:``訏谟定命,远猷辰告。''谓此句偏有雅人深致。谢道韫的话出自《诗经?大雅?荡之什?烝民》,大意是尹吉甫我作歌相赠,乐声和美如同清风吹拂;仲山甫临行长思,希望我的歌声抚慰着你的内心。这句诗也写的好,有动人的场景描写,温和的歌声,温和的风,有余音缭绕的回味,也在歌颂仲山甫杰出的才能和高尚的德行。谢玄的说法来自美的直觉,而谢道韫的说法善解人意,在拍叔叔谢安的马屁,而且拍的``穆如清风'',谢安十分舒服,于是以《诗经?大雅?荡之什》的第二首歌《抑》应和道:``訏谟定命,远猷辰告'',他不是在说自己这句最佳,而是说谢道韫这句好,让他壮志清发,不由自主的合唱,然后夸谢道韫选的诗句有``雅人深致''(高尚的人深远的志趣),同时也在委婉地提醒谢玄,要树立像仲山甫远大的志向和高尚的品德。
  师父,大师兄说的对!大师兄,二师兄说的对!哦,不对。谢玄说的对,谢道韫说的不对,谢安说的也不对。

\section{4.53}\label{section-231}

\begin{quote}
张凭举孝廉,出都,负其才气,谓必参时彦。欲诣刘尹,乡里及同举者共笑之。张遂诣刘,刘洗濯料事,处之下坐,唯通寒暑,神意不接。张欲自发,无端。顷之,长史诸贤来清言,客主有不通处,张乃遥于末坐判之,言约旨远,足畅彼我之怀,一坐皆惊。真长延之上坐,清言弥日,因留宿至晓。张退,刘曰:``卿且去,正当取卿共诣抚军。''张还船,同侣问何处宿,张笑而不答。须臾,真长遣传教觅张孝廉船,同侣惋愕。即同载诣抚军。至门,刘前进谓抚军曰:``下官今日为公得一太常博士妙选。''既前,抚军与之话言,咨嗟称善,曰:``张凭勃窣,为理窟。''即用为太常博士。
\end{quote}

  张凭:江东吴郡张氏子弟,在三国孙吴期间,江浙有``顾陆朱张''四大家族,有``顾厚、陆忠、朱武、张文''之说,吴郡张氏据说是张良之后,东汉末移居江苏吴县相人里,大概是行善积德,当时有谚语说:``相里张,多贤良,积善应,子孙昌''。张凭的前辈有张温,张翰等,年龄差不多的有2.51中的张玄之。张凭的爷爷张镇当过苍梧太守,他看到孙儿聪明,就和儿子开玩笑:``我不如你啊,你有聪明的儿子张凭,我却没有没有这样聪明的儿子。''(排调?40)谢玄也讲过类似的话,他认为自己的儿子不够人才,孙子谢灵运却打小风致标异,就对亲人感叹说:``想不到我谢玄居然生了个谢瑍,也想不到谢瑍居然生了个谢灵运。''我小时候还看过一个民间故事,有个儿子这样被父亲讽刺,就奋起反击,他对老头子说:``汝子不如吾子,汝父不如吾父!''
是啊,老人看自己的孙子,总比看自己的儿子要更顺眼一些。
  举孝廉:,汉武帝时,要求各郡国每年推荐孝廉各一人,就是举孝一人,察廉一人,对象是地方六百石以下的官吏和通晓经书的儒生。被举后,没有官职者授以官职,原为小官者升为大官。如果被推荐者后来犯了错误,要追查推荐者的责任。该制度还有不少细节,但无论它怎么设计,在专制体制之下,难免演变成``或以顽鲁应秀才,以黠逆应至孝,以贪饕应廉吏,以狡猾应方正\ldots{}\ldots{}富者乘其财力,贵者阻其势要,以钱多为贤,以刚强为上''。官僚制度中的伯乐相马,本来就不如比赛说话,比赛说话不如选举监察。
  出都:到建康。   必参时彦:必定厕身名流。彦,美士英才。
  刘尹:刘惔刘真长,丹阳尹,名士领袖,``清言冠世''。
  洗濯料事:有些费解,在《太平御览》中,原文倒了个个,``张遂径往诣刘,既前,处之下坐,通寒暑而已。真长方洗濯料事,神意不接。''这样就比较好理解,土著张凭拜访南渡的大名士刘惔,刘惔安排他坐在招待宾客的末位,寒暄了几句。刘惔顾自个去洗头发,处理杂事了,没有和他交流的意思。古人头发很长,要经常洗,洗的时间也不会短,官府的办公制度也特意安排沐浴时间,孔子据说更是每次上朝都要洗浴一番。
  长史:王濛,刘惔的老朋友。
  言约旨远:《孟子?尽心下》:``言近而指远者,善言也。''言语浅显而意味深远。约,简洁。唐代刘知几说:``言近而旨远,辞浅而义深,虽发语已殚,含意未尽。使人读者,望表而知里,扪毛而辨骨,睹一事于句中,反三隅于字外。''
作品是由作者和读者共同完成的,作者要举重若轻,信任读者,给读者空间。而意味深远主要来自于压缩、隐喻、模糊、色彩、音节和读者的学识、生活感触与情怀等。但是,文字本身应该浅显平易,不使用冷字僻词,语句拗口欧化,含义故弄玄虚。
  取卿:``取''同``聚'',会同你。
  抚军:简文帝司马昱,当时的职务是抚军大将军,二品,丹阳尹三品,所以刘惔自称下官。
  传教:通传官府法令的差人。
  勃窣:费解,《太平御览》作``劲粹''。生机勃勃的的勃是茂盛的意思。《说文解字》窣:穴中卒出也。窣有跳跃的意思,孔平仲《谈苑》:``如闭目窣身入水,顷刻间耳'',所以段玉裁在《说文解字中注》说,勃窣就是急行貌,所以当代许绍早在《世说新语译注》中说:``形容才华迸发而出。''但是也有很多人认为,吴方言中``谓搬移重物、随地转掇曰勃窣'',``今嘉定呼人体笨行步不轻脱曰勃窣''。谁知道这句话是说张凭是走得快还是走的慢,走得快是比喻他才气飞扬,走得慢是说他学识百转千回。对古人来说,也许这是个比较普通的词,隐含意也很明确,对于我们现代人来说,一辈子都不懂,一辈子都不会用上它,它是已经死亡的词语。
  理窟:是说他学问大,是义理之仓库。譬如魏晋时人称杜甫的老祖宗杜预为``杜武库'',言其学问渊博,无所不有。宋代刘克庄说晚年的自己:``理窟骚坛两罢休,倒持麈柄让名流。''
  太常博士:负责宗庙礼仪祭祀工作,七品,清要职位,所以张凭后来逐渐提拔,当了吏部郎,御史中丞,司空长史等。

\section{4.55}\label{section-232}

\begin{quote}
支道林、许、谢盛德共集王家。谢顾谓诸人:``今日可谓彦会。时既不可留,此集固亦难常,当共言咏,以写其怀。''许便问主人有《庄子》不,正得《渔父》一篇。谢看题,便各使四坐通。支道林先通,作七百许语,叙致精丽,才藻奇拔,众咸称善。于是四坐各言怀毕。谢问曰:``卿等尽不?''皆曰:``今日之言,少不自竭。''谢后粗难,因自叙其意,作万余语,才峰秀逸。既自难干,加意气拟托,萧然自得,四坐莫不厌心。支谓谢曰:``君一往奔诣,故复自佳耳。''
\end{quote}

  支道林、许询、谢安等人到王濛家集会清谈。
  盛德:语法使用有些费解,《左传?文公十八年》:``少皥氏有不才子,毁信废忠,崇饰恶言,靖譖庸回,服谗蒐慝,以诬盛德。''
西晋杜预注:``盛德,贤人也。''《世说新语?企羡?4》:``后
殷浩为长史,始到,庾公欲遣王使下都,王自启求住,曰:`下官希见盛德,渊源始至,犹贪与少日周旋。'\,''
这里盛德的用法很好理解,确实是贤人的意思,但``支道林、许、谢盛德''的用法很别扭,写成``支道林、许、谢各盛德''就好理解多了。其实把盛德去掉更好。
  顾谓:见2.33``丞相因觉,谓顾曰'',
2.61``简文入华林园,顾谓左右曰'',对四周众人说。   彦会:群英会。
  时既不可留,此集固亦难常:陶渊明《杂诗十二首》``人生无根蒂,飘如陌上尘。\ldots{}\ldots{}盛日不重来,一日难再晨'',王勃《滕王阁序》``胜地不常,盛筵难再。兰亭已矣,梓泽丘墟。''
  言咏:言谈吟咏。   写:抒发。
  正得《渔父》一篇:刚好有《庄子?渔父》一篇文稿。当时没有印刷术,纸张产量又不大,所以左思一篇文章出来,大家必须手抄,就成了洛阳纸贵。《庄子》有10几万字,王濛家里大约就抄了其中一篇《渔父》。
  《渔父》属于《庄子》的杂篇,一般认为不是庄子本人写的。其主要内容是孔子遇见一个打渔的隐士,他告诉孔子,不要不在其位而谋其政,各安其位才是最好的治理之道。做人不要多事,要顺应天道,注重真实,不拘泥于世俗条规。譬如真实的悲痛没有哭声而哀伤,真实的怒气未曾发作就有威严,真实的亲热未曾含笑而和善;譬如侍奉双亲目的在于达到适意,不必考虑使用什么方法;饮酒目的在于达到欢乐,不必讲究使用什么样的餐具;居丧目的在于致以哀伤,不必过问规范礼仪。
  谢看题:费解,看题也许是出题的意思,也许是把文章读给大家听。
  便各使四坐通:四坐,大家;通,阐述。我猜想谢安是这次玄谈的主持人,利用《渔父》这篇文章出了一些题目,叫大家选择陈述自己的理解和看法。
  叙致:叙、致同义,陈述,见4.42。
  各言怀毕:怀,怀抱,心意。《孔雀东南飞》:``感君区区怀''。
  少不自竭:``少''通``稍'',很少还有不竭尽的(几乎都尽了)。
  粗难:难,提问。粗难,对大家的阐述简要地提出一些问题,然后谢安自问自答。
  才锋:才华。刘勰《文心雕龙?诔碑》:``自后汉以来,碑碣云起。才锋所断,莫高蔡邕。''
  既自难干:此句费解,``既自''大概是``已经''的意思。陶渊明《归去来兮辞》:``既自以心为形役,奚惆怅而独悲'',东晋袁山松(袁崧)的《宜都山川记》:``既自欣得此奇观,山水有灵,亦当惊知己于千古矣''。既自应该解释为已经。``自''我以为在魏晋时候常常做语衬词,没有意义。学者纪果庵也指出:``如`自'、`本'、`正'、`固'等字,均为表现一种沉吟意味之助字,似为清谈中所最不能免去者。''干可以解释为到达。这句话也许是说,(谢安讲得好,大家感到)已经难以企及。
  加意气拟托:加上谢安把自己的情趣投向于文中渔父的情趣。《渔父》一文中孔子称其为``圣人''。
  萧然:洒脱貌。   厌心:心满意足。
  一往奔诣:奔诣,快跑貌。一往奔诣应该就是一向绝尘而去。
  故复自佳:故复,自然,当然,《太平御览》``言我不能歌,听我歌梅花。今年故复可,奈汝明年何'';自,无意义衬词。当然很好。
  《世说新语》的语言有一个特点,就是使用大量虚词,很接近于口语,这样使文章阅读方便,音调悠扬,格调舒缓。但由于时代变迁,我们现在阅读《世说》,依旧会遇到各种障碍,有些人自以为看懂了,其实是看错了。

\section{4.56}\label{section-233}

\begin{quote}
殷中军、孙安国、王、谢能言诸贤,悉在会稽王许。殷与孙共论《易象妙于见形》,孙语道合,意气干云。一坐咸不安孙理,而辞不能屈。会稽王慨然叹曰:``使真长来,故应有以制彼。''即迎真长,孙意己不如。真长既至,先令孙自叙本理。孙粗说己语,亦觉殊不及向。刘便作二百许语,辞难简切,孙理遂屈。一坐同时拊掌而笑,称美良久。
\end{quote}

  孙安国:孙盛,史学家,见4.25和4.31等。他的两个儿子孙潜(齐由),孙放(齐庄)我们在2.49和2.50见过。
  王、谢:程炎震说是王濛、谢尚,不是王羲之和谢安。
  会稽王:简文帝司马昱。
  《易象妙于见形》:《易经》有两派,一个是象数派,一个是义理派。具体我不懂,大约一个是用易算命,一个用易谈哲学。孙盛是象数派,因为他说:``易之为书,穷神知化,非天下之至精,其孰能与於此?世之注解,殆皆妄也。况弼以傅会之辨,而欲笼统玄旨者乎?故其叙浮义则丽辞溢目,造阴阳则妙颐无闻,至于六爻变化,群象所效,日时岁月,五气相推,弼皆摈落,多所不关。虽有可观者焉,恐将泥夫大道。''孙盛认为,易本来就是用来推算,可以穷究事物之神妙,了解事物之变化。而王弼的注解,就是摒弃了易的实际作用,而是用附会之词,空讲道理,文章辞藻不错,但是没有真正认识到易经参阴阳造化的作用,好看是好看,但没有向``大道''挺进。他写的文章《易象妙于见形》,题目的意思就是易经象数之学妙处就在于能够落实到现实生活中。
  孙语道合:古时候的人普遍迷信书本,迷信鬼神,史书上还真有些算命准的事例,孙盛随手拈来一些事例,他在讲故事和莫名其妙的天意,一下子哪里找得到推翻的证据,所以大家不服,但也无可奈何。因为东晋时候,玄学家继承的是王弼的玄学,对象数之学感到不舒服,但又说不过孙盛。打个比方,我看《史记?扁鹊仓公列传》,有些故事违背医学,明显是胡扯,但司马迁很认真地把故事当做真事记录下来,你拿什么去辩驳它啊。
  意气干云:和豪气干云差不多,意气风发直入云霄。
  真长:丹阳尹刘惔,由此印证其是当时第一流的玄学名家。
  殊不及向:孙盛被刘惔的名声吓住了,即使再复述一遍,因为没有气势,也感到很不如刚才阐述的那么精彩。
  辞难简切:大概就是辞简难切,言辞简洁,质疑有力。
  孙盛击败殷浩等人,由此知名。而大人物刘惔最后出场,一举制胜。他不会在玄学奥运会上玩假摔,不会雇一帮子水军在各论坛上造声势,也不会离了婚以后去活佛那里求心安。

\section{4.57}\label{section-234}

\begin{quote}
僧意在瓦官寺中,王苟子来,与共语,便使其唱理。意谓王曰:``圣人有情不?''王曰:``无。''重问曰:``圣人如柱邪?''王曰:``如筹算,虽无情,运之者有情。''僧意云:``谁运圣人邪?''苟子不得答而去。
\end{quote}

  僧意:其历史不清。
  瓦官寺:东晋南京著名寺庙,其中有三绝,斯里兰卡贡献的玉如来(可能吗?外国人也会把玉当成宝?狮子国就是一个传说中的国家),顾恺之的维摩诘画像,戴逵(雪夜访戴的戴安道)制作的五具佛像雕塑。
  王苟子:王濛的儿子王修。
  唱理:估计和看题意思差不多,发出论题,以便讨论。
  圣人有情不:圣人有没有情感?圣人指谁,孔子眼里的圣人大约是周公;庄子眼里的圣人肯定不是孔子,而是理想人格,在《庄子》中,特别是外篇和杂篇,作者常常拿孔子来做例子,而``孔子''是一个象征符号,一个有学问有见识的人,但还没有造就理想人格;魏晋时候一般读书人眼里的圣人大概是孔子,不是老庄。但现在他们讨论的圣人,是指孔子还是理想人格就不知道了。但他们讨论的是玄学,这次讨论的题目来自《庄子》。《庄子?德充符》:``惠子谓庄子曰:`人故无情乎?'庄子曰:`然。'惠子曰:`人而无情,何以谓之人?'庄子曰:`道与之貌,天与之形,恶得不谓之人?'惠子曰:`既谓之人,恶得无情?'庄子曰:`是非吾所谓情也。吾所谓无情者,言人之不以好恶内伤其身,常因自然而不益生也。'\,''
庄子的观点是``圣人无情'',意思是说,圣人不会妄情,而是顺应秉性,不会因为外界事物而伤害他的本性。所以王弼解释说:``圣人之情,应物而无累于物者也。''
  但是问题并没有这么简单,道家对此还有其他论述,而且一针见血,譬如老子说:``天地不仁,以万物为刍狗;圣人不仁,以百姓为刍狗。''圣人和天地一样冷酷无情,由此,玄学家何晏进而指出,圣人无喜怒哀乐之情。所以后来的理学家为了圆融这种说法,就解释说:圣人的精神境界是和天地一样的广大,对于他来说,现实世界没有主观和客观的分别,所以圣人没有专为他自身的利益而引起的感情,他的感情特点是无私。这里的无情,又解释为``无私''。还有理论家觉得还可以修正,又提出圣人的感情处于有情与无情之间。
  圣人如柱:僧意开始使用诡辩之术------比喻。理想人格怎么可以用比喻来讨论呢?比喻的特点就是只取一端。但是王修学养不够,就被僧意牵着鼻子走,落入其毂中。僧意问,圣人无情,那么是像一根高高的柱子吗?
  如筹算:有点像珠算,但算盘没发明之前,古人用竹签来计算。王修的意思是说,圣人无情,不像柱子那样无情,而像算签无情一样。算签能运作,人的感情就投影在其中,我们认识圣人是有情的。僧意立刻追击:谁在运作圣人?
  我们在辩论的时候,一般不能使用比喻,比喻使人被动。而且王修并没有深刻思考这个命题,如果他开始要求僧意解释什么是圣,什么是情,什么是有,僧意哪里能赢得这么轻松!
  天地固然无情,但圣人毕竟不是天地,他还带有人的属性。在《世说》中,玄学家还有一种观点,``太上忘情'',是有而忘之。

\section{4.58}\label{section-235}

\begin{quote}
司马太傅问谢车骑:``惠子其书五车、何以无一言入玄?''谢曰:``故当是其妙处不传。''
\end{quote}

  司马太傅:司马道子,简文帝司马昱的儿子,孝武帝司马曜的弟弟,谢安死后,司马道子执政。
  谢车骑:谢玄,死后追封车骑将军。谢玄死的也蛮早的,大概谢安死了两三年,谢玄就跟着去了。
  惠子:惠施,庄子的辩友。《庄子?天下篇》:``惠施多方,其书五车。其道舛驳,其言也不中。''学富五车的出典,说他著述有五车。但是惠施的著作没有流传下来,他的一些论述从转载记录看,主要是谈逻辑学,不太谈世界观问题。
  故当:应该。
  谢玄回答的意思是玄学妙不可言,所以惠施没有论述。当然,我们还可以理解为,惠施的文章散失了,其精妙的地方没有流传下来,就只能看到他怎样被庄子做靶子批评。
  这个故事是不是说明在东晋时候,惠施的作品也能读到?还是当时也是通过《庄子》来理解惠施的主要思想?
  《庄子》中讲述的故事一般不靠谱,他往往使用寓言手法。惠施的学术思想和他的作品没有流传下来,恐怕也不是什么妙不可言的问题,而是逻辑学和自然科学研究在中国文化中没有多少市场。我们不能全面准确地了解惠施的主要思想,但是名家的其他著作还是有一些的,但也没有引起当时和后人的多少重视。这也许是因为中国的学术思想主要是为政治和伦理道德服务,与政治无关的东西一向不大摆上台面,学者们没有多少探究事物本质的兴趣爱好,他们对真理漠不关心,而在统治术、人际关系和如何抚平自己烦躁不安的心灵上花费了太多的精力。
  令人回味的还有,庄子生前落魄,但身后却如此显赫;惠子生前常居高位,但他超越庄子的思想却无法流传和供后人学习。20世纪30年代,沈从文写道:``说句公道话,我实在是比某些时下所谓作家高一筹的。我的工作行将超越一切而上。我的作品会比这些人的作品更传得久,播得远。我没有方法拒绝。''也许是因为惠施思想虽然高妙,但文辞不够好吧。哦,也许不是,是天意弄人,任何人都没有方法拒绝。

\section{4.59}\label{section-236}

\begin{quote}
殷中军被废,徙东阳,大读佛经,皆精解,唯至事数处不解。遇见一道人,问所签,便释然。
\end{quote}

  350年,殷浩都督扬、豫、徐、充、青五州诸军事,主持北伐,其负责时间有近4年。当时石勒死亡后,北方极乱,但在这3年多时间里,殷浩缺乏粮草,内部关系又十分复杂。殷浩把主要精力放在排除异己上,结果洛阳还没到,就引发其部下的少数民族将军姚襄反叛,殷浩兵败,成为历史的罪人。当然,殷浩未必要负主要责任,而是一味听从朝廷的遥控指挥,对降者猜忌,试图通过北伐而消灭这些杂牌部队,而杂牌部队不甘心当炮灰,奋起反抗。对于起义将领怎么对待处理,总要有个消化过程,无非就是这么几招:把将领召回南京,许以高官,将兵分离;把将领和其部下安置一郡,缓缓图之;对将领推心置腹,委以重任,使其感恩戴德;对将领不温不火,挑几个伤亡不大的战役给他打,安其心思。但东晋朝廷昏聩,进退失据,多次派刺客暗杀起义将领姚襄,而姚襄偏偏又``雄武多才艺,明察善抚纳,士众爱敬之'',``上下咸允,人尽死力'',于是大好局面,毁于一旦。
  唯至事数处不解:4.43``殷中军读《小品》,下二百签,皆是精微,世之幽滞''。事数,就是佛教中的各种名相概念规则,又叫法数,如三界、四谛、五蕴、六度、八正道、十二因缘、五位七十五法之类。因为这些词语是用所谓``带数释''方式构成的,所以叫``事数''。这是佛教的基本概念,但当时没白话本,没佛教大辞典,其概念又相互交叉含混,自学起来麻烦是自然的事情。
  道人:僧人,大概就是指4.47中自信有料的康僧渊。
  说殷浩兵败后才读佛经,不过是他的一种姿态,``我已经看透了,看透了!''
  我看过一则小故事,说道学家曾国藩居然也纳妾,他向幕僚解释说:``纯为癣疾复发,夜间须人搔抑,并非溺于女色。''我晚上有时候背上也有点痒意,就自己抓两下,那敢有纳妾的想法啊?老曾啊,你身上痒,看看佛经就能止痒,不用娶小妾的。

\section{4.60}\label{section-237}

\begin{quote}
殷仲堪精核玄论,人谓莫不研究。殷乃叹曰:``使我解`四本',谈不翅尔!''
\end{quote}

殷仲堪:清谈家殷浩的一个侄子,当过荆州刺史,见1.40等。陈郡殷氏,以这两个人最有名,皆以玄学家、医生和个人的品操著名于世,但都败于军事。
  精核:精通。
  四本:他叔叔最精通才性的四本论,但殷仲堪好像没继承家学。但这句话有点奇怪,四本论是当时玄谈的一个主要话题,是请客的必备菜肴啊。《南齐书?王僧虔传》说道:``才性四本,声无哀乐皆言家口实,如客至之有设也。''连这个都没想清楚,就敢``庖厨不修,而欲延大宾者哉?''
不过也好理解,书话中有位悠哉兄,把自己的作品放在古往今来第三位,排在他前面的大概是曹雪芹和塞万提斯:``《燕梦园》我写的还不够好,比《红楼梦》和《唐吉可德》还差点。''``我叔叔最擅长的四本论我还差点,不然,嘿嘿,我还不止现在这样呢\ldots{}\ldots{}''
  翅:同``啻'',止,仅仅。

\section{4.61}\label{section-238}

\begin{quote}
殷荆州曾问远公:``易以何为体?''答曰:``易以感为体。''殷曰:``铜山西崩,灵钟东应,便是易耶?''远公笑而不答。
\end{quote}

  殷仲堪问僧人慧远。慧远是很有名的一个和尚,他提出,只要念阿弥陀佛就能进入西方净土,由此创立了净土宗。至今的民间佛教,基本都属于净土宗,但从另一个方面说,慧远还是白莲教的创始人,他的净土理论流变成乱世中弥勒佛会下界救众生,后来百姓往往依靠这个信仰而起义,宋以后的造反运动,白莲教功不可没。
  感应思想是《周易》的基本和核心:《周易?系辞传》:``易无思也,无为也,寂而不动,感而遂通天下之故,非天下之至神,其孰能与于此。''周易本来就是算命的书,古人认为,未发生的或已发生的但人们还未觉察到的事物,可以感应与其相类似的事物或事情,并由这类事物得到预测或预言。他们再进一步思考,事情或事物的成功失败,可以通过改变它先前的感应物(预兆)而改变事情或事物本身。譬如说,打仗前军旗断了,就象征着这次战役没有好结果;同时,如果打仗前你把敌人的军旗搞断,就说明你能打败敌人。这种联想性思考现在看来是很荒诞的,但它把宇宙当做一个整体,万物之间互相联系也很难说没有可取之处。譬如说混沌学的``蝴蝶效应''和``马蹄效应'',你就很难说它不科学,但是你要找出事物之间相互联系的规律,却必须遵守同类相比的原则,如果瞎扯就是迷信。《周易》从根本来说,就是一本瞎扯的书。现在有人拿周易看风水,也是利用感应思想,但是隋文帝杨坚很直接地说:``如果说我家的墓地没有好风水,我不当为天子;如果说我家墓地风水很好,我弟弟怎么会早夭呢!''史书上说,《周易》的老祖宗周文王的儿子武王出兵打商纣,多次占卜都没有好结果,军旗也被风吹断了,大家都吓坏了,吕尚毫不客气地说:``枯骨死草,何知吉凶!''
  ``铜山西崩'':相传汉武帝时候,宫殿里的大钟无故自鸣,东方朔说,恐怕有地震,因为铜采之于山,山是铜的妈妈,妈妈塌陷,儿子就受到感应发出声响。过了几天湖北来报,当地发生地震,山塌了。据说张衡制作了个地动仪,可以感应地震,很多学生是信的,我不信。
  笑而不答:慧远的感应论不仅仅局限于中国思想的天人感应说,``儒道九流学说,皆如糠秕'',而是把佛教的``业报''融入其中,他抛出了振聋发聩的高论:``佛有自然神妙之法,化物以权,广随所入,或为灵先转轮圣帝,或为卿相国师道士。若此之伦,在所变现,诸王君子,莫之为谁!''他的意思是说,中国的圣人可能是如来化身成的各种殊相,他还接着说,如来的投影,不仅仅是圣人,而且是世界万物,包括自然景物:``是故如来或晦先迹以崇基,或显生涂而定体,或独发于莫寻之境,或相待于既有场。独发类乎形,相待类乎影,推夫冥寄,为有待耶,为无待耶!''
世界万物皆包含在佛的感应范围,山河大地都是佛的影迹。你说,他会回答殷仲堪这个小儿科的问题吗?

\section{4.62}\label{section-239}

\begin{quote}
羊孚弟娶王永言女。及王家见婿,孚送弟俱往。时永言父东阳尚在,殷仲堪是东阳女婿,亦在坐。孚雅善理义,乃与仲堪道``齐物''。殷难之,羊云:``君四番后,当得见同。''殷笑曰:``乃可得尽,何必相同?''乃至四番后一通。殷咨嗟曰:``仆便无以相异。''叹为新拔者久之。
\end{quote}

  羊孚:见2.104,2.105,他反应敏捷,文采突出,可惜31岁就死了。其弟羊辅。
  王永言:琅邪王讷之。琅邪王氏取名有个特点,就是有很多``之''字,因为他们信奉天师道,之是信仰道教的标志,所以不用避讳。
  见婿:新娘婚后几天之内要回娘家省亲一次,一般是在第三日。女方家人此须准备午宴招待。
  东阳:王讷之的父亲王临之,曾任东阳太守。
  齐物:《庄子?齐物论》,庄周梦蝶就出于这一章。其大意是说各家学派对事物的看法纷纭不齐,皆由执我见所致。是非没有标准,没有绝对真理,正确的态度就是物我两忘,不言不辨,不存是非计较之心,才能超越各家的理论。当然,《庄子》为文随便,空话大话神话连篇,自相矛盾之处甚多,他到底是什么明确的态度我们不太清楚。
  难:在辩论中,羊是主,殷是客,殷负责提出质疑。   四番:四个回合。
  ``乃可得尽'':(``齐物论''的讨论)怎么能够穷尽呢,我们为什么定会得出相同的看法?乃可,岂可,《任诞?20》:``卿乃可纵适一时,独不为身后名邪?''。何必,为什么一定,《德行?26》:``百里奚亦何必轻于五羖之皮邪?''
  四番后一通:这里的``一通''有点难理解,``通''在《世说》清谈中往往是阐述的意思,本句大概就是四个回合后大家一沟通的意思。还有可能,本文的意思是``乃至四番后,(羊孚)一通'',或者``后''当作``羊'':``乃至四番,羊一通。''
  新拔者:拔,挺拔。新秀。殷比羊长一辈,岁数可能大上10几岁。殷自称仆,很谦虚的说法。
  这一则当中,我们可以发现东晋时期士族的婚姻状况。两晋南北朝乃至唐朝,士族多在内部通婚,从而形成相互支持,保证家族政治和社会地位长久不衰。即便是南方士族和北方士族,开始时也不太愿意互相通婚。南渡后琅邪王氏想拉拢江南士族,向吴郡陆氏求婚。陆氏回答道:``义不能为乱伦之始!''

\section{4.63}\label{section-240}

\begin{quote}
殷仲堪云:``三日不读《道德经》,便觉舌本间强。''
\end{quote}

  舌本间强:舌本,舌根;间强指人的身体上某一器官沉滞不灵。``舌本间强''出自《灵枢经?经脉第十》:``脾足太阴之脉\ldots{}\ldots{}连舌本,散舌下\ldots{}\ldots{}是动则病舌本间强。''
舌根发硬。据说殷仲堪老爹生病,他一心要治好父亲的病,苦读医书,以至于一只眼睛瞎了,所以他用医学术语来形容其玄学体悟。
  这句话后人多有引用,譬如宋代黄庭坚说:``士大夫三日不读书,则义理不交于胸中,对镜觉面目可憎,向人亦语言无味。''米芾说:``一日不书(练书法),便觉思涩,想古人未尝片时废书也。''就连毛泽东据说也讲过类似的话:``三天不学习,赶不上刘少奇。''
  《道德经》据说是老子所写,但根据近些年出土的文物,1973年马王堆帛书《老子》德经在前,道经在后,和今本《老子》意思有不少差别;1993年郭店楚简的《老子》版本居然就1000多字。由此可见,今本《老子》属于经过整理、累积的作品。当代杨叔子要求其学生熟读《老子》,``自己的硕士生、博士生必须能背诵《道德经》才能毕业''。读当然可以读,但要大学生看懂《老子》究竟说什么,太难了。至少我看不太明白,杨叔子也肯定看不太明白,自己不明白,还敢强行解释,还要学生背,这才是大师的风范。

\section{4.64}\label{section-241}

\begin{quote}
提婆初至,为东亭第讲``阿毗昙''。始发讲坐裁半,僧弥便云:``都已晓。''即于坐分数四有意道人,更就余屋自讲。提婆讲竟,东亭问法冈道人曰:``弟子都未解,阿弥那得已解?所得云何?''曰:``大略全是,故当小未精核耳。''
\end{quote}

  提婆:全名瞿昙僧伽提婆,西域僧人。
  东亭:东晋东亭侯王珣,王导的孙子,谢安弟弟谢万的女婿,著名书法家,伯远帖书法极佳。成语``如椽大笔''的主人公就是他,有一次他梦见有人给了他一支大笔。醒来后他就知道自己要负责起草朝廷的各项命令了。中国学者乃至中国人,信仰包容,僧道往往兼信,王珣更是``雅有信慧,住持正法,建立精舍,广招学众''。
  第:府邸。
  阿毗昙:佛经。383年,瞿昙僧伽提婆译出《阿毗昙八犍度论》(《发智论》)30卷,391年,译出《阿毗昙心论》4卷。有著作的人总是比较受人尊敬,伽提婆来到建康后,``晋朝王公及风流名士,莫不造席致敬''
  始发讲坐裁半:``裁''通``才'',才开始演讲到一半的时候。讲坐当做``讲座''解释很简单,但文言估计没有这种用法,``坐''可删,也可理解为:开始发表讲演,(大家)坐了才一半的时间。王珉就拉人自己开课的做法确实另类。
  僧弥:王珉字季琰,小名僧弥。王珣的弟弟,谢安的女婿,书法家。王珣和王珉后来都和谢家的女儿离婚了,王珉后来另娶。据《太平广记?报应九》中说:``晋琅琊王珉,其妻无子,尝祈观世音云乞儿。珉后路行,逢一胡僧,意甚悦之。僧曰:`我死,当为君作子。'少时,道人果亡,而珉妻有孕,及生能语,即解西域十六国梵音,大聪明,有器度,即晋尚书王洪明身也。故小名阿练,叙前生时,事事有验。''
王珉其实早有大儿子王朗,佛家不过是因为王家兄弟亲近佛教,故意编出这种乱七八糟的故事,利用他们的名声来做宣传罢了。可惜当时没有广告代言,不然王练可以从中国佛教协会拿多少广告费啊,反正传这种谣言衙门不管的。
  都已晓:魏晋时候开始用``都''来表示全部的意思,这是以前没有的用法。
  数四有意道人:数四表示为数不多;有意道人,有见解的僧人。僧人出入于高官家中,这是东晋以后的风尚,有时也被人诟病。
  法冈道人:僧人法冈,据说这次集会``名僧毕集''。法冈属于``有意''僧人,被王珉拉去听他的阐述,做出评价。
  所得云何:他的体悟阐述得怎么样?   故当:总应该。
  小未精核:精核,精通,``殷仲堪精核玄论''。大体上都对,(毕竟才听了个把小时),总应该稍稍不够精通。
  《高僧传》中说,这个故事说明伽提婆``宗致既明,振发义奥,\ldots{}\ldots{}其明义易启人心'',但本文的意思应该是王珉悟性比哥哥高出不少,一听便懂,而且为人特立独行。也有资料表明,僧伽提婆397年才到建康,当时王珉已死,这个故事是假的。佛教的故事大多是假的,为什么老百姓还这么喜欢它?``转轮圣王''李某一语破的:``话不说大点没人信!''

\section{4.65}\label{section-242}

\begin{quote}
桓南郡与殷荆州共谈,每相攻难。年余后,但一两番。桓自叹才思转退,殷云:``此乃是君转解。''
\end{quote}

  桓玄和殷仲堪是总角之交。殷仲堪出任荆州刺史,桓玄就在其治下的江陵生活(其爵位是南郡公),荆州本来就是其父桓温、其叔桓冲的根据地,桓玄在荆州官民畏惧他更过于殷仲堪,殷仲堪为拉拢桓玄父叔部下的支持,因而关系更加好了,这个故事就发生在殷仲堪出任荆州刺史期间。
  但一两番:一年前是常常没完没了的互相考问质疑,一年后辩论就一两个回合了。
  才思转退:转,渐渐。《雅量?28》``既风转急'',当时桓玄不到30岁,按道理说才思不会减退。
  是君转解:解,明白。是你渐渐领悟到玄学的根本,(与我的看法越来越接近了)。
  殷仲堪是玄学大家,三天必读《道德经》的。而《道德经》恰恰是本政治统治阴谋宝典,但殷仲堪在治理荆州的时候,已有人批评他``用计倚伏烦密,少于鉴略'',这也许说明计谋这种东西不能太复杂,环节不能太多。如果不抓住主要问题,伏笔太多,环节太多,难免会出现意外,其中一链断掉,满盘皆输。殷仲堪后来被桓玄弄死了。

\section{4.66}\label{section-243}

\begin{quote}
文帝尝令东阿王七步中作诗,不成者行大法。应声便为诗曰:``煮豆持作羹,漉菽以为汁。其在釜下然,豆在釜中泣;本自同根生,相煎何太急!''帝深有惭色。
\end{quote}

  这是中国人耳熟能详的典故,但这也许仅仅是个故事。曹丕要杀曹植,何必用写诗来做借口?能写诗就不杀,写不出来就杀,观众们在看生死时速,如此惊心动魄,如此不符合帝王的手段。帝王杀兄弟、子女、长辈还需要理由吗?自商纣王开始,子侄杀叔父,父母杀子女,子女杀父母,兄弟杀,奶奶杀,夫妻杀,外公杀,岳父杀,帝王家史何止七杀!以至于刘宋的末代皇帝刘准泣叹:``愿生生世世,再不生帝王家!''
  历史上据说最优秀的皇帝刘邦,他的灵魂是多么肮脏。他笑着对项羽说:``我愿意与你共享我父亲的肉羹!''他打败仗逃跑,把儿子老婆推下车赶路。他早就知道心爱的妃子将不得好死,但就仅仅是迷恋她的身体,根本不做安排,戚夫人后来被``断手足,去眼,煇耳,饮瘖药,使居厕中''。杀人魔王石勒曾经说:``朕若逢高皇,当北面而事之,与韩彭竞鞭而争先耳。''是啊,任何心狠手辣的皇帝,在刘邦面前就是一个渣。正如马基亚维利指出的那样:``君主必须是一头狐狸,以便认识陷阱;同时又必须是一头狮子,以便使豺狼惊骇。''看惯了中国历史变幻的小说家,终于借一代枭雄曹操之口喊出了无数豪杰的心声:``宁教我负天下人,休教天下人负我!''
  然而,曹丕不是刘邦,尽管他性格狭窄,但他还是一个有文化和残留人性的人。他不乏同情,弹着琴、吹着笛子唱道:``援琴鸣弦发清商,短歌微吟不能长'',``悲弦激新声,长笛吹清气。''他恋子之心和别的人没有差异,他和曹叡放下弓箭,不忍心射杀小鹿;他对死去的朋友也很悲痛,在王粲的墓前一声学驴鸣以悼念。他心中犹豫,在深夜起来徘徊:``漫漫秋夜长,烈烈北风凉。展转不能寐,披衣起仿徨。''曹植是杀还是不杀,他心中没有决断,没有快意恩仇,最终还是放下了屠刀,如果换成曹植,也许就下手了;他对司马懿有所戒备,但还是没有为子孙斩去荆棘,司马懿后来却下手了。他天资聪明,文学修养很高,创作、评论多有建树,在``立言''上影响深远,是中国历史上少有的文学皇帝,以至于陈寿感叹说:``若加之旷大之度,励以公平之诚,迈志存道,克广德心,则古之贤主,何远之有哉!''
  曹丕、曹植寿命都不长,都是刚过40就死了。

\section{4.67}\label{section-244}

\begin{quote}
魏朝封晋文王为公,备礼九锡,文王固让不受。公卿将校当诣府敦喻。司空郑冲驰遣信就阮籍求文。籍时在袁孝尼家,宿醉扶起,书札为之,无所点定,乃写付使。时人以为神笔。
\end{quote}

  264年,魏元帝曹奂下诏拜司马昭为相国,封晋王,加九锡。265年,司马昭去世,谥文王。同年司马炎称帝,追封司马昭为文帝。263年,阮籍已经去世,所以这个故事要向前推,但具体时间有多次说法,有人指出可能发生在258年左右,那次司马昭受曹髦封为晋公,据说9次辞让,没有接受九锡。
  九锡:古代天子对有大功的诸候大臣加以九锡,即赏赐车马、衣物等九种礼物,这是位极人臣的标志,王莽、曹操都受过九锡。蜀汉的李严也劝诸葛亮受九锡,诸葛亮说,只要能统一中国,十锡也可以。
  固让不受:古代的一种礼节,主人发出邀请,客人要``三辞'',礼辞、固辞、终辞。当对方对你发出第一次邀请时,你应该礼貌地婉言谢绝;当对方再次向你发出邀请时,你应该再次礼貌地婉言谢绝;当对方第三次向你发出邀请时,你才表示同意。提拔官爵、东西要三辞,大臣要辞职也要三辞。当然,三辞可以接受,也可以不接受,一般总是接受的多。其中的文书来往就很讲究了,所以要大手笔来写。
  敦喻:劝进,劝司马昭接受。
  郑冲:当时郑冲是三公之一,大儒,朝廷的代表人物。
  袁孝尼:袁准字孝尼,向嵇康学过广陵散的那个人。史书上说他``不耻下问,唯恐人不胜己也。世事多险,故恬退不敢求进'',那是很了不起的人物了。
  书札:札本来是指竹木简。估计东汉用竹木简打草稿,然后誊抄在帛或纸上。书札在这里就是草稿,随手写的意思吧。
  点定:修改。苏轼:``信笔直书,不加点定,殆是天成。''
  劝进文中说:``圣王作制,百代同风,褒德赏功,有自来矣'',伊尹、周公、吕尚都不如司马昭,``自是以来,功薄而赏厚者不可胜数,然贤哲之士犹以为美谈'',接着列数司马昭的功绩,``元功盛勋光光如彼,国士嘉祚巍巍如此'',理应``宜承圣旨,受兹介福,允当天人'',待天下统一后,期望司马昭``临沧州而谢支伯,登箕山而揖许由'',这样就可以做到天下独一无二,没有人能够再和司马昭相提并论的:``至公至平,谁与为邻!''这篇文章虽然不长,但一般不可能在醉酒时可以一蹴而就的,阮籍打过腹稿,里面的话也未必是违心之论
  也许是受到鲁迅演讲的影响,现在的一般读者认为阮籍和司马昭的关系不好,与嵇康一样对司马政权采取了不合作态度。而事实是,嵇康是夏侯家的女婿,站在曹家一边可以理解,但阮籍一向和司马昭交好亲密,恐怕不是委以虚蛇可以形容的。当时很多人看不惯阮籍,司马昭却很赏识阮籍,采取了保护措施。嵇康也说:``(阮籍)幸赖大将军每保持之。''而诋毁阮籍的何晏、钟会却被杀了。阮籍任步兵校尉,爵位关内侯,属于司马家的亲信才能担任这种岗位。司马昭曾向阮籍提出结儿女亲家,算是低就,阮籍想远离是非才以装醉不予答复,但也足见司马昭对阮籍的看重。阮籍拒绝儿女婚事后,司马昭也不以为忤,照样亲厚阮籍,``恒与谈戏,任其所欲,不迫以职事''。司马昭的儿子司马炎后来当皇帝的,阮籍的女儿本可以当皇后,这仿佛是次很可惜的机会,但正因为阮籍的远见卓识,避免了阮家一头扎进西晋时期``多险''、``多故''的政治漩涡。

\section{4.68}\label{section-245}

\begin{quote}
左太冲作《三都赋》初成,时人互有讥訾。思意不惬,后示张公。张曰:``此二京可三,然君文未重于世,宜以经高名之士。''思乃询求于皇甫谧。谧见之嗟叹,遂为作叙。于是先相非贰者,莫不敛衽赞述焉。
\end{quote}

  左思左太冲,是志气很高的人。他的家庭出身不高,父亲左雍原本是个小吏,女儿入宫做了司马炎的女人,才渐渐有了上升通道。他相貌差,口吃,悟性也不好,学书法、学音乐都不像有前途的样子。父亲左雍就对朋友说:``看左思现在的学识水平,还不如我小时候。''左思受到刺激,学习发愤起来。
  《三都赋》内容不全是描绘魏蜀吴的都城,而是写这些国家的具体情况,沿革风貌等,因为内容很多,实地考察多有不便,左思就去问旅居洛阳的各地学者,广泛收集资料,多次斟酌修改。据说他一写就是十年,家里到处放着纸笔,等待灵感来了就写一句,还为此申请当了秘书郎(管理图书)。他向出生名门的大才子陆机咨询吴国的情况,陆机草草应付,回头就写信和兄弟陆云说:``此间有伧父,欲作《三都赋》,须其成,当以覆酒甕耳。''
  《三都赋》用现在的文学标准来看,并不算一流的作品,太长,用词太生僻,好多字我都不认识,而且作家的感情和人格力量没有得以充分地表现。但它的特点是辞藻优美,对仗高超,词汇丰富,内容广博详实,是很好的历史文化、物产资源、语言修辞学教材。在没有语文、历史、地理教科书的魏晋时期,《三都赋》就可以充当这个角色。
  张公:西晋司空张华,其性格是``性好人物,诱进不倦,至于穷贱候门之士,有一介之善者,使咨嗟称咏,为之延誉''。在我国文化中,识人是很高的才能,是圣贤、君子的真正标志。在魏晋时期,人物识鉴更是最高的学问,魏国刘邵专门写了《人物志》,开宗明义就是说:``夫圣贤之所美,莫美乎聪明;聪明之所贵,莫贵乎知人。''这些话是有依据的。孔子说,君子不器,他认为君子不需要在某方面有特别的长处。樊迟问孔子什么是真正的智慧,子曰:``知人。''樊迟不是很明白,孔子接着解释:``选拔正直之人,罢黜不合正道的人,这样能使枉者归正(举直错诸枉,能使枉者直)。''后来清代曾国藩跟进:别人真正比不上君子的地方,就是``慧眼善识''(``君子之所人不及,在君慧眼善识人'')。
  此二京可三:东汉班固作《两都赋》,张衡作《二京赋》,《三都赋》可与之三足鼎立。
  皇甫谧:当时的大学者,一辈子写书而没有出仕,人称``书淫'',司马炎要他做官,他一口回绝,并上表借了一车內库藏书来读。前面介绍过,他写过《高士传》,他还写过史书、医书等,临死前写下遗书,要求``朝死夕葬,夕死朝葬,不设棺椁,不加缠敛,不修沐浴,不造新服,殡唅之物,一皆绝之'',``平生之物,皆无自随,唯赍《孝经》一卷,示不忘孝道'',能考虑到这些问题,是很通达的高人。
  非贰:非议。俗语``够二的''、``二傻''、``二愣子''等等,``二''天生就是贬义。
  敛衽:整理衣裳,表示敬意。《战国策》:``一国之众,见君莫不敛衽而拜,抚委而服。''
  作品要借重名人而得以畅销,自古皆然。高行健当年在台湾出版《灵山》,两三年只卖出180本,50本还是他自己购买送人的,得诺贝尔奖后10万本迅速卖光。未成名时的弗洛伊德《梦的解析》八年的销量也不过600册。伟大的哲学家维特根斯坦作品《逻辑哲学论》的出版多次遭到拒绝,出版商要求有名人作序,自己承担印刷费用等。维特根斯坦极为恼火,认为要求作者自费出书不是正派行为,``我的工作是写书,而世界必须以正当的方式接纳它''。后来还是依靠老师罗素的帮助才出版。
  有了这些先例,我们作者总要勉励自己,沉住气:我做到像左思这样10年写一篇文章吗?我写的是祸枣灾梨还是沾溉学子?

\section{4.69}\label{section-246}

\begin{quote}
刘伶著《酒德颂》,意气所寄。
\end{quote}

  刘伶是``竹林七贤''中最随心所欲的人,每当我们遇事思前想后,缩头缩颈的时候,我们不妨想想近2000年前的刘伶,杰出的先人是如何看待生命和生活的。
  他夫子自道写下《酒德颂》。其中说:
  像我这样的大人物,把天地的始终当做一天,把一万年当做片刻,把日月作为门窗,把极远的荒凉之地当做庭园间的道路。我行走没有轨迹,居住没有房屋,我以天为幕,以地为席,我随着放纵的心意到达生活的每一处。
  我停留时、我行走时都带着卮、瓢、榼、壶,酒就是我的一切,再也不知道其他什么事情。
  有些贵族公子,官员隐士,听到我的名声,议论我的行为。他们奋袂攘襟,怒目切齿,陈说礼法,非议像无边的刀锋一样迸发。
  而我这时候正对着酒槽捧着酒瓮,我正手举酒杯喝着浊酒,我正拨开胡须箕踞而坐,我正头枕着酒曲,臀垫着酒糟\ldots{}\ldots{}我无思无虑,其乐陶陶。
  我昏昏沉沉地喝醉,我突然之间清醒。我竖起双耳,四周一片安静,轰轰的雷霆我听不到;我睁大双眼,宏伟的泰山我也看不见;我感觉不到严寒酷暑,我没有利益欲望。我弯下腰,俯视着世间纷纷扰扰的一切,就像看着奔流不息的河流、海洋当中的点点浮萍。我身边带着两个仆人,他们就是我的干儿子。
  刘伶的很多故事,就是从这篇文章中推衍出来的,也许刘伶并没有那么多故事,因为有了这篇文章,所以诞生了很多佳话,凸显了他非凡脱俗的性格。
  幕天席地:刘伶脱光衣服在家里喝酒,俗人进屋后讥笑他。他说:我以天地作为房屋,以屋舍作为衣服。你们到我裤裆里来干什么?
  有贵介公子、缙绅处士\ldots{}\ldots{}奋袂攘襟,怒目切齿:俗人曾经和他争执,挽起袖子想动手,刘伶笑嘻嘻地说:``鸡肋岂足以当尊拳。''
  二豪侍侧焉,如蜾蠃之与螟蛉:他出行常常坐着独轮车(鹿车),一路走一路喝。叫仆人随身带着锄头,吩咐说:``我突然死了的话,就随地挖个坑埋了我。''(这句话在《酒德颂》里面很费解,但联系这个故事,大约就是仆人会随时帮我办丧事的意思。)
  刘伶因为寂寞而沉醉,因为旷达而潇洒,他那放荡诡异的行为,反映出他无限广阔的心胸。《庄子》里有个故事:有一个叫触的国家在蜗牛的左角上,有一个叫蛮的国家在蜗牛的有角上。两个国家为了争夺地盘发起战争,于是伏尸数万。后来白居易写诗说:蜗牛角上争何事,石火光中寄此身。刘伶就看透了人生,不过是寄托于石火光中罢了。

\section{4.70}\label{section-247}

\begin{quote}
乐令善于清言,而不长于手笔。将让河南尹,请潘岳为表。潘云:``可作耳,要当得君意。''乐为述己所以为让,标位二百许语。潘直取错综,便成名笔。时人咸云:``若乐不假潘之文,潘不取乐之旨,则无以成斯矣。''
\end{quote}

  乐令:尚书令乐广,当时任河南尹。他曾经和文章大家潘岳一起在权臣贾充手下干过,所以请潘岳代笔。但潘岳没当过这么大的官,经历也不痛,不能设身处地,就需要乐广提供思路。
  要当:应当。   所以:\ldots{}\ldots{}的原因。   标位:列举。
  直取错综:直接吸取重新排列。   名笔:名篇。
  《晋书》为人诟病的其中一点是,它成书仓促,缺乏严谨、确凿的史家态度,``忽正典而取小说'',大量引用野史资料,就把一些传说和故事当做史实,``其所载者大抵弘奖风流,以资谈柄''。它给晋人列传,不加辨析地直录《世说新语》、《搜神记》、《幽明录》,确实不可取。不过从另一个方面讲,它选用《世说新语》,把其中很多语言难题解决了,文字更加简练明快,譬如这则故事它是这样裁剪的:广善清言而不长于笔,将让尹,请潘岳为表。岳曰:``当得君意。''广乃作二百句语,述己之志。岳因取次比,便成名笔。时人咸云:``若广不假岳之笔,岳不取广之旨,无以成斯美也。''
  我们不能否认,《晋书》的文字更加简洁顺畅,克服了《世说新语》拗口、累赘、难懂之处。不过由于题材的不同,《世说新语》特殊的口语、虚词等用法消失了,它特别的韵律消失了。这种现象对我们创作、编辑多有提醒和警示。

\section{4.71}\label{section-248}

\begin{quote}
夏侯湛作《周诗》成,示潘安仁,安仁曰:``此非徒温雅,乃别见孝悌之性。''潘因此遂作《家风诗》。
\end{quote}

  夏侯湛:名将夏侯渊的曾孙。夏侯湛家族比较奇特,夏侯渊是曹操的亲戚,战功卓越,对曹操也有救命之恩。夏侯渊的侄女被张飞掳掠迎娶,生下的女儿做了刘禅的皇后。夏侯渊后来被刘备部将黄忠击败杀死,他很多儿子都在魏国当高官。而夏侯渊的次子夏侯霸在司马家族执政后,出走投靠刘禅,在蜀汉也成为重臣。夏侯霸的女儿是晋朝名臣羊祜的妻子,羊家和夏侯渊家族也多次联姻,也和司马家联姻。夏侯湛的祖父是夏侯渊的四子夏侯威,父亲夏侯庄的妻子是司马炎皇后羊氏的妹妹,夏侯湛的妹妹还是东晋元帝司马睿的母亲。夏侯湛家族确实可以说是``上面有人''------曹家、刘家、司马家、羊家等等。有这种盘根错节、无比复杂的家族关系,朝廷等自然有所防范,你可以为所欲为,但想进入中枢当特别的高官不太可能了。夏侯湛从小才华出众,写了诗歌宣扬家风。他也曾想过写史书,但看到陈寿写的《三国志》,就把自己写的《魏书》草稿给烧了。
  周诗:《诗经?小雅》有六首诗散佚,只留下篇名和内容概要。晋代盗墓者挖掘春秋年间的一个古墓,朝廷就得到了部分书简。文学家束皙整理了这些材料,编写了《诗经》中``有义无辞''的《南陔》、《白华》等六篇。这六首诗歌大抵是纪念父母养育之恩和天地盛德的作品。夏侯湛也取其中的大意,写了首诗歌,因为是模仿《诗经》的,所以叫《周诗》。
  别见:另见。
  夏侯湛《周诗》和潘岳的《家风诗》现在还保存着,古人的审美趣味和现在有所不同,这两首诗在当时受到一致的赞誉,葛洪在《抱朴子》中说:``近者夏侯湛、潘安仁并作《补亡诗》,《白华》、《由庚》、《南陔》、《华黍》之属,诸硕儒高文之赏才者,咸以古诗三百,未有足以偶二贤之作也'',把他们这两首诗歌高高挂在《诗经》三百的上面。这也是一个很有趣的文学现象,譬如唐代有个道士叫吴筠,在中国文学史上估计排名在万名之后吧,可是在《旧唐书》上,作者却把他高高供起:``(吴筠)词理宏通,文彩焕发,每制一篇,人皆传写。虽李白之放荡,杜甫之壮丽,能兼之者,其唯筠乎!''当然,潘岳的不少诗赋很优秀,但这两首诗不在其列。是啊,别人写了几首很孝的诗,不看僧面看佛面,我们怎么好意思批评他呢!

\section{4.72}\label{section-249}

\begin{quote}
孙子荆除妇服,作诗以示王武子。王曰:``未知文生于情,情生于文。览之凄然,增伉俪之重。''
\end{quote}

  孙子荆:东晋名士孙绰的祖父孙楚,王武子就是王济,两个人在2.24同时出现过,王济赏识孙楚,认为他``天才英博,亮拔不群''。王济死后,孙楚在他的葬礼上也学驴叫以示纪念。
  除妇服:为妻子服丧按仪礼要一年,但是后来大概变通为七天。所以孙楚在诗中说:``时迈不停,日月电流。神爽登遐,忽已一周。礼制有叙,告除灵丘。临祠感痛,中心若抽。''(``时迈不停,日月电流''大约出自孔融的《论盛孝章书》:``岁月不居,时节如流'',``日月电流''也许是孙楚的首创;神爽登遐就是魂魄归西的意思;孙楚的妻子大约葬在灵丘;``中心若抽''出自《诗经?黍离》``中心如醉''、``中心如噎''等,就是心如抽丝的意思。)
  ``未知''句:真不知是文由情生还是情由文生。你的诗歌因为感情而产生,我看了你的诗歌心中产生感情。我看了你的诗歌,感同身受,心中凄凉,从而更加珍惜自己夫妻间的感情。
  文学的意义究竟是什么?我想主要一条就是能唤起人们深刻的同情。我常常看不起女人和朋友看电影、电视剧时候感情被剧情而带动:``有没有搞错,人家是在演戏啊,至于当真吗!''但静下心来想想,文学乃至各类艺术,它唤醒的正是人们之间共同的欢乐和悲伤。从前有人说文学就是反映典型环境中的典型性格,文学是作家反映现实世界的工具,文艺是为人民服务、为社会服务等等。但作为一个真正的艺术家来说,他的创作根本目的就是一个------表达的需要,作为一个读者或者观众,他欣赏文艺作品,赞美它,正是因为它满足了人们的感情需要。而一般而言,一个人的道德水准和审美能力越高,他就越具有同情的能力,也能更好地鉴别艺术品的高低。

\section{4.73}\label{section-250}

\begin{quote}
太叔广甚辩给,而挚仲治长于翰墨,俱为列卿。每至公坐,广谈,仲治不能对;退,著笔难广,广又不能答。
\end{quote}

  太叔广:字季思,山东东平人,西晋太常博士。挚仲治:挚虞字仲治,西安人,皇甫谧的学生,西晋太常卿。汉代以后,把一年俸禄满二千石的高官称为卿。
  辩给:口才便利。给是富足的意思。   翰墨:笔墨,代指文辞。
  公坐:公众场合。
  挚虞反应、口才没有太叔广好,但思维比太叔广全面深远。太叔广说的没有流传下来,而挚虞的作品却保留下来,挚虞胜过了太叔广。
  西晋乱世变局,挚虞饿死,太叔广自杀。

\section{4.74}\label{section-251}

江左殷太常父子并能言理,亦有辩讷之异。扬州口谈至剧。太常辄云:``汝更思吾论。''

  江左殷太常父子:东晋江东名士、高官殷融、殷浩叔侄。殷融当过太常寺卿,殷浩扬州刺史。古时候把叔侄关系也笼统地称为父子关系,《礼记?檀弓上》中说:``丧服,兄弟之子,犹子也,盖引而进之也'',就是在服丧期间,侄子和儿子是一样的礼节。宋代王楙在《野客丛书?弟侄献言》中指出:``兄弟之子犹子也,古人视以父道\ldots{}\ldots{}而事叔亦以子礼,叔侄之分,与父子同。''有的古文不能以今意揣度,就像``天要下雨,娘要嫁人'',其中的娘本意是小姑娘的意思,而不是指自己的寡妇老妈。
  辩讷之异:口才好坏之别。在文学创作上,不少作家表现出文笔很好,但说话却很一般,从近了说,像沈从文、周作人都是不善言辞的人,从远了说,著名的有韩非、扬雄、左思等,甚至口吃得厉害。但你说不善言辞的人是不是反应很慢,也不见的,也许可以说,有浪漫气质的人,词锋锐利,而那些沉着的人,言辞就比较枯燥,不善言谈。
  73和74两则故事,都反映了作者的一种褒贬------辩论好不是真的好。

\section{4.75}\label{section-252}

\begin{quote}
庾子嵩作《意赋》成。从子文康见,问曰:``若有意邪,非赋之所尽;若无意邪,复何所赋?''答曰:``正在有意无意之间。''
\end{quote}

  庾子嵩:庾敳,东晋名臣庾亮(谥号文康)的叔父,西晋王衍``四友''(王敦﹑谢鲲﹑庾敳﹑阮修)之一。庾敳和阮籍一样当过陈留相(诸侯国的相国,相当于太守),一样无事事之心,以饮酒度日。他后来当了吏部郎这样重要的职位,估计也不办事,贪财好物,家产千万。后来执政的东海王司马越问他贪污受贿的情况,他不以为意地说,我家产有两千万,你需要的话拿去好了。永嘉之乱时,庾敳被石勒的兵马所杀。
  《意赋》:意就是思想、想法,《意赋》是一篇谈世界观的文赋,文采没有突出的地方,但意思很有意思。``至理归于浑一兮,荣辱固亦同贯。存亡既已均齐兮,正尽死复何叹。无咸定于元初兮,俟时至而后验。若四节之素代兮,岂当今之得远。且安有寿之与夭兮,或者情横多恋。宗统竟初不别兮,大德亡其情愿。蠢动皆神之为兮,痴圣惟质所建。真人都遣秽累兮,性茫荡而无岸。纵驱于辽廓之庭兮,委体乎寂寥之馆。天地短于朝生兮,亿代促于始旦。顾瞻宇宙微细兮,眇若毫锋之半。飘飖玄旷之域,深漠畅而靡玩。兀与自然并体兮,融液忽而四散。''文章前四句大约就是庄子齐物论的意见,宇宙始、终于混沌状态,荣辱、存亡没有什么差别。后四句是说,宇宙之初的``无''的空间概念,是依靠``时''这个时间概念来证明的,为什么说无是根本呢?就像四季的更换,有不是常态,不会永远。``且安''四句费解,大意也许在说,长寿与短命,爱恨与情仇,家族统系在圣人那里(大德)是不追求这些东西的(亡其情愿)。``蠢动''也出自庄子,意思是``出于本性的自然行动'',傻瓜和圣人的区别是材质的不同。得道的高人排斥了各种污秽拖累,其本性浩淼没有际涯,他驰骋寄托于寥廓寂寥之所。文章接着阐述真人的相对论世界观,看似永恒广阔的天地其实比朝生夕死的蜉蝣还有短小,亿万代比太阳升起的那一刻还有短促。我看宇宙很小很小,才及得上毛笔锋毫的一半。他想着宇宙的渺小,身体和思绪逍遥、沉醉于玄旷、广漠的领域,突然之间与自然规律合二为一,就像水滴加入河流,突然之间又分散了。
  像这种文章,我也是连猜带蒙好长时间,也算是处于有意无意之间吧。作者写出这种东西来,也许正处于坐忘之境。这种境界,大约就是摆脱责任、使命、痛苦、恐惧的良药和借口。总的来说,古代中国比较流行的的思想学说,儒、道、佛一个比一个更柔,一个比一个更弱,在强权暴力之下,儒家和佛教更是不要脸,往往贴上去舔统治者的痔疮,道家则跑到云雾之处了。
  非赋之所尽:这也是庄子的一个困境,言意之辨。《庄子?秋水》中说:``可以言论者,物之粗也;可以意致者,物之精也;言之所不能论,意之所不能察致者,不期精粗焉。''你既然说不清楚,你干嘛还要写呢?你不是说``知者不言,言者不知吗''?为什么言说,怎么言说肯定是思想家进退两难的问题,庄子选择了说,选择了乱说,又多次强调,我的本意不是这个,读者一定要``忘言''才能体会到我的真意。庾亮基本掌握了庄子的意思,对叔叔的文章提出质疑:你的文章有思想,这种思想文章怎么表达得清楚呢?你文章里提倡混沌,齐物,没有差别、空虚等等思想,那写它干啥?
  庾敳的回答应急、取巧,但仔细一品味,却超出了他的本意,引发了读者更为广阔的联想。有是实体,无是不是除了有之外的一切?它们之间有没有``间''?是不是只有实体与实体之间才有``间''?有意无意之间,也会令人联想起《庄子》的``材与不材'',``无用之用'',想起孔子的``从心所欲不逾矩'',佛教的``不即不离''等。从浅显处说,我有意于做事,事情做的多做的好,一般而言就当官了;我有意于当官,就会一门心思往上爬,结果利欲熏心,胡作非为。我有意于抒发感情,阐述思想,多写写,思想、文笔就越来越好;我一心要成名成家,用力太勤,反而无病呻吟,矫揉造作。往深里说,人生本来就是有限和空虚的,你有原则,就能游刃有余,顶天立地,你刻意要达到什么目标,往往会欲壑难填,迷失自我。即使在混沌的世界中,也有混沌的尺度,这也许就是《庄子》中说的那样:``一会儿上一会儿下,以和为度量,浮游于万物初始状态,给物命名而不是被物所驾驭。''

\section{4.76}\label{section-253}

\begin{quote}
郭景纯诗云:``林无静树,川无停流。''阮孚云:``泓峥萧瑟,实不可言。每读此文,辄觉神超形越。''
\end{quote}

  郭景纯:郭璞郭景纯,两晋著名诗人,中国风水学的老祖宗,后来他算命家的名头远远超过了他作为游仙诗的代表人物,但他的才能是多方面的,注释过不少典籍,算得上当时的大学者。如果说我国诸子百家中有个``方家''或者``神仙家'',郭璞当然可以算其中的翘楚。他不知道有什么手段,占卜的名头很大,史书上一本正经地记载了他自述的很多先知先觉的故事,而且看样子他并不是装神弄鬼,而是坚信自己,所以写了《葬经》、《洞林》这种神神叨叨的书。这就产生了一个问题:我们为什么认为史书上讲一般的事情就认为基本是真实的,而一涉及怪力乱神的东西,我们就产生很大的质疑?这并不简单的认为是文化背景的关系,而是我们对待史料态度。我们希望对过去的世界有一个合理的认知和解释,但对未来却充满疑虑。郭璞的结局很耐人寻味,不得不说,那个故事有很高的可信度,也许是他一辈子算的最准的一次,但我们的疑问并不会减少。
  王敦起兵谋逆,东晋朝廷和王敦事前都找当时任王敦参军的郭璞占卜。温峤、庾亮先找到郭璞,问郭璞,郭璞说看不清楚。温、庾继续叫郭占卜自己的命运,郭璞给出答案:``大富大贵''。温、庾回味说:``我俩希望朝廷镇压王敦,郭璞说我们将来大富大贵,这就说明王敦没有好下场。他对王敦谋逆事没说出答案,因为他有顾虑忌讳。''王敦找上郭璞占卜,这时候郭璞直接回答:``无成!卦凶!''王敦进而提问:``你看我能活多久?''郭璞回答:``看这个卦,你起事的话,祸不久的,如常住武昌不动,你的寿命很久很久。''王敦发怒:``你算算自己能活多久?''郭璞说:``今天中午!''于是王敦把郭璞给杀了。故事到这里还能够得到合理的解释,后面《晋书》讲的话带有很大的神话色彩:郭璞对行刑者说:``我将死在双柏树下,树上有个大鹊巢。''结果真是这样,郭璞当时49岁。故事有个悖论,遇到这样的事情,就算常人也能找到脱身之计,而有无数事例的神算为什么无法避开身死人手的命运?
具有非凡才能的郭璞,理应采取出世的态度,他为什么还入世如此之深?他写过很多诗歌和文章,悲伤是其中一个主题:``临川哀年迈,抚心独悲咤'',``悲来恻丹心,零泪缘缨流'',``遐邈冥茫中,俯视令人哀'',他甚至写了《愁诗》三首,说自己``独步闲朝,哀叹静夜''。他反反复复写这些东西,所以说无论郭璞有多少神迹,他终究不是一个哲学家,而是文学家,这从他``不持仪俭,形质颓索,纵情嫚惰''看出一端。
  阮孚:竹林七贤中阮咸的儿子阮孚阮遥集,《世说新语》后面会讲到,他父亲看上姑姑家一个漂亮的鲜卑族姑娘,就把她抢了回来做``人种'',生下的儿子就是阮孚。如果对两晋名人的事迹不了解的话,你很笼统地说``好酒,懒散,纵情声色'',一般不会错的。阮孚也是这样的人,他还当过东晋的吏部尚书、丹阳尹这样的高官,但并无突出的作为留在书上。
  ``林无静树'':首先是景物写实,干净利索地写出了山川景物风貌;其次是隐含世界不是静止的,一切永远在运动这样的哲理;第三是饱含人生的感慨,生命跃动、时间消逝、世事无常等情绪。形象、感情、哲理融于一体,当然是很好的诗句。郭璞的这首诗好像就这么两句,叫《幽思》。
  ``泓峥萧瑟'':河流清深,群山高耸,树木簌簌作响,览物之情,无法用语言表达。
  神超形越:身心超越于尘俗之上。即使我们普通人,也往往喜欢远登高山,近看河流,在森林与峡谷中行走,也许只有人类会对风景有这样强烈的兴趣。我们听风声,听雨声,看大雪,心中不免别样的情怀,绚烂、壮观的景物使我们的内心更加丰满。我们活着,并不仅仅是为了生存,在更多的时候,我们还愿意摆脱尘俗,把心灵放置于激荡和宁静之中。

\section{4.77}\label{section-254}

\begin{quote}
庾阐始作《扬都赋》,道温、庾云:``温挺义之标,庾作民之望。方响则金声,比德则玉亮。''庾公闻赋成,求看,兼赠贶之。阐更改``望''为``俊'',以``亮''为``润''云。
\end{quote}

  庾阐:庾阐庾仲初,庾亮的族侄,文学家,9岁就能写出文章,东晋时候的著作郎,掌修国史。
  扬都赋:写南京城的赋文,南京(建康)属于中国九州之一扬州的首府,东晋首都,所以叫扬都。他写这篇文章应该有政治目的,通过歌颂南京的盛况而体现东晋朝廷的中兴伟业。庾阐也许因为这篇文章,被当时人称之为``中兴之时秀''。这篇赋写了比较长的时间,有个故事说,《扬都赋》没写完的时候,庾阐离婚另娶谢家的女儿。他要求妻子晚上不要把油灯熄灭,而是放在瓮中,半夜里他灵感一来,马上就能拿出灯来,也就能记录下来了。
  ``温挺义之标'':
温峤、庾亮在道义上高标挺立,是百姓仰望的楷模;把他们比作声音,就像金石相击般高昂,比方他们的德行,就像美玉般剔透。《说文解字》中说,玉有五德:``润泽以温,仁之方也;理自外,可以知中,义之方也;其声舒扬尊以远闻,智之方也;不折不挠,勇之方也;锐廉而不忮,絮之方也。''大意是比喻德行的特点:温润,表里如一,嘹亮,不折不挠,不伤人等。
  庾公:庾亮重臣名士,文学爱好者,《晋书》中肯定庾亮``笔敷华藻,吻纵涛波,方驾搢绅,足为翘楚'',又听说是家族中的秀才夸他的,当然要拿来读读。
  赠贶:贶就是赠。
  因为庾亮要看,作为晚辈,就要注意避讳,所以亮要改成``润'',为了对仗押韵,前面的``望''要改成``俊''。从修辞来看,``玉亮''也是``玉润''更妥帖一些,但``人俊''还是``人望''评价更高一点。
  《扬都赋》全文没有保留下来,现在只有残篇。

\section{4.78}\label{section-255}

\begin{quote}
孙兴公作《庾公诔》,袁羊曰:``见此张缓。''于时以为名赏。
\end{quote}

  孙兴公:孙绰,东晋著名作家,,当过太学博士、著作郎、散骑常侍等,当时人一般认为他文章很好,德行有亏,见2.84等。
  庾公诔:写庾亮的诔文。当时写悼文有四种:诔、碑、哀、吊。一般来说,哀辞以悼早夭,吊文以慰亲属、诔文以彰德行,碑文以记功业,但也不能一概而论。孙绰写此类文章是著名的好手,《文心雕龙?诔碑》中说:``孙绰为文,志在于碑''。这次他分别写了《太尉庾亮碑》和《庾公诔》。《庾公诔》全文是:``咨予与公,风流同归。拟量托情,视公犹师。君子之交,相与无私。虚中纳是,吐诚悔非。虽实不敏,敬佩弦韦。永戢话言,口诵心悲。''文章大意是说,我和你是一样风流的人,你的气量、情怀我很能理解,我视你为师。我们俩的交情是无私的君子之交。你为人谦虚,善于听取意见,为人诚恳,善于改过。我虽然不够聪明,但很敬佩你的自省自警(弦韦:《韩非子?观行》西门豹之性急,故佩韦以自缓。董安于之心缓,故佩弦以自急)。我收集你的话语,口中诵读,不由悲从中来。
  袁羊:袁乔袁彦升,小名羊,当过东晋著作郎、尚书郎、桓温手下的司马等。
  见此张缓:此句费解。但当时作为名言,肯定很好懂。刘兆云说,张缓当作张奂,东汉张奂有才无德(他杀了《世说》第一名士的陈藩),好向死人献谀(杀了窦武、陈藩后,他认为上了宦官的当,要求为他们平反)。孙绰也有才无德,予以类比。因为这篇文章通过夸庾亮来抬高自己。张奂写作张缓,是因为袁乔的高祖父名涣,为了避讳。不过张奂在《后汉书》中,算是杰出的人物,刘兆云提供给我们另一个思路,袁乔是不是把庾亮比喻成张奂?把庾亮比作张奂,也不算辱没,而把孙绰比成张奂,总有点不伦不类。
  孙绰有才无德的事例,史书上并没有记载,只说他``诞纵多秽行''。他是当时的大手笔,却是一致公认的,东晋时期的道德评价和现在的标准不一样,我也不知道孙绰做了什么天怒人怨的事。刘兆云说袁乔看不起孙绰``有才无德'',有趣的是,孙绰也在简文帝面前说袁乔``不知者不负其才,知之者不取其体''(品藻?65),也是``有才无德''的意思。当然,孙绰也夸过袁乔``洮洮清便''
(品藻?36)。袁乔对孙绰作品的评价,其中的褒贬不太清楚。譬如也有学者指出,``张缓''来自于文中的``弦韦''(急缓),李天华说这句话的意思是:``弦韦这个典故在文章中用的如此之妙!''这也是有可能的,庾亮是外戚、辅政大臣,为人方正激进,这篇诔文通篇表扬庾亮,但也很委婉地提到了庾亮犯过的错误,庾亮在处理苏峻和北伐任务多有失误,但其特点是自身要求严,能改过,孙绰用弦韦的典故,很切合庾亮的实际情况。
  名赏:著名的品评。

\section{4.79}\label{section-256}

\begin{quote}
庾仲初作《扬都赋》成,以呈庾亮,亮以亲族之怀,大为其名价,云可三《二京》,四《三都》。于此人人竞写,都下纸为之贵。谢太傅云:``不得尔,此是屋下架屋耳,事事拟学,而不免俭狭。''
\end{quote}

  以亲族之怀:这个未必,前面说过,庾阐本来就是文章好手,《扬都赋》是苦心经营之作,从现存的残篇看,符合当时的政治需要,文采并不差,表扬的估计也不仅仅是庾亮、温峤两人,按照庾亮方正的性格,不会仅从亲友角度考虑的。
  名价:抬声价。   ``三《二京》'':与《两京赋》、《三都赋》媲美。
  屋下架屋:重复多余。   拟学:模仿。
  俭狭:简单狭隘。庾亮当政时,谢家还是小族,直到庾亮、王导、郗鉴死后,谢家才获得进入朝廷中枢的上升空间。谢安评论《扬都赋》,那时他才20来岁,名位不显,他一个小伙子为什么当众批评当时``建康纸贵''的名作呢?他的这种批评,锋芒是不是直指庾亮呢?俭狭的庾亮觉得俭狭的《扬都赋》是好文章。他抨击公认的大作家庾阐,算不算一种获得名望上升空间的渠道?当然,我们诋毁余秋雨,贾平凹和余华等作家,完全出于阅读的直觉,不入流的作家我们还不骂呢。

\section{4.80}\label{section-257}

\begin{quote}
习凿齿史才不常,宣武甚器之,未三十,便用为荆州治中。凿齿谢笺亦云:``不遇明公,荆州老从事耳!''后至都见简文,返命,宣武问:``见相王何如?''答云:``一生不曾见此人。''从此忤旨,出为衡阳郡,性理遂错。于病中犹作《汉晋春秋》,品评卓逸。
\end{quote}

习凿齿:字彦威,湖北襄阳人,祖上是土豪,属于寒门,所以能得到高位,在东晋实属不易,多因为袁乔推荐,桓温赏识,一年内提升习凿齿三次,最后升为治中,荆州是要害重镇,治中起码是五品的中高级官员,大致相当于办公厅主任,掌管文书。
  史才不常:唐代礼部尚书郑惟忠曾问掌管国史的刘知几:``自古以来,文士多而史才少。何也?''刘知几说:``史才须有三长,世无其人,故史才少也。''史学家比文学家更为难得,刘知几自己说:``余幼喜诗赋,而壮都不为,耻以文士得名,期以述者自命。''他解释说,史才包括``才、学、识'',文笔,学问和节操见识三者缺一不可。``彰善贬恶,不避强御,若晋之董狐,齐之南史,此其上也。编次勒成,郁为不朽,若鲁之丘明,汉之子长,此其次也。高才博学,名重一时,若周之史佚,楚之倚相,此其下也。''第一流的史学家是道德评价,像秉笔直书不怕死的董狐、南史氏;第二流像左丘明、司马迁,文采出众,善于编辑写作;第三流像史佚、倚相,学问广博。
  宣武甚器之:《晋书》中说:``(桓)温出征伐,凿齿或从或守,所在任职,每处机要,莅事有绩,善尺牍论议,温甚器遇之。''
  老从事:代指小吏。
  一生不曾见此人:他既然是桓温的亲信,有知遇之恩,和司马昱(丞相,会稽王)见了一面,就把司马昱放在这么高的位置,置桓温于何地?对司马昱的评价,谢安曾经说过很刻薄尖锐的话:``比白痴的晋惠帝,惟有清谈差胜耳!''《方正?49》中王濛还说:``别人说司马昱是傻子,现在看来真的是傻子。''那么习凿齿为什么做出这种回答?难道他这个时候已经``性理遂错''(精神错乱)?我们可以认为习是直观印象,随口说说的,他出身寒门,年纪轻,不精通门阀政治内幕(当时的贵族只知其家,不知其国),小小年纪去南京又受到司马昱等名士的款待,一下子骨子就轻了,``扬鞭语曰:`适自相公家来,
相公厚我,厚我!'\,''但往深里思考,这也许是习凿齿不搞人身依附的政治态度。在专制社会中,官僚体系的特点就是职务是上司的赏赐,上下级是主奴关系,你危难或者需要投机的时候背叛主子,那是知形势识大体,你平时在主子面前说政敌莫名其妙的好话,才见了一面就和主公的政敌眉来眼去,那就是首鼠两端,怀有二心。习不是骨子轻,而是骨头硬,开始就从内心认为自己是朝廷的人,而不是桓温的人,所以如此这般。
  忤旨:不合心意。《晋书》中还有一个故事。桓温叫一个四川的星象家占卜晋朝的国运。星象家不明就里,说晋朝起码还有50年的国运。桓温就送了星象家一匹绢、五千钱。星象家就自己体会:``给我绢,含义是叫我自裁(同字多义);给我钱,刚好买口棺材!''连夜请求习凿齿给他挖个墓地安葬。习巧妙地化解说:``俗话说,知星宿,衣不覆(那些算命占卜的,往往是很贫穷的,连衣服都买不起)。桓温送绢是和你利用谚语开玩笑,给钱是叫你回家罢了。''第二天星象家向桓温告别。桓温问他怎么就要走了?他就转述了习的意见。桓温笑着说:``习凿齿是怕你因误会而死,其实你是因误会而活着的。你读书30年,还不如去拜访了一次习凿齿。''桓温的问答很耐人寻味,同样机智非凡。我们仔细想想,到底是星象家还是习凿齿理解准确?答案自在其中。
  性理遂错:精神错乱。随后习凿齿被贬,中南政区办公厅主任变成了人口管理局局长,然后出任衡阳郡太守,想见先贤大哲和前尘往事,于是乎``未尝不徘徊移日,惆怅极多,抚乘踌躇,慨尔而泣。''大男人整天哭泣大概就是他精神病的表现吧。这句话也许真正表明,习凿齿``一生不曾见此人''是无心之言,他当时就是一个聪明干练的年轻人,没有老谋深算到真正了解桓温的心意,真正了解司马昱是绣花枕头。后来学问渐高,主见日深,又加上挫折,一条路走到黑,更不支持桓温的雄心野望了。
  品评卓逸:习指出,曹魏只是统治了几个州,直到司马炎``受禅''建立西晋时,曹魏仍然没有征服东吴,国家没有统一,曹魏从来没有做过天下之主,只是三国中的一个,不能单独称为一个朝代,直到西晋建立之后国家才完成真正的统一。司马家是击败了腐败无能的蜀汉、孙吴取得正统地位,而不是篡逆了曹魏得到名分的。这种说法与前人迥异,而且言之成理。所以该书叫《汉晋春秋》,而不是``汉魏晋春秋''。再进一步讲,该书也有精神激励作用,你东晋的老祖宗能够征服蜀、吴,现在苟安于南京,何以见先人?这也是一切历史都是当代史的意义吧。
  习后来腿瘸了,只能称为``半个人''。前秦苻坚占据襄阳,俘虏了两个名人。他高兴地说:``从前晋朝司马氏平定吴国,利在获得陆机、陆云二位才士;今日我平定汉南,也获得一个半人(释道安与习凿齿)。''不过后来苻坚兵败,习凿齿又回归东晋。他据说是习近平的祖先,现在已经有人溜须拍马写文章纪念习凿齿了。``四海习凿齿'',他当时确实是闻名海内的人物。

\section{4.81}\label{section-258}

\begin{quote}
孙兴公云:``《三都》、《二京》,五经鼓吹。''
\end{quote}

  孙兴公:孙绰。
  鼓吹:军乐队。言下之意是这五篇赋文是宣扬、解释诗、书、礼、易、春秋的作品。孙绰为什么这样说呢?这要从赋的流变来考量。西汉以司马相如为代表的大赋创作主要内容为虚构,它的创作目的是通过宏伟的想象力、绚烂的文字和炫耀宫廷生活来使君王愉悦,``不似从人间来,其神化所至邪''。到了东汉,士人以礼乐教化为己任的使命感膨胀,班固写《两都赋》的动机在于``或以抒下情而通风谕,或以宣上德而尽忠孝,雍容抑扬,著于后嗣,抑亦雅颂之亚也'',内容也从虚构转向写实,走上博学的道路。张衡创作《二京赋》,更加深化了作品为政治和教化服务的特点,创作以``美刺''意图为核心,尽量讽喻。他歌颂先人的成就,烘托国家仪式典礼的威严,批评现实生活的奢靡,强调礼德之治,反对谶讳之说,没有取悦君主的意思。而左思的《三都赋》写实度更高,他自己在序中说:``山川城邑则稽之地图,其鸟兽草木则验之方志。风谣歌舞,各附其俗;魁梧长者,莫非其旧'',指出前人的赋文``考之果木,则生非其壤;校之神物,则出非其所。于辞则易为藻饰,于义则虚而无征'',他尽量按自己的真实去描写魏蜀吴三国的历史沿革和风土人情,然后体现``客主之辞,正之以魏都,折之以王道,其物土所出,可得披图而校'',这本来就是一个政治话题和对时代的强烈参与感。所以说,这五篇赋文创作动机是宣扬作者的治国思想,为政治和道德价值服务,具有五经的意义。``五经''这个大将军出行,``五赋''就在边上吹拉弹唱,以壮声势。

\section{4.82}\label{section-259}

\begin{quote}
谢太傅问主簿陆退:``张凭何以作母诔,而不作父诔?''退答曰:``故当是丈夫之德,表于事行;妇人之美,非诔不显。''
\end{quote}

  谢太傅:谢安。   陆退:江东吴郡陆氏后人,张凭的女婿。
  张凭:``佳儿''、``理窟''张凭,被刘惔赏识而闻名。
  魏晋以来,人性提升,妇女地位上升,歌颂女性美德、体现对女性(家人)真挚的爱的文字作品逐渐多了起来,这是从前很难看到。估计以前很少有悼念母亲的文章,张凭写了,而且不写父亲,谢安觉得不合常规,故有此问。
  故当:总应当。
  男子的品德成就从事迹中自然体现出来,而妇女不能抛头露面,她们的美好情怀,没有悼念的文章,社会就不会知晓。所以中学生就学习了朱德的《回忆我的母亲》、史铁生的《我与地坛》和张承志的《无缘坂》,从前我的孩子总是说,爸爸妈妈一样好,现在他已经上初中了,前些天他说,妈妈比爸爸好一亿倍。是啊,孩子说的对。

\section{4.83}\label{section-260}

\begin{quote}
王敬仁年十三作《贤人论》,长史送示真长,真长答云:``见敬仁所作论,便足参微言。''
\end{quote}

  王敬仁:太原王氏,王修王苟子,名士王濛王长史的儿子,前已多见。
  《贤人论》:13岁小孩子写出来的东西,可能因为语言和背景,现在读起来很费解,就像王羲之写的那些便条,书法漂亮,但意思我没有读通过。《贤人论》谈《易》的一个问题,看不懂《易》没什么,如果说自己看懂了,才是怪事。王修写的东西虽有保存,但看了以后,不免感想:清谈是扯淡啊。
  参微言:参与清谈。

\section{4.84}\label{section-261}

\begin{quote}
孙兴公云:``潘文烂若披锦,无处不善;陆文若排沙简金,往往见宝。''
\end{quote}

  孙兴公:孙绰。潘:潘岳。陆:陆机。
  ``烂若披锦'':烂有闪光的意思。《诗经?女曰鸡鸣》:``士曰昧旦,子兴视夜,明星有烂。''把诗赋文章比喻成锦绣,古皆有之,王充《论衡?定贤》:``虽文如锦绣,深如河汉,民不觉知是非之分,无益于弥为崇实之化。''把文章比喻成金子,就不能说一篇文章就是一块金子,而只能说其中的几个句子像金子一样了。在《文学?89》中,孙绰继续评论道:``潘文浅而净,陆文深而芜。''这是对本文的说明和补充,意思有相似之处。
  潘岳、陆机在西晋已经是最著名作家,起码在东晋已经把他们相提并论了。他们的作品多而全面,即使从三千年的文学史角度观察,潘岳的悼亡诗、陆机的文论依旧有他们一席之地。潘岳、陆机文章特点的区别,主要在于他们的家庭出身、学术修养和气质的不同。陆机家庭背景显赫,写文章以学识统筹,多有模拟和总结阐述之作,相对艰深繁复;潘岳出身差一些,以才情驾驭文章,多有开创和清丽之作,清代陈祚明说:``安仁过情,士衡不及情;安仁任天真,士衡准古法。''陆机写文章宏观深沉一些,潘岳则微观浅静一些,南北朝时期的钟嵘在《诗品》评价说:``陆才如海,潘才如江''。譬如陆机写自己孩子死了:``呼子子不闻,泣子子不知。叹息重榇侧,念我畴昔时。三秋犹足收,万世安可思。殉没身易亡,救子非所能。含言言哽咽,挥涕涕流离'',他不写孩子生前音容笑貌等细节,一写就非常讲究对仗,展开哲理的思考。潘岳妻子死了,就写道:``望庐思其人,入室想所历。帏屏无仿佛,翰墨有余迹。流芳未及歇,遗挂犹在壁。怅恍如或存,回遑忡惊惕\ldots{}\ldots{}''用很多生活的细节来表现自己的感情,流畅自然。刘勰在《文心雕龙?熔载》中说:``士衡(陆机)才优,而缀辞尤繁;士龙(潘岳)思劣,而雅好清省。''按照文学欣赏的角度出发,简洁明了比繁复要好,所以张华才评价陆机:``人之作文,患在不才;至子为文,乃患太多也。''而别人对潘岳的评价是``清绮绝伦''。
  潘岳因为依附贾谧而被杀,夷三族。陆机因为当了河北大都督,兵败被杀。

\section{4.85}\label{section-262}

\begin{quote}
简文称许掾云:``玄度五言诗,可谓妙绝时人。''
\end{quote}

  简文:司马昱。
  许掾:许询许玄度,隐居在萧山的名士,多次征辟为司徒掾,不就。前已多见。在东晋时与孙绰一起被称为``一代文宗'',``清风朗月,辄思玄度''。
  妙绝时人:曹丕说建安七子中刘桢的诗歌``其五言诗之善者,妙绝时人''。随着时间的推移,许询诗歌的评价降低,在南北朝钟嵘的《诗品》中,许询和孙绰被评为``下'',不过曹操的诗歌也是``下''。
魏晋100来年中,据张廷银考察,保存下来的玄言诗大约有250首左右。现在一般认为,作为玄言诗的代表作家,许询的诗歌没有值得赞许的地方,只是当时特定的文化背景下,才被认为好。而且东晋时期也说不上来谁是大诗人,许询算是矮子中的大个。东晋末的大诗人陶渊明的作品应该主要在南朝时候创作。不过即使像陶渊明这样的诗人,也接受了许询等人的影响,写过不少玄言诗,阐述释道思想。陶渊明诗歌的好处,不在于他的玄言诗如何好,而是他的田园诗写的好。但我们不能忽视,玄言诗也是陶渊明诗歌的一部分,他继承了这部分遗产,进行了开拓,他的玄言诗是田园诗的基础,他的田园诗,有玄言诗的成分。而后来诸多作家那些出色的山水诗、理趣诗,也是对玄言诗的继承和发展。而那些创作手法的改变,也正是诗人们吸取了玄言诗的教训。从这个意义上说,那些``理过其辞,淡乎寡味''的玄言诗,正孕育着诗歌的好几个流派。我们观察它们,理清诗歌的发展脉络,就会心领神会:``哦,她原来在那里'',``哦,小时候她曾经是个丑丫头。''
  那么,让我们看看理趣诗的美人吧。杨万里《过松塬晨炊漆公店》:``莫言下岭便无难,赚得行人空喜欢。正入万山圈子里,一山放出一山拦。''

\section{4.86}\label{section-263}

\begin{quote}
孙兴公作《天台赋》成,以示范荣期,云:``卿试掷地,要作金石声。''范曰:``恐子之金石,非宫商中声。''然每至佳句,辄云:``应是我辈语。''
\end{quote}

  孙兴公:孙绰在台州(临海郡)章安县当过县令,天台山脉基本在绍兴新昌县和台州天台县,宋代传奇人物济公据说就出生在天台。道教洞天赤城山,李白笔下的天姥山,都属于天台山脉。因为工作,我多次去过天台山国清寺,觉得很幽雅沉静,它旧,简朴而不失韵味,没有簇新的建筑物。遗憾的总是在庙里坐个把小时就离开了,没有走过景区。孙绰有没有去过也很难说,因为他在文中说:``然图像之兴,岂虚也哉!''``天台山好不好,有图有真相,怎么会是假的呢?''我们可以理解为他在看图作文,也可以理解为天台山当时已经有人在作画了。
  范荣期:范启范荣期(传说春秋时候有隐士荣启期,人生三乐:``吾得为人,是一乐也。吾既得为男矣,是二乐也。吾既已行年九十矣,是三乐也。''),东晋名士,曾任秘书郎、黄门侍郎,都是天子近臣,清要之职。
  天台赋:《游天台山赋》,列入萧统《文选》,著名赋文。《天台赋》的特点之一是写了不少玄学的内容,如``太虚辽阔而无阂,运自然之妙有,融而为川渎,结而为山阜'',``散以象外之说,畅以无生之篇'',``悟遗有之不尽,觉涉无之有间'';``浑万象以冥观,兀同体于自然''等。特点之二是音节铿锵有力,词章优美,如``赤城霞起而见标,瀑布飞流以界道'',``双阙云竦以夹路,琼台中天而悬居'',``害马既去,世事多捐。投刃皆虚,目牛无全''等。当然,《天台赋》也有一些冷字僻词,这是当时的风气,就像现在有些学者好端端不说人话,就故意佶屈聱牙以炫博争奇,看似严密渊博,实则浮夸荒芜。
  金石声:指音乐中的编钟和编磬,都是打击乐器和礼器,形容文章嘹亮典雅,文采出众。``掷地有声''的出典。
  非宫商中声:不和谐的音调。魏晋时期士人风采的特点之一就是傲放,``傲放为邈世'',葛洪在《抱朴子》中反复说,``世故继有,礼教渐颓,故让莫崇,傲慢成俗'',``今人何其愦慢傲放如此乎!''所以曹植说:``人人自谓握灵蛇之珠
, 家家自谓抱荆山之玉''
(《与杨德祖书》)'',孙绰叫范启学习自己的作品,范启又看不起孙绰,互不买账。
  应是我辈语:这是我们这类人说的话啊。同道不孤,``我辈中人''的出典。
  孙绰在《文学》中多次出现,但是作者并没有把他放在一起表述,《世说新语》行文就是这样故意错落有致,如果把同一个主人公的故事一股脑儿倾倒给读者,读者未免感到局促和紧张,这是一种写作直觉。

\section{4.87}\label{section-264}

\begin{quote}
桓公见谢安石作简文谥议,看竟,掷与坐上诸客曰:``此是安石碎金。''
\end{quote}

  372年9月,东晋简文帝司马昱病死,他辅政20多年,当皇帝不到一年。其病重期间,一天下四次诏,要大司马、平北将军、徐兖二州刺史桓温入朝辅政,桓温上书自称老病,推荐了吏部尚书谢安和侍中王坦之。373年,桓温病死。
  谥议:``\,`一德不懈曰简,道德博闻曰文。'《易》简而天下之理得,观乎人文,化成天下,仪之景行,犹有仿佛。宜尊号曰太宗,谥曰简文'',``《易》简而天下之理得'',``观乎人文,化成天下''都是引用《易》原文,他说:《易》文字简易,而能包含天下的道理,把握现实社会中的人伦秩序,以此来教化天下。司马昱有高尚的品德,很符合《周易》中讲的道理。
  碎金:优美的短文。文章好叫碎金,口才好叫玉唾,书法好叫银钩。谢安曾是桓温下属,年纪也小一些,所以桓温称谢安的字。

\section{4.88}\label{section-265}

袁虎少贫,尝为人佣,载运租。谢镇西经船行,其夜清风朗月,闻江渚间估客船上有咏诗声,甚有情致;所诵五言,又其所未尝闻,叹美不能已。即遣委曲讯问,乃是袁自咏其所作咏史诗。因此相要,大相赏得。

  袁虎:袁宏袁彦伯,小名虎,文学家,有捷才,成语``倚马可待''的主人公,被推许为``一时文宗'',``当今文章之美,故当共推此生''。他姓袁,确是东汉以来的高姓大族,不过父亲去世早,家道中落,因谢尚赏识而为其幕僚。他是历史学家,``性直亮'',在桓温手下干过,看不惯桓温欲行曹操事,多有劝诫,``故位不显也。''
  载运租:运载租谷。古代官僚生活地与田地往往分离,租谷要运到居住地。``范文正公(范仲淹)在睢阳﹐遣尧夫(儿子范纯仁)于姑苏取麦五百斛。尧夫时尚少,既还,舟次丹阳,见石曼卿,问寄此久近。曼卿曰﹕`两月矣。三丧在浅土,欲丧之西北归,无可与谋者。'尧夫以所载舟付之﹐单骑自长芦径而去。到家拜起﹐侍立良久。文正曰﹕`东吴见故旧乎?'曰﹕`曼卿为三丧未举,留滞丹阳。时无郭元振(唐代郭元振,好义博济),莫可告者。'文正曰﹕`何不以麦舟付之?'尧夫曰:`已付之矣。'\,''
  谢镇西:``小颜回''谢尚谢仁祖,谢安的堂哥,镇西将军,前以多见。
  经船行:``经''费解,《世说新语?德行》:``华歆、王朗俱乘船避难'',《世说新语?任诞》:``时戴在剡,即便夜乘小船就之''。
  估客:估同``贾'',商客。唐?卢纶《晚次鄂州》:``估客昼眠知浪静,舟人夜语觉潮生。''
  委曲:费解,当作``部曲''。
  咏史诗:两晋咏史以左思出名,袁宏的作品有两首,从艺术角度而言,并没有突出的成就,不过其中的感慨却深沉,譬如他列举汉代名人,``趋舍各有之,俱令道不没'',他们的志趣和行为不同,但道是相同的;他也谈是自己的困惑,``无名困蝼蚁,有名世所疑'',我不想做蝼蚁,但木秀于林,风必吹之;谈了自己的抱负,``吐音非凡唱。负此欲何之'',我有嘹亮的歌声,远大的抱负,但又能怎么样呢?这种歌咏志士的慷慨志气和徘徊在人生的尴尬境地,任何一个有抱负的人都能够体会到。
  相要:相邀。   赏得:欣赏投合。
  世有伯乐然后有千里马,谢尚与袁宏的故事后来多有歌咏,譬如李白说``登舟望秋月,空忆谢将军。
余亦能高咏,斯人不可闻'',他感慨谢尚的难遇;阎尔梅说``江天寥落租佣睡,谁唱袁宏咏史诗'',他感慨人才的难得。

\section{4.89}\label{section-266}

\begin{quote}
孙兴公云:``潘文浅而净,陆文深而芜。''
  孙兴公:孙绰。此文与4.84类似。问题是``芜''究竟是贬义还是褒义,因为芜有两个解释:茂盛和杂乱。孙绰讲了这句话,后世被引用,《北史?儒林传序》:``南人约简,得其英华;北学深芜,穷其枝叶。''这里的深芜有广博的意思。刘勰在《文心雕龙?才略》说:``陆机才欲窥深,辞务索广,故思能入巧而不制繁。''简约是魏晋时候审美趣味,但不等于说,繁复就一定不好,西方的哲学著作就很繁复,繁复然后精深。短篇小说和长篇小说,孰优孰劣,不能笼统评价;评价一个大作家,也不能用像孙绰这样简单的话语就概括他的特点,一个不复杂的作家,就不能称为大作家。我们还是把芜解释成中性词,繁复。
  世界简单而又复杂。潘岳、陆机、左思是西晋最重要的作家。
\end{quote}

\section{4.90}\label{section-267}

\begin{quote}
裴郎作《语林》,始出,大为远近所传。时流年少,无不传写,各有一通。载王东亭作《经王公酒垆下赋》,甚有才情。
\end{quote}

  裴郎:裴启,或又名裴荣,字荣期,东晋名士,山西闻喜裴家子侄。春秋隐士荣启期在两晋有很高的声誉,被西晋皇甫谧列入《高士传》,陆云也写过《荣启期赞》等。著名砖画``竹林七贤与荣启期''是南北朝墓葬画中的常见题材。裴启或者前文中的范启,取这样的名和字很容易理解。耐人寻味的是,前朝隐士两三百总有的,为什么躲在小角落里的荣启期在当时反复被歌咏?也许是古代的隐士高人心中往往有不平之气,爱发牢骚和讥讽社会,真正快乐的很少。而荣启期能够在乱世中``自宽'',高扬快乐的旗帜,黄连树下弹琵琶,深得魏晋南北朝人
  《语林》:比《世说新语》更早出的志人小说,该书散佚有残篇。362年中,此书一出,``大为远近所传'',``时人多好其事,文遂流行''。正因为它的流行,对《世说》很有启发,《世说》对它继承、模仿、发展,它的很多内容,被《世说》直接或者稍加修改后植入,按照现在的版权意识,《世说》要被判罚款和登报道歉,不能再版了。
  在没有版权法的古代,这么一部流行作品散佚,据《世说》中说,主要是谢安不满。在《语林》中,裴启夸赞谢安的仇家东亭侯王珣风姿,自我表扬谢安很赏识裴启(``载入己事,尤乖体例''),都触犯了谢安的忌讳。《世说新语?轻诋》中,谢安就说:``我没有表扬过裴启,他自我吹捧,写的都是假的'',于是``自是众咸鄙其事矣'',``谢相一言,挫成美于千载;及其所与,崇虚价于百金'',就这样,裴启也当不成官了,成为``处士''。当然,《世说新语》比《语林》更为丰满周到,也是《语林》散佚的原因。
  时流年少:世俗之辈和少年。
  各有一通:通,费解,也许就是《世说新语》中常见``通''的意思,解释发挥。就像我写的这些内容,就是``时流中年''对《世说新语》的``通''。不过解释为``人手一册''最方便。但联系后文,``通''解释为``阐述发挥''比较合适,人人通讲,于是形成``裴氏学'',解释为``册''没有先例。
  载王东亭:``载''为衍文,因为王珣看了《语林》,写了一篇《经黄公酒垆下赋》。``而有人于谢坐叙其`黄公酒垆',司徒王珣为之赋''。裴启写了``黄公酒垆''的故事,王戎的后代王珣因此写了赋文,这篇赋是对``黄公酒垆''故事的``通''。
  黄公酒垆:《伤逝?2》:王濬冲为尚书令,著公服,乘轺车,经黄公酒垆下过。顾谓后车客:``吾昔与嵇叔夜、阮嗣宗共酣饮于此垆。竹林之游,亦预其末。自嵇生夭、阮公亡以来,便为时所羁绁。今日视此虽近,邈若山河。''这个故事应该是《世说》直接从《语林》那里抄来的。王戎是王珣的族祖父辈,裴启写这个故事,是歌咏王戎的风采,王珣写赋,是对祖父辈的向往和赞美。
  我们曾经相聚,多少年过去了,地方犹存,离我很近,但我和历史上或者现实中的朋友们已经遥隔山河。吊古之情,览物之心,也是我们喜欢《语林》或者《世说新语》的原因。帕斯捷尔纳克《日瓦戈医生》中写道:``因此在早春的时候,朋友们常常聚首,我们把晚会当作别离,把酒宴当作遗言,为的是用这内心的痛苦,温暖那凄凉无情的生活。''
 深得魏晋南北朝人的共鸣,既然我们在向黑暗而坠落,也要痛饮地下的甘泉。

\section{4.91}\label{section-268}

\begin{quote}
谢万作《八贤论》,与孙兴公往反,小有利钝。谢后出以示顾君齐,顾曰:``我亦作,知卿当无所名。''
\end{quote}

谢万:谢安的弟弟谢万石,见2.77等,因为他后来兵败贬为庶人,所以谢安出山以维持家门。
  《八贤论》:又名《八贤颂》,赞美渔父、屈原、司马季主、贾谊、楚老、龚胜、孙登、嵇康八个人,现文散佚,存残篇,如``皎皎屈原,玉莹冰鲜。舒采翡林,摛光虬川'',``楚老潜一,寂玩无为。含贞内外,载葺羽仪'',写嵇康``邈矣先生,英标秀上。希巢洗心,拟庄托相。乃放乃逸,迈兹俗网。锺期不存,奇音谁赏''等。
  孙兴公:名士兼好友孙绰。
  往反:清谈辩难。谢万认为这八个贤人,隐居的四个高明,做官的四个要差一些。孙绰不赞同这个说法。
  小有利钝:偏义复词,主要指失败。孙绰提出,只要精神超脱,见识远大,隐居和当官没有高下之分,优劣之别。这个观点比谢万的容易自圆其说。但真要较真的话,孙绰的这种和稀泥也不堪一击,就像他说什么``周孔即佛,佛即周孔'',儒、释两道出发点一样,不过是``内外名之耳''。一个隐居的贤者,怎么可能与一个对人类发展有推动意义的伟人相提并论呢?事实上,四位隐居的正因为四位做官的才得以被历史记载,隐居者作为做官者的配角而出现。一个人的真正价值,在于他的社会事业和道德品质,而不是他如何洞察世情和颐享天年。庄子的种种说法,排解心胸、灌浇块垒自然是很好的,但并不等于说,庄子的思想代表着什么真理和生活准则。
  顾君齐:顾夷,吴郡顾家子弟。事迹不详,看来是谢万的好友。
  当无所名:``名'',被称道的意思,《后汉书?班超传》:``闻此必恐而谋泄,死无所名,非壮士也。''这句话的意思是:我如果写的话,你就会知道,你写的《八贤论》不会被后人所称道(我写的肯定比你好)。李天华提出,``顾曰我亦作'',``顾''和``作''为衍文,原文应该是``曰:`我亦知卿当无所名'\,''------我就知道我的文章你也不会称道。
  看不起别人的作品或看法,这往往是两晋傲放之士的惯例,无论是谢万、孙绰还是顾夷。
举报 回复 楼主:linyi518 时间:2014-01-03 11:53:14
  4.92桓宣武命袁彦伯作《北征赋》,既成,公与时贤共看,咸嗟叹之。时王珣在坐,云:``恨少一句。得写字足韵当佳。''袁即于坐揽笔益云:``感不绝于余心,溯流风而独写。''公谓王曰:``当今不得不以此事推袁。''
  解释:
  369年3月,侍中、大司马、都督中外诸军事、扬州牧、平北将军、徐兖二州刺史桓温兴起第三次北伐,攻打前燕,袁宏当时在桓温幕府中任记室,起草文章。
  《北征赋》:现散佚有残篇,如``于时天高地涸,木落水凝,繁霜夜洒,劲风晨兴。日暧暧其已颓,月亭亭而虚升'',``鱼托水而成鲲,木在山而有松'',``于量背梁山,截汶波,泛清济,傍祝阿'',``闻所闻于相传,云获麟于此野。诞灵物以瑞德,奚授体于虞者。悲尼父之恸泣,似实恸而非假,岂一物之足伤,实致伤于天下。感不绝于余心,溯流风而独写''等。
  得写字足韵:最后一句应当押包含韵母为 ê、ie、üe
的字韵,以使文章饱满。
  ``感不绝于余心'':我心中无限感慨,迎着北风独自抒发情怀(或者是按照古人的风范独自抒发情怀)。
  以此事推袁:写赋这种事情,首推袁宏。
  袁宏起家因为谢尚的赏识相得,他在桓温幕府很长时间,桓温礼遇他,但因为他耿直,``每不阿屈,故荣任不至'',直到桓温死后,谢安才得以提拔袁宏。不过桓温死后的碑铭,依旧是相交甚深的``文宗''袁宏写的,他高度评价了温``固天时纵,生民之杰者'',比之于伊尹、吕尚。

\section{4.93}\label{section-269}

\begin{quote}
孙兴公道曹辅佐才如白地明光锦,裁为负版绔,非无文采,酷无裁制。
\end{quote}

  孙兴公:孙绰。
  曹辅佐:曹毗字辅佐,三国曹休曾孙,和庾阐一样写过《扬都赋》,以文采著名,自称``吐辞则藻落扬班,抗心则志拟高鸿,味道则理贯庄肆,研妙则颖夺豪锋'',当过著作郎、下邳太守、太学博士、光禄勋等。
  白地明光锦:一种以白色为底的高档织锦,上多有动物图案。
  负版:《孟子?生于忧患》``舜发于畎亩之中,傅说举于版筑之中'',建筑工人。明光锦是高档布料,却成了普通工人的裤子。
  酷:当做``苦'',苦于。
  孙绰是说,曹毗辞章飞扬,但是缺乏修剪(要么是气质不好),太泛滥,就变得很普通。曹毗现存的作品不多,其文章的繁复不太看得出来,但言不可自休总是有才华之人的通病。

\section{4.94}\label{section-270}

\begin{quote}
袁彦伯作《名士传》成,见谢公,公笑曰:``我尝与诸人道江北事,特作狡狯耳,彦伯遂以著书。''..
\end{quote}

  谢安辅政后,提拔袁宏为吏部郎(袁宏《与谢仆射书》:``闻见拟为吏部郎,不知审尔,果当至此,诚相遇之过。''),而后又提拔为东阳太守,在桓温手下干了10来年只利用不重用的袁宏可谓是谢安的铁杆心腹了。不过373年桓温去世,376年袁宏也去世了。
  袁宏为吏部郎,谢安是吏部尚书,东晋做官比较轻松,以办事拖沓和无事事之心为美,没有人来查你上班有没有淘宝、看电影和炒股票,所以组织部一帮子人经常一起聊天。谢安、袁宏这样的高明之人,话题不是求田问舍,而是谈前朝名士,袁宏是有心人,就把谢安的聊天内容记录下来,炼成了历史学家,写了诸如《名士传》、《后汉纪》等著作。
  《名士传》:正始名士夏侯玄、何晏、王弼,竹林名士七贤,中朝名士裴楷、乐广、王衍、庾顗、王承、阮瞻、卫玠、谢鲲。今该书散佚,《世说新语》刘孝标的注中保留了其中的一些片段,对这些人的评价,既有谢安的观点,也有袁宏的观点。
  江北事:永嘉南渡以前的往事。
  狡狯:玩笑游戏。《太平广记》:``会稽石亭埭有大枫树,其中空朽,每雨,水辄满溢。有估客载生鳣至此,聊放一头于枯树中,以为狡狯。''陆游说:``晋人谓戏为狡狯,今闽语尚尔。''

\section{4.95}\label{section-271}

\begin{quote}
王东亭到桓公吏,既伏阁下。桓令人窃取其白事,东亭即于阁下更作,无复向一字。
\end{quote}

  王东亭:``如椽大笔''的王珣,他出生琅琊王氏高门,王导的孙子,二十来岁时候到桓温手下当官,被桓温赏识,迅速提拔任主簿,``军中机务并委珣焉''。
  到:或当做``为'',或者直接解释成到任、就任。
  伏阁下:《说文解字》:``伏,司也'',服侍。《后汉书?曹世叔妻传》:``时《汉书》始出,多未能通者,同郡马融,伏于阁下从昭(班昭)受读。''古代高级官员的官署往往称阁,``王珣在桓温帐下听用''。
  白事:白,报告。有事报告的公文。
  阁下更作:下,词缀,无意。在官衙重新写了一份。
  无复向一字:无复,不再有,《世说新语?德行?27》``殆无复坐处'',《言语?27》``无复疑虑''等;向,以前的。不再有原来那份报告的一个字。
  古时候的公文多用赋体,文章短。这个故事是说王珣才思敏捷,词汇丰富,从容不迫。大王珣37岁的桓温开年轻人这种玩笑,心态值得玩味。
举报 回复 楼主:linyi518 时间:2014-01-04 22:39:43
  4.96桓宣武北征,袁虎时从,被责免官。会须露布文,唤袁倚马前令作。手不辍笔,俄得七纸,殊可观。东亭在侧,极叹其才。袁虎云:``当令齿舌间得利。''
  解释:
  369年桓温北征第三次前燕,其目的一般人认为是为了谋私利,大概因为文字秘书袁宏``至于辩论,每不阿屈'',被免官,但依旧在帐下听用。
  会须:正需,应当。李白《将进酒》:``烹羊宰牛且为乐,会须一饮三百杯。''
  露布文:不缄封的文书,常指檄文或捷报。
  俄:《说文解字》俄,行顷也。片刻。   东亭:主簿王珣,后封东亭侯。
  当令齿舌间得利:当令,自当是,《世说新语?俭啬?10》``士当令身名俱泰''。袁宏在自嘲,我流年不利,官场失意,也就是在你口中还能得到几句奖励。
  这个故事看似简单,但有较强的写作技巧。桓温罢免袁宏,依旧使用,是重其才,突显袁宏文章的杰出。袁宏罢官后没有意气消沉,依旧才华横溢,借助倚马这个具体平台,确是洋洋洒洒,出口成章。作者又通过同样有捷才的王珣高度评价袁宏,起到衬托作用。袁宏自嘲,为人潇洒。桓温、袁宏、王珣的形象跃然纸上。
  从此以后,袁宏与曹植并驾齐驱,七步成诗和倚马可待双壁交辉,成为自古捷才的典范。

\section{4.97}\label{section-272}

\begin{quote}
袁宏始作《东征赋》,都不道陶公。胡奴诱之狭室中,临以白刃,曰:``先公勋业如是,君作《东征赋》,云何相忽略?''宏窘蹙无计,便答:``我大道公,何以云无?''因诵曰:``精金百炼,在割能断。功则治人,职思靖乱。长沙之勋,为史所赞。''
\end{quote}

  东征赋:东征赋不是指征讨东部的赋,而是东行赋,《尔雅》``征,迈行也'',书写作者一路东行的所见所思。这篇赋现在保存下来了,应该是残篇。譬如这个故事和桓温的故事都说明了这一点,现有的文章不存在``悉称过江诸名望'',没头没尾的。
  陶公:陶侃,见2.47,东晋封长沙郡公。
  胡奴:陶侃第十子陶范的小名,官至江州刺史、光禄勋等。
  云何:为什么。《诗?扬之水》:``既见君子,云何不乐?''云,发语词,无意。后面的``何以云无'',就是说的意思了。
  忽略:袁宏的全篇应该包括``过江诸名望'',陶侃是东晋杰出人士,``机神明鉴似魏武,忠顺勤劳似孔明'',功勋卓著,袁宏文章的第一稿没有他,也许因为陶侃没有门第,``望非世族,俗异诸华'',是最下等的人。在东晋,看不起下等人是节操高的表现,称之为``方正''。
  ``精金百炼'':曹操《百辟刀令》``百炼利器,以辟不祥,摄服奸宄者也'',曹植说曹操赐的刀``陆斩犀革,水断龙角''。陶侃中年前坎坷失意,他能取得这么大的成就,也离不开生活对他的百般锤炼。
  ``功则治人'':治人,管理人。陶侃后来是是荆、江两州刺史,都督荆江雍梁交广益宁八州诸军事等,在他的管理下,``自南陵迄于白帝数千里中,路不拾遗''。他44岁前还不过是一个小县的县令,但在西晋末年,由于时事和得人赏识,踏上军旅,从此利刃出鞘,``作籓于外,八州肃清;勤王于内,皇家以宁'',他临死前还上表说:``欲为陛下西平李雄,北吞石季龙\ldots{}\ldots{}良图未叙,于此长乖!''
  这个故事应该仅仅是故事,是为了嘲笑陶侃的后人和反映袁宏的捷才吧。故事还有一个版本,配角是桓温,他对袁宏不提及自己的老爹桓彝不满,也想发难,袁宏也一下子抛出六句,说得桓温潸然泪下。桓彝同样是久居南方的小族门户。而袁宏虽然家境不好,但出生是东汉袁尚、袁汤等``三公''这样的高大门第。

\section{4.98}\label{section-273}

\begin{quote}
或问顾长康:``君《筝赋》何如嵇康《琴赋》?''顾曰:``不赏者作后出相遗,深识者亦以高奇见贵。''
\end{quote}

  顾长康:``云兴霞蔚''顾恺之,江东顾家杰出子弟,深得桓温和谢安的赏识,前已多见。
  筝赋:现存残篇:``其器也,则端方修直,天隆地平。华文素质,烂蔚波成。君子喜其斌丽,知音伟其含清。罄虚中以扬德,正律度而仪形。良工加妙,轻缛璘彬。玄漆缄响,庆云被身。''筝现在还有,但晋代的筝估计不太一样,应该是竹制,十二弦,唐以后采用木制,十三弦。文章的片段描写了筝的大体形状和特点,整个长方形,上部弧线形,中空,底部平整。油漆黑色,绘图以云水为主。烂蔚、斌丽和璘彬差不多,都是有光彩的意思。顾恺之有``三绝'',画、文、痴,文绝(才绝)就是其中之一。
  琴赋:《琴赋》有全篇,近二千字,写了琴的用材、制琴工艺、外观、演奏情状、琴曲发展、风格特色,指出声无哀乐的音乐观等。
  作后出相遗:我的文章因为后出而被遗弃。
  以高奇见贵:因为高明杰出而被推崇。
  现在认为,我国最优秀的赋,不是枚乘、司马相如和张衡这样的大赋,而主要是抒情小赋,如《归去来兮辞》、《滕王阁序》、《阿房宫赋》等,这可能要从几个方面分析,作者人格,创作目的,阅读通畅,层次清楚,篇幅简短。像顾恺之、嵇康的这类赋,没有任何高奇,没有文学价值,属于被淘汰的作品。现在有些作者写华山赋、绍兴赋之类的东西,当做什么盛世华音,但令人遗憾的是,它们根本没有意义。

\section{4.99}\label{section-274}

\begin{quote}
殷仲文天才宏赡,而读书不甚广博。亮叹曰:``若使殷仲文读书半袁豹,才不减班固。''
\end{quote}

  殷仲文:荆州刺史殷仲堪的堂弟,桓玄的姐夫,见2.106。因为和桓玄关系好,桓玄当皇帝失败后,他就不得志了,后以谋反罪被处死。
  天才宏赡:天资宏大丰富。晋书中说他``善属文,为世所重''。
  广博:此处一些注家有另外的看法,认为可能是:``读书不甚广,傅亮叹曰'',也有人提出,按《晋书》记载,傅亮当做谢灵运。
  傅亮:东晋末与南朝宋大臣,他``博涉经史,尤善文词'',拥戴刘裕当皇帝,成为顾命大臣,后被他所拥立的刘义隆所杀。
  袁豹:东晋末年大臣,袁氏子弟,曾任著作佐郎、丹阳尹等,``好学博闻,喜谈雅俗''。
  现在的文学家不用读太多的书,而那个时候文学的概念和现在的不一样,那时候还涵盖史学、朝廷的文书等,应该指的是一切文字工作,所以要博学。现在如果博学的话,就当不成文学家,顶多也就是文学理论批评家。现在的文学创作,可能就是``学三年天下无敌,学十年寸步难移''。

\section{4.100}\label{section-275}

\begin{quote}
羊孚作《雪赞》云:``资清以化,乘气以霏。遇象能鲜,即洁成辉。''桓胤遂以书扇。
\end{quote}

  羊孚:羊孚羊子道,桓玄记室参军,但他反对桓玄当皇帝,见2.104等。
  雪是文学创作的重要题材,羊孚的《雪赞》道出了人们为什么喜欢雪的一些原因。雪是纯洁的象征,天地间的清气所化,以清澈为资本,元朝诗人刘将孙说:``天地间清气为六月风,为腊前雪,于植物为---梅,于人为仙,于千载为文章,于文章为诗。''雪乘着大气纷纷扬扬,不事雕琢而自然成态,没有冰的坚硬,没有霜的凌厉,它柔和,它丰厚。我们向往一个纯洁光辉的世界,雪清扫了世界,清扫后的世界又都是洁白的雪,``千峰笋石千株玉,万树松萝万朵云''。雪使想象中天堂才遍处的玉树琼花一夜降临人间,它覆盖下的一切是那样熠熠生辉。羊孚的四句小诗,道出了雪的本质、形态和作用,象征着人的精神追求,所以得到了桓胤的喜欢。冈布里奇说:``一切艺术都起源于人的心灵,即源于我们对世界的反应,而并非是视觉世界本身。''雪就是这样。
  桓胤:他祖父是桓温的弟弟桓冲,在叔叔桓玄谋逆前后,他都是很风光的,当过中书令、吏部尚书这样的高官。后因桓玄事败被贬,后被指谋反罪与殷仲文一道被处死。《晋书》上说桓胤``少有清操,虽奕世华贵,甚以恬退见称'',喜欢羊孚的《雪赞》是一个证明吧。

\section{4.101}\label{section-276}

\begin{quote}
王孝伯在京,行散至其弟王睹户前,问``古诗''中何句为最。睹思未答。孝伯咏:```所遇无故物,焉得不速老!'此句为佳。''
\end{quote}

  王孝伯:身无长物的王恭,痛饮酒熟读离骚的王恭,见1.44等。
  王睹:大名士王濛的孙子,王爽字季明,小名睹,官至侍中,相当于宰相的高官,爽有明亮的意思。王爽后来因王恭起兵攻击司马道子兵败而一起被杀。王恭吃了五石散后要吹冷风,所以就不进弟弟家的门了。
  古诗:《古诗十九首》,五言之冠冕,人生之悲歌。
  ``所遇无故物'':《古诗十九首》中的《回车驾言迈》,这首诗大意是说,驾车远行,路途漫长,冬去春来,百草重生,一眼望去,旧事物已经过去,新者日新,故者速故,人生真是又漫长而又短暂。生物有盛有衰,而且由盛而衰,人啊,总归不能长生不老,树立一生的事业应该早一些。即使人与万物同化为一声叹息,归于尘埃,但美好的名声久久流传在值得珍惜。王恭被杀的时候说:``我信错了人,所以落到这个地步。但我的本心却忠于社稷!只希望百代之下还有知道我王恭这个人。''
王恭潇洒,他在行散的时候,脑海里就在默诵着这首诗和这一句,表露了复杂的生命感慨和立功建业之心。当年他出道的时候任佐著作郎,感叹说:``仕宦不为宰相,才志何足以骋!''以疾辞,也是这个道理。
  有人问我,人是什么。我说,人是世界上唯一能够试图理解人生和世界的生物。探求虽然沉重,但格外高贵。帕斯卡尔在《思想录》中说:人只不过是一根芦苇,是自然界最脆弱的东西;但他是一根能思想的芦苇。用不着整个宇宙都拿起武器来才能毁灭,一口气、一滴水就足以致他死命了。然而,纵使宇宙毁灭了他,人却仍然要比致他死命的东西高贵得多;因为他知道自己要死亡,以及宇宙对他所具有的优势,而宇宙对此却是一无所知。

\section{4.102}\label{section-277}

\begin{quote}
桓玄尝登江陵城南楼,云:``我今欲为王孝伯作诔。''因吟啸良久,随而下笔。一坐之间,诔以之成。
\end{quote}

  397年,王恭、殷仲堪起兵讨伐权臣王坦之之子、谢安之婿王国宝,宰相司马道子同意并下令赐死王国宝和其族弟王绪,王恭罢兵。
  398年,王恭联合桓玄与殷仲堪,又决定讨伐王坦之另一个儿子、桓温的女婿王愉及司马尚之,这次王恭部将、淝水之战名将刘牢之被策反,王恭兵败被杀。
  东晋末年地方与中央的诸多纷争,归根到底是朝廷想削弱门阀的力量,而门阀想保住地位,就出镇各州,建立自己的势力。照一般人想,司马家想一家独大,应该派遣自家人出镇各州,但由于八王之乱的教训,自己家族的人更加靠不住,所以当时一派乱象,只有到了宋代才最终解决了王权与兵权之间的矛盾------文官为地方最高首脑,武官战时调派,将不知兵,兵不知将。
  桓玄``博综艺术,善属文'',多次与王恭合作,他执政以后,为王恭平反,朝廷追赠王恭为侍中、太保。
  一坐之间:坐下片刻作文。
  桓玄的诔文大意是说:王恭是天地的灵气所育,但老天爷的心思不可揣摩,灵秀的人福气却不深厚。王恭的走狗反咬一口,朝廷的豺狼欢喜跳跃。山间高大的梧桐树已摧毁,林间挺拔的修竹已凋残。我怀念这个身系国家安危的贤人,叹息国家丧失了一个很好的管理者。我写下这篇诔文,来纪念这位像爰旌目这样名声芳郁的义人。
  荆州城南楼很有名,大概是平原大江,视野开阔,不少人写过诗歌和文章。登高是乐府的古题,文学的常见题材,即可翛然出尘,怀宇宙之想,也可伤时感事,坎坷归途,也可歌颂爱情,寄托思念等。李白说:``登高壮观天地间,大江茫茫去不还'',范仲淹说:``明月高楼休独倚,酒入愁肠,化作相思泪'',李梦阳说:``不是青天遮眼目,望尽天下十八州''。

\section{4.103}\label{section-278}

\begin{quote}
桓玄初并西夏,领荆、江二州,二府,一国。于时始雪,五处俱贺,五版并入。玄在听事上,版至,即答版后,皆粲然成章,不相揉杂。
\end{quote}

  西夏:古时中原称为中夏,桓玄当时是荆州、江州的刺史,位于中原的西部,故称西夏,南部古越国之地称为南夏。
  二府,一国:后将军府、都督府,后将军、都督等可以开府。所谓开府就是可以自选僚属,不用朝廷批准。一国指封国,桓玄是南郡公,封国在广州。
  399年末,桓玄消灭荆州殷仲堪、杨佺期。400年,向朝廷求领荆江二州刺史。朝廷下诏给桓玄加都督江州及扬州豫州共八郡诸军事,领江州刺史,晋国三分之一左右的疆域已经归桓玄统治了,自幼有大志向和雄豪风流的桓玄,自然是意气风发,有了更大的志向。
  于时始雪:这句话与内容无关,闲笔使短文有了情趣,也许在古人眼里,下雪还是吉兆。
  五版并入:晋代纸张虽然已经广泛使用,但官府的公文还是使用简的,所以称为版,直到404年桓玄当了皇帝,下令停止使用简牍,从此以后,黄纸成为朝廷文书的正式书写材料。
  听事:官衙大厅。代指处理政务。
  与汉代文人不同,汉末以后的学者摆脱经学的束缚,独立意识增强,不强调因袭古人的说法和论述,往往看重发表己见和写作的独创性。他们不但不因袭古人,还鄙视因袭自己。夏侯湛看到陈寿《三国志》后,
烧掉自己写的《魏书》,何晏看到王弼写的《老子注》后搁笔,谢安讽刺大作家庾阐的《扬都赋》是``屋下架屋'',王恭写文书,``更作无复向一字'',这里的桓玄也是这样,同一件事情的回信一定要变幻文字,决不雷同,而不是像我们现在一样,公文就是``八股'',以循旧抄袭应付。公文依旧画葫芦倒还罢了,一些学者还把抄袭来的东西当做职称评选的著作,这如果成为一种风气,就很容易形成落后无耻的学风。

\section{4.104}\label{section-279}

\begin{quote}
桓玄下都,羊孚时为兖州别驾,从京来诣门,笺云:``自顷世故睽离,心事纶蕰。明公启晨光于积晦,澄百流以一源。''桓见笺,驰唤前,云:``子道,子道,来何迟!''即用为记室参军。孟昶为刘牢之主簿,诣门谢,见云:``羊侯,羊侯,百口赖卿。''
\end{quote}

  402年,执政太傅司马道子的儿子司马元显奉诏讨伐桓玄,桓玄起兵``勤王'',起初桓玄以为他不受士族和将士支持,谁知士族对司马家的皇权集中十分抵触,与桓玄合作,桓玄一路没有受到抵抗,3月进取南京,黜司马道子,杀司马元显。于是桓玄要朝廷加封自己为总百揆,侍中、都督中外诸军事、丞相、录尚书事、扬州牧,领徐州刺史。这时候兖州刺史的副手羊孚羊子道就来拜访桓玄,表示对桓玄的支持。桓玄得到了士族的支持,投桃报李,采取了大量措施继续坚持抬高士族的地位,``及玄初至也,黜凡佞,擢俊贤,君子之道粗备,京师欣然。''
  诣门:上门拜访,因为桓玄职位太高,要先预约写求见贴。
  自顷:近来。
  世故睽离:世事变故,我俩暂别。睽也是离的意思。连绵词。
  心事纶蕰:心念国事,心思沉积烦扰。``纶蕰''有的本子上是``沦蕴'',意思是一样的。
  ``启晨光于积晦'':亚里斯多德讲过,一个能够看到两种不同领域的相似之处的人是天才,再能把这些相似之处联系起来是需要特殊天赋的。美国文学理论家哈罗德?布鲁姆说:诗本质上是比喻性的语言,集中凝练故其形式兼具表现力和启示性。羊孚保留下的作品很少,但他无疑是使用比喻的行家里手。这句话是说,桓玄你是照射迈迈长夜中的一缕晨光,是澄净浑浊时势中的一股清流。因为你英勇的革命举动,使东晋的政坛显得那么光明和清澈。但桓玄目标远大,想当皇帝,令士族失望,不过这个时候羊孚已暴病死亡。
  刘牢之:刘本是谢玄帐下北府军官,淝水之战成名,后与后燕战败罢官。398年,王恭重新启用他,结为兄弟。但经过司马元显策反,刘反戈一击,击败并取代王恭的职务。402年,司马元显派刘牢之征讨桓玄,这时候桓玄又策反刘牢之,导致执政的司马道子和司马元显垮台。桓玄执政后,解除刘牢之兵权,将其调离京口要他去会稽当太守。刘牢之感觉大祸临头,又打算当吕布。其参军刘袭说:``一个人一生当中三次反叛,怎能够立足于天地之间!''扬长而去。名将刘裕说:``你本来有劲卒数万,对桓玄是望风降服。现在他威震天下,朝野人情皆已归附,广陵您到得了吗?''于是将士纷纷脱离部队,其中应该就包括孟昶。刘牢之自杀了事,桓玄下令开棺斩首示众。
  孟昶:东晋末名臣,以俊美闻名,与其弟被人称为``双珠''。他和刘裕联合,击败桓玄后一路提拔,当上了丹杨尹、吏部尚书、尚书左仆射等。
  谢:谢辞,告辞。
  百口赖卿:刘牢之死后桓玄还枭首示众,其原来的部下就紧张了。孟昶希望通过羊孚来保全全家的性命,当年王导也是这样请求周顗。看样子羊孚确实做到了,桓玄并没有把孟昶一家怎么样。甚至还曾经打算重用孟昶,不过孟昶曾经的同事刘迈对桓玄说:``我没听过孟昶有什么特别才能,只知他们父子常互相赠诗吹捧。''
刘迈以直率著名,所以孟昶没有被启用。
  桓玄这个人从小头角峥嵘,志向远大,文采出众,心狠手辣,这些都是领袖的必备要素,但是他真正执政以后其缺点就暴露出来了,其才能不足以成大事,譬如说,他当上丞相后造了几艘轻巧小船,装满了自己心爱的字画珍玩。左右不解,桓玄说:``书画服玩既宜恒在左右,且兵凶战危,脱有不意,当使轻而易运。''他和刘裕部队决战时,数万对数千,居然还``惧有败衄,常漾轻舸于舫侧''。他不是枭雄,而是小人。
\#\#文学第四 小结
  《世说新语》当然是杰出的文学作品,但第四章``文学'',谈不上什么文学性,用现在的话说,应该是``学问'',它涵盖了一切文字作品理解、使用、产生过程和作品交流等林林总总。``世说''的编目虽然按孔门四科为基础,但概念也有了发展。在孔子的时代,文学主要是指``文献学''------理解和注解典籍。而到了《世说新语》,文学的范围扩大,文艺创作列入其中。编著者已经清楚地认识到这个问题,所以从1至65条,记录了各种文献的使用、注解、讨论等轶事,66条至最后,则主要记录各种写作的趣闻轶事。看了这些故事,我们能发现一些饶有趣味的东西。
  譬如学问之道要自信和谦虚。服虔在郑玄面前侃侃而谈,王弼在何晏面前侃侃而谈,张凭对刘惔阐述经义\ldots{}\ldots{}后进在前辈面前并无胆怯,大学者对小人物褒奖有加,``神伏''、``披襟解带''。人无贵贱,学有先后,正是有这种互相讨论的求学之道,达者为师的评价体系,故人们的眼界日益开阔,知识日益增长。
  学问之道讲究独立。魏晋时人自我意识扬长,他们反复争论才性之关系,本来就是在追问人的成长之道,他们千奇百怪的行为,归根到底是在表现自己的``才''或``性''。庾敳不把庄子放在眼里,``了不异人意''!袁宏在船上独自歌咏,孙盛和殷浩争论以至于辱骂打斗。有些人在学问之道上根本不讲情面,就算是朋友或者同僚,你不如我就是不如我,讲话不来客套。简文帝怕桓温怕的要死,但听说他清谈聚会一日一卦,就开口讽刺,顾夷对谢万说:如果我写,就没你什么事情了!支遁对王濛说:你了无长进!孩子是自己的好,观点是自己的好,文章是自己的好,这种感受不是傲慢,而是主体意识的觉醒。
  文章之道贵在捷思。曹植七步成诗,袁宏倚马可待,阮籍宿醉上表,王珣、桓玄的写作速度很快,文辞不重复,这些都是文学创作中久远的佳话。这种捷才未必就是天生的,而往往需要反复练习才做得到。古诗文创作本来就是个技术活,元代的欧阳玄一天可以写梅花诗上百首,清代的乾隆一生写过数万首诗,他们诗歌的水平不论,由于特殊的文化与社会背景,古人诗赋的写作比我们现在想象的要来的容易的多,因为诗赋写作是他们的生活方式和生存技能。熟然后快,快自然有美感,很快很快更有特殊的美感,就像跑步,运动员的姿势大体相似,谁最快谁就最美。
  文章之妙需要品味。作品不仅仅属于作者,更属于读者。魏晋时期人们喜欢品评人物,也喜欢品评文学。《世说新语?文学》选录了不少文字评价的故事,这篇文章写得怎么样,这篇诗歌中哪句为佳,其重点不是作品本身,而是读者如何利用作品来表现自己的情怀。只有真正为人所用的作品,才是优秀的作品。谢玄选择``杨柳依依'',自然是诗人的情怀,谢安选择``远猷辰告'',则是政治家的气度,王恭一路歌咏``所遇无故物,焉得不速老'',其风采宛若眼前,真是神仙中人,羊孚写《雪赞》,桓胤把它写在扇子上,故事的重点并不是羊孚,而是知音能赏的桓胤,不然把故事写入``言语''就行了。经过魏晋士人有意识地品评,后来的作者也注重锤炼词句,明代谭元春说:``必一句之灵,能回一篇之运,一篇之朴,能养一句之神,乃为善作'',清代刘熙载说:``(诗歌)语语微妙,便不微妙,须是一路坦易中忽然触着,乃是令人神远。''品味妙语佳句,就是体会文章之``神'',``令人神远''。
  我们喜欢文学,因为文学告诉我们世界的另一种真实。这种真实,来自于我们的精神世界。世界的事实,并不仅仅是事实本身,更高的事实不是``世界就是如此'',而是``世界应该如此''。


\end{document}
